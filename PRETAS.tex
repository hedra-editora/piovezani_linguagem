\begin{itemize}


\item \textbf{A linguagem fascista}, a partir de uma perspectiva histórica e da exposição do uso da linguagem pelo regime nazista, traça um paralelo entre dois casos emblemáticos da linguagem fascista: os discursos de Benito Mussolini e de Jair Bolsonaro A comparação entre seus desempenhos oratórios expõe ao leitor as propriedades dessa linguagem, seus recursos e seu funcionamento, mas também sua conservação e suas transformações ao passar da Itália do século \textsc{xx} ao Brasil do século \textsc{xxi}. Com base nos estudos do filólogo alemão Victor Klemperer, perscrutam"-se aqui os usos linguísticos mais característicos e os aspectos fundamentais da oratória fascista para, assim, compreender esse sistema de produção de crenças, devoções e fanatismos, sejam eles dedicados ao Führer, ao Duce ou ao Mito.
  
\item \textbf{Carlos Piovezani} é linguista, professor associado do
Departamento de Letras e do Programa de Pós"-Graduação em Linguística da
Universidade Federal de São Carlos e Pesquisador do \versal{CNP}q. Foi
coordenador do \versal{PPGL/UFSC}ar e coordena o Laboratório de Estudos do
Discurso (\versal{LABOR/UFSC}ar) e Grupo de estudos em Análise do
discurso e História das ideias linguísticas (\versal{VOX/UFSC}ar). É
autor de \emph{A voz do povo: uma longa história de discriminações}
(Vozes, 2020) e de \emph{Verbo, Corpo e Voz} (Editora \versal{UNESP}, 2009) e
organizador, entre outras, das seguintes obras: \emph{Saussure, o texto
e o discurso} (Parábola, 2016), \emph{História da fala pública} (Vozes,
2015), \emph{Presenças de Foucault na Análise do discurso} (Ed\versal{UFSC}ar,
2014) e \emph{Legados de Michel Pêcheux} (Contexto, 2011). Foi Professor
convidado da École des Hautes Études en Sciences Sociales (\versal{EHESS}/Paris)
e Professor visitante da Universidad de Buenos Aires (\versal{UBA}). O autor
agradece ao \versal{CNP}q pelo financiamento de suas pesquisas.

\item \textbf{Emilio Gentile} é historiador, professor emérito da Università
La Sapienza de Roma e membro da Accademia Nazionale dei Lincei. É
considerado um dos mais importantes historiadores do fascismo em todo o
mundo. Desde a década de 1970, já publicou dezenas de obras
incontornáveis para o estudo do fascismo e traduzidas em vários idiomas.
Entre outras, é autor das seguintes publicações: \emph{Le origini
dell'ideologia fascista} (Laterza, 1975), \emph{Il mito dello Stato
nuovo} (Laterza, 1982), \emph{The Sacralization of Politics in Fascist
Italy} (Harvard University Press, 1996), \emph{Fra democrazie e
totalitarismi} (Laterza, 2001), \emph{Politics as Religion} (Princeton
University Press, 2006), \emph{E fu subito regime. Il fascismo e la
marcia su Roma} (Laterza, 2012), \emph{In Italia ai tempi di Mussolini}
(Mondadori, 2018), \emph{Quien ès fascista} (Alianza Editorial, 2019).
Gentile recebeu, entre outras distinções, o prêmio \emph{Hans Sigrist}
da Universidade de Berna (2003) e a condecoração \emph{Renato Benedetto
Fabrizi} da Associazione Nazionale dei Partigiani d'Italia (2012).

\end{itemize}

