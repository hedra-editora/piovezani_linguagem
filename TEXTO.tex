\chapter[A linguagem fascista, \emph{por Carlos Piovezani}]{A linguagem fascista}


\hfill{\footnotesize ``É preciso falar a língua que o povo entende''}

\hfill{\footnotesize \textsc{j.\,g.}}


\bigskip

\noindent{}A frase não foi dita por alguém que se preocupasse em fazer do povo
protagonista de ações e decisões políticas. Diferentemente do que talvez
pudéssemos imaginar, foi Joseph Goebbels quem a pronunciou. O ministro
da Propaganda do Terceiro Reich e um dos braços direitos de Hitler
postulou a necessidade de os líderes nazistas usarem uma linguagem
popular desde o começo da ascensão totalitária. Na mesma ocasião,
Goebbels disse ainda: ``Quem quiser se comunicar com o povo tem de olhar
na fuça do povo''. A frequente referência ao povo não significava de
modo algum que o nazismo tivesse um real interesse em ouvir a sua voz.

Falar às massas para mais bem calar o povo: esta não seria a primeira
nem a última vez que assistiríamos a esse perverso expediente. Muitos
dos que se dirigem às multidões com o propósito de falar exclusivamente
em nome do povo acabam por lhe calar a voz. Trata"-se aí de um fenômeno
que se repete na história, mas não sem profundas transformações em sua
ocorrência em tempos e lugares distintos. Isso porque durante muitos
séculos a exclusão da voz e da vez das classes populares se deu por um
desprezo quase absoluto de suas dores e queixas, de suas revoltas e
reivindicações.

Na medida em que quase nunca são ouvidos, empobrecidos e marginalizados
tornam"-se mais suscetíveis a ouvir os que se apresentam falando em seu
nome. Assim foram produzidas as crenças, as devoções e os fanatismos
dedicados ao \emph{Duce}, ao \emph{Führer} e ao \emph{Mito}. Nesse
processo, os usos de uma \emph{linguagem fascista} desempenham
um papel fundamental. Vítima, espectador e seu mais fecundo analista, o
filólogo judeu, Victor Klemperer, testemunhou o poder dessa linguagem:
``O nazismo se embrenhou na carne e no sangue das massas por meio de
palavras, expressões e frases impostas pela repetição, milhares de
vezes, e aceitas inconsciente e mecanicamente''.\footnote{Klemperer, V.
  \emph{\textsc{lti}. A linguagem do Terceiro Reich}. Rio de Janeiro:
  Contraponto, 2009, p.\,55.}

Já terminada a guerra, Klemperer ainda hesitava em publicar suas
reflexões sobre a linguagem nazista. Decidiu fazê"-lo, ao ouvir o que uma
trabalhadora de Berlim lhe dissera ainda em um campo de refugiados.
``Ela era uma pessoa amável e logo percebeu que compartilhávamos o mesmo
pensamento político. Revelou que seu marido fora preso por ser
comunista. Com uma ponta de orgulho, contou que também passara um ano na
prisão. Por que você foi presa? --- perguntei. Porque empreguei certas
palavras\ldots{}''\footnote{Klemperer, 2009, p.\,424.}

Ao abrir a boca, podemos conquistar ou arruinar nossa liberdade, podemos
ganhar a vida ou sucumbir à morte. Vários mitos repetem essa ideia,
porque ao fazê"-lo os seres humanos respiram, se alimentam e falam.
Quando falamos, nossas palavras podem abrir ou fechar portas, podem
ampliar nossos horizontes ou aniquilar nossos sonhos. No livro do
\emph{Gênesis}, é de lábios abertos que vêm o sopro da vida, as palavras
da tentação e o anúncio da queda. A fala cria a existência e sua
finitude, gerando assim nossa própria humanidade. Nossa linguagem,
portanto, não está somente a serviço do que é útil, belo e justo. Ela
pode, ao contrário, servir ao que há de mais nefasto na condição humana:
o ódio por seu semelhante, visto como seu pior inimigo. Depois de
assistir de perto às atrocidades do nazismo e aos terríveis usos de sua
linguagem, Klemperer não têm dúvidas em afirmar que as ``palavras podem
ser como minúsculas doses de arsênico: são engolidas de maneira
despercebida e parecem ser inofensivas; passado um tempo, o efeito do
veneno se faz notar''.\footnote{Klemperer, 2009, p.\,55.}

Mais do que inofensivo, o veneno se apresentava como um suave remédio
indicado supostamente para curar as moléstias das classes trabalhadoras
e o mal"-estar das classes médias. Seu modo de usar compreendia a
administração constante de altas e variadas doses de linguagem nazista.
Já mencionamos a receita desta última prescrita pelo Dr. Goebbels: falar
ao povo e em nome do povo, falar ao povo como o povo fala e pode
entender. Com vistas a mais bem compreender em que consiste a linguagem
do fascismo, vejamos algumas das principais características dos usos
linguísticos do terceiro Reich e alguns dos aspectos mais fundamentais
da oratória de Hitler, segundo as descrições de Klemperer.

Os nazistas empregavam palavras estrangeiras com certa frequência em
seus discursos, uma vez que muitas pessoas não as entendiam muito bem e
se sentiam ``mais impactadas justamente porque não compreendiam bem o
significado''. Este não seria o único aparente paradoxo da linguagem
nazista: pregar que é preciso falar de modo que o povo entenda e usar
termos estrangeiros que boa parte da população não entendia. A esse se
somavam estes outros: sua língua compreendia ``desde a absoluta pobreza
de espírito até sua abundância exuberante''; a incessante repetição na
produção de seus textos e a constante introjeção inconsciente em sua
recepção. Uma das passagens de Klemperer que mais bem ilustra essa
tentativa de conjugar repetição e inconsciência é a seguinte:
``\emph{Mein Kampf}, de Hitler, insiste em afirmar a necessidade de
manter a massa na ignorância e explica claramente como intimidá"-la
contra qualquer reflexão. Um dos principais recursos para isso é
martelar sempre, repetidamente, as mesmas teorias simplistas que não
podem ser rebatidas.''\footnote{Klemperer, 2009, p.\,280.}

A linguagem nazista se caracteriza também por alterar o sentido das
palavras e a frequência de seu uso. É o que emblematicamente se deu com
o significado de ``fanático'' e com a assiduidade de ``povo''. De
desvairado, ``o termo `fanático' estava assumindo um novo sentido,
passando a significar uma feliz mescla de coragem e entrega
apaixonada''. Por sua vez, a palavra ``\emph{Volk} (povo) era empregada
nos discursos e nos textos com a mesma naturalidade com que se coloca
uma pitada de sal na comida''.\footnote{Klemperer, 2009, p.\,75 e 116.}

Já na imagem do corpo do orador nazista, pouca coisa talvez se
destacasse mais do que seus picos de energia e sua rigidez, o aspecto
mecânico de seus gestos e o empinamento de sua postura. Em sua voz tanto
quanto no tom de sua escrita, ressaltam"-se o estilo mais do que
enfático, com o qual se tenta excitar e exortar as massas, e as atitudes
irônicas e ferinas, com as quais menosprezam e insultam suas vítimas e
seus oponentes. A propósito desses traços da linguagem nazista,
Klemperer atesta que

\begin{quote}\looseness=-1
a linguagem do Terceiro Reich (\textsc{lti}) não fazia distinção entre
oralidade e escrita. Para ela, tudo era discurso, arenga, alocução,
invocação, incitamento. O estilo do ministro da Propaganda não
distinguia a linguagem do discurso e a linguagem dos textos, razão pela
qual era tão fácil declamá"-los. \emph{Deklamieren} (declamar) significa
literalmente falar alto sem prestar atenção ao que se diz. Vociferar. O
estilo obrigatório para todos era berrar como um agitador berra na
multidão.

A \textsc{lti} transforma tudo em apelo e exclamação e usa \emph{ad
nauseam} o que podemos chamar de aspas irônicas. Na \textsc{lti}, o
emprego irônico predomina largamente sobre o neutro, pois ela odeia a
neutralidade. Precisa ter sempre um adversário a ser
rebaixado.\footnote{Klemperer, 2009, p.\,66 e 133. \textsc{lti} é a sigla
  para a linguagem do Terceiro Império, tal como os próprios nazistas
  também se referiam ao \emph{Reich}.}
\end{quote}

Não dar margem à consciência crítica e sempre eleger um inimigo comum
são outras duas características que se amalgamam no pensamento e na
linguagem do Terceiro Reich. Além do gosto pela declamação e pela
vociferação, os nazistas se valiam também de formas e de conteúdos
superlativos para tentar embotar a crítica: ``os superlativos são a
forma linguística mais usada pela \textsc{lti}, o que é fácil de compreender,
pois o superlativo é o melhor instrumento à disposição do orador e do
agitador, a forma propagandística por excelência''. Às formas mais
canônicas do superlativo se juntam conteúdos que vão na mesma direção.
Durante doze anos, os pronunciamentos de Hitler teriam produzido uma
recepção total e ideal, segundo a imprensa do Reich, porque suas
manchetes costumavam repetir à exaustão o seguinte estereótipo oficial:
``O mundo escuta o \emph{Führer}''. De modo análogo, ``quando se vencia
uma batalha grande, dizia"-se que fora `a maior batalha da história
universal'.''\footnote{Klemperer, 2009, p.\,335--336.}

\looseness=-1
Já a eleição do inimigo comum eleito pelos nazistas é mais do que
conhecida e imensamente lamentada. Desde a ascensão de Hitler, os judeus
foram vítimas de genocídios e de violências físicas e simbólicas de toda
sorte. Menos evidente é a associação entre o ódio aos judeus e a aversão
à crítica. Klemperer trata dessa associação, ao assim formular o que
chama de ``a lei suprema de Hitler'': ``Não permitas que teu ouvinte
chegue a formular qualquer pensamento crítico. Trata tudo de forma
simplista! Se falares de diversos adversários, alguém poderia ter a
ideia de que talvez seja tu que estejas errado. Reduza todos a um
denominador comum, junte"-os, crie uma afinidade entre eles! O judeu se
presta muitíssimo bem a uma operação desse tipo, muito clara e
compatível com a mentalidade popular''. Para o eficaz desempenho dessa
operação, os usos linguísticos cumprem uma importante função, além de
serem indícios das paixões e afetos dos nazistas: ```Judeuzinho' e
`peste negra', são expressões de escárnio e desprezo, mas também de
horror e medo angustiado: essas duas formas estilísticas estarão sempre
presentes quando Hitler se referir aos judeus em discursos e
alocuções.''\footnote{Klemperer, 2009, p.\,272.}

Há uma presença constante dessas e de outras formas linguísticas, como
adjetivos e locuções adjetivas profundamente depreciativos, que de modo
quase invariável acompanhavam o uso do substantivo ``judeu'', tais como
``\emph{gerissen} (ladino), \emph{listig} (manhoso), \emph{betrügerisch}
(fraudulento), \emph{feige} (covarde), \emph{plattfüssig} (que tem pés
chatos), \emph{krummnasig} (que tem nariz aquilino), \emph{wasserscheu}
(que tem medo de água)''. Com essas formas da língua, os nazistas e seus
partidários aviltam tanto a alma quanto o corpo dos judeus. Ao fazê"-lo,
revelam ter uma concepção ao mesmo tempo abjeta e tola do judaísmo.
Segundo Klemperer, nessa concepção

\begin{quote}
reside grande parte de sua força, pois é a partir dela que ele se une à
plebe mais embrutecida, que em plena era da industrialização nem sequer
faz parte do proletariado fabril, a uma parte da população rural e
sobretudo à massa pequeno"-burguesa apinhada nas grandes cidades. Para
esses homens e mulheres, a pessoa que se veste de maneira diferente ou
fala de outra forma não é uma outra pessoa, e sim um animal de outro
curral, com o qual não pode haver acordo, que se deve odiar e enxotar a
pontapés.\footnote{Klemperer, 2009, p.\,273.}
\end{quote}

A diferença é reduzida a medo, repulsa e chacota, e o diálogo, a ódio,
violência e extermínio. Não deveríamos jamais subestimar o letal poder
da linguagem, porque ele concorre decisivamente para estabelecer ``um
sentimento instintivo de antagonismo que se opõe a tudo o que é estranho
e desperta animosidade tribal'', para instaurar uma ``consciência de
raça que está alojada no estágio primitivo do desenvolvimento humano, e
que só será superada quando a horda humana aprender a não mais ver na
horda vizinha um bando de animais diferentes.''\footnote{Klemperer,
  2009, 274.}

Depois desse sobrevoo por alguns traços da linguagem do terceiro Reich,
passemos agora a examinar certas propriedades da oratória de Hitler. Uma
dessas propriedades é a junção que ele promove em seus pronunciamentos
entre histeria e violência da linguagem: ``O \emph{Führer} pronuncia
algumas frases diante de uma grande assembleia. Cerra o punho, contorce
o rosto, sua fala lembra mais o urro de um animal, está mais para um
acesso de cólera do que para um discurso''. Para Klemperer, essa atitude
de Hitler seria o sintoma mais agudo de sua insegurança, uma vez que ele
``aparenta ser o todo"-poderoso, e talvez seja; porém, a impotência de
seu ódio aparece nos gestos e no timbre da voz''. Os excessos lhe
indicariam uma falta. Picos vocais enérgicos e repetições gestuais mais
do que agitadas são sintomas da falta de segurança do \emph{Führer}:
``Alguém anunciaria assim, tão reiteradamente, um reinado milenar e a
eliminação dos opositores, caso se sentisse seguro quanto à duração
desse reino e ao extermínio dos opositores?''\footnote{Klemperer, 2009,
  p.\,77.}

\looseness=-1
Estaríamos equivocados, caso acreditássemos que tais excessos, agitações
e inseguranças decorressem de alguma negligência na preparação e na
execução dos discursos de Hitler. Ao relatar a experiência de uma escuta
pelo rádio de um programa da propaganda nazista, Klemperer destaca
justamente todo o esmero dispensado aos preparativos para emoldurar
perfeitamente o desempenho oratório do \emph{Führer}: ``Primeiro, o
barulho das sirenes soava como um uivo por toda a Alemanha; depois o
minuto de silêncio, também em todo o país. Em seguida, sem grande
originalidade, mas realizado com muita perfeição, vinha tudo o que era
necessário para criar uma moldura para o discurso de Hitler''. Eis como
Klemperer descreve mais detalhadamente essa experiência de escuta:

\begin{quote}
Em uma fábrica Siemens, durante minutos, barulho ensurdecedor da
empresa, marteladas, ruídos em geral, alvoroço retumbante, assobios e
rangidos. Ouve"-se então a sirene, o canto e o silenciar gradativo das
rodas, que vão parando. Emergindo do silêncio, a voz grave de Goebbels
anuncia mensagem. Somente então surge Hitler, durante três quartos de
hora. Foi a primeira vez que ouvi um discurso completo dele. Na maior
parte a voz soava muito agitada, esganiçada, às vezes, rouca. Só que
dessa vez muitas passagens eram pronunciadas em tom de lamúria, como se
ele fosse o pregador de uma seita. Prega a paz, elogia a paz, quer o
`sim' da Alemanha, não por ambição pessoal, mas para poder preservar a
paz, repelindo os ataques de uma cambada internacional de negocistas
desenraizados que, em nome dos lucros de maneira inescrupulosa, atiçam
povos uns contra os outros, milhões de pessoas\ldots{}

Tudo isso, inclusive o estudado tom de voz, as interrupções calculadas
para enfatizar (`os judeus!'), tudo eu já conhecia de longa data. Mas, a
despeito do caráter repisado e da hipocrisia evidente, que até mesmo um
surdo perceberia, o ritual tinha sua eficácia renovada por causa da
originalidade de detalhes bem concebidos\ldots{}\footnote{Klemperer, 2009, p.\,86--87.}
\end{quote}

As coisas ditas, os modos de dizer e mesmo os silêncios de Hitler
concorrem para estabelecer e reforçar o que seria sua estreita ligação
com Deus: ``o \emph{Führer} reiterou o seu relacionamento estreito com a
divindade, sua condição especial de eleito e filho de Deus, sua missão
religiosa''. Ainda quando de sua triunfal ascensão, Hitler faz um
pronunciamento em junho de 1937, no qual afirma o seguinte: ``A
Providência nos conduz, agimos conforme a vontade do Onipotente. Ninguém
pode fazer a história dos povos e do mundo se não contar com a bênção da
Providência Divina''. Antes disso, logo no início de 1932, Hitler havia
feito um discurso no Palácio de Esportes, que motivou a seguinte
anotação de Goebbels em seu diário: ``No final, seu discurso assume um
\emph{pathos} oratório maravilhoso e inacreditável, que termina com um
`amém'. Dá a tudo um ar tão natural que as pessoas se comovem e se
sensibilizam profundamente\ldots{} as massas do Palácio de Esportes ficam
tomadas por um frenesi\ldots{}''. A nota de Goebbels é assim comentada por
Klemperer: ``O `amém' mostra claramente que esse é um discurso de
caráter religioso e pastoral. E a frase `dá a tudo um ar tão natural',
redigida por um ouvinte mais experiente, permite concluir que Hitler fez
muito uso da retórica, conscientemente''. Ao comentário se acrescenta
ainda o seguinte: ``Quem se dispuser a ler as receitas para sugestionar
as massas, ensinadas pelo próprio Hitler em \emph{Mein Kampf}, não terá
dúvidas sobre a sedução deliberada, baseada no registro
religioso''.\footnote{Klemperer, 2009, p.\,188.}

Às coisas ditas se somam as formas de dizer e os silêncios na construção
dos laços estreitos entre Hitler e o campo religioso: ``O fato de que
culmine em uma dimensão religiosa deve"-se, pois, em primeiro lugar, a
determinadas expressões linguísticas tomadas especificamente do
cristianismo e também, em grande medida, porque seus discursos assumem o
tom e a ênfase de uma prédica''. O \emph{Führer} discursa outra vez e
Goebbels novamente lhe dedica uma anotação em seu diário: ``Todos se
emocionam profundamente quando no acorde final do discurso ouve"-se a
poderosa `oração de graças' holandesa, cuja última estrofe é coberta
pelo repicar dos sinos da Catedral de Königsberg. Graças ao rádio, esse
hino é transmitido pelo espaço celestial para toda a Alemanha''. Uma
oração, os dobres dos sinos e as ondas do rádio elevam a fala de Hitler
à condição de uma manifestação da própria divindade. Ele fala com certa
frequência, porém, não o fez incessantemente. A intermitência, o
silêncio e repercussão da doutrina pela boca de outrem convergem para a
produção de um efeito religioso:

\begin{quote}
O \emph{Führer} não pode nem deve falar todos os dias. A divindade tem
de ocupar um trono sobre as nuvens, e seus pronunciamentos devem ser
feitos mais pela boca dos sacerdotes do que pela própria. No caso de
Hitler, essa é uma vantagem adicional, pois seus amigos e servidores
podem erigi"-lo em salvador de maneira ainda mais firme e com maior
desenvoltura, venerando"-o ininterruptamente e em coro. Sob a prepotência
reinante, ninguém pôde contradizê"-lo jamais.\footnote{Klemperer, 2009,
  p.\,190.}
\end{quote}

Não ser contestado para ser mais bem obedecido, eis o ideal regime de
escuta dos que falam a língua fascista. Para alcançar tal objetivo, os
nazistas não se valeram somente de um único meio. Não foi apenas a
eleição de Hitler como um salvador, mediante o que ele próprio e seus
sacerdotes diziam, mediante o modo como o faziam e ainda mediante o
silêncio quase divino do \emph{Führer}, que contribuiu para arrefecer o
pensamento crítico de muitos e para silenciar a voz da objeção de outros
tantos. Outro recurso retórico fundamental para abalar os públicos
ouvintes e lhes provocar uma recepção anuente foram as constantes e
súbitas mudanças de estilo oratório no interior de um mesmo
pronunciamento. Klemperer chamou esse recurso de combinar alternada e
repentinamente estilos quentes e frios, simples e elevados, de ``ducha
escocesa'':

\begin{quote}
O ápice da retórica nazista não está nessa contabilidade que separa
cultos e incultos, e impressiona as massas com pedaços de erudição. O
grande desempenho, aquele em que a maestria de Goebbels é incomparável,
consiste em misturar elementos estilísticos heterogêneos. Não, esse não
é o termo exato. O que Goebbels faz é saltar subitamente de um extremo a
outro, do erudito ao plebeu, do tom sóbrio e racional para o
sentimentalismo das lágrimas contidas com esforço, da simplicidade de um
Fontane ou da vulgaridade berlinense para o tom patético do defensor da
fé e do profeta. O efeito é como uma reação da pele, fisicamente eficaz,
similar àquele produzido pela ducha escocesa e seu choque térmico:
primeiro quente, depois frio. O sentimento do ouvinte nunca está em
repouso, é constantemente jogado de um lado para outro, de modo que o
espírito crítico não tem tempo de se recompor.\footnote{Klemperer, 2009,
  p.\,384--385.}
\end{quote}

A despeito da eficiência da ducha escocesa oratória aplicada pelos
nazistas, Klemperer conseguiu manter o espírito crítico para mais bem
descrever, interpretar e compreender a linguagem fascista. Duas outras
ocasiões de fala lhe deram ensejo para efetuar uma comparação. Na manhã
de um domingo de outubro de 1932, atendendo ao convite que lhe fora
feito pelo Consulado italiano da cidade de Dresden, ele foi ao cinema
para assistir ao filme \emph{Dez anos de fascismo}. Era a primeira vez
que Klemperer via e ouvia o \emph{Duce} falar, em um filme que
considerou ser uma obra"-prima. Vejamos qual foi sua impressão a respeito
do Mussolini orador:

\begin{quote}
Mussolini discursa do alto do balcão do palácio de Nápoles para a
multidão espalhada no chão; imagens do povo e grandes imagens do orador
se alternam, apresentando as palavras de Mussolini e a aclamação da
multidão à qual se dirige. Pode"-se ver como o \emph{Duce}, relaxando
depois de cada pequena pausa, volta a dar à face e ao corpo uma
expressão de energia máxima e de tensão. A exaltação do pregador aparece
no tom de voz de ritual eclesiástico, lançando frases curtas, como
fragmentos litúrgicos, diante das quais obtém reações emocionadas de
todos, sem qualquer esforço mental, mesmo que não captem o sentido das
palavras, ou justamente por não terem capacidade para captá"-lo.

A boca é gigantesca. De vez em quando faz gestos tipicamente italianos
com os dedos. A massa, exultante, grita entusiasmada. Assobia de maneira
alvoroçada, especialmente quando o nome do inimigo é mencionado. Repete
o gesto de saudação fascista, o braço estendido para o alto.\footnote{Klemperer,
  2009, p.\,102.}
\end{quote}

Quatro meses mais tarde, Klemperer ouviria pela primeira vez a voz de
Hitler. Ele já havia se tornado chanceler da Alemanhã, mas em março de
1933 ocorreriam eleições que lhe confirmariam ou não o cargo. Houve
preparativos em larga escala para aquele pleito, o que incluiu até mesmo
o incêndio do Parlamento. Naquele cenário, o sucesso eleitoral de Hitler
era dado como certo. Foi nesse contexto que ele falou ao vivo na rádio
de Königsberg ``demonstrando certeza na vitória''. Na fachada do hotel
da principal estação ferroviária de Dresden, havia um alto"-falante que
transmitia esse seu discurso. Ali ``uma multidão delirante se
apertava''. O pronunciamento de Mussolini no Palácio de Nápoles e este
discurso de Hitler em Königsberg foram comparados por Klemperer. Eis uma
passagem que resume essa sua comparação:

\begin{quote}
Que diferença havia em relação ao modelo de Mussolini!

Era visível o esforço físico que o \emph{Duce} fazia para imprimir
energia às frases, o empenho para manter a massa aos seus pés, mas em
sua língua materna ele se expressava livremente, a ela se entregava,
apesar do desejo de dominar. Mesmo tropeçando entre a oratória e a
retórica, o \emph{Duce} permanecia um orador, sem contorções nem
convulsões. Hitler queria sempre aparecer, ou como bajulador ou cheio de
sarcasmo --- dois registros pelos quais gostava de transitar. Hitler
falava, ou melhor, gritava em convulsões. Até mesmo no máximo de
exaltação é possível manter certa dignidade e algum bem"-estar interior,
um sentimento de autoconfiança e de estar em harmonia consigo mesmo e
com os demais. Esses aspectos faltavam a Hitler, que desde o começo era
um retórico consciente, retórico por princípio. Não se sentia seguro nem
mesmo em situação de triunfo, fustigava com seu linguajar aos
adversários e as ideias contrárias. Não tinha compostura, sua voz não
possuía musicalidade, o ritmo das frases açoitava a si mesmo e aos
demais.
\end{quote}

Sua condição de espectador e de vítima do regime nazista contribuiu para
que Klemperer se tornasse um precursor e o mais fecundo analista da
linguagem do Terceiro Reich. A proximidade com as atrocidades
totalitárias lhe proporcionou a observação e o exame de fenômenos,
fatores e aspectos que talvez pudessem passar despercebidos. Foi a
conjunção entre a adjacência e a sagacidade que lhe permitiu notar como
a linguagem foi decisiva para a ascensão de Hitler e para sua
consolidação no poder.\footnote{Outra testemunha da época, o médico
  foniatra, Alexis Wicart, afirmou o seguinte a propósito de Hitler: ``O
  surgimento do \emph{Führer} é uma conquista vocal e infeliz dele, caso
  sua grande voz se torne rouca. Com efeito, Hitler tem somente a
  veemência do gesto e do verbo, aos quais se acrescentam as forças
  vivas do despotismo'' (Wicart, Alexis. \emph{Les puissances vocales}.
  2 volumes. Paris: Éditions Vox, 1935, p.\,5).} Assim, o filólogo judeu
alemão pôde observar o papel desempenhado pelos usos e recursos da
língua, do corpo e da voz dos nazistas para introjetar as minúsculas e
poderosíssimas ``doses de arsênico'', que envenenou boa parte dos
alemães e tornou possível um dos maiores genocídios já vistos na
história ocidental. Com base em suas descrições, podemos perceber que
não lhe escapou a importância das tecnologias de linguagem: o microfone
e os alto"-falantes, o rádio e o cinema.\footnote{Não sem certo lamento,
  como se a intermediação tecnológica implicasse a perda de uma aura,
  Wicart também notara a presença da tecnologia de linguagem e a
  modificação que ela produzira: ``Sem alto"-falantes e sem microfones,
  foram feitos os verdadeiros discursos de Hitler, pronunciados de
  improviso, em meio às dificuldades das lutas em que esteve envolvido,
  antes de chegar ao poder. Agora, ele utiliza o alto"-falante e o
  microfone, em circunstâncias específicas, nas quais se encontra diante
  de multidões de mais de 200.000 espectadores'' (Wicart, 1935, p.\,125).}

Próximo dos eventos e dos atores, Klemperer viu de perto o que se lhe
apresentavam como as semelhanças e as diferenças entre o \emph{Duce} e o
\emph{Führer}. Por um lado, ``\emph{Führer} é a tradução alemã de
\emph{Duce}, a camisa marrom é uma variação da camisa negra italiana, a
saudação alemã é uma imitação da saudação fascista. A própria cena do
discurso do \emph{Führer} diante do povo reunido copia o modelo de
propaganda italiano''. Por outro, havia uma diferença em relação ao
modelo de Mussolini, pois, enquanto o ``\emph{Duce} permanecia um
orador, sem contorções nem convulsões'', por sua vez, ``Hitler falava,
ou melhor, gritava em convulsões''.\footnote{Klemperer, 2009, p.\,102 e
  106.} A despeito de sua pertinência e de todo esclarecimento que
essas indicações aportam, não podemos perder de vista que a proximidade
que lhes enseja também compreende limites. Não vemos as mesmas coisas
nem o fazemos de modo idêntico, se as olhamos de perto ou de longe, se
as inscrevemos em uma ou outra temporalidade histórica. De certo modo, a
contiguidade e a curta duração tendem a nos fazer enxergar mais as
diferenças entre fenômenos, processos e agentes, ao passo em que a
distância e as médias e longas durações históricas facilitam a
identificação de suas semelhanças.

Após o valioso trabalho de Klemperer nos ter apresentado uma análise da
linguagem do Terceiro Reich e dos recursos retóricos mobilizados por
Hitler, podemos aqui anunciar a proposta deste livro: um estudo
histórico de dois casos emblemáticos da \emph{linguagem
fascista}. Sua leitura proporcionará um cotejamento entre os discursos
de \emph{Benito Mussolini} e de \emph{Jair Bolsonaro}.
A comparação entre seus desempenhos oratórios permitirá ao leitor mais
bem conhecer as propriedades dessa linguagem, seus recursos e seu
funcionamento, mas também sua conservação e suas transformações, ao
passar da Itália das primeiras décadas do século \textsc{xx} ao Brasil do século
\textsc{xxi}. Assim, poderemos avançar na compreensão dos fascistas de ontem e
dos de nossos dias e de seus antigos e atuais modos de falar às massas
para mais bem calar o povo e seus porta"-vozes.

Para calar a voz do povo, para silenciar opositores e para fazer aceitar
o aniquilamento de seus adversários, o fascismo de ontem investia
principalmente um carisma distintivo e tradicional em seu orador, já o
de nossos tempos vale"-se sobretudo de um carisma \emph{pop} daquele que
fala às massas. Mais ou menos esquematicamente, se poderia dizer que
aquele falava com grande energia depositada em seu corpo, em sua língua
e em sua voz e com uma aparente firmeza de caráter e de valores, para se
impor como um superior sobre o povo, enquanto este, por seu turno, tenta
lhe falar de modo natural e autêntico, ainda que também enérgico, e de
forma simples e clara, para se aproximar do povo, como se fosse uma
pessoa comum.

Esse esquematismo na exposição de fascismos europeus de ontem e do
brasileiro de hoje não deve recobrir a complexidade dos fenômenos nem a
condição compósita de cada líder fascista. Isso não impede que possamos
destacar os traços mais predominantes de cada um. De modo semelhante à
opção pelo estilo oratório da ``ducha escocesa'', há na propaganda
fascista do começo até meados do século \textsc{xx} o uso do modelo do ``grande
homem comum'', ou seja, de um líder que se apresenta como ``alguém que
sugere tanto onipotência quanto a ideia de que é apenas um de
nós''.\footnote{Adorno, Theodor. A teoria freudiana e o padrão da
  propaganda fascista. Disponível no \textit{Blog da Boitempo}, 25 de outubro de 2018.}
Na que recentemente emergiu e grassou entre nós, foi, antes, forjado um
perfil comum, autêntico e autoritário.

Em um caso, usa"-se uma língua mais ou menos próxima e distinta da do
povo para lhe dirigir a palavra e comandá"-lo. Já no outro, fala"-se a
língua do povo para falar ao povo, supostamente no intuito de fazê"-lo em
seu nome e certamente a fim de mais bem conduzi"-lo. O \emph{Duce} e o
\emph{Führer} se apresentavam como líderes que exerciam um comando
vertical com base em sua autoridade fabricada pela violência física e
simbólica, agindo, se comportando e falando como homens de uma elite, ao
passo em que o \emph{Mito} se apresenta como um porta"-voz, que exerce
uma condução horizontal com base em sua tramada proximidade com as
classes populares, produzida pelo que seria sua coragem de dizer a
verdade sem papas na língua, agindo, se comportando e falando como um
homem do povo.

Em ambos os casos, nos diferentes fascismos, o povo é ``apenas uma
ficção teatral''. Contudo, essa ficção e suas práticas de linguagem,
seus regimes de fala e de escuta não ocorrem de modo idêntico em
diferentes contextos históricos. ``Não precisamos mais da Piazza Venezia
ou do estádio de Nuremberg. Em nosso futuro, desenha"-se um populismo
qualitativo de tevê ou internet, no qual a resposta emocional de um
grupo selecionado de cidadãos pode ser apresentada e aceita como `a voz
do povo'\,''.\footnote{Eco, Umberto. \emph{O fascismo eterno}. Rio de
  Janeiro / São Paulo: Record, 2018, p.\,56--57.} Essa passagem de
Umberto Eco se encontra no décimo terceiro traço do que ele designa como
``o fascismo eterno''. Trata"-se ali de indicar que uma das
características do fascismo é sua condição de um ``populismo
qualitativo'', ou seja, um regime de governo no qual ``os indivíduos
enquanto indivíduos não têm direitos, e o `povo' é concebido como uma
qualidade, uma entidade monolítica que exprime a `vontade comum'. Como
nenhuma quantidade de seres humanos pode ter uma vontade comum, o líder
se apresenta como seu intérprete''.\footnote{Eco, 2018, p.\,55.} Esta é a
razão pela qual nesse populismo fascista o povo nada mais é do que ``uma
ficção teatral''.

De nossa exposição das principais características da linguagem do
terceiro Reich e dos aspectos mais fundamentais da oratória de Hitler,
do anúncio de nosso objetivo neste livro e dessa passagem de Eco que
acabamos de ler, poderiam derivar as seguintes questões: quais são as
demais características do fascismo? Como se pode distingui"-lo do
populismo e da demagogia? É excessivo chamar as práticas políticas do
bolsonarismo e os desempenhos oratórios de Bolsonaro de fascistas? Como
nosso propósito aqui não é o de responder diretamente a essas perguntas,
mas o de expor dois estudos históricos de caso --- o de Mussolini e o de
Bolsonaro, como oradores que se propuseram a falar ao povo --- para mais
bem compreender o que disseram, suas maneiras de dizer, os meios de
fazer circular as coisas ditas e as formas de infundir a crença nos que
lhes ouviram, nos limitaremos a comentá"-las e a eventualmente
responder"-lhes, respeitando a finalidade deste nosso estudo. Para tanto,
comecemos com a primeira.

Em inúmeras passagens de seu livro, Klemperer aponta a onipresença de um
anti"-intelectualismo entre os nazistas. Em seu programa pedagógico,
Hitler ``coloca o preparo físico em primeiríssimo lugar'', enquanto ``a
formação intelectual e seu conteúdo científico ficam por último, sendo
admitidos a contragosto, com desconfiança e desprezo''.\footnote{Klemperer,
  2009, p.\,39--40. Entre outras tantas, poderíamos ainda citar as
  seguintes: ``o nacional"-socialista rejeita profundamente a atividade
  de pensar sistematicamente, pois, por instinto, de preservação, ele
  precisa execrar o pensamento sistemático''; ``filosofar é uma
  atividade da razão, do raciocínio lógico, que o nazismo considera o
  pior inimigo''. (Klemperer, 2009, p.\,170 e 232).} Uma faceta
correlata ao desprezo pelos universos intelectual, científico e
artístico é ``o culto da ação pela ação'': ``a ação é bela em si e,
portanto, deve ser realizada antes de e sem nenhuma reflexão. Pensar é
uma forma de castração. Por isso, a cultura é suspeita na medida em que
é identificada com atitudes críticas''.\footnote{Eco, 2018, p.\,47--48.}
Para o fascismo, a complexidade deve ser banida, e a réplica crítica e o
desacordo são vistos como uma traição.

Também seria uma propriedade fascista a doutrina da guerra total. Nela,
``tudo é espetáculo bélico, o heroísmo militar pode ser encontrado em
qualquer fábrica, em qualquer porão. Crianças, mulheres e idosos morrem
a mesma morte heroica, como se estivessem no campo de batalha, com
frequência usando o mesmo uniforme desenhado para jovens soldados no
\emph{front}''.\footnote{Klemperer, 2009, p.\,42.} Assim, o pacifismo
passa a ser ``conluio com o inimigo''. Isso porque no fascismo ``não há
luta pela vida, mas antes `vida para a luta'; o pacifismo é mau porque a
vida é uma guerra permanente''.\footnote{Eco, 2018, p.\,52.} Nessa
obsessão pela guerra total, se delineia outro traço da cosmovisão
fascista: o gosto pela uniformização. Deseja"-se a padronização dos
pensamentos, das ações e das formas de expressão. O fascismo ``cresce e
busca o consenso utilizando e exacerbando o natural medo da
diferença''\footnote{Eco, 2018, p.\,49--50.} e produz igualmente uma tal
uniformização da linguagem que os brutamontes da Gestapo e os judeus
perseguidos, adeptos e opositores, beneficiários e vítimas, todos
seguiam os mesmos modelos linguísticos em suas conversas.\footnote{Klemperer,
  2009, p.\,51.}

A crença na irracionalidade das massas e a necessidade de se lhes impor
um líder e um salvador são outros dois traços do fascismo. Entre ambas,
há um vínculo sólido e estreito. Ao fato de que a ``doutrina nazista
acredita na estupidez das massas, consideradas incapazes de
raciocinar'', responde o modo como o \emph{Führer}, o \emph{Duce} e o
\emph{Mito} se apresentam: ``Todos os meios de comunicação tinham
anunciado: `Cerimônia das 13 às 14 horas. Na décima terceira hora Hitler
comparecerá para encontrar os trabalhadores'. É a linguagem do
evangelho. O Senhor, o Salvador vem para os pobres e para os que
perderam o rumo. Adolf Hitler, o salvador, aparece para os trabalhadores
na cidade de Siemens''. Mas reponde igualmente a forma como são
recebidos, como líderes infalíveis, nos quais se deve depositar uma fé
cega e inquebrantável. Já no pós"-guerra, Klemperer, quando trabalhava em
um centro de reabilitação dos arrependidos adeptos do Terceiro Reich,
reencontra um ex"-aluno e antigo membro do Partido Nazista. O diálogo
entre eles se encerra assim:

\begin{quote}
\forceindent{}--- Então por que você não está sendo reabilitado?

--- Porque não solicitei nem posso fazê"-lo.

--- Não entendo.

Pausa. Depois, com dificuldade, os olhos baixos:

--- Não posso negar, eu acreditava nele.

--- Mas é impossível que continue a acreditar. Você vê no que deu; e,
agora, todos os crimes hediondos estão expostos à luz do dia.

Pausa ainda mais longa. Então, bem baixinho:

--- Concordo com tudo o que o senhor diz. Foram os outros que não o
compreenderam que o traíram. Mas nele, \textsc{nele}, eu ainda
acredito.\footnote{Klemperer, 2009, p.\,87 e 198 (destaque do autor). Eco
  também ressalta esse vínculo entre incapacidade das massas e dominação
  do líder como um dos traços do fascismo: ``O líder sabe que sua força
  se baseia na debilidade das massas, tão fracas que têm necessidade e
  merecem um `dominador'.'' (2018, p.\,53).}
\end{quote}

Esboçar brevemente alguns pontos de aproximação e de distanciamento
entre fascismo, populismo e demagogia exige concebê"-los como produções
históricas. Parece haver uma quase invariância neste fenômeno: sujeitos,
grupos e classes sociais bem estabelecidos, que pressupõem ou dão a
impressão de pressupor uma série de incapacidades da participação
popular mais ou menos direta nas decisões e ações políticas, pretendem
reduzir essa participação, conservar um \emph{status quo} ou retroagir a
um estado de menor dinâmica democrática e igualitária. Para fazê"-lo,
elegem um líder para falar às massas, adulá"-las e manipulá"-las, de modo
que apoiem a abolição dessa dinâmica que lhes favorece. Ante a
relativamente constante reaparição disso que parece ser um mesmo
fenômeno, é preciso conhecer a especificidade de suas diferentes
emergências em tempos, lugares e contextos diversos.

O mais antigo registro conhecido da palavra ``demagogia'' se encontra na
comédia \emph{Os cavaleiros}, de Aristófanes, que foi pela primeira vez
encenada em 424 a. C. Sua presença nesse texto indica que já havia um
uso relativamente corrente da palavra para designar a prática política
sob a forma dos debates e das deliberações na assembleia. A princípio,
tratava"-se, portanto, de um termo mais ou menos neutro ``que se carrega
de traços negativos pelo modo como os novos políticos, provenientes das
camadas baixas, praticam a \emph{demagogia}''. Numa das passagens da
peça de Aristófanes, cujo título se refere à classe mais bem quista por
seu autor, isto é, os cavalheiros atenienses, um servo incita deste modo
um salsicheiro, com pretensões políticas e com o apoio dos aristocratas,
a enfrentar um emergente líder popular chamado Paflagon: ``Conquiste o
povo com saborosas iguarias verbais. Você tem todos os requisitos para a
demagogia: uma voz repugnante, origens baixas e vulgaridade. Você tem
tudo de que precisa para fazer política''.\footnote{Aristófanes citado
  por Canfora, Luciano. Demagogia. In: \emph{Serrote}, vol. 10
  (Suplemento ``Alfabeto'' letra D). Rio de Janeiro: \textsc{ims}, 2012, p.\,12.}

Infelizmente, os princípios igualitários que marcam a democracia e sua
abertura a porta"-vozes populares encontraram já em seu nascimento
perversos e poderosos detratores. O próprio termo ``democracia'' foi
concebido para menosprezar o regime de governo baseado na igualdade
entre os cidadãos e na vontade popular: ``a palavra democracia foi
inventada para afirmar que o poder de uma assembleia de homens iguais só
podia ser a confusão de uma turba informe e barulhenta''.\footnote{Rancière,
  Jacques. \emph{O ódio à democracia}. São Paulo, Boitempo, 2014. 117.}
Ao lado de Aristófanes, outros célebres e anônimos antidemocratas
contribuíram para deslocar a carga semântica negativa de democracia para
o termo ``demagogia'' e seus correlatos ``demagógico'' e ``demagogo''.
Naquele contexto, a participação popular era concebida pelos
aristocratas como uma decadência. Desde fins do século V a. C. até o
período helênico, um dos principais atores da política ateniense é
justamente o povo, a massa dos que não eram proprietários de terra ou de
outros bens econômicos. Por isso, ``ganha cada vez mais força a ideia de
que a política em si, numa cidade regida por democracia, só pode ser
demagogia em sentido pejorativo''.\footnote{Canfora, 2012, s.\,p.\,15}

Além disso, são nítidos os vínculos entre povo, palavra e poder na visão
antidemocrática. Quanto mais a noção pejorativa da democracia se impõe,
mais ``fica claro que o veículo privilegiado da demagogia é a palavra.
Nas tentativas de sua definição é latente, mas bem perceptível, a
identificação entre demagogo e hábil orador. Os modos de se falar ao
povo na assembleia democrática seriam os sinais incontestes da
decadência política. Em \emph{Constituição de Atenas}, Aristóteles não
esconde sua aversão pelos modos e meios de Cléon, um líder popular:
``Cléon se apresenta com veste apertada ao corpo, como em roupa de
trabalho --- quando antes se falava com decoro, imóvel e com a túnica
descendo até o chão. Ele ergue a voz e ofende os
adversários.''\footnote{Aristóteles citado por Canfora, 2012, p.\,15.} No
final da República romana, os textos de Cícero sugerem uma divisão em
dois eixos distintos das qualidades oratórias: um primeiro no qual as
falas são suaves, leves, urbanas e elegantes, e um segundo em que elas
são ácidas, veementes, ásperas e grosseiras. Ao primeiro eixo pertencem
as falas dos patrícios dirigidas aos patrícios, ao passo em que no
segundo se inscrevem as de porta"-vozes populares endereçadas aos
plebeus.

Essa maneira de se falar ao povo da República romana era chamada de
\emph{eloquentia popularis}. A discriminação mais fundamental de que
eram objeto os oradores dessa eloquência era a falta de
\emph{urbanitas}. Essa noção significava o que era próprio de Roma, uma
vez que \emph{urbani} designava as pessoas da cidade eterna e
compreendia igualmente a língua e os gostos de Roma. Nesses dois últimos
campos, os porta"-vozes populares, em sua maioria não romanos, se
encontravam muito desprovidos: ``seu primeiro defeito era o de `mal'
pronunciar o latim. Havia na Cidade uma pronúncia especificamente
romana, feita de nuances insensíveis e inacessíveis aos que não haviam
vivido em Roma desde sua infância. Era o sotaque que definia o latim
legítimo''. O orador popular e estrangeiro era identificado, ao cometer
``erros'' linguísticos e prosódicos, ou seja, ao dar indícios de que não
conhecia os idiomatismos e as entonações romanas. Tratava"-se de uma
série de nuances que justificavam um distanciamento, que ``era ao mesmo
tempo insuportável e insuperável e que se tornava decisivo nos
obstáculos à promoção e à integração dos oradores da eloquência popular:
o som de suas vozes então operava essa distinção''.\footnote{David,
  Jean"-Michel. \emph{Eloquentia popularis} et \emph{urbanitas}. Les
  orateurs originaires des villes italiennes à Rome à la fin de la
  République''. In: \emph{Actes de la Recherche en Sciences Sociales},
  Paris: Seuil, 1985, vol. 60, n. 1, p.\,70 e 72.}

Guardadas as devidas diferenças e proporções, os oradores da
\emph{eloquentia popularis} eram então concebidos e depreciados pela
elite romana como já o haviam sido, séculos antes, ao longo da República
de Roma, os tribunos da plebe. Ante a revolta dos plebeus, que até então
estavam destituídos de direitos e que sofriam de opressões e explorações
diversas, o estabelecimento do tribunato do povo surgia ao mesmo tempo
como uma conquista popular e como uma concessão aristocrática. A partir
daí, os sem voz e voto teriam ao menos seus porta"-vozes no Senado. Uma
relativa conquista emancipatória e certa concessão apaziguadora que não
ficariam imunes às reações conservadoras. Caio Márcio, o Coriolano,
personifica emblematicamente essas reações. Antes de Cícero e de modo
bem menos discreto, Coriolano já havia depreciado o povo e seus
tribunos. Por meio da pena de Shakespeare, em sua tragédia sobre o
patrício e militar romano, Coriolano, seus amigos e seus partidários
desdobram uma enorme série de declarações elitistas.

Eis aqui somente algumas dessas afirmações: ``Arranque a língua do
populacho, para que ele não mais lamba o veneno destilado por seus
bajuladores''; ``Vejam o rebanho que vocês, tribunos, conduzem. Eles
merecem ter voz? Vocês que são suas bocas, por que não reprimem a fúria
de seus dentes?''; ``Veja aí essa ignóbil e fedorenta multidão'';
``Esses tribunos imbecis, que, ligados aos vis plebeus, detestam tua
glória, Coriolano''; ``Se o povo ama sem motivo, ele também odeia sem
fundamento''; ``Vejam os tribunos do povo, as línguas das bocas
vulgares. Eu os desprezo''. No momento em que, após muita insistência de
seus próximos para que ele vá se dirigir ao povo, no intuito de
convencer seus membros a elegê"-lo cônsul de Roma, Coriolano nos brinda
com mais esta lapidar afirmação eivada de preconceitos:

\begin{quote}
Pois bem! Que eu abandone meu coração e minhas inclinações naturais para
ceder ao espírito de uma cortesã. Que minha voz viril e guerreira, que
fazia coro com as cornetas das batalhas, se torne estridente como o
falsete de um eunuco ou como a voz de uma mocinha que embala um bebê em
seu berço. Que a língua suplicante de um mendigo se mova entre meus
lábios e que meus joelhos, cobertos de ferro e que jamais se dobraram
sobre meu estribo, se prostrem tão baixo quanto os dos miseráveis que
recebem esmolas.\footnote{Shakespeare, \emph{op.\,cit.} Ato \textsc{iii}, cena 1
  e 2.}
\end{quote}

Há uma longa história de discriminações da voz e da escuta do povo. Aos
preconceitos e exclusões das práticas populares de linguagem que se
praticavam na Antiguidade se somaram tantos outros na Idade Média, na
Era Moderna e na Contemporaneidade. Essa persistente história de
detratações da voz e da escuta populares será permeada por surgimentos
tardios, escassos e não poucas vezes ambivalentes de legitimidades
conquistadas por porta"-vozes do povo e pelos próprios sujeitos das
classes populares.\footnote{Piovezani, Carlos. \emph{A voz do povo}: uma
  longa história de discriminações. Petrópolis: Vozes, 2020.} A
despeito de sua condição tardia, escassa e ambígua, tais conquistas
conheceram certo recrudescimento a partir da emergência e consolidação
da noção de soberania popular, oriunda das Revoluções Francesa e
Americana do século \textsc{xviii}. No século seguinte, na esteira dessas
conquistas, assistiríamos ao surgimento e ao fortalecimento de uma
oratória popular e mesmo de uma eloquência proletária.

Desde então, apesar do caráter quase inabalável dos preconceitos contra
as camadas populares, falar mais ou menos como uma pessoa do povo parece
ter se tornado, em alguma medida, um direito, um trunfo e, por vezes,
até mesmo uma necessidade. De modo relativamente simultâneo, mas não em
igual proporção, alguns sujeitos de classes e grupos sociais
desfavorecidos conquistaram certa legitimidade de expressão para seus
próprios meios de intervenção pública; os oradores puderam contar com
uma nova possibilidade de eloquência e/ou de empregá"-la como um
estratagema retórico, se apresentando como oriundos do povo ou como seus
legítimos porta"-vozes, tendo em vista a contiguidade entre seus modos de
elocução; não apenas à fala pública dirigida aos setores desvalidos, mas
também àquela endereçada a auditórios mais gerais praticamente se impôs
a obrigação de incorporar padrões e índices populares. Uma conquista a
que se ascendeu, uma expressão que emergiu e uma tendência que se fixou,
assim parece ter surgido um novo quadro da oratória contemporânea. Os
oradores populares puderam se beneficiar em alguma medida dessa
conquista, mas os populistas também se aproveitaram dela intensa e
extensivamente.

O populismo nasce no tempo e no lugar das sociedades de massa. Trata"-se,
inicialmente, de uma noção burguesa e aristocrática, que decorre da
identificação dos poderes e perigos concentrados nas multidões. O final
do século \textsc{xix} e as primeiras décadas do século \textsc{xx} são marcados pelo
avanço de vanguardas culturais e movimentos políticos e sociais que
reclamam não somente uma democracia representativa, sob a forma do
sufrágio universal, compatível com o direito à propriedade, mas também e
principalmente reivindicam práticas e direitos igualitários, ações de
democracia direta e participação popular nas decisões políticas e até a
abolição do estado e da propriedade privada. É nesse contexto que surgem
a ideia de massa, seu mais frequente emprego no plural, as massas, e os
temores que elas passarão a despertar no pensamento conservador.

Era assim que um de seus principais teóricos, o psicólogo social francês
Gustave Le Bon, caracterizava aquele cenário: ``O período em que
entramos será realmente a era das massas''. Le Bon descreve e contribui
para aumentar o medo que pretensamente seria instaurado pela desordem,
pelo caos e pela anarquia. Entre os anos de 1880 e a Primeira Grande
Guerra, teriam se tornado visíveis os signos de uma inquietante potência
das multidões, que, por sua vez, era o anúncio da destruição da
civilização: ``A história ensina"-nos que, no momento em que as forças
morais que são o fundamento das sociedades perderam o seu domínio, as
multidões inconscientes e brutais, justamente qualificadas de bárbaras,
encarregam"-se de realizar a dissolução final''.\footnote{Le Bon, Gustave.
  \emph{Psicologia das multidões}. Lisboa: Roger Delraux, 1980, p.\,9.}
A inquietação com a qual Le Bon e alguns de seus contemporâneos se
preocupavam era o medo do desenvolvimento da democracia política. Essa
democracia manifesta"-se desde as duas últimas décadas do século \textsc{xix} sob
a forma do que se via como a violência política e social das greves e do
progresso dos movimentos trabalhistas e socialistas.

A perturbação social e política burguesa e aristocrática se aliava então
a vários conhecimentos científicos de seu tempo. Em Cesare Lombroso e
sua antropologia criminal, Le Bon vai encontrar as origens
criminológicas que as massas humanas conceberiam e nutririam. Já
Jean"-Martin Charcot, cujas sessões Le Bon tinha o hábito de frequentar,
lhe oferece a ideia de histeria patológica das multidões. Finalmente, de
Émile Durkheim, Le Bon retoma o propósito de descobrir as leis sociais
de uma unidade mental das massas.\footnote{Courtine, Jean"-Jacques. A voz
  do povo: a fala pública, a multidão e as emoções na aurora da era das
  massas. In: \emph{História da fala pública}: uma arqueologia dos
  poderes do discurso. Petrópolis: Vozes, 2015. p.\,265--266.} Essa
junção entre o medo e a ciência concorrerá para o fato de que as ideias
centrais sobre as multidões atravessem quase todo o século \textsc{xx}
praticamente intactas e forneçam durante muito tempo as explicações para
o comportamento dos indivíduos no interior das massas. Há uma concepção
naturalista dos traços atribuídos às multidões por Le Bon. Seus estados
psicológicos e suas características mentais, tais como a impulsividade,
a irritabilidade, a inconstância, os exageros e simplificações dos
sentimentos, a credulidade, ``são observáveis em seres pertencentes às
formas inferiores de evolução, como o selvagem e a criança''. A estes
dois seres inferiores, se acrescenta a histérica: ``As multidões são, em
todos os lugares do mundo, femininas; mas as mais femininas dentre todas
elas são as multidões latinas''.\footnote{Le Bon, p.\,21.}

Dada essa sua concepção das massas, a prática de falar em público para
Le Bon é antes de tudo composta pelo corpo e pela voz do orador e pela
escuta irracional, volúvel e instável dos ouvintes compactados na
multidão. As ideias e as palavras importariam bem pouco: ``no que
respeita ao verbo, basta que as palavras sejam marteladas sem cessar; e
no que concerne as ideias, basta fazê"-las penetrar com força na alma das
multidões''; basta fazer penetrar nessa alma de primitivos, de mulheres
e de crianças, da forma mais simples e mais imaginativa possível, as
``imagens impressionantes que preenchem e obcecam o espírito. Conhecer a
arte de impressionar a imaginação das massas é o mesmo que conhecer a
arte de governá"-las''.\footnote{Courtine, 2015, p.\,285--286.}

O que aprendemos sobre \emph{a linguagem fascista}, ao ler
\emph{A psicologia das multidões}? Não interessam a Le Bon a observação
e as recomendações sobre recursos argumentativos e figuras de linguagem,
sobre o léxico, a sintaxe ou a enunciação. Sua preocupação se concentra,
antes, na ``misteriosa potência'' das palavras e em sua ``verdadeira
magia'' capaz de converter as imagens que elas evocam em emoções
intensas e em ações irrefletidas:

\begin{quote}
A multidão somente pode ser impressionada pelos sentimentos excessivos.
O orador que pretende seduzi"-la deve abusar das afirmações violentas.
Exagerar, reafirmar, repetir e jamais tentar demonstrar nada mediante um
raciocínio são os procedimentos de argumentação familiares aos oradores
das massas populares.\footnote{Le Bon, p.\,27.}
\end{quote}

O livro de Le Bon não é evidentemente um tratado de retórica, um
compêndio de oratória ou manual de estilo, mas também não deveria ser
considerada uma obra científica de psicologia coletiva, tal como ela o
foi durante muito tempo. Ele deve ser, antes, concebido como um tratado
político, que se situa na longa tradição das artes de governar. Seu
texto materializa os temores suscitados pela efervescência do movimento
operário nas sociedades democráticas, se propõe como um meio de conjurar
seus perigos, repercute os apelos burgueses e aristocráticos pelos
líderes fortes e anuncia o terrível e promissor destino da propaganda
política populista. Por essa razão, \emph{A psicologia das multidões}
pode ser lida como o delineamento de um programa político autoritário e
reacionário, como um manual de técnicas de comando e de domesticação das
massas e ainda como uma contribuição ao pensamento que busca o controle
das populações. Noutros termos, trata"-se de uma obra a que o populismo
também nascente não poderia ficar indiferente.

Não poderia e não ficou. O populismo se aproveitou de Le Bon e dos
avanços das técnicas de propaganda. Seu objetivo era o de obter o
controle e a manipulação das aglomerações humanas no início da era das
massas. Assim, ele passou a fazer jus à condição de um dos inventores
das formas modernas de propaganda política.\footnote{Cf.: Tchakhotine,
  Serge. \emph{Le viol des foules par la propagande politique}, Paris:
  Gallimard, 1939.} Nesse sentido, sua relação com a
\emph{linguagem fascista} é fundamental: Hitler inspirou"-se
amplamente em \emph{Psicologia das multidões}, ao produzir seu próprio
\emph{Mein Kampf}; e Mussolini fez da obra de Le Bon seu livro de
cabeceira.\footnote{Ver, respectivamente: Goonen, Jay. \emph{The Roots of
  Nazi Psychology}, Lexington, The University Press of Kentucky, 2013,
  p.\,92; e Gentile, Emilio, aqui mesmo no capítulo ``Mussolini fala às
  massas''.} Como se já não bastasse, as massas tornadas objeto de
reflexão se inscrevem ainda na modernidade econômica do \emph{business}
norte"-americano, que nos anos de 1920 criam a promissora e bem"-sucedida
fórmula do \emph{mass medias}. Ao lado da publicidade dos produtos do
mercado e da propaganda política dos partidos socialistas e comunistas,
as lideranças populistas se autorizaram cada vez mais a falar em nome
das massas formadas principalmente pelas classes trabalhadoras.

Há, portanto, algo análogo entre a demagogia, o populismo e a linguagem
fascista.\footnote{Para mais informações sobre demagogia, populismo e
  fascismo, ver: Canfora, Luciano. Demagogia. Palermo, Selerio, 1994;
  Ferreira, Jorge (Org.) \emph{O populismo e sua história}: debate e
  crítica. Rio de Janeiro: Civilização Brasileira, 2001; Paris, Robert.
  \emph{As origens do fascismo}. São Paulo: Perspectiva, 1993. Sobre a
  detratação de políticas populares como populistas e sobre a
  legitimidade de populismo como a própria lógico do funcionamento
  político, ver: Rancière, Jacques. \emph{O desentendimento: política e
  filosofia}. São Paulo: Editora 34, 1996; Laclau, Ernesto. \emph{A
  razão populista}. São Paulo: Três Estrelas, 2013; e Souza, Jessé.
  \emph{A elite do atraso.} Da escravidão à lava"-jato. Rio de Janeiro,
  Leya, 2017.} Demagogos, populistas e fascistas dirigem"-se ao povo e
supostamente o fazem em nome de suas causas. Falam ao povo e o fazem
mimetizando seus meios de expressão. Fazem"-no com o propósito de se
tornarem cada vez mais capazes de incutir suas crenças e mobilizar as
ações das massas populares. Enquanto Mussolini, além de populista, foi
precursor do fascismo e um de seus tipos mais bem acabados, Bolsonaro é
um populista e um ``fascista \emph{wannabe}'', uma vez que consiste no
líder populista que mais quer e que mais se aproxima do fascismo na
história, ao reativar em seu populismo traços fascistas indeléveis: a
violência anunciada como fator de regeneração social, a segregação de
grupos fragilizados, a mobilização exponencial das mentiras e o flerte
com a ditatura.\footnote{Finchelstein, Federico. Bolsonaro é o líder
  populista que mais se aproximou do fascismo na história. Entrevista
  concedida ao \emph{The Intercept}, 07 de julho de 2020.}
Com tamanhas semelhanças, não poderíamos prescindir de compreender suas
eventuais interconexões, considerando todos os perigos que eles envolvem
isolada e, sobretudo, conjuntamente. Mas é ainda mais urgente e
necessário diferenciá"-los e concebê"-los como os fenômenos históricos que
efetivamente eles são.

Em 2019, o fascismo completara 100 anos de seu nascimento. Por essa
ocasião, a \emph{\textsc{bbc}} de Londres entrevistou o historiador italiano
Emilio Gentile, autor do capítulo ``Mussolini fala às
massas'' deste livro. Naquela circunstância, o jornalista Angelo
Attanasio que o entrevistou assim apresentou seu entrevistado: ``O
historiador italiano Emilio Gentile é um dos maiores especialistas em
fascismo do mundo''. Com o lastro de uma vida acadêmica inteira dedicada
ao estudo do fascismo, Gentile afirma que se deve ``distinguir entre o
fascismo histórico, que é o regime que, a partir da Itália, marcou a
história do século 20 e se estendeu à Alemanha e a outros países
europeus no período entre as duas guerras mundiais, e o que é
frequentemente chamado de fascismo depois de 1945, que se refere a todos
aqueles que usam da violência em movimentos de extrema direita''.
Segundo o historiador italiano, a extrema"-direita se caracteriza por se
opor ``aos princípios da Revolução Francesa de igualdade e liberdade,
mas sem necessariamente ter uma organização totalitária ou uma ambição
de expansão imperialista''. São esses últimos aspectos que tornam um
movimento ou um regime fascista: ``Sem o regime totalitário, sem a
submissão da sociedade a um sistema hierárquico militarizado, não é
possível falar de fascismo''.\footnote{Gentile, Emilio. 100 anos do
  fascismo, \textit{\textsc{bbc} News}, 24 de março de 2019.}
Com base nessas afirmações de Gentile, Attanasio lhe dirige estas
questões, que são assim respondidas:


\begin{quote}\parindent=0em
\textbf{\textsc{bbc}: Então, quando se pode falar de ``fascismo''?}

\textsc{gentile:} Podemos falar de fascismo ao nos referir ao fascismo
histórico, quando um movimento de massas organizado militarmente tomou o
poder e transformou o regime parlamentar em um Estado totalitário, ou
seja, em um Estado com um partido único que procurou transformar,
regenerar ou até criar uma nova raça em nome de seus objetivos
imperialistas e de conquista.

\medskip

\noindent\textbf{Isto é, somente quando nos referimos a esta experiência
específica?}

Sim, para o período histórico entre as duas guerras
mundiais, quando ainda havia a vontade de conquistar e se expandir
imperialmente por meio da guerra. Se estas características ainda
estivessem presentes hoje, poderíamos falar em fascismo. Mas me parece
completamente impossível. Mesmo aqueles países que aspiram a ter um
papel hegemônico procuram fazer isso por meio da economia, e não da
conquista armada.

\medskip

\noindent\textbf{O senhor acha que existe o perigo de um retorno do fascismo?}

Não, absolutamente, porque na história nada volta,
nem de um jeito diferente. O que existe hoje é o perigo de uma
democracia, em nome da soberania popular, assumir características
racistas, antissemitas e xenófobas. Esses movimentos se definem como uma
expressão da vontade popular, mas negam que este direito possa ser
estendido a todos os cidadãos, sem discriminações entre os que pertencem
à comunidade nacional e aqueles que não lhe pertencem.

\medskip

\noindent\textbf{Donald Trump, Vladimir Putin, Jair Bolsonaro, Viktor
Orbán e outros líderes políticos foram chamados de fascistas por suas
políticas de imigração ou seu nacionalismo. é correto defini"-los assim?}

Se afirmamos isso, poderíamos dizer então que todos
o são, porque são homens e brancos. Mas, ao mesmo tempo, não
entenderíamos a novidade destes fenômenos. Não se deveria aplicar o
termo ``fascista'' para todos os contextos, mas entender quais são as
causas que geraram e fizeram proliferar estes fenômenos. Em todos esses
países, esses movimentos extremistas se afirmaram com base no voto
popular.

\medskip

\noindent\textbf{O senhor acha então que a palavra ``fascismo'' está sendo
abusada para definir estes governos?}

Na minha opinião, é um grande erro, porque não nos
permite compreender a verdadeira novidade destes fenômenos e o perigo
que eles representam. E o perigo é que a democracia possa se tornar uma
forma de repressão com o consentimento popular. A democracia em si não é
necessariamente boa. Só é boa se realiza seu ideal democrático, isto é,
a criação de uma sociedade onde não há discriminação e na qual todos
podem desenvolver sua personalidade livremente, algo que o fascismo nega
completamente. Então, o problema hoje não é o retorno do fascismo, mas
quais são os perigos que a democracia pode gerar por si mesma, quando a
maioria da população --- ao menos, a maioria dos que votam --- elege
democraticamente líderes nacionalistas, racistas ou antissemitas.
\end{quote}

Sem dúvida, a história não se repete. Se ``todos os grandes fatos e
personagens da história universal aparecem como que duas vezes'', ao que
se acrescenta que ocorrem ``uma vez como tragédia e a outra como
farsa'',\footnote{Marx, Karl. \emph{O 18 Brumário de Louis Bonaparte}.
  Lisboa: Edições Avante, 1982, p.\,21.} essa dupla aparição e seu duplo
aspecto, o do drama da primeira e o da impostura da segunda, já seriam
suficientes para rejeitar uma pura e simples repetição. Subscrever essa
ideia não corresponde ao encerramento de um debate nem à impossibilidade
de lhe produzir uma inflexão. O debate permanece aberto, está
efervescente no Brasil de nossos dias e envolve não apenas antagonismos
frontais e manifestos, mas também posições ao mesmo tempo relativamente
divergentes e bastante próximas.\footnote{No mesmo mês de fevereiro deste
  ano de 2020, a revista \emph{Carta Capital} publicou dois textos nos
  quais se materializam estas duas posições distintas. Em ``Não há
  `fascismo' no Brasil, mas `malignidade', diz sociólogo'', Antonio
  Cattani, professor da \textsc{ufrgs}, ressalta as especificidades do que
  ocorreu na Itália nas primeiras décadas do século \textsc{xx}, enquanto em ``Os
  maus modos do neofascismo brasileiro'', Tales Ab'Sáber, professor da
  \textsc{unifesp}, sustenta haver traços suficientes para chamarmos o que se
  reúne em torno do bolsonarismo de um neofascismo.
  Matérias publicadas em 4/02/2020 e 21/02/2020, disponíveis no site da \textit{Carta Capital}.}
Estas últimas parecem se distinguir pela disposição distinta da lógica
concessiva: ``Ainda que haja `malignidade' no bolsonarismo, não se trata
de fascismo, porque o fascismo foi um fenômeno histórico preciso''
\emph{versus} ``Ainda que não se trate da repetição de um fenômeno
histórico preciso, o bolsonarismo consiste em um neofascismo
brasileiro''.

Há um nosso fascismo nacional comum.\footnote{Dentro e fora do contexto
  brasileiro, filósofos, sociólogos e historiadores já postularam a
  existência desse fascismo impregnado no cotidiano: ``Enfim, o inimigo
  maior, o adversário estratégico do Anti"-Édipo: o fascismo. E não
  somente o fascismo histórico de Hitler e de Mussolini --- que tão bem
  souberam mobilizar e utilizar o desejo das massas ---, mas o fascismo
  que está em nós todos, que martela nossos espíritos e nossas condutas
  cotidianas, o fascismo que nos faz amar o poder, desejar esta coisa
  que nos domina e nos explora.'' (Foucault, Michel. Introdução à vida
  não"-fascista. In: Gilles Deleuze e Félix Guattari.
  \emph{Anti"-Oedipus}: Capitalism and Schizophrenia, New York, Viking
  Press, 1977, p.\,\textsc{xii}). Já sobre o cenário brasileiro, ver: Sodré,
  Nelson Werneck. \emph{O fascismo cotidiano}. Belo Horizonte: Oficina
  de Livros, 1990.} De algum modo, ele sempre esteve entre nós, sob a
forma de opressões diárias sofridas por pobres e marginalizados, por
negros e indígenas, por mulheres e \textsc{lgbtq}s, decorrentes do sistema
escravocrata e de enormes injustiças e desigualdades sociais, de
xenofobias e aporofobias, de patriarcalismos, misoginias e homofobias de
toda sorte, gestados e reproduzidos por nossa história. Ele sempre
esteve entre nós, sob a forma dos assassinatos de membros e de
lideranças dos movimentos negros e indígenas e dos movimentos
militantes, estudantis, ecologistas e sindicais, sob a forma dos
assassinatos de meninos e meninas das favelas e dos rincões brasileiros,
principalmente durante a ditadura, mas também em períodos chamados de
democráticos. Mais recentemente, a ele se acrescentou uma escalada de
práticas e discursos de ódio e, finalmente, uma ascensão ao poder
político da presidência da República. Esta última é uma das maiores
catástrofes de nossa história recente: ``a inacreditável ascensão
política do fascista tupiniquim, muito ignorante e violento, Jair
Bolsonaro, com seus apoiadores, verdadeiros fanáticos da burrice
histórica e da estupidez anti"-humanista''.\footnote{Ab'Sáber, Tales.
  \emph{Michel Temer e o fascismo comum}. São Paulo: Hedra, 2018, p.\,28.}
A história não se repete, mas as comparações entre fenômenos históricos
e agentes políticos costumam esclarecer zonas mais ou menos obscuras dos
fatos sociais ou concorrem para mais bem compreendê"-los. Em contextos
dramáticos, elas se tornam ainda mais necessárias e urgentes.

Ora, não são poucas as correspondências entre o que ocorre hoje conosco
e as propriedades do fascismo que indicamos acima na esteira
principalmente de Klemperer e de Eco. Se há algo de pertinente e
relevante em apontarmo"-las aqui, com mais forte razão, deve haver em
cotejá"-las com as especificidades do contexto brasileiro contemporâneo.
Assim como no fascismo europeu da primeira metade do século \textsc{xx}, também
em nosso fascismo comum, são fundamentais as relações entre o pensamento
e as experiências de ódio e entre a linguagem e as ações violentas. Em
um passado muito recente, seus adeptos expressaram seu pensamento ``como
\emph{passagem ao ato}, de bater panelas para calar o adversário na
linguagem --- em uma metáfora muito concreta, já no limite da ação
física, do desejo evidente de bater, usar a força e calar''.\footnote{Ab'Sáber,
  2018, p.\,54.} Tratava"-se já de uma verdadeira política do direito ao
ódio e do ódio como exercício legítimo da política. Nesse processo, a
\emph{linguagem fascista} desempenha um papel decisivo:

\begin{quote}
O passo final das clivagens fascistas, das suas certezas que legitimam a
violência e o extermínio, a tortura e o escárnio dos adversários
políticos e seus gozos de massa, de sua falsa identidade de uma
superioridade qualquer, e de sua vida prática que busca a ação e que
recusa fortemente qualquer conhecimento mediado, criativo ou crítico de
algum modo, é uma ampla curvatura descendente no plano da linguagem, o
carregamento excitado das palavras que tende ao concreto de seu valor, o
desprezo aberto por outras palavras que devem ser recusadas, negadas, o
deslocamento do plano do léxico e da semântica para outro centro
gravitacional cuja natureza política é interessada e imensamente triste.
(\ldots{}) Para além da violência direta, o fascista extirpa, como um
cirurgião carniceiro do simbólico, mundos e mais mundos de
possibilidades de sentido e de experiência, que desfalecem em conjunto
com a morte programada do outro na cultura. A cultura programática da
morte e do extermínio, é cultura da morte de palavras, e com elas, de
sentidos.\footnote{Ab'Sáber, 2018, p.\,155--161.}
\end{quote}

O fascismo e sua linguagem podem ser fascinantes. Eis aí um de seus
maiores riscos. As forças negativas, os discursos e ações do ódio e as
políticas do medo e da morte podem subsumir no encanto e na beleza
promovidos pela performance da força e da saúde, pela pertença a uma
comunidade pura e harmônica e pela franqueza e autenticidade da
expressão de seu líder e de seus membros. Em condições históricas e
sociais distintas, as pulsões e os afetos, os pensamentos e as ações
fascistas não são os mesmos, nem são idênticos os recursos e os
fascínios de sua linguagem. Quais foram tais recursos e como puderam
produzir crenças e seduções, adesões e repetições na Itália de Mussolini
e no Brasil de Bolsonaro? Responder a estas e a outras questões
relativas à linguagem fascista é o objetivo que buscamos aqui alcançar,
mediante a leitura independente ou conjunta de ``Mussolini
fala às massas: do socialismo revolucionário ao regime fascista'', de
Emilio Gentile, e de ``Bolsonaro fala às massas: do baixo
clero político à presidência da República'', de nossa própria autoria.

\asterisc

Em sua defesa de Helena, Górgias alega sua inocência. Abandonara Menelau
e seguira com Páris não por imprudência ou por seu próprio desejo, mas
``ou bem em função das intenções do acaso, das vontades dos deuses e dos
decretos da necessidade, ou bem por ter sido raptada com violência, ou
bem por ter sido persuadida pelos discursos''. Mais adiante, outras
passagens de seu texto permitem depreender uma hierarquia de forças
entre os fatores que teriam motivado a ida de Helena: ``se aquele que a
persuadiu, que construiu uma ilusão em sua alma, foi o discurso, também
não será difícil defendê"-la contra esta acusação, e destruir a
inculpação da seguinte forma: o discurso é um grande soberano que, por
meio do menor e do mais inaparente dos corpos, realiza os atos mais
divinos''.\footnote{Górgias. Elogio de Helena. In: Cassin, Barbara.
  \emph{O efeito sofístico}. São Paulo: Editora 34, 2005, p.\,295--297.}


\looseness=-1
Mais do que o acaso, os deuses ou as necessidades, são os discursos que
mais bem seduzem e movem os homens e as mulheres. É o domínio do
discurso que ``cria uma necessidade na alma que ele persuade de ser, a
uma só vez, persuadida pelas coisas ditas e condescendente face às
coisas que são feitas''. Desde os antigos, há entre os amantes da
linguagem uma consciência de seus poderes e perigos, de sua condição de
veneno e de remédio: ``assim como tal droga faz sair do corpo um tal
humor, e que umas fazem cessar a doença, outras a vida, assim também,
dentre os discursos, alguns afligem, outros encantam, fazem medo,
inflamam os ouvintes, e alguns, por efeito de uma má persuasão, drogam a
alma e a enfeitiçam''.\footnote{Górgias, 2005, p.\,299--300.}

Entre seus amantes modernos, essa consciência não se perdeu. Nós podemos
encontrá"-la nas reflexões de Roland Barthes sobre as constitutivas
relações entre o poder e a linguagem. A respeito do primeiro, ele afirma
o seguinte: ``o poder está presente nos mais finos mecanismos do
intercâmbio social; não somente no Estado, nas classes, nos grupos, mas
ainda nas modas, nas opiniões correntes, nos espetáculos, nos jogos, nos
esportes, nas informações, nas relações familiares e privadas, e até
mesmo nos impulsos liberadores que tentam contestá"-lo''. Já no que se
refere à segunda, Barthes diz que o poder habita seu interior e a
explora: ``o poder é o parasita de um organismo trans"-social, ligado à
história inteira do homem. Esse objeto em que se inscreve o poder, desde
toda a eternidade humana, é: a linguagem --- ou, para ser mais preciso,
sua expressão obrigatória: a língua''.\footnote{Barthes, Roland.
  \emph{Aula}. São Paulo: Cultrix, 2000, p.\,11--12.}

\looseness=-1
Uma célebre passagem desse mesmo texto de Barthes vem então logo em
seguida: ``a língua, como desempenho de toda linguagem, não é nem
reacionária, nem progressista; ela é simplesmente: fascista; pois o
fascismo não é impedir de dizer, é obrigar a dizer''. O uso da língua
aprisiona, porque nela ``servidão e poder se confundem
inelutavelmente''. Se a liberdade é a ``potência de subtrair"-se ao
poder'' e a de ``não submeter ninguém'', então somente poderia haver a
experiência absoluta da liberdade fora da linguagem. Isso nos seria
impossível, pois não há vida humana fora da linguagem. Barthes indica
uma única saída: ``só resta, por assim, dizer, trapacear com a língua,
trapacear a língua''.\footnote{Barthes, 2000, p.\,14--16.} É ``no interior
da própria língua'' que o fascismo da língua pode e deve ser combatido.

Ao fascismo da língua e a todo e qualquer fascismo, se devem contrapor
as trapaças da língua, as participações democráticas e as liberdades dos
desejos. ``É bom que os homens, no interior de um mesmo idioma, tenham
várias línguas''. Muitas línguas, diversas vozes e múltiplas funções, e
todas elas ``promovidas à igualdade''. Elas formariam, assim, uma fonte
da qual todos os humanos poderiam beber livremente e conforme a verdade
de seus desejos. ``Essa liberdade é um luxo que toda sociedade deveria
proporcionar a seus cidadãos: tantas linguagens quantos desejos houver.
Que uma língua, qualquer que seja, não reprima outra: que o sujeito
conheça, sem remorso, sem recalque, o gozo de ter a sua disposição duas
instâncias de linguagem, que ele fale isto ou aquilo segundo as
perversões, não segundo a Lei''.\footnote{Barthes, 2000, p.\,24--45.} Sem
essa utopia, sem essa resistência forjada com a linguagem e na própria
linguagem, o fascismo grassa como desejo que deseja sua própria
repressão, seja no Estado totalitário, seja no mercado neoliberal, seja
ainda em interconexões de suas facetas.

Se no fascismo cotidiano, os poderes se imiscuem na linguagem para
tentar nos impor o que deveríamos pensar, fazer e dizer, com mais forte
razão, se estabelecem relações constitutivas entre a língua e as
opressões, entre os discursos do ódio e as ações de extermínio, quando
os fascistas ou neofascistas assumem os lugares de poder no Estado. Dois
ilustrativos episódios ocorridos ainda no começo dos estados
totalitários da Itália e da Alemanha comprovam"-no. Na noite do dia 22 de
junho de 1925, Mussolini pronunciou um discurso no mausoléu de Augusto.
Esse pronunciamento registra pela primeira vez o uso do termo
\emph{totalitário} entre os fascistas. Mas, essa não é a única razão de
ele ter se tornado conhecido. Por um lado, sua fala ``situa"-se numa
longa cadeia de discursos, que narram as ações, sem cessar de
produzi"-las''; e, por outro, manifesta pela boca de seu líder supremo um
dos principais traços do fascismo: ``negar aos outros partidos a
legitimidade, o direito de existirem ou de tornarem"-se fatores positivos
de governo''.\footnote{Faye, Jean"-Pierre. \emph{Introdução às linguagens
  totalitárias}. São Paulo: Perspectiva, 2009, p.\,57.} As ações
violentas e os assassinatos já haviam começado, mas esse discurso lhes
dava uma justificativa e um acobertamento de Estado.

Usar sem reserva um último e radical ato de linguagem: o comando à
execução. Era isso que passara a circular nos discursos de Hermann
Göring desde o início de fevereiro de 1933. Ele era ministro do Interior
da Prússia e fora figura fundamental para que Hitler alcançasse o posto
de chanceler da Alemanha. A partir do dia 07 daqueles mês e ano, Göring
``dirigia"-se verbalmente à polícia da Prússia para anunciar que
acobertaria qualquer um que fosse levado a `puxar sua arma' no combate
`contra a ralé e a canalha internacional' ou, numa linguagem mais clara,
contra o que se denominava então os partidos social"-democrata e
comunista alemães''. Dez dias mais tarde, Göring lança um decreto no
qual precisa que ``a polícia deve evitar qualquer processo contra as
`associações nacionais', \emph{\textsc{sa}}, \emph{\textsc{ss}} e \emph{Capacete de Aço},
mas deve ao contrário, se necessário, `fazer uso de suas armas sem
hesitação'\,''.\footnote{Faye, 2009, p.\,148--149.}

A \emph{linguagem fascista} usa a linguagem humana para calar a
linguagem humana. Fala da pureza da raça ou das pessoas de bem para
calar a crítica e as diferenças. Fala às massas populares, que, de fato,
menospreza, para tentar conduzir e calar o povo e seus porta"-vozes. Fala
para justificar e fomentar o aniquilamento do adversário, transformado
em inimigo. Os perigos do fascismo estão além da linguagem. Agressões e
extermínios ultrapassam as ações linguísticas. Mas as versões fascistas
da história promovem um aumento progressivo na aceitação de discursos de
ódio e de atos violentos, tanto pelo que contam quanto por suas maneiras
de contar suas narrativas. Na constituição de nossa história, há ``um
efeito de produção de ação pelo relato''. As ideologias políticas
fascistas, mais do que outras, colocam em sua origem uma narração, em
nome da qual se insurge contra verdades factuais: ``é uma narração que
não é verdadeira, mas se faz terrivelmente ativa''.\footnote{Faye, 2009,
  p.\,17, 18 e 121.}

Encerremos este já longo início de uma história da
\emph{linguagem fascista} que pretendemos contar aqui com a
dupla lição que nos fora ensinada por antigos e modernos amantes da
linguagem. Não se deve subestimar o poder destrutivo da palavra, mas
tampouco se deve desdenhar e abrir mão de sua potência crítica e
criadora. Uma função fundamental da linguagem é a narração. Carregada
pela base material e psíquica das sociedades, a narração ``não apenas
toca a história, mas efetivamente a engendra''. A história é antes de
mais nada uma narração. ``Porque ela só se faz contando"-se, uma crítica
da história só pode ser exercida contando como a história se produz ao %116 e??
narrar"-se''.\footnote{Faye, 2009, p.\,\textsc{xxiv} e 116.} Passemos, então, sem
mais demora, ao exame e à crítica dos discursos de Mussolini e de
Bolsonaro, com este precioso ensinamento de Jean"-Pierre Faye:
compreender a maneira pela qual a opressão se tornou aceitável é o
primeiro gesto de nossa própria libertação.

\bigskip
\hfill\textit{Carlos Piovezani}

%\hfill{}São Carlos, agosto de 2020

\chapter[Mussolini fala às massas, \emph{por Emilio Gentile}]{Mussolini fala às massas \subtitulo{Do socialismo revolucionário ao regime fascista}}

\section{O nascimento de um orador}

\noindent{}No dia 4 de abril de 1900, um aluno do colegial, que estudava na Régia
Escola Normal para Meninos, dissertava sobre o tema: ``O ensino de
história na escola elementar''. O estudante iniciava seu texto,
observando que a educação ``deu gigantescos passos em direção à
modernidade, ao se desvencilhar do sistema jesuítico, baseado na
memorização, de modo que atualmente o aprendizado não consista mais em
uma questão de quantidade, mas de qualidade, embora isso não se tenha
efetivamente refletido no trabalho do governo''. Contudo, no ensino da
história, esse sistema jesuítico ainda permanecia atuante na prática do
``ditado histórico'', o que implicava uma ``lamentável sobrecarga da
memória''. A história, sustentava o estudante, ``não deve ser um objeto
de estudo a ser ensinado como outros de natureza distinta, porque seu
método de aprendizagem, que, à primeira vista, parece fácil, não o é
efetivamente''. Assim, para fazer a devida opção por um apropriado
método de ensino de história, seria preciso responder às seguintes
questões: ``De qual história necessita a maioria da população? Como lhe
dar um fundamento na cultura escolar? O que fazer para que ela seja
assim compreendida pelos estudantes?''.

O manual de história era uma ferramenta necessária, mas era
indispensável que o professor expusesse claramente o tema histórico
tratado, porque o livro ``dirige"-se à inteligência, ao passo que o
professor deve se endereçar ao sentimento dos estudantes, no intuito de
estimulá"-lo e transportá"-lo aos patamares mais puros e elevados. Assim,
o texto do livro, por meio da inteligência, e o professor, por meio do
sentimento, atingem o mesmo objetivo: a educação do coração''. O ``bom
livro de história'' deveria conter ``textos ao mesmo tempo instrutivos e
saborosos, para que não fatiguem nem entediem o estudante''. Ele não
deveria ``mencionar os eventos de menor importância, esses eventos que
não produzem modificações na atividade e na consciência humana''. Não
deveria se limitar a muitas datas e nomes, coisas que são rapidamente
esquecidas, mas deveria buscar compreender esquemas históricos e
geográficos que mostrassem claramente aos estudantes como se
desenrolaram os fatos. Os educadores que ``realmente sentem a
importância de sua missão devem utilizar a história como um talismã
didático'', sem jamais esquecer que ``o escopo mais importante é o de
estimular e de enobrecer o sentimento'':

\begin{quote}
O professor não deve seguir estritamente o texto, porque suas próprias
lições devem sempre se manifestar na exposição de seu conteúdo. O timbre
de sua voz deve ser belo e sedutor, seus movimentos, agradáveis, e seus
gestos, apropriados. Ele tem de conseguir renovar a cena histórica
diante dos olhos dos alunos, de forma que, se alguma passagem lhes
escapar, as perdas serão mínimas, porque o mestre os terá atraído e eles
estarão admirados e mesmo fascinados. Uma comemoração, uma efeméride
poderá fornecer o argumento para que o professor ministre sua aula de
história, de modo tanto mais produtivo quanto mais atual ela for e
quanto mais ela for ouvida com o amor que o mestre depositou em sua
fala. Em suma, o professor deve manter uma devida distância do texto que
expõe, para que os alunos não apenas digam ``O professor é bom'', mas
para que, antes, digam ``O professor é excelente''.

Somente trabalhando desse modo, tal como expus aqui, é que a história
será eficaz na educação do povo. Somente dessa maneira é que ela criará
o sentido de uma observação humana; somente assim ela produzirá o
raciocínio qualitativo, que, ao indicar os vícios e as virtudes dos
homens, os encoraja a sempre praticar ações salutares ao seu
caráter.\footnote{Mussolini, Benito. \emph{Opera Omnia}, vol. \textsc{i},
  organizado por Edoardo e Duilio Susmel, Florença: La Fenice,
  1951--1963, p.\,226--227.}
\end{quote}

O estudante chamava"-se Benito Mussolini e ele ainda nem sequer havia
completado dezessete anos. Seu pai era um metalúrgico, expoente do
Partido Socialista na região de Emilia"-Romagna, ateu e anticlerical. Sua
mãe era professora primária e católica devota. A família Mussolini vivia
modestamente em um conjunto habitacional da cidade de Predappio, na
província de Forlì. Aos noves anos de idade, Benito foi posto em um
colégio salesiano na cidade de Faenza, mas permaneceu por lá somente
dois anos, porque rapidamente manifestou uma personalidade rebelde e
violenta. Um relatório do instituto salesiano o descreve com os
seguintes termos:

\begin{quote}
Ele só tem nove anos, mas já demonstra, como muitos rapazes, um caráter
rebelde, distante e pouco inclinado à disciplina. O primeiro ano se
passou sem incidentes e ele recebeu uma generosa ajuda, na esperança de
que conseguíssemos melhorá"-lo um pouco. (\ldots{}) Ele sempre quer ser o
primeiro entre os primeiros. Nos exames escritos, superou todos seus
colegas. Com uma leitura, ele consegue memorizar qualquer lição. Certa
vez, ao ser repreendido porque parecia ter saltado uma parte de uma
poesia, ele prontamente respondeu: ``Eu memorizei exatamente o que eu
li''.

De índole apaixonada e indisciplinada, não soube se adaptar à vida no
colégio, onde ele foi frequentemente punido, e desde cedo buscou mostrar
sua necessidade de estar no mundo exterior para viver, sentir e conhecer
a vida. Isso contrasta com cada uma das regras de ordem e disciplina do
Instituto.

Além disso, nada o agrada: em meio a tanta gente, se sente mais triste e
mais sozinho. Ele quer sempre estar só. Os jogos não o atraem. Parece
que a formação de sua personalidade já se iniciou.

Uma motivação pessoal determina, e nisso consiste um primeiro
desdobramento da alma, sua busca por vingança, caso ele seja ofuscado
por um colega mais velho, o que ele não pode suportar e o que o impele à
vingança.

Uma aguda consciência de si, que por vezes é hereditária, não lhe é
menor.

Ele se rebela contra qualquer punição e correção, de sorte que o diretor
Don Giovanni Battista Rinaldi foi forçado, muito a contragosto, em
agosto de 1894, a pedir a seus pais que, tão logo fossem encerrados os
exames finais, o retirassem do Colégio, em razão de seu temperamento,
que não se coaduna absolutamente com o regime da instituição, com um
sistema de educação ao qual se deve submeter seriamente aquele que
estiver em um colégio Salesiano.\footnote{Mussolini, \emph{Opera Omnia},
  vol. \textsc{i}, p.\,242.243.}
\end{quote}

Benito continuou seus estudos em um colégio laico na cidade de
Forlimpopoli. Em 28 de janeiro de 1901, o diretor do colégio Vilfredo
Carducci o encarregou com a organização das homenagens a serem feitas no
teatro da cidade a Giuseppe Verdi, morto no dia anterior, em Milão.
Todavia, conforme relatou um colega de classe, que estava presente
naquela ocasião, o jovem orador de 18 anos se entusiasmou e em seu
discurso pouco tratou do grande artista homenageado, falando, antes e
muito, de sua paixão patriótica pelo \emph{Risorgimento},\footnote{Nota
  do tradutor: o \emph{Risorgimento} consiste no processo da história
  italiana durante o qual houve a unificação nacional da Itália. Esse
  processo de unificação foi concluído com a proclamação do reino da
  Itália, em março de 1861.} de suas decepções com o que ocorreu na
Itália, após sua unificação, e das desigualdades que distanciavam a
classe dirigente do proletariado. O diretor do colégio, inquieto com a
visada política que então tomara a homenagem, se aproximou do jovem, no
intuito de contê"-lo ``com um gesto afetuoso. Mas, já era tarde demais.
Todos sabiam que, apesar de tudo, o diretor gostava muito daquele
estudante. Além disso, àquela altura, os aplausos já haviam tomado conta
de toda a cena''.\footnote{Pini, G.; Susmel, D. \emph{Mussolini. L'uomo e
  l'opera}. I. \emph{Dal socialismo al fascismo (1883--1919)}, Florença:
  La Fenice, 1953, p.\,56.}

No dia 30 de janeiro, o jornal de Bolonha \emph{Il Resto del Carlino},
em uma crônica sobre as homenagens prestadas a Verdi, por ocasião de sua
morte, informava que ``o estudante Benito Mussolino (sic) pronunciou um
discurso muitíssimo aplaudido de comemoração dedicado ao grande mestre
Verdi''. Dois dias depois, em uma publicação da sucursal de Forlimpopoli
do jornal do partido socialista \emph{Avanti} afirmava que ``no teatro
municipal o companheiro e estudante Mussolini prestou homenagens a
Giuseppe Verdi, pronunciando um discurso bastante aclamado''.\footnote{Mussolini,
  \emph{Opera Omnia}, vol. \textsc{i}, p.\,244--245.}

Três anos mais tarde, um pronunciamento de Mussolini era novamente
citado na imprensa. No dia 23 de março de 1904, o jornal socialista
suíço \emph{Le Peuple de Genève} relatava que no dia 18 daquele mês, na
cervejaria \emph{Handwerk}, o aniversário da Comuna de Paris havia sido
comemorado pelos socialistas suíços e por socialistas de outros países
europeus, que então viviam na Suíça: ``O discurso para os italianos foi
pronunciado pelo companheiro Mussolini, que com grande eloquência
investiu contra os detratores da Comuna e traçou o caminho pelo qual
deve seguir a classe operária para garantir as liberdades necessárias à
sua completa emancipação''.\footnote{Trecho citado por Gagnebin, B.
  ``Mussolini a"-t"-il rencontré Lenine a Genève en 1904?''. In: Monnier,
  L. (Org.). \emph{Genève et l'Italie}, Paris/Genebra: Bibliothèque
  Romande, 1969, p.\,290.} Mussolini estava já há dois anos na Suíça,
para onde havia emigrado em busca de trabalho e aventura e onde se
tornou rapidamente jornalista, propagandista e agitador socialista junto
aos trabalhores italianos igualmente imigrados. Naquela ocasião, ele
ainda redigiu uma breve crônica a respeito da comemoração genebrina,
publicada no dia 27 de março de 1904, no periódico \emph{Avanguardia
Socialista}:

\begin{quote}
No último dia 18 de março, os grupos socialistas de Genebra comemoraram
o aniversário da Comuna de Paris. Na cervejaria \emph{Handwerk}, havia a
habitual multidão cosmopolita. Wyss fez um discurso em alemão; Tomet, em
francês; e em italiano falou este seu correspondente. Os vários grupos
cantaram hinos revolucionários. Houve belas e luminosas projeções que
ilustravam os principais episódios da Comuna. Nos confraternizamos com
companheiros russos, que responderam às nossas intervenções e aos nossos
hinos com gritos de ``Viva o Proletariado italiano. Viva o
Socialismo!''\footnote{Mussolini, \emph{Opera omnia,} vol. \textsc{i}, p.\,79--80.}
\end{quote}

Naquela noite de comemoração na cervejaria, também se fez um discurso em
russo aos russos que ali estavam presentes. É quase certo que foi Lênin
quem o pronunciou.\footnote{Gentile, E. \emph{Mussolini contro Lenin},
  Roma"-Bari: Laterza 2017, cap.\,\textsc{i}.}

\section{Constante na metamorfose}

A dissertação sobre o ensino da história redigida pelo jovem Mussolini e
os relatos e registros da homenagem a Verdi e da comemoração do
aniversário da Comuna de Paris são os primeiros documentos sobre a
formação de Mussolini que já contêm embrionariamente algumas das
características essenciais de seu estilo oratório, tal como elas se
desenvolveriam nas quatro sucessivas décadas de sua militância política,
vivida em fases diversas e bastante contrastantes umas com outras: a
passagem do socialismo internacionalista e antimilitarista ao
intervencionismo nacionalista (1900--1914); o período da Primeira Guerra
Mundial até o fascismo libertário, passando pelo antistalinismo e pelo
fascismo individualista (1915--1920); em seguida, a época de sua opção
pelo fascismo monárquico, estatutário e totalitário (1921--1943); e,
finalmente, a fase de seu fascismo republicano socialista, durante os
dois últimos anos de sua vida (1943--1945).

Recusando decisivamente o método anti"-histórico, que foi frequentemente
empregado por diversos especialistas na vida e no pensamento de
Mussolini, mediante o qual se encontram traços do \emph{Duce}
totalitário já em sua militância marxista, apresentaremos alguns
exemplos do desempenho oratório mussoliniano, passando por diferentes
fases de sua fala política, desde sua militância socialista, em sua
juventude, até o início do regime fascista. Ao fazê"-lo, retraçaremos as
características peculiares de seu estilo oratório, indicando os motivos
fundamentais de suas concepções de vida, de sociedade, de história e de
política, que permaneceram constantes no curso de sua metamorfose
ideológica:

\begin{enumerate}
\def\labelenumi{\alph{enumi})}
\item
  concepção da política: \emph{subjetivamente}, ele a concebe como arte,
  ou seja, como intuição individual das circunstâncias apropriadas que
  podem ser elaboradas pela vontade do homem político;
  \emph{objetivamente}, como manifestação da força e do conflito entre
  as ambições e interesses de grupos sociais antagonistas: as classes
  sociais, as nações, os Estados;
\item
  redução das ideias a mitos, que são instrumentos capazes de suscitar e
  canalizar as paixões das massas e de conquistar sua fé, impelindo"-as à
  ação;
\item
  desprezo pelas massas, mas também apreciação realista de sua
  importância como força política da sociedade moderna, sem, contudo,
  depositar nenhuma confiança em sua capacidade de evoluir em direção a
  modos de consciência coletiva autônoma e a uma autonomia de governo;
\item
  visão da história como um ciclo criador de baixa energia, mas também
  como uma atividade criadora do Estado e da civilidade, sem, contudo,
  projetar nenhum fim em sua perpétua evolução;
\item
  possibilidade de regeneração social ou de revolução por meio da
  iniciativa de grandes líderes, concebidos como \emph{novos homens},
  que conseguem impor suas vontades de poder, ao agirem além das regras
  morais comuns;
\item
  pessimismo a respeito da natureza dos homens, considerados egoístas e
  inclinados ao mal, caso não sejam submetidos e disciplinados por um
  poder superior, que lhes imponha uma disciplina que forje o
  caráter.\footnote{Cf. Gentile, E. \emph{Le origini dell'ideologia
    fascista (1918--1925)}, Bolonha: Il Mulino, 1996, p.\,61 e seguintes.}
\end{enumerate}

Com base na trama constante desses motivos, Mussolini elaborou,
inicialmente, sua ideologia no interior do socialismo marxista
revolucionário e, mais tarde, sua ideologia do fascismo, em sua primeira
versão libertária (1919--1920) e, posteriormente, em sua versão
totalitária (1921--1943). Ainda que a combinação constante desses motivos
constitua o fio de continuidade que acompanha as metamorfoses da vida
política mussoliniana, isso não significa que no Mussolini socialista já
estivesse incubado o Mussolini fascista, de modo a dar a impressão de
que o socialismo marxista mussoliniano seria um tipo de socialismo não
verdadeiramente marxista, mas uma prefiguração do fascismo. Isso posto,
é preciso afirmar que, sem dúvida, tanto o Mussolini marxista quanto o
Mussolini fascista partilhavam da mesma aversão à democracia parlamentar
e ao liberalismo. Caso tivesse desaparecido em 1913, Mussolini passaria
a ser conhecido na história como um líder do socialismo revolucionário
italiano, que havia estabelecido e mesmo imposto à direção do Partido
Socialista uma concepção revolucionária do marxismo e da luta de classes
como a vanguarda da consciência do partido revolucionário do
proletariado. No momento em que, graças à intransigência de Mussolini,
então com 29 anos de idade, a corrente revolucionária assume a direção
do Partido Socialista italiano, durante o congresso nacional de 1912,
Lênin escreve em seu \emph{Pravda} que o socialismo italiano ``havia
tomado o bom caminho''.\footnote{Lenin, V. I. \emph{Sul movimento operaio
  italiano}, Roma: Riuniti 1962, p.\,81 e seguintes.}

Todavia, sem a Primeira Grande Guerra, que provocou uma verdadeira
metamorfose no Mussolini socialista, a partir dos meses de julho e
agosto de 1914, não teria havido um Mussolini intervencionista e
fascista e nem mesmo o fascismo. Se Mussolini tivesse morrido no fronte
no outono de 1915, ele entraria para a história como o ex"-dirigente do
Partido Socialista e diretor do \emph{Avanti}, depois de ter sido
expulso do \textsc{ps} em novembro de 1914, e como um socialista
intervencionista, tal como era a maioria dos socialistas europeus
daquela época. Já se tivesse morrido em 1919, ele passaria para a
história como intervencionista e veterano de guerra, que, depois do fim
do conflito mundial, havia tentado em vão retomar um papel político na
direção de um movimento denominado \emph{Fasci di combattimento}
(Conjuntos de combate), fundado em 23 de março de 1919, com um programa
republicano genérico, libertário, antiestatal e individualista, que até
o fim da década de 1920 teve um número exíguo de partidários e uma
presença violenta e ruidosa, mas marginal na luta política italiana.

Com efeito, depois de nove meses de seu nascimento, em dezembro de 1919,
o movimento fascista contava somente com 800 membros em toda a Itália.
Na ocasião das eleições de novembro de 1919, Mussolini obtém menos de
cinco mil votos; e, ao final de 1920, os fascistas somavam ainda somente
alguns poucos milhares, dispersos pelo território italiano. Durante esse
período, Mussolini fez muitos pronunciamentos, mas, sem o concurso de
outros fatores, sua oratória permanecia sem maiores poderes de sugestão
sobre as massas.

Sem a repentina explosão do \emph{squadrismo},\footnote{Nota do tradutor:
  o \emph{squadrismo} foi um fenômeno político e social de constituição
  de milícias fascistas, que atuaram na década de 1920 contra os
  adversários do então nascente regime de Mussolini. O termo ainda
  designa a ideologia desse movimento e suas ações.} entre 1920 e 1921
--- uma explosão que se produziu independentemente de Mussolini e
transformou seu minúsculo movimento em um movimento armado de massa ---
provavelmente o ex"-socialista revolucionário, intervencionista e
libertário permaneceria na condição de jornalista e jamais se tornaria o
líder de um regime totalitário. Foi graças, antes, ao \emph{squadrismo}
armado, do que aos talentos de jornalista e de orador de Mussolini, que
o exíguo movimento dos \emph{Fasci di combattimento} se tornou um
movimento de massa. Em maio de 1922, a marcha fascista já havia
ultrapassado os cem mil membros e Mussolini foi eleito deputado nas
eleições de 1921, com mais de duzentos mil votos.

Além disso, entre agosto e novembro de 1921, ocorre uma outra
metamorfose mussoliniana. Ela acontece quando o Mussolini fascista
libertário, antiestatal e individualista, que até havia flertado com a
anarquia, depois de ter visto naufragar o projeto de transformar o
movimento fascista em um partido parlamentar, por meio da desmobilização
de suas facções armadas, se converte ao \emph{squadrismo}. Logo após
essa conversão, ele assume a direção nacional do partido fascista,
criado em novembro de 1921, como um partido armado e abertamente
antidemocrático.\footnote{Cf. Gentile, E. \emph{Storia del partito
  fascista. 1919--1922. Movimento e milizia}, Roma"-Bari: Laterza, 1989,
  p.\,314 e seguintes.}

Depois de 1921, como líder do partido fascista, Mussolini se vale,
sobretudo, de discursos dirigidos às massas cada vez mais numerosas de
fascistas e igualmente àquelas de seus simpatizantes, para elaborar uma
nova ideologia do fascismo estatal e antidemocrático, modelada com base
na experiência concreta do \emph{squadrismo}, que havia imposto um
controle ditatorial nas províncias italianas, nas quais esse movimento
armado havia destruído o predomínio do Partido Socialista mediante ações
violentas. O novo fascismo encarnado em um partido dotado de milícia, ao
se identificar com a nação, buscou eliminar o Estado liberal para
instituir o Estado fascista, concentrando no fascismo o monopólio do
poder. Como de fato ocorreria mais tarde, a marcha sobre Roma conduziu o
líder do partido fascista e armado à condição de chefe do governo do
Estado italiano e o transformou em um Estado totalitário.\footnote{Cf.
  Gentile, E. \emph{Soudaine, le fascisme}, Paris: Gallimard, 2015.}

Durante aquele período de adesão ao \emph{squadrismo}, Mussolini teve de
lutar para conquistar e consolidar seu papel de líder do fascismo e ele
obteve êxito nessa luta graças aos seus talentos de grande orador.
Certamente, ele não apenas era o mais hábil e o mais eficiente entre os
jovens dirigentes do \emph{squadrismo}, mas também era o mais competente
e o mais eficaz entre todos os demais dirigentes dos partidos italianos.
Além de ser o mais jovem, ele foi o único entre todos os líderes do
Estado italiano que conseguiu com seu desempenho oratório estabelecer um
contato direto e frequente com a população de todas as regiões da Itália
e instituir desde os primeiros meses de governo um amplo consenso
popular.

Essas sintéticas observações sobre a trajetória política mussoliniana,
que insistem em destacar suas metamorfoses relacionadas às sucessivas
situações históricas, sempre novas, inesperadas e imprevisíveis, buscam
propor um exame da oratória de Mussolini, que não seja anacrônico, ou
seja, que não projete retrospectivamente as sombras do líder totalitário
no fascista libertário, no fascista intervencionista e até mesmo no
socialista revolucionário. Nossa intenção consiste, antes, em demonstrar
como o desempenho oratório mussoliniano foi essencial nas diferentes
fases de sua trajetória política para suscitar a sugestão carismática na
própria pessoa de Mussolini, mesmo em contextos políticos distintos e em
públicos bastante diversos e até mesmo opostos uns aos outros. A
extraordinária singularidade da experiência oratória de Mussolini,
quando comparada às de outros oradores carismáticos do começo do século
\textsc{xx}, reside precisamente em sua biografia política particular, com suas
sucessivas metamorfoses.

Uma atividade oratória que perdurou ininterruptamente por quase quatro
décadas somente pode ser examinada aqui por meio de alguns de seus casos
mais emblemáticos.\footnote{A oratória de Mussolini não dispõe de um
  estudo sistemático e completo, que examine os aspectos linguísticos,
  estilísticos, culturais e ideológicos em seus diferentes contextos
  históricos. Entre as publicações realizadas durante o período
  fascista, a apologia prevalece sobre a análise crítica, enquanto nas
  publicações posteriores ao fim do fascismo, predominam os estudos
  linguísticos, que, por razões culturais e ideológicas, se limitam aos
  traços grotescos e às contradições dos discursos mussolinianos. Nós
  nos limitaremos a mencionar aqui os estudos mais interessantes,
  contemporâneos a Mussolini ou publicados após sua morte: Ardau, G.
  \emph{L'eloquenza mussoliniana}, Milão: Mandadori, 1929; Gustarelli,
  A. (Org.) \emph{Mussolini oratore e scrittore}. Milão: Vallardi, 1935;
  Bianchi, L. \emph{Mussolini oratore e scrittore}, Bolonha: Il Mulino,
  1937; Adami, E. \emph{La lingua di Mussolini}, Modena: Società
  Tipografica Modenese, 1939; Ellwanger, H. \emph{Sulla lingua di
  Mussolini}, Milão: Mondadori, 1941; Leso, E. ``Aspetti della lingua
  del fascismo. Prime linee di ricerca'', in \emph{Storia linguistica
  dell'Italia nel Novecento}, Roma, 1973; Lazzari, G. \emph{Le parole
  del fascismo}, Roma: Argileto, 1974; \emph{La lingua italiana e il
  fascismo} (Vários autores), Bolonha, Treccani, 1977; Simoni, A.
  \emph{Il linguaggio di Mussolini}, Milão: Bompiani, 1978;
  \emph{Parlare fascista. Lingua del fascismo, politica linguistica del
  fascismo} (Vários autores), in ``Movimento operaio e socialista'', n.
  1., janeiro"-abril de 1984; Golino, E. \emph{Parola di duce. Il
  linguaggio totalitario del fascismo e del nazismo}, Milão: Rizzoli,
  2011.} É necessário, porém, começar com casos do período em que
Mussolini era um socialista revolucionário, porque foi então que,
juntamente com a carreira de jornalista, emergiu entre 1912 e 1914 a
originalidade do Mussolini orador. Foi justamente como orador que ele
obteve grande sucesso no interior do Partido Socialista e entre as
massas proletárias. Em seguida, seguiremos a trajetória mussoliniana
durante os anos do período posterior à Primeira Guerra, entre 1921 e
1922, quando a oratória de Mussolini cumpriu uma função decisiva em sua
ascenção à condição de líder do fascismo, então organizado com suas
milícias, de tal sorte que, com a violência de suas facções, em poucos
meses, o movimento conseguiu se impor como o mais forte partido italiano
e conquistar o poder. A oratória de Mussolini foi também absolutamente
decisiva nos dois primeiros anos depois da ascensão do fascismo ao poder
para transformar o líder de um partido armado em chefe do governo mais
popular em toda história da Itália; tal transformação ocorreu, tal como
demonstram vários exemplos, graças ao contato direto de Mussolini com as
massas populares de diversas regiões italianas. Finalmente, no intuito
de tratar do líder totalitário, examinaremos o exemplo mais
significativo, que é certamente o do dia 09 de maio de 1926, quando da
sacada do Palácio de Veneza o \emph{Duce} anunciou ``o retorno do
Império nas mortais colinas de Roma''.

\section{O orador de história monumental}

Nos primeiros vinte anos de sua vida política, de 1902 a 1920, apesar
das diversas metamorfoses ideológicas, a oratória mussoliniana conserva
traços constantes. O mais importante é o laço entre a história e a
atualidade, tal como havia sido mencionado pelo estudante Benito em sua
dissertação escolar, característica que esteve sempre presente na
eloquência de Mussolini durante todo o período de sua militância
política. O objetivo dessa conexão era o de influenciar os públicos de
seus discursos, tocando em seus sentimentos, e, assim, formar suas
consciências. Inicialmente, socialista, e, mais tarde, fascista, o
orador Mussolini sempre concebeu o auditório de seus pronunciamentos
como um conjunto de estudantes ao qual era necessário ensinar ``o
verdadeiro e o justo'', por meio de exemplos da história conectados à
atualidade.

Mussolini --- para empregarmos aqui uma distinção feita por Friedrich
Nietzsche, um filósofo que teve grande influência na formação de sua
cultura e personalidade --- era adepto e entusiasta da ``história
monumental'', ao passo que tanto era indiferente à ``história
antiquária'', que presta um culto reverente ao passado, quanto
desprezava a ``história crítica'', que considera o passado a partir de
observações desvinculadas das necessidades da vida contemporânea. ``Se o
homem pretende criar coisas grandes, ele precisa do passado, que se
impõe mediante uma história monumental'', sem se curar completamente das
mutilações que o passado lhe infligiu, porque ``grandes segmentos do
passado são esquecidos, desprezados e fluem como uma torrente cinzenta
ininterrupta, de modo que apenas fatos singulares adornados se alçam por
sobre o fluxo como ilhas''.\footnote{F. Nietzsche, \emph{Sull'utilità e
  il danno della storia per la vita}, Milão: Adelphi, 1974, p.\,21--23. A
  influência da ideia de ``história monumental'' sobre a oratória
  mussoliniana já fora observada por Hermann Ellwanger em \emph{Sulla
  lingua di Mussolini}, \emph{op.\,cit.}, p.\,37 e seguintes. A propósito
  da influência de Nietzsche sobre Mussolini, ver: Nolte, E. Marx und
  Nietzsche im Sozialismus des jungen Mussolini, in \emph{Historische
  Zeitschrift}, outubro de 1960, p.\,304 e seguintes.}

Na oratória mussoliniana, os ``fatos singulares adornados'' citados em
seus discursos pronunciados em plena efervescência política para ``criar
coisas grandes'' eram os grandes acontecimentos do passado, que haviam
produzido ``modificações na atividade e na consciência humana'', tal
como escrevera o estudante Benito. Na oratória do socialista Mussolini,
esses acontecimentos eram, principalmente, a Comuna e, mais geralmente,
a Revolução Francesa. Já para o Mussolini intervencionista, eram as
guerras da França revolucionária e o \emph{Risorgimento}. Finalmente,
durante o fascismo, era sobretudo a história da Roma antiga. Mas, nas
diferentes etapas, foi constante a convicção de Mussolini na evocação
histórica como instrumento poderoso para sugerir e influenciar os
públicos ouvintes de seus pronunciamentos, ao lhes dar o sentimento de
participar da atualidade, ao mesmo tempo em que estariam imersos no
fluxo da história e criando ``coisas grandes'' no presente, inspirados
pelas ``grandes coisas'' do passado. ``Evoquemos o passado,
apreendamo"-lo, e sintamos a `continuidade' da vida, para que possamos
encontrar nos dias de outrora as razões de nossos dias e em nossos dias
os elementos necessários para construirmos o amanhã''.\footnote{Mussolini,
  \emph{Opera Omnia}, vol. \textsc{ii}, p.\,58.}

Nos pronunciamentos socialistas e fascistas de Mussolini, o poder
evocador da história prescindia de precisão filológica, que a abordagem
dos discursos mussolinianos negligenciava facilmente, para mais bem
exprimir julgamentos e projetar imagens verbais, que se poderiam
imprimir no espírito das massas. Os apelos à história eram um meio
empregado por Mussolini para legitimar sua ação política, quando ele
estava comprometido em sua luta pelo poder, tanto no interior do Partido
Socialista quanto, mais tarde, no movimento fascista. Além disso, a
frequente referência à história, nos discursos socialistas, mas também
nos fascistas, tinha o objetivo de demonstrar que suas ideias, suas
escolhas e decisões não eram ditadas pelas circunstâncias, mas provinham
de uma reflexão sobre a realidade contemporânea à luz da experiência
histórica e estavam na boa direção do curso da história. Enfim, a
``história monumental'', especialmente durante a fase fascista de
Mussolini, foi o meio para imediatamente estabelecer uma comunhão
emotiva com as massas das várias cidades da Itália, visitadas pelo
\emph{Duce} desde os primeiros meses de sua ascensão ao poder até as
vésperas da Segunda Guerra Mundial. Em cada uma dessas cidades,
Mussolini começa por se identificar com a massa mediante a inserção de
sua visita entre os grandes acontecimentos históricos vividos pela gente
daquele lugar, acontecimentos exaltados como momentos de revelação da
grandeza do povo italiano, que finalmente havia encontrado no
\emph{Duce} seu intérprete e seu artífice.

\section{O povo aprendiz}

No que concerne a audiência de suas falas públicas, Mussolini sempre
teve a postura do mestre --- tal como ele mesmo o havia esboçado em sua
dissertação da juventude --- que deve suscitar os sentimentos mais
elevados no povo que o escuta. Para tanto, a atividade oratória era
indispensável, conjuntamente com o jornal, mas ainda mais do que este
último, porque seu público consistiria em uma sociedade na qual a
população contava ainda com um grande percentual de analfabetos.

Desde os primeiros anos de sua militância socialista na Suíça, as
atividades de Mussolini, constituídas de uma intensa atividade
jornalística, compreendem uma tão ou mais intensa atividade como orador,
em comícios ou em conferências, nas quais ele tratava de assuntos
históricos e culturais, sempre tentando, mesmo que as discussões
versassem sobre os temas mais diretamente relacionados ao presente,
situar os acontecimentos atuais em uma perspectiva histórica. Sua
concepção do povo então se limitava ao proletariado e sua prática
oratória era, principalmente, dedicada a expor clara e nitidamente sua
compreensão revolucionária do socialismo, em violenta polêmica contra a
direção de tendência reformista do Partido Socialista. O Mussolini
orador, ao falar aos círculos restritos de militantes ou ao falar em
comícios mais amplos, se dirigia às massas de operários, ou seja, se
endereçava ao povo proletário, com a postura do mestre que deve fascinar
seus alunos para elevar seus sentimentos e consciências.

Essa atitude era consequência de sua concepção do público de seus
pronunciamentos. A multidão, a massa, enfim, o povo, concebidos como uma
entidade coletiva, seriam incapazes, por sua própria natureza de
coletividade anônima, de se alçar a melhores condições de vida sem a
direção de um líder e de uma elite organizada sob a forma de um partido.
Nesse sentido, em 1909, Mussolini confessava o seguinte: ``Meu
temperamento e minhas convicções me levam a preferir um pequeno grupo
resoluto e audaz a uma massa numerosa, mas caótica, amorfa e
vil''.\footnote{Mussolini\emph{, Opera omnia}, vol. \textsc{ii}, p.\,75.}

De seus tempos de socialista revolucionário, Mussolini manteve sempre a
ideia de que o partido deveria primar por sua vanguarda, em detrimento
da extensão numérica de seus componentes, mesmo que houvesse centenas de
milhares de outros membros na composição do quadro partidário. Uma das
pedras angulares de sua concepção política era a crença, conforme ele a
expressou em 1908, de que a história ``não é nada mais do que uma
sucessão de elites dominantes''; e conforme ele reiteraria em 1913: ``A
luta na sociedade humana sempre foi e sempre será uma luta de minorias.
Reivindicar a maioria absoluta --- quantitativamente --- é um absurdo.
Jamais se lamentará que a maioria do proletariado não esteja na direção
das organizações econômicas e políticas. E os demais? A luta de classes
é, em última instância, uma luta de minorias. A maioria segue, se
submete'',\footnote{Mussolini, \emph{Per l'intransigenza del socialismo},
  in ``Avanti!'', 29 marzo 1913.} porque ``a massa é somente
quantidade; é inércia. A massa é estática; as minorias são
dinâmicas''.\footnote{Sobre o Mussolini socialista, conferir: Megaro, G.
  \emph{Mussolini dal mito alla realtà}, Milão: Il Mulino, 1947; De
  Felice, R. \emph{Mussolini il rivoluzionario (1883--1920)}, Torino:
  Einaudi, 1965; Gregor, J. A. \emph{The Young Mussolini and the
  Intellectual Origins of Fascism}. Berkely/Los Angeles: University of
  California Press, 1979; Gentile, E.; Di Scala, S. M. (Org.)
  \emph{Mussolini socialista}, Roma"-Bari: Laterza, 2015.}

O Mussolini socialista costumava declarar abertamente que preferia a
``qualidade'' à ``quantidade'': ``Ao rebanho obediente, resignado e
idiota, que segue o pastor e se desespera diante do primeiro uivo dos
lobos, preferimos o pequeno grupo, resoluto e ousado, que dá fundamento
racional às suas convicções. Esse pequeno grupo sabe o que quer e vai
diretamente em direção a seu objetivo'', afirmou em 1910, ao difundir os
propósitos de sua propaganda socialista. Esses propósitos eram
difundidos, antes, por meio de conferências culturais do que por
discursos com frases de efeito.\footnote{Mussolini, \emph{La nostra
  propaganda}, in ``La Lotta di Classe'', 12 de fevereiro de 1910.}

A despeito de seu desprezo pelas massas, Mussolini considerava a
propaganda que lhes era endereçada como algo indispensável ao
desenvolvimento de um partido socialista revolucionário, no qual cabia
aos seus líderes a função pedagógica de educar culturalmente o
proletariado. Por essa razão, o jovem revolucionário dava considerável
``importância ao elemento teórico e doutrinário na existência do
socialismo'', tal como declarou em 30 de maio de 1908:

\begin{quote}
É a cultura, é a sua máxima difusão que deve preparar a nova alma, é a
cultura que dará ao elemento humano a capacidade de sair da vida bestial
do mundo ordinário, a capacidade de compreender a beleza de uma ideia e
de se interessar pelas grandes questões. A influência da literatura
socialista será ainda maior quando o operário se dedicar ao livro como a
um fiel amigo e, assim, buscar atingir a elevação de sua própria
inteligência e a libertação da escravidão de seu espírito. É com esse
esforço determinado e consciente que a classe operária marcará uma nova
e luminosa etapa na história da humanidade.\footnote{Mussolini,
  \emph{Socialismo e socialisti,} in ``La Lima'', 30 de maio de 1908.}
\end{quote}

Mussolini continuava sempre a se valer de sua atividade oratória, mas
passou a fazê"-lo ainda com maior intensidade a partir de 1909, na
condição de secretário da Federação Socialista de Forli, na Romagna. Em
10 de abril de 1910, na ocasião do primeiro congresso da Federação,
Mussolini, em seu relatório sobre sua atividade de secretário, sublinha
seu trabalho de propaganda, enumerando as conferências que havia feito
em várias circunstâncias, e afirma que ele havia pretendido ``explicar
durante essas conferências os princípios do socialismo, antes, a um
público de socialistas do que a um habitual auditório de curiosos'':

\begin{quote}
O trabalho de proselitismo já foi realizado ou, em todo caso, já
caminhou a passos largos em nossa região; agora se trata de dar uma
consciência socialista aos que estão inscritos em nosso Partido. Não
precisamos nos preocupar somente com a ``quantidade'', mas também, e
isso é ainda mais importante, com a ``qualidade''. (\ldots{}) Não creio que
alguém entre os que ouvem minhas conferências diria algo como: `Somos
poucos!' E se esse fosse o caso, devo dizer que eu não poderia fazer
nada além do que faço para retê"-lo. Não poderia fazê"-lo por respeito ao
meu cérebro, à minha condição de guardião da propaganda e de fonógrafo
ambulante. Devo ler uma infinidade de jornais, muitas revistas e muitos
livros para que eu me mantenha bem informado sobre o movimento
socialista e intelectual contemporâneo; e ler tudo isso toma um tempo
considerável. Muito melhor são as conferências de pensamento denso e
para poucos do que um rosário de tagarelices, superficialidade e
arroubos retóricos. Continuarei com minhas funções nessas ocasiões e não
abrirei mão de minha fala a cada vez que ela for necessária.\footnote{Mussolini,
  \emph{Opera Omnia}, vol. \textsc{iii}, p.\,70--71.}
\end{quote}

De 1910 a 1912, a oratória mussoliniana esteve exclusivamente dedicada
aos socialistas em reuniões locais ou aos trabalhadores em situações de
agitação política. Os resumos desses seus pronunciamentos, publicados
pela imprensa, terminam sempre com a seguinte menção: ``fortes
aplausos''. Mussolini passa a obter maior notoriedade, mesmo em nível
nacional, a partir de 1911, quando foi um dos principais organizadores
das manifestações de protesto contra a guerra com a Líbia. Essas suas
ações o conduziram a uma condenação a alguns meses de prisão, mas lhe
garantiram igualmente a conquista de um renome além dos limites de sua
província.

\section{Um orador meio estranho}

O sucesso local da oratória mussoliniana foi confirmado com o
considerável aumento no número de membros da Federação socialista de
Forli. Todavia, o primeiro discurso proferido por Mussolini em um
congresso socialista nacional foi um fiasco. No dia 20 de outubro de
1910, na condição de dirigente de uma província e de militante da ala
revolucionária do partido, Mussolini tomou a palavra no \textsc{xi} Congresso do
Partido Socialista, realizado naquele ano em Milão, em ofensiva contra a
direção da ala reformista do partido. Era, então, seu primeiro discurso
diante de uma plateia de âmbito nacional. Ele começou dizendo que faria
apenas algumas ``declarações telegráficas, antes de mais nada para não
prolongar aquela discussão, que mais parecia um debate acadêmico ou um
conselho ecumênico''. Em seguida, atacou os reformistas, que então
estavam na direção do Partido Socialista, sustentando que o partido
deveria adotar uma política de intransigência revolucionária. O breve
discurso foi frequentemente interrompido por rumores e gritos hostis; e,
ao seu final, houve somente ``alguns aplausos''.\footnote{Mussolini,
  \emph{Opera Omnia}, vol. \textsc{iii}, p.\,208--211.}

Porém, apenas dois anos mais tarde, no \textsc{xiii} Congresso Socialista, que
ocorreu em Reggio Emilia, em julho de 1912, o jovem socialista da
Romagna, então com 29 anos, conseguiria uma clamorosa revanche. Seu
discurso contra os reformistas, apesar de algumas interrupções, ruídos e
tumultos, foi acompanhado e seguido de ``vivos e prolongados aplausos e
ainda de numerosas congratulações''.\footnote{Mussolini, \emph{Opera
  Omnia}, vol. \textsc{iv}, p.\,160--170.} Com seu discurso, Mussolini conduziu a
ala revolucionária à vitória na disputa pela direção do Partido
Socialista e, em seguida, conquistou ainda a direção do jornal do
partido, o \emph{Avanti}, de cuja chefia editorial ele mesmo se tornou o
titular. Repentinamente, o jovem romagnolo se alçou às luzes da ribalta
política nacional com o sucesso oratório que teve naquele congresso.

O jovem orador causou uma forte impressão nos numerosos participantes do
congresso e mesmo entre seus adversários, tanto por sua personalidade
quanto, sobretudo, por seu inédito e insólito estilo oratório. Os
comentários sobre seu pronunciamento são particularmente significativos
não apenas porque indicam como a personalidade de Mussolini suscitava as
emoções carismáticas, sem a intervenção da manipulação via propaganda,
mas também porque todos eles se concentram na originalidade de seu
discurso, que parecia perturbar a longa tradição da retórica política
daqueles tempos e mesmo a própria tradição retórica do Partido
Socialista.\footnote{Todas as citações dos comentários dos jornais sobre
  o discurso de Mussolini foram extraídos de: Mussolini, \emph{Opera
  Omnia}, vol. \textsc{iv}, p.\,292 e seguintes.} \emph{Il Messaggero}, jornal
liberal de Roma, comentava assim aquele pronunciamento: ``O primeiro
orador da tarde foi o prof. Mussolini, o ardente revolucionário de
Forli. Ele pronunciou um discurso paradoxal''. Por sua vez, o cotidiano
moderado de Florença, o \emph{Nuovo Giornale}, afirmava que a sessão
havia começado ``de maneira mais tempestuosa e violenta possível.
Mussolini, o ardente professor romagnolo, subira à tribuna: ele é
absolutamente intratável, ele é tão intransigente que continua a falar
mesmo com os contínuos rumores da audiência''. Já o expoente da corrente
liberal, o \emph{Corriere della Sera}, ressaltava o sucesso do orador
``magro, áspero, que falou franca e sinceramente e que agradou os
membros do congresso, porque estes sentiram que havia nele um intérprete
de seus próprios sentimentos. (\ldots{}) Os insistentes aplausos provinham da
grande maioria do congresso. Por ter falado com grande preocupação e
ardor, Mussolini desceu da tribuna pálido e extenuado, mas continuamente
aplaudido e parabenizado por vários colegas''. \emph{Il Secolo},
cotidiano democrático de Milão, escreveu que o revolucionário socialista
discursara ao congresso, e que seus integrantes ``haviam ficado
encantados com o discurso de Mussolini'',

\begin{quote}
um original agitador romagnolo, que não repetiu as razões de sua própria
condição de revolucionário a partir do velho arsenal de companheiros
dessa tendência e que, tal como ocorre com seus rigorosos estudos que se
tornam a fonte de sua cultura variada, no frequente contato com as
massas operárias durante a bem"-sucedida campanha na Romagna, alcançou o
calor de sua fé e seu irredutível instinto revolucionário. Depois de
Lazzari, que insistia demais na mesma nota e o fez durante muitos anos,
batendo sempre na mesma tecla, Mussolini foi o único que pôde justificar
seu desapreço por outras tendências e explicar como a corrente
revolucionária se distingue claramente das demais. Todos os outros que
discursaram bem ou mal não conseguiram este efeito: mostrar que, no
fundo, todos eles possuem uma mesma alma.
\end{quote}

Outro jornal liberal de Torino, o \emph{La Stampa}, assim o definia:
``Mussolini, um dos mais notáveis líderes revolucionários e também um
dos mais intransigentes. Com gesto vivíssimo e com oratória violenta,
ele nega o direito de autonomia ao grupo parlamentar. Os companheiros de
Mussolini, que o seguem com grande e calorosa simpatia, aplaudiram
bastante seu discurso''. Já o \emph{La Vita}, um jornal romano,
ironizava ``o reconhecido, autorizado e respeitado chefe dos
ultra"-intransigentes. Esse chefe é certamente Mussolini, que é também
chamado de Benito. Tão logo alguém o vê, já compreende que é
intransigentíssimo. (\ldots{}) Ao se pronunciar, ele recebe estrondosos
aplausos. Mussolini sorri e anuncia que passará a apresentar a festiva
programação do dia.''. Finalmente, o já mencionado cotidiano florentino,
o \emph{Nuovo Giornale}, no dia 10 de julho de 1912, resumia o discurso
de Mussolini ao mesmo tempo com palavras de crítica e de admiração:

\begin{quote}
A teoria do prof. Mussolini\ldots{} é meio estranha. Mas, ela é sustentada
por um homem sutilmente dialético, fecundo e soberbo: um verdadeiro tipo
de pensador original, que sempre pretende encontrar uma nova via e que
começa a se impor, abrindo docemente essa via e a insuflando com os
talentos de sua engenhosidade e de sua rude oratória, que agrada
bastante os rudes romagnolos. (\ldots{}) O socialismo deveria, portanto, se
tornar puramente um partido político, que visa a subverter as atuais
instituições para fazer funcionar maximamente seu programa. Como se pode
ver, a teoria é estranha e renega os benefícios de que as classes
sociais e a sociedade em geral podem usufruir da legislação. Porém, não
lhe faltam adeptos.
\end{quote}

\section{A revelação de um homem}

Os comentários mais entusiastas sobre os pronunciamentos de Mussolini
provêm dos jornais da esquerda revolucionária. Paolo Valera, um
sindicalista revolucionário, qualificava Mussolini como ``o cérebro do
socialismo revolucionário'', dotado de um temperamento que não conhecia
a instabilidade emocional: ``Ele é todo feito de bronze. É um homem
cheio de ideias, um homem repleto de ações''. Em \emph{La Propaganda},
um jornal sindicalista de Nápoles, encontramos a seguinte afirmação:

\begin{quote}
Um dos mais belos discursos foi o pronunciado por Benito Mussolini, o
revolucionário romagnolo, diretor do \emph{A Luta de classes}. Ele tem
uma eloquência pessoal, sem fraseologias, afiada como navalha e sem
nenhuma vulgaridade. Foi um dos maiores sucessos.

Em sua fala, ora mais dinâmica ora mais desacelerada, havia uma vibração
abarrotada, principalmente, de verdade e de bondade. Ele não se
apresenta como um magistrado do Ministério público que pretende condenar
réus, mas como um soldado íntegro que deve executar uma obra
fundamental, mesmo que para isso tenha de perder um pedaço de sua alma.
Até mesmo os adversários devem tê"-lo escutado com todo o respeito.

Na tribuna, drapeada de vermelho ao fundo, a figura desse jovem discreto
e reflexivo, com dois olhos brilhantes, com uma chama de bondade
espalhada por todo seu rosto, que acompanharia sua fala e seus gestos
agitados, surgiu do fundo do teatro para expressar aos presentes toda a
inquitação de seu pensamento, em que se podia sentir que ele continha a
palpitação de seu coração, caso fosse decidido que se deveria retardar o
caminho para o socialismo.

Ao descer da tribuna, entre as entusiasmadas aclamações, seus olhos
brilhantes e sinceros não expressavam maior contentamento. Ele apenas
havia cumprido seu dever, em defesa dos interesses do Partido.\footnote{Mussolini,
  \emph{Opera Omnia}, vol. \textsc{iv}, p.\,294--297.}
\end{quote}

Quando, em dezembro de 1912, Mussolini foi nomeado diretor do
\emph{Avanti!}, seus companheiros de Forlí exultaram diante do sucesso
de seu ``\emph{Duce} incorruptível''.\footnote{Mussolini, \emph{Opera
  Omnia}, vol. \textsc{iv}, p.\,297.} Tal como uma extraordinária revelação, o
Mussolini orador surgiu aos olhos de Leda Rafanelli, uma intelectual
anarquista. Ao ouvir o pronunciamento de Mussolini feito por ocasião do
aniversário da província, ela o julgou melhor orador do que jornalista e
comentou em abril de 1913 com os seguintes termos sua intervenção:

\begin{quote}
Benito Mussolini, diretor do \emph{Avanti!}, é um socialista destes
nossos tempos heróicos. Ele sente o vigor do socialismo e lhe deposita
ainda toda sua confiança, com um \emph{élan} repleto de força e
virilidade. Ele é um homem. Eu o ouvi pela primeira vez no dia 18 de
março, por ocasião da comemoração do aniversário da província, depois de
já tê"-lo conhecido através da leitura de seus artigos, cuja escrita traz
formulações e ideias originais. Mas, eu o achei ainda melhor e mais
forte como orador do que como jornalista. Ele é diferente de todos os
oradores que pretensamente conduzem seus públicos ao êxtase. Seu
pronunciamento não correspondeu a uma comemoração ordinária; foi, antes,
a rememoração do sagrado período da Revolução e, ao mesmo tempo, a
apresentação de repentinas revelações e o estabelecimento de preciosas
relações. Em cada uma de suas palavras, em cada um de seus gestos, sua
personalidade forte se expressava e despertava a atenção mesmo daqueles
que poderiam estar mais distraídos. (\ldots{}) Munido de um arsenal de
ideias, de dados e de documentos que ilustravam sua fala, conseguiu,
ainda assim, ser conciso, com um poder de síntese absoluto. Ele falava e
transmitia perfeitamente suas ideias. Nem sequer uma única vez pretendeu
dar ensejo aos aplausos, não o fez nem sequer em uma única das frases
que se formavam em seus lábios, sempre animadas por uma íntima vibração
de todos os seus sentimentos. Essas frases não eram adornadas por
nenhuma palavra supérflua. A história da província foi exposta de modo
vivo e verdadeiro, foi esculpida sob a forma de curtas formulações,
dirigidas às centenas de ouvintes, como talvez ela jamais tenha sido
revivida em todas as comemorações oficiais anteriores, que haviam
ocorrido nestes últimos anos.

E Benito Mussolini revelou"-se a mim quase como um artista da palavra e
do pensamento, um artista colossal e despojado de refinamentos, que cria
seus esboços e os desenvolve, sem dispor do tempo para poli"-los e
torná"-los mais bem acabados. Mas a impressão que ele produz permanece
gravada em nossos corações; o eco de suas palavras permanece gravado em
nossos cérebros. Ele afirmou que estava pessimista em relação aos
homens, mas que cria sempre na força da massa sublevada pela explosão
diante da vontade burguesa de se enriquecer ainda mais durante a guerra
e de perpetuar sua dominação. Ele suspendeu por um momento sua fala,
provocando nossa expectativa, e, com sua voz plena de esperança, disse
que a futura Revolução está próxima, certamente muito próxima\ldots{}

Este ano, a província de Forlí teve uma digna comemoração com Benito
Mussolini. Se a desejada Revolução não vier, esta província e nós mesmos
estaremos entre aqueles que mais sofrerão muitos retrocessos.\footnote{Citado
  por De Felice, R.; Goglia, L. \emph{Mussolini. Il Mito}. Roma"-Bari:
  Laterza, 1983, p.\,94--96.}
\end{quote}

No congresso nacional do Partido Socialista, ocorrido em Ancona, em
abril de 1914, teria havido novamente outro estrondoso sucesso pessoal
do Mussolini orador. Já no momento em que subiu ao alto da tribuna, foi
recebido com uma grande ovação de todos os congressistas, que o
aplaudiram de pé. A essa eufórica recepção, o orador imediatamente
replicou: ``Eu os cumprimento com minhas mais cordiais saudações,
anunciando imediatamente que serei telegráfico. Vou lhes falar somente
alguns minutos, mesmo porque creio que seja desnecessário duplicar aqui
oralmente o relatório escrito que certamente os senhores viram, leram e,
espero, sobre o qual tenham meditado''.\footnote{Mussolini, \emph{Opera
  Omnia}, vol. \textsc{vi}, p.\,163--168.} Ao encerrar seu segundo discurso
naquele congresso, no qual ele solicitou e obteve a aprovação de sua
proposta de interditar a pertença à franco"-maçonaria aos militantes
socialistas, Mussolini, com seu estilo oratório original, exprimiu a
concepção do partido revolucionário e recebeu uma longa e intensa série
de aplausos de todos os membros do congresso:

\begin{quote}
O Partido é uma organização de soldados, de guerreiros, e não de
filósofos e ideólogos; por conseguinte, como guerreiros, nós não podemos
pertencer a um exército e ao mesmo tempo a um outro, do qual somos
adversários. Mais luz! Eis o grito que soltou Goethe, ao morrer. Ele
lamentou que não mais pudesse ver a luz. De nossa parte, como
socialistas, dizemos: sempre mais luz e, assim, eliminaremos as trevas.
Hoje, nós queremos travar nossos combates nas praças, sob a luz do sol,
olhando uns aos outros no fundo de nossos próprios olhos.\footnote{Mussolini,
  \emph{Opera Omnia}, vol. \textsc{vi}, p.\,169--173.}
\end{quote}

O principal jornal conservador italiano, \emph{Il Giornale d'Italia},
comentou assim o sucesso pessoal de Mussolini: ``As atitudes desse
orador atraente e sombrio, com uma eloquência explosiva e autoritária,
feita de frases refletidas e de incisões programáticas calculadas, foram
muito bem recebidas por toda a assembleia''.\footnote{Mussolini,
  \emph{Opera Omnia}, vol. \textsc{vi}, p.\,477--480.}

Entre 1912 e 1914, a atividade oratória, em conjunto com os artigos
quase diários publicados no \emph{Avanti!}, foi bastante importante para
ampliar a popularidade de Mussolini entre as massas proletárias
socialistas e para aumentar a estima que lhe dedicavam não somente os
revolucionários italianos, mas também os intelectuais democráticos
radicais. Na ocasião de um pronunciamento de Mussolini sobre o ``valor
histórico do socialismo'', em 1914, Giuseppe Prezzolini estava entre o
público ouvinte. Ele era um dos jovens intelectuais da vanguarda
italiana do começo do século \textsc{xx} e fundador da revista \emph{La Voce}
(1909--1913), que o também jovem Mussolini lia assiduamente e com a qual
passaria mais tarde a colaborar, tornando"-se, assim, amigo pessoal de
Prezzolini. Ao recordar a impressão que o orador Mussolini havia causado
em seu público naquelas circunstâncias, Prezzolini escreveu o seguinte:

\begin{quote}
Eu não fui o único a ser profundamente tocado por ele. Ninguém o
interrompeu. Ninguém ousou tomar a palavra. Uma espécie de silêncio
sombrio dominou o público, que estava fascinado por aquela nova
eloquência, seca, tensa, decidida e pungente, composta por algo que era,
antes, os sons vigorosos de uma trombeta e não uns meros gorjeios. Além
disso, ao encerramento de seu discurso, explodiram os aplausos. Toda
aquela ovação vinha das pessoas que jamais o haviam visto e ouvido
antes. Talvez, fossem, inicialmente, avessos a Mussolini. Mas todos
saíram de lá absolutamente convencidos.\footnote{Prezzolini, G.
  \emph{Mussolini oratore}, in Gentile, E. (Org.), \emph{Mussolini e La
  Voce}, Florença: Sansoni, 1976, p.\,216.}
\end{quote}

Um comentário similar foi feito por outro jornal, desta feita, sobre
outro pronunciamento de Mussolini, realizado em Rovigo, no dia 21 de
março de 1914, quando tratou do tema ``Do capitalismo ao socialismo'':
``o orador muito culto e muito forte falou durante uma hora e meia,
mantendo durante todo esse tempo a total atenção de seu público. Ele
desenvolveu magistralmente seu tema e demonstrou sua grande cultura, seu
pensamento denso e sua acuidade crítica fora do comum. Mesmo os
dissidentes julgaram"-no um grande orador''.\footnote{Mussolini,
  \emph{Opera Omnia}, vol. \textsc{vi}, p.\,472.}

\section{O «Duce» carismático}

Em menos de dois anos, o socialista romagnolo havia se tornado um mito
entre os jovens revolucionários. Ademais, mesmo seus adversários mais
implacáveis do Partido socialista, os reformistas, não resistiam ao
fascínio produzido pela eloquência de Mussolini. Em seu comentário sobre
o discurso de Mussolini em Ancona, o reformista Giovanni Zibordi, um de
seus adversários mais obstinados, reconheceu seus méritos e sua decisiva
atuação para o incremento da circulação do jornal socialista e para o
aumento do número de membros do partido e afirmou que Mussolini ``é
muito querido por todos, em função de suas qualidades, que ultrapassam a
tática política: a fé, a retidão, o caráter e o amor pela verdade, com
os quais Mussolini fala na tribuna, mostram que ele está sempre em busca
de uma verdade superior!''\footnote{Prezzolini, \emph{op.\,cit}., p.\,224.}
O próprio Zibordi temia, contudo, a ascendência que Mussolini tinha
sobre a massa proletária, e observava que no Partido socialista, tal
como ele escreveu em 01 de agosto de 1914, o diretor do \emph{Avanti!}
havia instituído ``uma ditadura que se baseia individual e coletivamente
na psicologia ou, melhor, no sentimentalismo''. Zibordi ainda acrescenta
o seguinte:

\begin{quote}
Com o prestígio irresistível de sua combatividade áspera, mas ao mesmo
tempo elevada, que arrasta as multidões, sem ser vulgarmente demagógico,
com certas qualidades pessoais, tal como a de um crente fervoroso e de
um soldado, Mussolini faz as massas engolirem qualquer coisa que ele
queira.\footnote{Prezzolini, \emph{op.\,cit}., p.\,226.}
\end{quote}

Por sua vez, Mussolini estava consciente dessa ascendência que ele
exercia sobre as massas socialistas, mas também dava mostras de sua
preocupação em afastar o risco de se tornar um objeto de idolatria. Na
ocasião de um pronunciamento no Teatro municipal de Cesena, no dia 03 de
maio de 1914, diante da ovação com a qual o público o acolhera, ele
passou a dizer que ``os aplausos eram destinados à sua fé e à obra que
fora realizada pelo ideal comum, porque eu mesmo os considero como a
expressão de uma repugnante idolatria, completamente avessa à minha
própria natureza. O dia em que eu perceber que me tornei um ídolo, eu
mesmo destruirei essa imagem''.\footnote{Mussolini, \emph{Opera Omnia},
  vol. \textsc{vi}, p.\,180--182.} Se crermos no que dizem as crônicas do
\emph{Avanti!}, pensaremos que os pronunciamentos de Mussolini,
realizados seja em congressos seja em comícios, suscitaram sempre e em
todos os lugares um grande entusiasmo. Foi exatamente isso o que teria
ocorrido no dia 04 de junho de 1914, quando da realização de eleições em
Milão, nas quais Mussolini era um dos candidatos ao conselho municipal.
Ele teria sido recebido ``por longas séries de aplausos entusiasmados e
por gritos de `Viva, viva o socialismo!'; e, ao concluir sua
intervenção, saudando a vitória do ``grande exército dos
trabalhadores'', teria sucedido um imenso aplauso, que durou muitos
minutos e que fora acompanhado por gritos de ``Viva Mussolini! Viva o
\emph{Avanti!} Viva o socialismo!''\footnote{Mussolini, \emph{Opera
  Omnia}, vol. \textsc{vi}, p.\,204--205.} Poucos dias mais tarde, em 11 de junho,
no encerramento da violenta agitação da ``semana vermelha'', que
Mussolini havia concebido e saudado como o prelúdio da revolução, em um
comício na Arena de Milão, ``uma imensa e delirante ovação se seguiu ao
término do magnifico discurso de nosso diretor. Foi um grande momento de
comoção e de entusiasmo, que não se conseguiria descrever
aqui''.\footnote{Mussolini, \emph{Opera Omnia}, vol. \textsc{vi}, p.\,215--217.}

``Aplausos vibrantes, que se estenderam por muitos minutos'', acolheriam
também o primeiro discurso de Mussolini contra a guerra, pronunciado em
Milão, no dia 29 de julho de 1914, na Câmara do Povo. Mussolini o
encerrou com os seguintes brados: ``Abaixo a guerra! Basta de guerra!
Viva a solidariedade internacional do proletariado! Viva o
socialismo!''. A reação do público: ovação interminável, chapéus e
lenços se agitavam e ecoavam novamente os gritos de ``Viva! Viva o
socialismo!''\footnote{Mussolini, \emph{Opera Omnia}, vol. \textsc{vi}, p.\,290--293.}

Na condição de diretor do \emph{Avanti!}, após o ataque a Sarajevo, já
prevendo uma guerra em toda a Europa, Mussolini proclamara que o Partido
socialista conservaria uma neutralidade absoluta. Mas, durante o mês de
agosto, quando a Segunda Internacional se afundou, porque quase todos os
partidos socialistas europeus se pronunciaram a favor de seus países,
que haviam entrado em guerra, sua decisão de neutralidade absoluta
começa a vacilar.

Uma vez mais, a partir do dia 04 de agosto, ao repercutir a morte de
Jean Jaurés, que havia sido assassinado por um nacionalista, Mussolini
confiou à sua oratória, assim como também ao seu jornal, a tarefa de
fazer com que o público conhecesse sua posição a respeito dos
acontecimentos. Convidado a falar como o primeiro orador, acolhido com
uma ``longuíssima e frenética salva de palmas, que durou muitos
minutos'', manifestação ``que se repetiu de modo ainda mais vibrante e
intenso, no momento em que Mussolini ascende à tribuna da presidência'',
o diretor do \emph{Avanti!} começa dizendo que teria preferido ser
substituído por outro colega para falar, ``porque não tive tempo de
reorganizar minhas ideias'':

\begin{quote}
Estamos em um período bastante intenso da vida. Em uma época
absolutamente vertiginosa. Os fatos se sucedem com tamanha celeridade e
são tão excitantes que não nos resta tempo para, imersos no cotidiano da
vida política, reordenarmos nossas próprias ideias. Vivemos em um
momento solene da história do mundo. O amanhã é algo desconhecido. É um
tempo perturbador este que nós atravessamos e que deveria ter sido
previsto. Por várias vezes, pensei comigo mesmo: ou a Europa se desarma
simultaneamente ou assistiremos à conflagração de uma guerra europeia.
\end{quote}

Mussolini encerrou seu discurso, dizendo que Jaurés havia sofrido ``a
mais bela das mortes que pode ser desejada por aqueles que, como nós,
oferecem suas vidas ao ideal da emancipação. Este também é o modo como
nós gostaríamos de morrer!''. O jornal comentou que ``o final inspirado
e magnífico do discurso de Mussolini suscitou uma onda incontrolável de
entusiasmo. Todos os presentes o saudaram e as frementes salvas de
palmas se estenderam por muitos minutos''.\footnote{Mussolini,
  \emph{Opera Omnia}, vol. \textsc{vi}, p.\,301--304.}

Todo aquele entusiasmo com a eloquência de Mussolini se repetiria
novamente no dia 09 de setembro daquele mesmo ano. Desde então,
convencido do dever de abondonar a neutralidade aboluta, para se
pronunciar a favor da intervenção italiana na Grande Guerra, em apoio à
França e à Ingleterra e contra o militarismo e o autoritarismo dos
impérios centrais, Mussolini passou a advertir a massa socialista
milaneza, reunida no Teatro do Povo de Milão, de que eles ``continuavam
no bom caminho, no caminho do socialismo. Nós não pretendemos afirmar
com isso que nossas ideias não podem mudar, porque os únicos que não
mudam de ideia são os loucos. Se amanhã estivermos diante de um novo
acontecimento, nos caberá decidir o que faremos! (Calorosíssimos
aplausos)''.\footnote{Mussolini, \emph{Opera Omnia}, vol. \textsc{vi}, p.\,361--363.}

\section{Discursos do traidor}

A influência carismática de Mussolini sofreu um profundo abalo, quando
ele proclamou abertamente em seu jornal o abandono da neutralidade
absoluta. No congresso do partido, realizado em Bolonha, entre os dias
19 e 20 de outubro, durante o qual se impôs ao Partido socialista a
necessidade de aceitar a intervenção armada, Mussolini viveu pela
primeira vez a experiência da perda do carisma. Seu discurso foi
frequentemente interrompido por gritos hostis, sobretudo, no momento em
que definiu a neutralidade ``como uma atitude impassível e como uma
indiferença cínica diante de todos os beligerantes'', dizendo ainda que
o Partido socialista não poderia ``se esconder atrás de frases feitas e,
assim, hipotecar o futuro. Os socialistas devem estar a postos e não
somente como espectadores desse grande e trágico drama''. No decurso
daquela reunião, segundo o relato do jornal \emph{Corriere della Sera},
Mussolini teve ``de se defender permanentemente dos ataques dos adeptos
da neutralidade'', anunciando sua demissão da direção do \emph{Avanti!},
caso a maioria não aceitasse sua ordem de adotar uma neutralidade
condicional.

No dia 21 de outubro de 1914, ao entrar no salão em que ocorria a
assembleia da seção socialista, Mussolini seria uma vez mais
``longamente aclamado e convidado à tribuna da presidência, em meio a
uma entusiasta e frenética ovação. De todos os cantos do recinto,
ouviam"-se os gritos de ``Viva Mussolini! Viva o socialismo!''. Mussolini
agradeceu as manifestações, proclamando o seguinte:

\begin{quote}
A minha fé é imutável. Dentro de alguns dias, organizarei uma
conferência em Milão, na qual explicarei integral e claramente minhas
ideias. O grande público me julgará. Estou tranquilo. Eu cumpri meu
dever, cumpri todo o meu dever, e, por isso, estou certo de que o tempo,
ele que é, de fato, sempre cheio de bom senso, me dará razão.
\end{quote}

O discurso, de acordo novamente com o jornal, foi pronunciado ``com voz
em alto volume, foi aclamado com um entusiasmado conjunto de aplausos''
e foi ainda acompanhado de gritos que lhe prestavam solidariedade e que
solicitavam o cancelamento de sua demissão da função de diretor.
Mussolini respondeu que se opunha ``a qualquer declaração que tivesse
alguma nuance ou algum aspecto de fetichismo pessoal'': ``Sempre fui um
feroz adversário de toda e qualquer forma de fetichismo''. E, assim,
sempre segundo o jornal, concluiu seu pronunciamento com uma ``fala
vibrante'' e com o grito de ``Viva o socialismo! (A assembleia se
levanta e o aplaude freneticamente)''.\footnote{Mussolini, \emph{Opera
  Omnia}, vol. \textsc{vi}, p.\,417--418.} Este seria o último sucesso oratório do
Mussolini socialista.

No início de novembro, quando circulava o rumor de que o antigo diretor
do \emph{Avanti!} estava na iminência de começar a publicar um novo
jornal para apoiar a campanha intervencionista, a derrocada do carisma
de Mussolini no interior do Partido Socialista foi repentina, tal como
havia sido repentina sua conversão da neutralidade à intervenção.
Mussolini se deu conta dessa derrocada no dia 10 de novembro, por
ocasião de uma dramática assembleia na seção socialista de Milão, aquela
mesma que ainda há alguns poucos dias o havia aclamado. Desde sua
entrada naquele ambiente, ele já compreendera que a maioria, partidária
da neutralidade, lhe era hostil, mas, mesmo assim, enfrentou
corajosamente aquela oposição e, mais do que isso, ele a provocou com o
exórdio de seu discurso: ``Ouçam"-me, ao invés de me aplaudirem. Serei
breve e falarei, como de costume, sem reticências e sem insinuações''.
Ele foi imediatamente acusado pela maioria socialista de estar
vinculando aos adeptos da neutralidade absoluta a ``uma camisa de
força'', da qual Mussolini afirmava que ``era preciso se desvencilhar
para ser mais livre''. Ele exortou seus companheiros a fazer o mesmo, a
não serem dogmáticos, porque somente se ``somos cérebros pensantes e não
cérebros ruminantes é que poderemos discutir''. Em seguida, bruscamente,
diante daquele público que ele mesmo havia incitado, desde ao menos dois
anos atrás, ao internacionalismo e ao antipatriotismo, Mussolini
proclamou que era culpa do Partido Socialista o fato de que se ignoravam
os problemas nacionais, quando na verdade era preciso reconhecer que a
nação ``representa uma etapa do progresso humano que ainda não foi
superada. (\ldots{}) O sentimento da nacionalidade existe e não pode ser
negado. O velho antipatriotismo é algo já ultrapassado e os próprios
luminares do socialismo, Marx e Engels, escreveram páginas sobre o
patriotismo que escandalizariam todos vocês!'' Finalmente, depois de ter
explicado que seu intervencionismo continuava fundamentado no socialismo
porque ele era revolucionário e porque ele buscava destruir o
imperialismo e o autoritarismo alemão para acelerar a revolução social
na Europa, ele começava assim a concluir seu pronunciamento:

\begin{quote}
Agora, termino minha fala, até porque eu nem mesmo tinha a intenção de
lhes dirigir a palavra. No entanto, terei outros meios de exprimir meus
pensamentos. (Uma voz irrompe da plateia, dizendo: ``Com o novo
jornal?''). Sim, com o novo jornal. Isso me ajudará a poder falar todos
os dias.
\end{quote}

Naquele mesmo momento a assembleia passa a improvisar uma entusiástica
demonstração de apoio ao \emph{Avanti!}. Por todos os lados, ecoam os
gritos: ``Viva o \emph{Avanti!} Viva o jornal de nosso Partido!'' Bacci,
membro dos quadros dirigentes daquela seção do \textsc{ps}, intervém gritando:
``Viva o Socialismo!'' E seu grito é seguido de uma longa ovação. Uma
vez encerrada aquela comovida aclamação, Mussolini pode continuar a
falar e, de fato, concluir seu discurso: ``O que lhes digo é que o dever
do socialismo é o de sacudir a Itália dos padres, a Itália dos que
quiseram submetê"-la ao Império Austro"-germânico, a Itália dos
monarquistas. E encerro lhes assegurando que, apesar de seus protestos e
de suas vaias, a guerra vai oprimir todos nós''.\footnote{Mussolini,
  \emph{Opera Omnia}, vol. \textsc{vi}, p.\,427--429.}

No dia 15 de novembro era publicado o primeiro número do jornal
mussoliniano \emph{Il Popolo d'Italia}, com um editorial de Mussolini
cujo título era \emph{Audácia!} e que tanto por seu estilo quanto por
seu conteúdo não podia dar outra impressão à massa socialista que não
fosse a de uma renegação e de uma traição ao socialismo
internacionalista, antimilitarista e antibelicista. Esse socialismo era
professado pelo ex"-diretor do \emph{Avanti!} até a poucas semanas
atrás.\footnote{Mussolini, \emph{Opera Omnia}, vol. \textsc{vii}, p.\,7} Mussolini
terminava o texto com uma provocação ao Partido Socialista, lançando um
grito ao seu novo público, que não era mais o proletariado, mas ``os
jovens da Itália'':

\begin{quote}
Não me importo com os perversos e com os idiotas. Os primeiros
permanecerão na lama de sua maldade e os últimos em sua nulidade
intelectual. Eu avanço! Cabe a vocês, jovens da Itália, seguir avançando
pelo mesmo caminho. Cabe a vocês fazê"-lo, jovens das fábricas e
oficinas, jovens das universidades, jovens seja por sua idade seja por
seu espírito, jovens que pertencem à geração cujo destino está
comprometido com o dever de fazer a história. É a vocês que eu lanço meu
grito de esperança, seguro que estou de que haverá entre todos vocês uma
ampla ressonância desse grito e uma grande vontade de fazê"-lo ecoar.

Grito uma palavra que eu jamais teria pronunciado em tempos ordinários e
que hoje faço ecoar com voz firme e decidida, sem fingimento e com
convicção; é uma palavra terrível e fascinante; é a palavra:
\emph{guerra!}
\end{quote}

Nove dias mais tarde, a seção socialista de Milão, depois de uma
tumultuada assembleia, tomou a decisão, decretada pelos principais
dirigentes do Partido Socialista, os mesmos com os quais Mussolini havia
conquistado em 1912 a direção do partido, de expulsá"-lo por
``indignidade política e moral'', com a justificativa de que ele teria
recebido dinheiro dos empresários da indústria militar para fundar seu
jornal. Mussolini estava presente naquela ocasião e deu novamente provas
de sua audaciosa coragem, diante de um auditório que lhe era ferozmente
hostil, que não mais se submetia ao seu charme carismático e que o
acusava abertamente de traição. É isso, ao menos, o que foi relatado em
\emph{Il Popolo d'Italia}. Seu breve pronunciamento, menos que uma
defesa, foi um ataque constituído de recursos e expressões que deveriam
produzir grande eficácia oratória, mas que, de fato, não suscitaram esse
efeito sobre aquela maioria que lhe era tão hostil:

\begin{quote}
Meu destino está decidido e vocês parecem querer selá"-lo com um ato
solene. (Vozes diziam: ``Mais alto! Fale mais alto!''. Diante daquela
insistência, o orador não pode se conter e reagiu batendo nervosamente
um copo sobre a mesa). Se ele está decidido, se vocês pensam que sou
indigno de continuar militando entre vocês (``Sim! Sim!'', gritam em
coro os mais exaltados), me expulsem logo, mas tenho o direito de exigir
que me apresentem uma real acusação. (\ldots{}) Vocês sairão daqui com uma
consciência pesada (Vozes em coro e alto volume gritaram: ``Não!
Não!''). Vocês acreditam que se livrarão de mim, mas eu lhes digo que
vocês estão enganados. Se hoje vocês me detestam é porque ainda me amam,
é porque\ldots{} (Os aplausos e as vaias interrompem novamente o orador).

Mas, vocês não se livrarão de mim. Doze anos da minha vida inteiramente
dedicados ao Partido são ou deveriam ser uma garantia suficiente de
minha fé socialista. O socialismo é algo que está enraizado em meu
sangue. O que divide atualmente não é algo trivial; é uma grande questão
que divide todo o socialismo. (\ldots{}) Não redimirei ninguém, não terei
nenhuma piedade por qualquer um dos reticentes, por qualquer um dos
hipócritas, por qualquer um dos vis! E vocês verão que eu estarei sempre
ao seu lado. Vocês não deveriam acreditar que a burguesia é entusiasta
de nosso intervencionismo.

Não pensem que rasgando minha carteirinha vocês eliminarão minha fé
socialista. Vocês me impedirão somente de trabalhar pela causa do
socialismo e da revolução! (Aplausos calorosos em reação àquelas
palavras pronunciadas com grande energia e com uma entonação que
transmitia a mais profunda convicção). Mussolini desce da tribuna e abre
passagem pela imensa sala, enquanto ao seu redor se forma uma massa
feroz de detratores que investe sobre ele, uma massa afligida pelas
poucas e incisivas palavras daquele que teve a força de enfrentar sua
explosão manifesta de ódio, daquele que teve a coragem para efetuar um
novo ato de fé. Ato de fé que se tornava ainda mais solene e ainda mais
belo, à medida justamente que ele era cada vez mais
contestado.\footnote{Mussolini, Opera Omnia, \textsc{vii}, p.\,39--41.}
\end{quote}

A derrocada do carisma socialista de Mussolini demonstra que o carisma é
eficaz somente se corresponde às convicções e valores próprios do
público ao qual se dirige o orador e que o apoia com sua confiança. Não
há tal eficácia quando o carisma contrasta com essas convicções e esses
valores. O carisma socialista de Mussolini estava ligado à confiança em
sua integridade revolucionária, na sinceridade e na fé de que ele dava
provas. No momento em que essa sua imagem foi corrompida por sua
conversão ao intervencionismo, o carisma mussoliniano que reinava sobre
as massas foi rapidamente dissolvido.

\section{Em busca do carisma perdido}

Durante sete anos, de novembro de 1914 a novembro de 1921, Mussolini
esteve presente no cenário nacional com \emph{Il Popolo d'Italia}, mas
sem desempenhar nenhum protagonismo. Ele cumpriu um papel ativo durante
os meses da campanha intervencionista, até 24 de maio de 1915, porém,
nos dois primeiros anos depois do fim da Grande Guerra, sua oratória não
produzia nenhum fascínio sobre as massas. Seu público se limitava aos
intervencionistas, aos combatentes e aos veteranos de guerra, mas, mesmo
entre eles, seu desempenho retórico era quase desprovido de carisma. De
1918 a 1920, o líder carismático do nacionalismo revolucionário gerado
pela experiência da Grande Guerra foi Gabriele D'Annunzio, por meio de
sua oratória e da \emph{Impresa di Fiume}.\footnote{Nota do tradutor: A
  \emph{Impresa di Fiume} consistiu na rebelião de alguns setores do
  Exército Real, com o propósito de ocupar a cidade adriática de Fiume,
  que era então disputada pelos Reinos da Itália e da Iugoslávia.
  Organizada por uma frente política predominantemente nacionalista e
  liderada pelo poeta Gabriele D'Annunzio, a expedição chegou àquela
  cidade em 12 de setembro de 1919, proclamando sua anexação ao Reino da
  Itália.}

O conjunto de partidários arregimentados por Mussolini com o movimento
\emph{Fasci di Combattimento}, fundado em 23 de março de 1919, também
foi escasso, mesmo quando o fascismo já se tornara um movimento de
massa, valendo"-se cada vez mais de contingentes armados nas províncias
da fronteira oriental, entre a Planície Padana e a Toscana. Nesse
contexto, Mussolini teve que lutar para afirmar seu papel como líder,
ante a ascensão de novas lideranças nos esquadrões de massa. Isso porque
o diretor de \emph{Il} \emph{Popolo d'Italia} não tinha tido um papel
pessoal particularmente relevante na transformação dos \emph{Fasci} num
movimento de massa. Essa transformação acontecera mediante a iniciativa
autônoma de líderes fascistas locais na organização das equipes armadas,
que, entre o final de 1920 e a primavera de 1921, destruíram
violentamente grande parte das organizações políticas e sindicais do
proletariado.

No verão de 1921, Mussolini tentou afirmar sua autoridade sobre o novo
esquadrão fascista, lançando a ideia de transformar os movimentos do
fascismo num partido político e, assim, desmantelando aqueles esquadrões
e assinando um pacto de pacificação com o Partido Socialista. Os líderes
dos grupos armados, opostos à proposta de paz, se rebelaram e os membros
dos esquadrões acusaram Mussolini de trair o fascismo, tal como ele já
fizera com o socialismo.\footnote{Cf. Gentile, E. \emph{Storia del
  partito fascista}. Roma/Bari: Laterza, cap.\,\textsc{iv}.} Um dos líderes da
revolta antimussoliniana, Dino Grandi, negou a Mussolini, como escreveu
no jornal fascista de Bolonha \emph{L'Assalto}, em 6 de agosto de 1921,
``o direito exclusivo de dispor, com a autoridade de um mentor e com um
\emph{pater familias} da memória romana, de \emph{nosso} movimento, ao
qual todos nós demos nossa alma, nossa juventude e nossa vida''.\footnote{Grandi,
  D. \emph{Pensieri di Peretola}, in ``L'Assalto'', 6 de agosto de 1921.}
No mesmo jornal, no dia 13 de agosto daquele mesmo ano, com um título de
página inteira, no qual se dizia \textsc{o fascismo não é homem, é idéia},
Grandi acusou Mussolini de querer ser reconhecido como líder de um
movimento que ele, na verdade, não conhecia:

\begin{quote}
Se Mussolini tivesse mais frequentemente descido de seu pedestal e
tivesse estado entre nós, entre as nossas massas, para nos ajudar a
coordenar, não por meio de seu espírito, mas de sua presença real, as
múltiplas e variadas tendências que o nosso movimento estava assumindo
no seu rápido e tumultuoso desenvolvimento, parte desse nosso movimento
não teria sido deixada para trás e hoje ele não sofreria com julgamentos
que não lhe atribuem a devida honra. Esses julgamentos, além de
desvalorizar todos nossos pensamentos fundamentais, distorcem as
verdades mais elementares.\footnote{Grandi, D. \textsc{Il Fascismo non è
  un uomo, è un'idea}, in ``L'Assalto'', 13 de agosto de 1921.}
\end{quote}

Diante da revolta da massa esquadrista, Mussolini reagiu com a renúncia
ao comitê executivo do movimento fascista, declarando o seguinte: ``Do
meu ponto de vista, a situação é de uma óbvia simplicidade: se o
fascismo não me seguir, ninguém será capaz de me obrigar a seguir o
fascismo''.\footnote{Mussolini, \emph{Fatto compiuto}, in ``Il Popolo
  d'Italia'', 3 de agosto de 1921.} Ele ainda respondeu a Grandi,
refutando a acusação de que agia como se fosse proprietário do fascismo:

\begin{quote}
Não preciso rebater a acusação vazia de que eu seria uma espécie de dono
do fascismo italiano. Sou ``\emph{Duce}'', como se costuma dizer. Deixei
que usassem essa palavra, porque, se é bem verdade que ela me
desagradava, já que não gosto de palavras e posturas solenes, também é
verdade que outras pessoas gostavam de empregá"-la. Mas, se eu sou um
``\emph{Duce}'', sou um \emph{Duce} de um constitucionalismo
absolutamente escrupuloso. Eu nunca impus nada a ninguém. Eu sempre
aceitei discutir com todos, até mesmo com aqueles que lidam com a
política de modo desconcertante, até mesmo com aqueles que estão
infectados pelas doenças malignas que se disseminam cronicamente entre
os antigos partidos. A política é uma arte. Portanto, é um contínuo
processo de aprendizagem. Ela é constituída ainda de muita intuição e só
quem possui esse belo dom é que tem condições suficientes para
praticá"-la. Apesar de possuir esse dom, sempre me senti um amigo entre
amigos e jamais um patrão diante de subalternos. Para mim, o fascismo
não é um fim em si mesmo. Sempre foi um meio para restaurar o equilíbrio
nacional, para restaurar certos valores espirituais negligenciados, para
estabelecer as bases para uma reconstrução nacional, tal como a que
parte da premissa indestrutível da intervenção, como Vittorio Veneto a
concebeu. Muito disso foi alcançado. Por isso, o fascismo pode se
dividir, se fragmentar, se enfraquecer e definhar. Se for necessário
vibrar martelos poderosos para acelerar sua ruína, vou me adaptar a essa
necessidade ingrata. Porque o fascismo que não é mais liberação, mas
tirania; porque o fascismo que não mais protege a nação, mas defende
interesses privados e as castas mais sombrias, surdas e inescrupulosas
que existem na Itália, porque o fascismo que assume essa aparência
continuará sendo o fascismo, mas não é mais aquele pelo qual, durante um
amargo período, uns poucos de nós enfrentamos a cólera e o chumbo de
multidões. Esse não será mais o fascismo que foi concebido por mim, em
um dos momentos mais sombrios da recente história italiana. O fascismo
pode viver sem mim? Claro, mas eu também posso viver sem o fascismo. Há
espaço para todos na Itália: até mesmo para trinta fascismos, o que
significa, então, que há possibilidade de não existir nenhum
fascismo.\footnote{Mussolini, \emph{La culla e il resto}, in ``Il Popolo
  d'Italia'', 7 de agosto de 1921.}
\end{quote}

A situação em que Mussolini se encontrava no verão de 1921 era
dramaticamente semelhante à do outono de 1914. Mas, desta vez, ele não
se arriscou a ficar isolado novamente e executou uma manobra para
estabelecer um acordo com os líderes dos esquadrões, em detrimento de um
pacto de apaziguamento, e para acentuar a transformação do movimento
fascista em uma milícia. Mussolini optou por preservar a organização sob
a forma de esquadra como estrutura política e paramilitar do novo
partido fascista nacional, estabelecido em 7 de novembro de 1921 no
Congresso nacional dos \emph{Fasci} \emph{di combattimento}.

Em seu discurso, Mussolini assumiu a tarefa de definir as
características essenciais do novo fascismo antidemocrático, estatista e
imperialista. Juntamente com as manobras nos bastidores para se
reconciliar com os líderes rebeldes, o discurso foi decisivo para
conquistar uma ascendência carismática sobre a massa dos esquadristas,
que já tinham cantado em manifestações de revolta antimussoliniana:
``Uma vez traidor, sempre traidor''. Um dos maiores jornalistas
italianos daquela época, Ugo Ojetti, que estava presente naquele
congresso, traçou um quadro bastante eficaz do Mussolini orador e de seu
original estilo verbal e corporal:

\begin{quote}
Eu nunca tinha ouvido um discurso de Mussolini. Ele é um exímio orador,
mestre de si mesmo, que se posiciona sempre de frente para o seu
público. Cada um de seus períodos, cada linha de seu discurso é
proferida com a expressão facial que lhe convém. Emprega os gestos com
parcimônia. Muitas vezes, ele só gesticula com a mão direita, mantendo a
mão esquerda no bolso e o braço esquerdo apertado junto à lateral de seu
corpo. Outras vezes, principalmente, quando se trata do final do seu
pronunciamento, ele coloca as duas mãos no bolso: este é o momento
estatuário, em que resume tudo o que disse. Nos raros momentos em que
essa cristalizada figura de orador se distende e se liberta, seus braços
chegam a girar na altura de sua cabeça. Seus dez dedos se agitam como se
estivessem procurando cordas vibrantes no ar e suas palavras jorram
sobre a multidão, como se fosse uma catarata. Um momento de suspensão se
produz\ldots{} e o Mussolini orador retorna imediatamente à cena, franzindo a
testa e se certificando, com dois de seus dedos, de que o nó da gravata
elegante não se deslocou da devida posição. Esses momentos de
gesticulação tumultuada não se dão nas passagens mais comoventes, mas
ocorrem, principalmente, ao cabo de argumentações mais lógicas, como se
fossem uma forma de representar para a massa uma série de outros tópicos
que ele menciona e enumera, como se, por uma questão de brevidade, seus
gestos imitassem o grande número das coisas de que ele fala. Mussolini
emprega esse recurso de modo extremamente eficaz.

Além dessa gesticulação muitíssimo eficiente de um excelente orador,
Mussolini tem outras três qualidades para conquistar o público. A
primeira é a de que sempre produz períodos completos. Assim, ele nunca
deixa uma frase solta, sem o complemento de uma oração principal. A
segunda consiste na abundância de suas definições morais, que são
pitorescas e incisivas e que, por isso, se inscrevem facilmente na
memória de seu público. A terceira qualidade é a produção de afirmações
fluídas, peremptórias, mas que têm, ao mesmo tempo, algo de repousantes,
que produzem na maioria das pessoas uma sólida confiança. Nessas
afirmações, não há neblina, não há cinza, pois o mundo inteiro é ali
reduzido a preto e branco. Não há margem para dúvidas. E as que poderiam
surgir entre alguns da multidão permanecem guardadas para eles mesmos.

Seu discurso está chegando ao fim. Tomado pela fadiga, seu rosto
expressa seu cansaço e parece se tornar mais ossudo e rígido. E assim
que ele termina e vai para a escada para descer da tribuna, o deputado
Capanni o agarra pela cintura e levanta acima da multidão, com o gesto
típico do padre que levanta as imagens e objetos sagrados.

Ao meu lado, dois jovens de camisa preta têm lágrimas nos olhos. Se
Mussolini visse essas lágrimas, ele ficaria ainda mais orgulhoso do que
já estava com todos os aplausos da multidão.\footnote{Citado por De
  Felice, R.; Goglia, L. \emph{Mussolini. Il mito}. Roma: Laterza, 1983,
  p.\,108--109.}
\end{quote}

No encerramento de seu discurso, Mussolini tratou de sua própria relação
pessoal com o fascismo, admitindo que havia cometido erros por seu ``mau
temperamento'': ``Na nova organização, eu quero desaparecer, porque vocês
têm de se afastar desse meu mal e têm de andar por si mesmos. Só assim,
enfrentando os problemas e assumindo as responsabilidades, é que
venceremos as grandes batalhas''.\footnote{Mussolini, \emph{Opera Omnia},
  vol. \textsc{xvii}, p.\,222.} Após seu pronunciamento, há um dissimulado abraço
da reconciliação com Grandi, chefe da revolta antimussoliniana.
Mussolini é, então, reconhecido e aclamado como chefe do novo partido,
mesmo que o congresso tivesse oficialmente elegido Michele Bianchi, um
fiel amigo de Mussolini, desde os tempos do intervencionismo, como
secretário geral do \textsc{pnf}. ``Os fascistas --- publicou o \emph{Avanti!}, em
sua edição do dia 9 de novembro daquele ano --- ouviram seu líder por
mais de três horas. Mussolini disse coisas absolutamente novas para a
maioria deles. Então houve quase uma espécie de loucura coletiva, houve
um enorme entusiasmo, sob a forma de gritos, músicas e aplausos sem fim.
O \emph{Duce} é beijado, abraçado e coroado com flores''.\footnote{\emph{Avanti}!,
  9 de novembro de 1921.}

\section{O «Duce» da nova Itália}

Entre as coisas ``absolutamente novas'', ou quase novas, ditas por
Mussolini em seu discurso estava a exaltação da vocação imperialista do
fascismo:

\begin{quote}
Não pode haver grandeza nacional, se a própria nação não for movida por
uma ideia de império. Nós falamos de ``império'' do ponto de vista
espiritual e econômico, de uma necessidade instintiva de todos os
indivíduos, porque todo indivíduo é imperialista em certo sentido,
quando tenta abrir caminho pela vida e quando um povo consegue se livrar
dos golpes sofridos em sua carne viva. Isso porque, quando o povo se
fecha em sua própria morada, quando só enxerga a si mesmo, quando se
embrutece ensimesmado, então, esse é o momento em que esse povo se
aproxima fatalmente da decadência e da morte.
\end{quote}

No mito fascista do império, Mussolini incluía a apologia da Igreja
Romana, a proclamação do estado nacional como uma unidade compacta e
governada pela vontade de poder e a defesa da raça: o fascismo tinha
antes de tudo de ``resolver o problema da raça'' porque ``se a Itália
estivesse cheia de doentes e loucos, a grandeza seria algo impossível de
ser alcançado. Além disso, os fascistas devem se preocupar com a saúde
da raça porque a raça é o material com o qual pretendemos construir
nossa história''. Finalmente, Mussolini enfrentou o problema das massas,
que ele tinha ostensivamente desprezado, desde que foi expulso do
Partido socialista, durante o período do fascismo libertário:

\begin{quote}
Dizem por aí: precisamos conquistar as massas. Há quem diga também: a
história é feita por heróis; outros dizem que é feita pelas massas. A
verdade está no meio termo. O que a massa faria se não tivesse seu
próprio líder conectado com o espírito do povo e o que faria o poeta se
não tivesse o material para forjar sua poesia? Nós não somos
anti"-proletários, mas não queremos criar um fetichismo da massa, como se
ela fosse nossa suprema majestade. Queremos servi"-la, educá"-la, mas,
quando ela fizer algo errado, nós teremos de puni"-la. Devemos prometer o
que sabemos ser matematicamente necessário para mantê"-la potente.
Queremos elevar seu nível intelectual e moral, porque queremos incluí"-la
na história da nação. Porque não pode haver desenvolvimento da economia
nacional com um proletariado indisciplinado, malárico e temerário.
Dizemos às massas que, quando os interesses da nação estão em jogo,
todos os egoísmos, sejam eles do proletariado sejam eles da burguesia,
devem permanecer em silêncio.
\end{quote}

No curso do ano de 1922, especialmente após o verão, o Partido fascista,
depois de ter derrotado todos os partidos da oposição, tornou"-se a
agremiação política mais numerosa e mais forte da Itália, sobretudo por
sua organização paramilitar, que conseguia impor um domínio ditatorial
em muitas províncias do país. No norte e centro da Itália, o Partido
fascista desafiou abertamente o governo com a ocupação de cidades
inteiras e com a imposição forçada de demissão de quadros das
administrações locais que não fossem simpáticos ao fascismo. O desacato
mais descarado ao Estado de direito foi a publicação do regulamento para
a organização da ``Milícia para a segurança nacional''. Sua publicação em
``\emph{Il Popolo d'Italia}'', no dia 12 de outubro, estabeleceu uma nova
ordem que unificava os grupos armados fascistas, dando forma
institucional à identificação do Partido fascista com os esquadrões
fascistas.

Naquela época, os discursos dirigidos a dezenas de milhares de fascistas
reunidos nas cidades onde o fascismo grassava e pronunciados no auge de
eventos espetaculares, tais como desfiles cívicos, procissões, rituais
de juramento dos esquadrões, cerimônias de entrega de bandeiras e
homenagens a fascistas mortos, produziram grandes efeitos e
proporcionaram a Mussolini ocasiões de consolidar e estender seu carisma
à massa dos fascistas. Essa massa, que o aclamava em todas as
circunstâncias e por todo o país com um exaltado entusiasmo, era sempre
advertida para que estivesse organizada com estrita disciplina e para
que estivesse às ordens dos líderes e, evidentemente, do \emph{Duce}.

Italo Balbo, então chefe de 26 anos do fascismo da cidade de Ferrara e
um dos líderes da milícia fascista, descreveu assim, em seu diário, o
discurso que Mussolini proferiu em Udine, no dia 20 de setembro de 1922,
no Teatro Social:

\begin{quote}
Hoje ele falou mais seco e decisivo do que nunca. Chegou a hora de falar
claramente para as massas. Mussolini me disse que devemos queimar
etapas. Seu discurso esclareceu a posição dos fascistas em relação à
monarquia. {[}\ldots{}{]} Antes de começar a falar, ele passou sob um arco de
centenas de bandeirolas. Primeiro, houve certos rumores. Depois, um
silêncio religioso. {[}\ldots{}{]} Sua voz no princípio é baixa, como se
fosse uma emboscada para os ouvintes. Logo em seguida, o discurso se
torna cortante, incisivo. Nas expressões de seu rosto, passam ondas de
pensamento fulminante. Seus gestos são hipertônicos. Às vezes, ele se
movimenta como se estivesse buscando uma palavra no interior do coração.
Logo depois, o punho se eleva de sua postura ereta e se projeta sobre a
multidão. O público está cativado, dominado, fascinado. Então, brotam as
palavras solenes, os compromissos definitivos, os programas de guerra e
as frases que já passam a ter uma importância histórica.\footnote{I.Balbo,
  \emph{Diario 1922}, Milano 1932, p.\,153--154.}
\end{quote}

Os discursos de Mussolini foram o principal instrumento para expor, com
grande ressonância na imprensa e entre a população, a ideologia política
do fascismo, especialmente a concepção do Estado fascista, tal como
Mussolini a estava desenvolvendo, enquanto o Partido fascista se
preparava para a conquista do poder. Por essa razão, esses discursos
tornaram"-se os textos doutrinários fundamentais do Partido fascista.

\section{Discursos da véspera}

Essa função foi, principalmente, desempenhada pelos ``discursos da
véspera''. Foi assim que ficaram conhecidos os pronunciamentos proferidos
por Mussolini nas semanas que antecederam a ``marcha sobre Roma'',
realizados no dia 20 de setembro em Udine, no dia 24 daquele mesmo mês
em Cremona, no dia 4 de outubro em Milão e no dia 24 do mesmo mês em
Nápoles. Nesses discursos, Mussolini proclamou mais abertamente do que
nunca, diante de grandes multidões, que o fascismo se identificava com a
nova Itália que tinha sido gerada pela Grande Guerra. Declarou também do
mesmo modo que apenas os fascistas eram a encarnação e os intérpretes da
nação e que, portanto, somente eles tinham o direito de tomar o poder
para derrubar o Estado de direito e para construir um Estado fascista.

Nestes discursos, a retórica mussoliniana da ``história monumental'' foi
expressa nas formas mais originais e eficazes: o \emph{Duce} situava os
eventos em que o fascismo havia sido um protagonista na história da
Itália contemporânea como o ápice da revolução nacional italiana
iniciada pelo \emph{Risorgimento}. Essa revolução, que fora traída pelo
estado liberal, teria sido retomada com grande vigor pelos
intervencionistas e era, agora, continuada pelos fascistas em seu maior
objetivo: a conquista de Roma e o estabelecimento de um Estado nacional
forte, unido e disciplinado. Esses discursos de Mussolini protestavam,
enfim, o comando do fascismo na conquista imperial e na expressão da
grandeza da Itália.

No primeiro dos ``discursos da véspera'', o mais importante para a
construção desse novo conteúdo fascista e de suas formas oratórias,
pronunciado em Udine no aniversário da conquista de Roma pelo exército
italiano em 1870, Mussolini fez sua estreia, exibindo o que mais tarde
se tornaria um de seus mais habituais recursos retóricos do período
fascista, a saber, o hábito de manifestar seu descontentamento pela fala
pública e pela fala pública eloquente, em particular, e sua predileção
pelos fatos e pelas ações. Assim, o \emph{Duce} se valia de seus
discursos para definir o que eram ``fatos e ações'' e para distinguir
esses seus próprios discursos das formas tradicionais de eloquência,
estabelecendo o que deveria ser o estilo de oratória fascista:

\begin{quote}
Com este discurso que pretendo fazer diante de vocês, abro uma exceção à
regra que impus a mim mesmo: isto é, limitar as manifestações de minha
eloquência ao mínimo possível. Ah, se fosse possível estrangulá"-la, como
aconselhou um poeta, se fosse possível eliminar essa eloquência
verborrágica, prolixa, inconclusa e demagógica, que tem nos desviado de
nosso caminho por tanto tempo! Mas, estou seguro de que vocês não
esperam de mim um discurso que não seja primorosamente fascista, um
discurso esquálido e áspero, duro e franco.
\end{quote}

Com a pretensão de conquistar Roma, Mussolini exaltou o caráter romano
do fascismo, colocando"-o na esteira de Mazzini e Garibaldi, que ansiavam
pela conquista da capital para torná"-la a sede de uma nova Itália, unida
e forte. Ao mesmo tempo, ele enaltecia o catolicismo como uma
manifestação espiritual e religiosa da tradição imperial romana:

\begin{quote}
Por isso, elevemos nossos pensamentos a Roma com um espírito puro e
despojado, elevemos nossos pensamentos a esta que é uma das poucas
cidades do mundo que tem uma verdadeira alma. Porque foi em Roma, entre
aquelas sete colinas cheias de história, que se ergueu um dos maiores
prodígios espirituais de que a história se recorda. Foi em Roma que
transformamos uma religião oriental em uma religião universal. Esta é
uma outra forma daquele império que as legiões consulares de Roma haviam
estendido até os confins da terra. Façamos de Roma a cidade do nosso
próprio espírito, uma cidade purificada, desinfetada de todos os
elementos que corrompem e degradam seus cidadãos. Tornemos Roma o
coração pulsante, o espírito vivo da Itália imperial com que
sonhamos.\footnote{Mussolini, \emph{Opera Omnia}, vol. \textsc{xviii}, p.\,411--421.}
\end{quote}

Mussolini continuaria afirmando que os fascistas eram dignos de assumir
o poder, porque eram os intérpretes mais autênticos da vontade nacional.
Por essa razão, tinham de ser um modelo para a disciplina coletiva que
queriam impor a todos os italianos, exercendo a violência, se
necessário:

\begin{quote}
Precisamos impor a mais estrita disciplina a nós mesmos, porque senão
não teremos o direito de impô"-la à nação {[}\ldots{}{]} A disciplina deve ser
aceita. Quando não aceita, deve ser imposta. Rejeitamos o dogma
democrático de que devemos prosseguir eternamente de discursos em
discursos, de debates em debates, cada vez mais liberais. Em determinado
momento, a disciplina deve se expressar sob a forma de um ato imperial
de força. Somente obedecendo, somente tendo o orgulho humilde, mas
sagrado, de obedecer, é que se conquista o direito de comandar. Quando a
disciplina está presente em nosso espírito, podemos impor a disciplina a
todos os demais. Os fascistas de toda a Itália devem estar bem
conscientes disso. Não devem interpretar a disciplina como o
procedimento de uma ordem administrativa ou como o medo de líderes que
temessem a revolta de seu rebanho. A disciplina não é isso, porque nós
não somos líderes como os outros. Dada sua força, nossas massas jamais
poderiam ser chamadas de rebanho. Somos uma milícia, mas precisamente
porque nós mesmos nos demos esta constituição especial, devemos fazer da
disciplina a pedra angular suprema da nossa vida e da nossa ação.
\end{quote}

Chamando a massa fascista de ``milícia'' e as massas dos partidos opostos
de ``rebanho'', o \emph{Duce} fascista expressou explicitamente sua
atitude em relação às massas, que se tornaram numerosas no partido e nos
sindicatos fascistas, sobretudo após os esquadrões terem destruído as
organizações socialistas proletárias:

\begin{quote}
Vocês sabem que eu não gosto desta nova divindade: a massa. Ela é uma
invenção da democracia e do socialismo: porque são muitos, devem estar
certos. Nada disso. O oposto é o que mais geralmente ocorre, ou seja, o
número é contrário à razão. De qualquer forma, a história mostra que
sempre foram as minorias, pequenas, a princípio, as responsáveis pela
produção de profundas convulsões nas sociedades humanas. Nós não
adoramos a massa, mesmo que ela tenha as mãos calejadas. Em vez de
adorá"-la, nós examinamos os fatos e as ideias sociais e trazemos os
novos elementos e novas concepções ao povo italiano. Nós não poderíamos
rejeitar essas massas. Elas vieram e virão até nós. Nós deveríamos
recebê"-las aos pontapés? Deveríamos nos perguntar se elas são sinceras
ou se são dissimuladas? Se vêm ao nosso encontro movidas pela convicção
ou pelo medo? Essas questões são praticamente inúteis, porque as massas
ainda não encontraram o caminho para penetrar nas profundezas do
espírito. {[}\ldots{}{]} Por outro lado, a burguesia deve perceber que o povo
também faz parte da nação. Deve reconhecer a importância dessa massa de
trabalho e não pode pensar que se alcançará a grandeza da nação, se esta
massa de trabalho estiver inquieta e ociosa. A tarefa do fascismo é
incorporá"-la de modo orgânico à nação. Porque a nação precisa da massa
assim como o artista precisa da matéria"-prima para forjar suas obras de
arte.

Somente com uma massa que esteja integrada à vida e à história da nação
é que poderemos nos dedicar à construção de uma política externa.
\end{quote}

Mussolini concluiu o primeiro ``discurso da véspera'' declarando: ``Nosso
programa é simples: queremos governar a Itália. Agora as coisas estão
muito claras: é preciso demolir toda a superestrutura
socialista"-democrática''. Ainda no encerramento dessa sua fala, para
definir o papel do Estado, ele se vale dessa formulação clara e simples:
``O Estado não representa um partido, o Estado representa a comunidade
nacional, inclui todos, protege a todos, se sobrepõe a todos e se opõe a
qualquer um que não respeite sua imprescritível autoridade''.

Já no dia 4 de outubro, no terceiro ``discurso de véspera'', pronunciado
em Milão, Mussolini precisou com uma franqueza brutal o que entendia por
``autoridade'', tal como ela seria aplicada no futuro Estado fascista:

\begin{quote}
Não temos grandes obstáculos a superar, porque a nação está conosco. A
nação se sente representada por nós. Certamente não podemos plantar a
árvore da liberdade nas praças públicas, porque não podemos dar plena
liberdade àqueles que se aproveitarão dela para nos matar. Esta é a
loucura do estado liberal: ele dá liberdade a todos, mesmo àqueles que a
usarão para derrubá"-lo. Nós não vamos dar essa liberdade. Nem mesmo se o
pedido por essa liberdade vier sob a forma daquela já desbotada ``Carta
magna'' dos princípios imortais. {[}\ldots{}{]}

Nós dividimos os italianos em três categorias: os italianos
``indiferentes'', que permanecerão em suas casas à espera das mudanças; os
``simpatizantes'', que podem circular; e finalmente os ``inimigos'' dos
italianos. Estes não vagarão livremente por aí.\footnote{Mussolini,
  \emph{Opera Omnia}, vol. \textsc{xviii}, p.\,433--440.}
\end{quote}

\section{O «Duce» da revolução}

O que foi anunciado por Mussolini no dia 4 de outubro seria
constantemente reafirmado pelo \emph{Duce} fascista, desde que se tornou
chefe do novo governo, após a mobilização revolucionária que ficou na
história como ``a marcha sobre Roma''. O advento de Mussolini, um político
de origem popular, o mais jovem primeiro"-ministro da história da Itália
unificada, com um passado político bastante agitado, sem qualquer
experiência de governo, um deputado cujo mandato se estendeu por pouco
mais de um ano, um chefe de gangues armadas que governaram toda a
Itália, que se proclamavam como parte da nação e que tratavam todos os
partidos oponentes como inimigos internos da Itália, foi um evento que
imediatamente atingiu o imaginário popular, despertando grande
curiosidade pela própria personalidade do \emph{Duce}.

No dia 16 de novembro de 1922, quando Mussolini apresentou o novo
governo à Câmara, o número de observadores ali reunidos era ``o maior
que já se vira, ao menos nos últimos trinta anos, em que estou na
imprensa política'', escreveu um cronista.\footnote{\emph{Nell'aula
  rigurgitante}, in ``La Stampa'', 17 de novembro de 1922.} A estreia
do primeiro"-ministro se deu com modos, linguagem e tons sem precedentes
nos últimos setenta anos do parlamento italiano. Desde sua primeira
apresentação, Mussolini não se dirigiu à Câmara com o habitual ``Honrados
colegas'', mas com um tratamento bem mais direto, sob a forma de um
``Senhores'', seguido de um discurso que nunca tinha sido feito por um
presidente daquele Conselho. Seu pronunciamento era marcado por uma
linguagem hostil e desdenhosa e por um conteúdo repleto de repetidos
insultos e ameaças à própria Câmara e de um desprezo manifesto por todo
Parlamento.\footnote{\emph{Atti} \emph{parlamentari della Camera dei
  Deputati, Legislatura \textsc{xxvi}, 1a Sessione 1921--1923. Discussioni},
  \emph{Tornata del 16 novembre 1922}, Tipografia da Câmara dos
  Deputados, Roma, 1923, p.\,8390--8394.} Este seu primeiro comunicado
consistiu na reivindicação de sua revolucionária ascensão ao poder e os
contornos dessa declaração eram os de uma aberta ameaça:

\begin{quote}
Afirmo que a revolução tem seus direitos. Mas gostaria de acrescentar,
mesmo porque todos aqui já sabem disso, que estou aqui para defender o
máximo potencial da revolução dos ``camisas negras'', para demonstrar que
essa revolução é uma força fundamental do desenvolvimento, do progresso
e do equilíbrio na história da nação.

Recusei"-me a acumular ganhos, e eu podia ter ganhado muito. Ao invés
disso, eu me impus limites. Eu disse a mim mesmo que a sabedoria
significa não desistir da luta depois que se consegue a vitória. Com
trezentos mil jovens armados, determinados a fazer tudo e quase
misticamente prontos a acatar minhas ordens, eu poderia ter punido todos
aqueles que difamaram e tentaram manchar o fascismo. (Manifestações de
aprovação e aplausos vindos do lado direito da Câmara).

Eu poderia fazer deste salão surdo e cinzento uma sede de nossos
estandartes \ldots{} (Mais aplausos à direita da Câmera. Alguns ruídos e
comentários. Modigliani: ``Viva o Parlamento! Viva o Parlamento!''. Ruídos
; Barulhos e intervenções vindos da direita. Aplausos também vindos da
extrema esquerda).

Eu poderia fechar o Parlamento e estabelecer um governo exclusivamente
de fascistas. Eu poderia\ldots{}, mas eu não pretendo fazer isso. Ao menos,
não logo no início de nosso governo.

Criei um governo de coalizão, sem a intenção de ter uma maioria
parlamentar, da qual, aliás, agora poderia perfeitamente prescindir
(Aplausos na extrema direita e na extrema esquerda). Criei esse governo
para reunir toda ajuda possível a nossa nação ofegante, para reunir
todos aqueles que, acima das nuances dos partidos, tenham o mesmo
objetivo de salvar nossa nação.
\end{quote}

Mussolini disse que o desejo de seu governo era o de restaurar a
disciplina no país, proclamando, por um lado, que o Estado ``não
pretende, a princípio, abdicar de ninguém'' e, por outro, que qualquer um
que ``se oponha ao Estado será punido''. Além disso, afirmava ainda que,
uma vez que ``falar não é, evidentemente, suficiente, o Estado
providenciará a seleção e o aperfeiçoamento das forças armadas. O Estado
Fascista provavelmente se constitui como a única polícia perfeitamente
organizada, de grande mobilidade e de elevado espírito moral''. Já no
encerramento de seu discurso, novamente introduzido pelo direto
``Senhores'', Mussolini retornou ao tom desdenhoso, brutal e ameaçador
do início: ``Eu não quero, por enquanto, governar contra a Câmara. Mas a
Câmara deve saber ocupar sua real posição. Ela está sujeita à
dissolução, algo que pode ser feito daqui a dois dias ou daqui a dois
anos''. O final de sua intervenção é marcado por uma invocação a Deus, em
que o \emph{Duce} diz o seguinte: ``Que Deus me ajude a levar meu árduo
trabalho a um fim vitorioso''.\footnote{Mussolini, \emph{Opera Omnia},
  vol. \textsc{xix}, p.\,22.}

A oratória de Mussolini na condição de presidente do Conselho rompeu com
toda a tradição da eloquência dos governos italianos, ao se valer de seu
estilo assertivo, autoritário, ameaçador e até mesmo violento. O que
caracterizou, em particular, essa sua oratória foi a virulenta franqueza
com a qual Mussolini alternou, sobrepôs e misturou essas duas posições
que ele ocupava, a de primeiro"-ministro e a de líder do fascismo. Assim,
ele proclamava publicamente que o governo fascista se identificava com o
Estado, que o partido fascista se identificava com a nação e que o
fascismo se identificava com o \emph{Duce}. O ciclo se fechava, por sua
vez, com a própria identificação de Mussolini com o fascismo, tal como
ele mesmo afirmava. Um exemplo da construção dessa identificação ocorreu
no dia 31 de maio de 1923, quando, ao se dirigir ao público do Primeiro
congresso de mulheres fascistas do Triveneto, Mussolini declarou:

\begin{quote}
Fico muito feliz de poder dizer a vocês, mulheres fascistas, e aos
fascistas de toda Itália, que a tentativa de separar Mussolini do
fascismo ou de separar o fascismo de Mussolini é a tentativa mais
inútil, mais grotesca e mais ridícula que se pode imaginar.

Não é por orgulho que digo a vocês que entre este que lhes dirige a
palavra e o fascismo se constitui uma identidade única. Mas, quatro anos
de história têm mostrado claramente que Mussolini e o fascismo são dois
aspectos da mesma natureza, são dois corpos de uma mesma alma ou duas
almas de um mesmo corpo.

Não posso abandonar o fascismo, porque eu o criei, eu o levantei, eu o
disciplinei, eu o fortaleci e eu o tenho ainda em minhas mãos. Sempre o
terei em minhas mãos! Portanto, é absolutamente inútil que as velhas
aves de rapina da política italiana me façam sua corte de merda. Sou
muito inteligente para cair nesse tipo de emboscada de comerciantes
medíocres, nessa armadilha de feirantes de um lugarejozinho qualquer.\footnote{Mussolini, Opera Omnia, vol. \textsc{xix}, p.\,226--227.}
\end{quote}

A mesma forma de identificação ideal entre ele e seu público foi adotada
por Mussolini em seu discurso endereçado aos veteranos da Grande Guerra.
Ali, ele se autonomeou de modo idealizado como o \emph{Duce} perfeito,
aquele que levou a Itália dos combatentes ao poder, aquele que, na
condição de um intervencionista, foi um soldado e um homem ferido na
guerra: ``mais do que o chefe do governo, é o camarada que fala com
vocês, o soldado que é honrado por ter comido nas trincheiras, por ter
travado uma guerra, depois de tê"-la desejado pelas necessidades da
história. Vocês representam a mais alta aristocracia da nação'', disse
ele aos condecorados com medalhas de ouro, no dia 8 de janeiro de 1923.
Nesse mesmo discurso, Mussolini ainda os exaltou, dizendo que eles eram
``o testemunho vivo do prodígio realizado por um povo que lutou unido,
como já não mais fizera há vários séculos''. Acrescentou, finalmente, o
seguinte: ``O eclipse de nossa linhagem se encerrou e todas as nossas
virtudes adormecidas, mas não extintas, passaram ao primeiro plano e nos
deram a vitória imortal''.\footnote{Mussolini, \emph{Opera Omnia}, vol.
  \textsc{xix}, p.\,94--95.} Foi praticamente com esse mesmo discurso que se
dirigiu aos mutilados milaneses, no dia 11 de março:

\begin{quote}
Eu vim aqui falar com vocês, principalmente, como companheiro de
trincheira e de sacrifício. Quando estou diante de vocês, eu me
reconheço em vocês e revivo aquelas que são certamente as páginas da
minha vida de que mais me lembro: as páginas escritas na trincheira,
onde pude ver com meus próprios olhos a sangrenta labuta da raça
italiana, onde pude ver seu espírito de devoção, onde pude ver rebrotar
de suas raízes milenares, mas que pareciam perdidas, a maravilhosa flor
de nossa magnífica história. Todos nós nos reconhecemos. Considero os
combatentes, os mutilados, as famílias dos que caíram em combate como a
grande, a mais pura e intangível aristocracia da nova Itália.\footnote{Mussolini,
  \emph{Opera Omnia}, vol. \textsc{xix}, p.\,166--170.}
\end{quote}

No dia 18 de março, ao se endereçar aos cegos de guerra, em Roma,
Mussolini reiterou sua identificação com seu público, dizendo o
seguinte: ``Eu mesmo lutei e fui ferido. O governo considera todos vocês
como o que há de melhor e de mais nobre entre os italianos. Eu lhes digo
isso hoje como chefe de governo e como companheiro de
trincheiras.''\footnote{Mussolini, \emph{Opera Omnia}, vol. \textsc{xix}, p.\,183--184.}

Além disso, desde os primeiros dias, após sua ascensão ao poder, e nos
dois anos seguintes, antes que passasse a se empenhar mais
resolutamente, a partir 3 de janeiro de 1925, na demolição do regime
liberal e na construção do regime de partido único e totalitário, os
discursos de Mussolini, realizados no Parlamento, mas, sobretudo, os
proferidos em outros contextos, tinham como motivo dominante a afirmação
de que a chegada do fascismo ao poder era um acontecimento irrevogável.
No dia 8 de janeiro, quando se dirigia aos condecorados com medalhes de
ouro, que se encontravam reunidos em Roma, proclamou: ``Não volta atrás!
O que aconteceu é irrevogável! Todas as antigas classes, todos os velhos
partidos, os velhos homens e suas crenças antiquadas foram varridos pela
revolução fascista e nenhum prodígio pode juntar esses cacos, que devem
ir para o museu de coisas mais ou menos veneráveis''.\footnote{Mussolini,
  \emph{Opera Omnia}, vol. \textsc{xix}, p.\,95.}

Já no dia 10 de fevereiro de 1923, na Câmara, referindo"-se àqueles que
teriam se iludido com o fato de que poderiam ``combater impunemente o
Estado e o fascismo'', Mussolini repetiu que o Estado fascista,
diferentemente do Estado liberal, agia deste modo: ``o Estado fascista
não apenas se defende, ele ataca. E aqueles que pretendem difamá"-lo no
exterior ou prejudicá"-lo aqui dentro mesmo da nossa Itália devem saber
que o desempenho dessa sua função envolve riscos muito altos. Os
inimigos do Estado fascista não poderão ficar surpresos, se eu os tratar
severamente, se eu os tratar exatamente como inimigos''.\footnote{Mussolini,
  \emph{Opera Omnia}, vol. \textsc{xix}, p.\,129.} Alguns meses mais tarde, no
dia 8 de junho, falando ao Senado, disse que ``para defender essa forma
tão especial de governo, para defender o Fascismo, há um poderosíssimo
exército de voluntários''.\footnote{Mussolini, \emph{Opera Omnia}, vol.
  \textsc{xix}, p.\,265.} Vinte dias depois, em Roma, numa reunião com os
trabalhadores do \emph{Poligrafico dello Stato}, Mussolini reiterou que
seu governo ``nasceu de uma grande revolução, que se desenvolverá ao
longo de todo o século''.\footnote{Mussolini, \emph{Opera Omnia}, vol.
  \textsc{xix}, p.\,115.} Novamente, no dia 15 de julho daquele mesmo ano, quando
a nova reforma eleitoral apresentada pelo governo, cujo objetivo era
garantir uma maioria parlamentar segura e estável, estava em discussão
na Câmara, Mussolini reiterou que o fascismo estava determinado a se
manter no poder de qualquer forma: ``Temos poder e nós o manteremos. Nós
vamos defendê"-lo contra qualquer um. Fizemos a revolução com este firme
desejo de tomarmos e de nos mantermos no poder''. Para isso, o fascismo
se perpetuaria ``até que todos estejam resignados a este fato consumado,
até que todos reconheçam sua bela armadura e sua bela alma
guerreira''.\footnote{Mussolini, \emph{Opera Omnia}, vol. \textsc{xix}, p.\,317.}

\section{O homem do povo}

A marcha sobre Roma foi imediatamente seguida pelo nascimento e pela
disseminação, em grande medida, espontânea, de um mito mussoliniano
entre a população. Para promover a disseminação desse mito, o próprio
Mussolini deu sua contribuição: ele foi o primeiro presidente do
Conselho que, poucos meses depois da chegada ao poder, passou a visitar
todos os lugares da Itália, passando por cidades e regiões
negligenciadas ou praticamente ignoradas por seus predecessores. Durante
uma visita à Sardenha, depois de prestar homenagem a Garibaldi na ilha
de Caprera, disse ao povo de Sassari que sua viagem não era um exercício
ministerial, mas ``uma peregrinação de devoção e amor pela sua magnífica
terra'':

\begin{quote}
Disseram"-me que, de 1870 até hoje, é a primeira vez que o chefe do
Governo fala com o povo de Sassari reunido nesta sua grande praça
pública. Vocês foram esquecidos, foram esquecidos por muito tempo! Em
Roma, se sabia e não se sabia que a Sardenha existia. Mas, desde que a
guerra revelou sua importância à Itália, todos os italianos devem se
lembrar da Sardenha não apenas por palavras, mas por ações.

Eu os saúdo, filhos magníficos desta ilha sólida, desta esquecida ilha
de ferro. Eu saúdo com meu corpo e com meu espírito todos vocês. Não é o
chefe do Governo que fala com vocês, não é o chefe do Governo que está
aqui: é o irmão, o camarada, o que dividiu a trincheira com muitos de
vocês. Por isso, gritemos juntos: Viva o rei! Viva a Itália! Viva a
Sardenha!\footnote{Mussolini, \emph{Opera Omnia}, vol. \textsc{xix}, p.\,264--266.}
\end{quote}

Em todos esses lugares por que passava, Mussolini proferia um discurso
para a população local, que se reunia, principalmente de modo
espontâneo, chegando muitas vezes a dezenas de milhares de ouvintes. A
chamada ``fábrica do consenso'' ainda não estava funcionando. Entre maio e
outubro de 1923, Mussolini fez essas viagens e esses discursos em todas
estas regiões: Lombardia, Vêneto, Emília e Romagna, Sardenha, Sicília,
Campânia, Abruzzo, Piemonte, Ligúria, Toscana e Úmbria. No ano seguinte,
ele repetiria esse percurso. Assim, em pouco tempo, tal como observou um
jornalista francês, no final de 1924, o prestígio pessoal do ditador
conheceu um aumento constante, que o levou ao topo da idolatria popular.\footnote{De Nolva, R. \emph{Le mysticisme et l'esprit révolutionnaire
  du fascisme}, in ``Mercure de France'', 1 de novembro de 1924.}

Mussolini estabeleceu assim um contato direto com o povo italiano,
criando uma conexão com diferentes classes sociais de diversas regiões
italianas. Produzia"-se a sensação física e imaginária de que agora todos
os italianos estavam mais próximos do poder e podiam ser ouvidos pelo
\emph{Duce}. Para as pessoas de modo geral, Mussolini aparecia como um
chefe de governo que tinha um novo estilo. Ele teria chegado ao poder
por um movimento revolucionário, mas estava sempre pronto para impor
disciplina ao seu partido. Mussolini era considerado, ao mesmo tempo, um
revolucionário e um ditador, ainda alguém que teria demonstrado
qualidades de um bom administrador, como realismo e senso de proporção.
Para a opinião pública burguesa, ele era o salvador da pátria, quem a
livraria da anarquia, o cavaleiro que matara o dragão vermelho do
comunismo na Itália e que salvara o Ocidente do bolchevismo. Nas classes
populares que não haviam sofrido violência fascista, as manifestações de
simpatia eram direcionadas ao ``filho do povo'', que havia se tornado
chefe do Governo, sem esconder, mas, antes, de fato, mostrando suas
origens populares. Portanto, ele fora imediatamente investido de
confiança e se tornou um depositário da esperança, graças ao seu
trabalho de luta contra as injustiças e males da existência.

Em quase todos os primeiros discursos às multidões, o primeiro"-ministro
Mussolini insistia em sua origem popular, especialmente quando se
dirigia às massas operárias, que naquele momento ainda eram
majoritariamente hostis ao fascismo. No dia 6 de dezembro de 1922, em
Milão, falando aos trabalhadores metalúrgicos lombardos, Mussolini se
valeu de um exórdio patético:

\begin{quote}
Eu não descendo de ancestrais aristocráticos e ilustres. Meus ancestrais
eram camponeses que trabalhavam na terra, e meu pai era um ferreiro que
dobrava o ferro quente na bigorna. Às vezes, quando criança, eu ajudava
meu pai em seu trabalho duro e humilde. Agora, tenho uma tarefa muito
mais dura e muito mais difícil: a tarefa de dobrar as almas. Aos vinte
anos, trabalhei com meus próprios braços. Eu era operário e pedreiro.
Digo isso a vocês não para lhes pedir sua simpatia, mas para lhes
mostrar que nunca fui e que não posso ser um inimigo das pessoas que
trabalham. Eu sou inimigo daqueles que, em nome de falsas e grotescas
ideologias, querem enganar os trabalhadores e levá"-los à
ruína.\footnote{Mussolini, \emph{Opera Omnia}, vol. \textsc{xix}, p.\,57--59.}
\end{quote}

Pouco mais de um mês mais tarde, no dia 23 de janeiro de 1923, Mussolini
insistiria em sua harmonia ideal com a classe trabalhadora, ao se
dirigir aos trabalhadores do \emph{Poligrafico} em Roma:

\begin{quote}
Eu me orgulho de ser filho de trabalhadores, eu me orgulho de ter
trabalhado com meus próprios braços. Eu conheci os humildes trabalhos
dos verdadeiros trabalhadores. Quando trabalhei, a jornada de trabalho
era de doze horas, hoje, é de oito. Essa conquista é intocável.
{[}\ldots{}{]}

Estes aplausos que vocês me dedicam são muito vivos e espontâneos para
serem aplausos de conveniência e cortesia. Vocês sentem que minhas
palavras entram nas suas almas, que minhas palavras são o eco dos
sentimentos que vocês sentem já há muito tempo. Peço a vocês que
continuem trabalhando com toda a tranquilidade e com a mais perfeita
disciplina.\footnote{Mussolini, \emph{Opera Omnia}, vol. \textsc{xix}, p.\,114--116.}
\end{quote}

O grupo de trabalhadores que mais sofreu com as ações fascistas foi a
classe operária de Turim, sede da \textsc{fiat} e de outras indústrias
importantes. Em agosto de 1917, durante uma semana, a cidade piemontesa
foi palco do maior protesto contra a Grande Guerra. Houve uma série de
confrontos entre manifestantes e as forças públicas de ordem, de que
resultaram mortos e feridos.\footnote{Cf. Spriano, P. \emph{Torino
  operaia nella grande guerra (1914--1918)}, Torino: Einaudi, 1960.} Em
Turim, com Antonio Gramsci e seu jornal \emph{Ordine Nuovo}, uma facção
comunista bastante sólida havia se formado no seio do Partido
Socialista. Foi a criação dessa facção que deu ensejo em janeiro de 1921
ao nascimento do Partido Comunista italiano. Em dezembro de 1922, os
\emph{squadristi} fascistas, em retaliação aos movimentos operários de
Turim, executaram um massacre, matando 14 (ou talvez mais) trabalhadores
antifascistas.\footnote{Cf. De Felice\emph{,} R. \emph{I fatti di Torino
  del dicembre 1922,} in ``Studi Storici'', n. 4, 1963.} Foi ainda em
Turim, no dia 24 de outubro de 1923, que Mussolini visitou a \textsc{fiat}
e falou aos trabalhadores, lhes dirigindo palavras de louvor, mas também
de advertência mais ou menos velada: ``Não pensem que vocês podem se
livrar da vida, da alma e da história da nação'':

\begin{quote}
Mesmo se vocês quisessem fazer isso, vocês não conseguiriam, pois não
podemos renegar nossa própria mãe. Querendo ou não, somos todos
italianos e devemos ter orgulho de ser italianos, não apenas pelas
glórias do passado, glórias muito nobres, sobre as quais não podemos
viver como descendentes parasitas e degenerados que desfrutam das
conquistas de seus antepassados. Devemos ter orgulho, principalmente,
dessa nova Itália, que está progredindo e que tem aqui, nesta sua
fábrica, conseguido alcançar o primado europeu.

É por esta nova Itália que lhes peço o cumprimento ordeiro e silencioso
do seu dever como trabalhadores e cidadãos. Somente com trabalho e
colaboração de todos os envolvidos na produção de nossas indústrias é
que aumentará o bem"-estar individual. Proclamo solenemente, onde não há
esse empenho, o que reina é a miséria individual e a ruína da nação.
\end{quote}

Depois da advertência e da exposição do preceito, segundo o qual somente
aqueles que são disciplinados, empenhados e silenciosos é que podem ser
considerados bons trabalhadores e bons cidadãos, Mussolini voltou a
declarar que a prova de sua solidariedade para com os trabalhadores era
sua própria origem popular. Ele o fez, carregando, novamente, sua fala
com recursos patéticos e com mais uma ameaça:

\begin{quote}
Porque eu trabalhei com meus próprios braços, porque vim do povo e
porque tenho orgulho dessa minha origem, saúdo vocês, não com a falsa
simpatia dos demagogos, dos vendedores de ilusões, mas com a rude
sinceridade de um trabalhador, de um homem que não quer enganar vocês,
de um homem que imporá a disciplina necessária a todos, mesmo a seus
amigos, mas, acima de tudo, a seus adversários.\footnote{Mussolini,
  \emph{Opera Omnia}, vol. \textsc{xx}, p.\,55--56.}
\end{quote}

\section{O preceptor da nova Itália}

Em todos os lugares e em cada um de seus discursos para a multidão,
Mussolini insistiu no tema da disciplina como o dever de todos os
italianos, começando pelos próprios fascistas, que haviam assumido a
tarefa de impô"-la para tornar o estado mais forte e para promover a
prosperidade e a grandeza da nação. Como primeiro"-ministro e líder do
fascismo, Mussolini se apresentava como um tutor do povo italiano, como
se o país fosse uma imensa escola à qual ele pessoalmente tinha de
ensinar as regras da boa vida civil. Era a ele que cabia ensinar aos
italianos os deveres que tinham que seguir, mesmo que não fossem
fascistas. As advertências disciplinares eram válidas para todos, e
ninguém podia escapar, caso fosse desobediente, de uma severa punição.

A ameaça de punição era dirigida, evidentemente, aos italianos que
persistiam em se opor ao governo de Mussolini e não se resignavam à
irrevogável conquista do poder pelo Partido fascista. Esse foi outro
motivo recorrente nos discursos mussolinianos para a população reunida
nas praças das cidades que visitou em 1923. Uma vez que Turim era não
apenas a sede de um forte grupo comunista, mas também e acima de tudo a
capital monárquica do \emph{Risorgimento} e do liberalismo da união
nacional, foi ali que Mussolini declarou com maior ênfase que a chegada
do fascismo ao poder não era uma mudança normal no governo parlamentar,
mas um grande evento histórico irrevogável, que marcava o início de um
novo regime e de uma nova era. Apresentando"-se à multidão de Turim como
fiel servidor do rei, ``um soldado fiel, um líder fiel e abnegado'',
Mussolini se referiu às boas vindas que as massas lhe deram em todas as
cidades. Isso para afirmar que seu governo tinha forças para impor a
disciplina, mas que tinha igualmente o consentimento do povo para
fazê"-lo com uma rigidez legítima. Endereçando"-se aos administradores de
Turim, no dia 24 de outubro, ele disse que essa legitimidade autorizava
até mesmo a limitação da liberdade:

\begin{quote}
Sem a necessidade de recorrer à força, pode haver consenso. E por quê?
Por uma razão muito simples. Nós não somos ambiciosos, somos ainda menos
vaidosos, menos ainda assumimos posições infalíveis. Somos simplesmente
trabalhadores que impuseram uma disciplina e que, por isso, têm o
direito de impô"-la àqueles que são recalcitrantes no meio do povo
italiano.

Porque, senhores, a liberdade sem ordem e sem disciplina significa
dissolução e catástrofe. (Aplausos). O povo italiano, certamente mais
conscientes e saudáveis do que aqueles que pretendem representá"-lo,
valoriza as vantagens desse regime que impõe a necessária disciplina.

Nós não estamos num momento fácil, senhores, especialmente na Europa, e
quando o navio da nação em que estamos carregados é atingido pelas ondas
da tempestade, a disciplina deve ser muito rígida.

Quando chegarmos ao porto e ao nosso destino, poderemos dar liberdade
razoável à tripulação. Não antes disso, porque seria um crime contra a
própria nação.\footnote{Mussolini, \emph{Opera Omnia}, vol. \textsc{xx}, p.\,48.}
\end{quote}

À advertência dirigida aos estudantes do consciente e saudável povo
italiano, Mussolini acrescentou novamente a ameaça de punir os
opositores de seu regime, quando falou, poucas horas mais tarde, à
multidão reunida na vasta \emph{Piazza Castello}. Ali lembrou seu
público de que um ano antes, em Nápoles, ele havia lançado o grito aos
fascistas que os levou à ``marcha sobre Roma'':

\begin{quote}
E três dias depois tomamos a cidade eterna e começamos a limpeza e o
trabalho policial que ainda não acabou e que deve continuar. Garanto que
esse trabalho continuará de forma inflexível, tenaz e sistemática.
Agora, nós não nos apoderamos de Roma por nossa ambição, não foi por
nosso próprio interesse, não foi pela miserável vaidade de tantas
pessoas. Nós nos apoderamos de Roma e nós a manteremos, lutando contra
qualquer um que seja inimigo da nação. Nós vamos mantê"-la até que o
trabalho que nós começamos esteja completo, até que todas as oposições
mais ou menos insignificantes e miseráveis sejam eliminadas para
sempre.\footnote{Mussolini, \emph{Opera Omnia}, vol. \textsc{xx}, p.\,50.}
\end{quote}

\section{Diálogo com o povo imortal}

Outro motivo constante nos discursos de Mussolini era uma espécie de
aplicação do modelo do mestre como defensor de sentimentos nobres e
elevados, tal como aquele que o jovem estudante Benito delineara em sua
tarefa escolar mais de vinte anos antes. Além de se apresentar como
chefe do governo, líder da revolução fascista, campeão dos veteranos de
guerra, homem do povo e tutor da nação italiana, Mussolini se expunha às
multidões como o visionário de uma nova grande Itália. Sob o comando do
\emph{Duce}, a grande nação estava prestes a repartir pelas estradas que
conduziriam a uma nova grandeza imperial. No estilo oratório típico da
``história monumental'', nos discursos às multidões das várias regiões,
Mussolini sempre exaltou o povo italiano que, após muitos séculos sem
protagonismo, agora voltava a ser um grande povo, sob a orientação do
fascismo e de seu líder e movido pelo renascimento do romanismo
imperial.

``Sinto toda a poderosa fonte da vida que agita a nova geração da raça
italiana'', disse ele em Milão, no dia 30 de março de 1923, aos alunos
de uma escola especial de emigração. Acrescentou ainda o seguinte:
``Vocês certamente já pensaram algumas vezes sobre o que poderia ser
chamado de um prodígio na história da raça humana. Não se trata de
retórica, quando se diz que o povo italiano é o povo imortal, o povo que
sempre encontra uma fonte para nutrir suas esperanças, sua paixão e sua
grandeza.\footnote{Mussolini, \emph{Opera Omnia}, vol. \textsc{xix}, p.\,192.} Em
21 de abril de 1923, no \emph{Campidoglio}, na entrega das medalhas do
Instituto Nacional da Fita Azul, Mussolini exaltou nos soldados ``a nova
e poderosa aristocracia da nova Itália'', dizendo o seguinte:

\begin{quote}
Gostaria de recordar à memória de todos vocês o prodígio desta velha e
sempre jovem raça italiana, o prodígio de sua poderosa renovação. Onde
estão os podres patifes que cuspiram em nós na véspera da guerra,
declarando que o povo não saberia lutar? Em qual caverna esconderam sua
vergonha? O povo italiano têm uma única origem e uma única bandeira e,
por isso, luta unido e num perfeito compasso.\footnote{Mussolini,
  \emph{Opera Omnia}, vol. \textsc{xix}, p.\,203--204.}
\end{quote}

Na representação oratória de Mussolini, o prodígio do renascimento da
raça italiana coincidiu com o advento do fascismo no poder. Para
alegá"-lo, o \emph{Duce} afirmava que a tradição do \emph{Risorgimento}
foi revivida pela experiência vitoriosa da Grande Guerra. Os fascistas e
seu líder seriam os representantes mais genuínos dessa exitosa renovação
da tradição. Nos encontros com as multidões, Mussolini realizou várias
vezes uma espécie de rito de comunhão, que o uniu física e misticamente
com o ``povo imortal''. Essa perfeita união entre o \emph{Duce} e seu
povo se dera justamente na primavera de renascimento da raça italiana,
como ele disse em Piacenza, no dia 17 de junho:

\begin{quote}
Sempre que me distancio de Roma, onde os restos de pequenas castas
políticas ainda se iludem sobre sua vitalidade, e me misturo no meio do
povo, tenho diante de meus olhos a visão esplêndida e magnífica de uma
primavera incomparável.

Aqui, nesta cidade histórica, o sangue da nova geração pulsa, pulsa aqui
mais do que em qualquer outro lugar. Aqui o povo, em todas as suas
categorias, compreendeu que neste momento a disciplina, a harmonia e o
trabalho são elementos necessários para a reconstrução da Pátria. Aqui
há esse consenso, e não apenas a força. Aqui está o povo que se une a
mim e ao Governo, porque legislamos acima de todos os interesses das
castas e das categorias particulares. Aqui o povo sabe que o bem supremo
da nação está acima das vontades singulares e individuais. Sabe que
agora existe uma magnífica vontade coletiva no poder: a vontade coletiva
de todo o povo italiano. Hoje nosso povo é compacto e solidário, hoje
nosso povo se une em torno do fascismo, porque sabe que o fascismo
representa o prodígio da raça italiana. Dessa raça que se reencontrou
consigo mesma, que se redimiu e que voltou a ser grande.\footnote{Mussolini,
  \emph{Opera Omnia}, vol. \textsc{xix}, p.\,272--273.}
\end{quote}

Um aspecto peculiar da oratória dos primeiros meses de governo de
Mussolini foi o diálogo com a multidão, derivado do estilo inventado por
Gabriele D'Annunzio na instauração do Estado Livre de Fiume. O diálogo
não era efetivamente uma forma de envolvimento mais direto com a
multidão na oratória de Mussolini, mas algo que lhe proporcionava uma
maneira imediatamente eficaz de se proclamar como o intérprete e o
representante do ``povo imortal''. Nesses diálogos, Mussolini pedia a
esse povo a consagração de sua autoridade por meio de uma investidura
direta e, por outro lado, lhe pedia juramentos de lealdade. É isso que
podemos observar em seu pronunciamento do dia 24 de junho de 1923,
quando ele falou pela primeira vez da sacada do \emph{Palazzo Venezia},
se dirigindo a uma multidão de veteranos reunidos em Roma, por ocasião
do Festival dos Combatentes. Ao final do desfile e da homenagem ao
Soldado Desconhecido, Mussolini disse estas palavras:

\begin{quote}
Depois dessa demonstração formidável de força e saúde, minhas palavras
são absolutamente supérfluas. Não tenho a intenção de fazer um discurso
para vocês. Mas tenho o direito de interpretar o que aconteceu nesse
nosso encontro, para o qual vocês vieram para ouvir a minha fala. Sua
presença é um gesto de solidariedade para com o governo nacional. Nós
não vamos falar palavras desnecessárias. Ninguém de nós está atentando
contra a liberdade do povo italiano. Mas, mesmo assim, eu lhes pergunto:
deve haver liberdade para mutilar a vitória do povo? (Gritos de ``Não!
Não!'') Deveria haver liberdade para sabotar a nação? (Gritos de ``Não!
Não!'') Deve haver liberdade para aqueles que planejam perturbar as
instituições que nos governam? (Mais gritos de ``Não! Não!'') Repito o
que disse explicitamente antes. Não me sinto infalível. Eu sou um homem
como vocês.\footnote{Mussolini, \emph{Opera Omnia}, vol. \textsc{xix}, p.\,276.}
\end{quote}

A identificação do líder com o ``povo imortal'', como seu único
intérprete legítimo, não ocorreu apenas nas ocasiões em que Mussolini
fez discursos dirigidos a categorias específicas, como os combatentes e
os fascistas, evidentemente. Sua produção foi efetuada para forjar a
assimilação de Mussolini com todo o povo italiano, com todo o povo
vigoroso que o aclamava, mesmo que parte daquelas multidões não fosse
fascista. Assim fez o \emph{Duce} em seu discurso para a massa durante
sua visita a Palermo em 5 de maio de 1924:

\begin{quote}
Agora, povo de Palermo, eu quero falar com vocês. Este é um antigo
costume, que remonta ao tempo em que os tribunos falavam ao povo em suas
arengas. Mas é também um costume moderno, porque foi revivido em Fiume
(Gritos de ``Viva D'Annunzio!'')

Bem, povo de Palermo, se a Itália lhes pede a disciplina necessária, se
a Itália exige o trabalho harmonioso e a devoção absoluta à pátria, o
que vocês respondem, povo de Palermo? (Todas as pessoas, de modo súbito
e veemente, respondem"-lhe com um formidável ``Sim!'')

E se amanhã for necessário que corra novamente a avalanche de sangue do
seu peito, se for necessário eliminar tudo o que já não tem razão de
existir, vocês estarão prontos para marchar? (A multidão explode em um
novo e poderoso ``Sim!'').

Povo de Palermo!

Vocês são mesmo dignos de sua história e de sua glória. Vocês são mesmo
dignos de sua história e de sua glória. Vocês são realmente o povo de
Garibaldi! Como nem todas as batalhas já foram travadas, o trabalho de
redenção e de reconstrução ainda não terminou.\footnote{Mussolini,
  \emph{Opera Omnia}, vol. \textsc{xx}, p.\,260--261.}
\end{quote}

Em outras ocasiões, como no discurso para a população da Sardenha, no
dia 11 de junho, Mussolini introduzia, ainda que de modo mais ou menos
indireto, outra das razões para sua exaltação ao ``povo imortal'': a
necessidade de regenerar o caráter dos italianos, após séculos de
declínio e de escravização, para torná"-los um povo novo, disciplinado e
educado pelo fascismo. Para torná"-lo pronto para marchar como um
exército em busca da conquista de sua grandeza imperial:

\begin{quote}
Eu olhei em seus rostos, vi suas feições, fixei meus olhos nos seus
olhos. Ao fazê"-lo, hoje reconheço que vocês são excelentes filhos da
raça italiana. Essa raça italiana já era grande, quando outros povos
ainda nem sequer tinham nascido. Esta raça italiana que legou três vezes
ao mundo bárbaro e atônito a beleza de sua civilização. Queremos tomar
novamente a forma desta raça italiana, para enfrentarmos todas as
batalhas necessárias com disciplina, com trabalho e com fé.\footnote{Mussolini,
  \emph{Opera Omnia}, vol. \textsc{xix}, p.\,264--266.}
\end{quote}

Desde os primeiros meses de governo, o \emph{Duce}, nesse seu papel de
tutor, nutre o ambicioso projeto de também se tornar o forjador de um
novo povo italiano, movido cada vez mais por sua própria visão de uma
Itália fascista unida, forte e poderosa. Em seu caráter coletivo, essa
nova Itália incorporaria a ideia do \emph{Duce} como o grande artista
que moldou a matéria humana, como Mussolini frequentemente gostava de se
descrever. Mas para um empreendimento tão gigantesco, era necessário
tornar irrevogável a ascensão ao poder, concentrando"-o totalmente em
suas próprias mãos, era necessário transformar o próprio governo,
nascido de uma ameaça insurrecional e de legitimidade parlamentar, num
regime exclusivamente fascista.

\section{Discursos do regime}

Apesar das conclamações de parte da população para que as promessas de
acabar com a violência fascista e de restaurar o respeito à lei fossem
cumpridas, Mussolini nunca parou de repetir que o advento do fascismo
era um evento irrevogável, que a milícia fascista, legalizada, era a
força armada cujo dever era defender ``os desenvolvimentos inexoráveis
da revolução fascista'', que o fascismo estava determinado a impor a
todos os italianos uma disciplina ditatorial para iniciar a criação de
um novo povo, moldado de acordo com os desígnios do \emph{Duce}. Depois
de reiterar essas proposições em todos os discursos, dentro e fora do
Parlamento, Mussolini passou a colocar cada vez mais em prática tudo o
que havia afirmado desde os últimos dias de outubro de 1922, quando
impôs a comemoração do primeiro aniversário da ``marcha sobre Roma'',
com manifestações públicas do Estado, segundo um programa que havia sido
aprovado por seu próprio governo.

O \emph{Duce} participou pessoalmente de todas as celebrações que
aconteceram de Milão a Roma, refazendo o caminho tomado pelo fascismo
para conquistar o poder. No primeiro discurso, feito em 28 de outubro em
Milão, diante de uma imensa multidão de fascistas fortemente armados,
que formavam as legiões da Milícia Voluntária para a Segurança Nacional,
Mussolini narrou, com uma ênfase épica, uma breve crônica do fascismo,
destacando sua rápida e esmagadora ascensão ao governo, num período de
apenas três anos. Nessa crônica, exaltou ainda ``a força invencível'' do
fascismo, representada por seus soldados. Dirigindo"-se, particularmente,
a eles, buscou excitar ainda mais seu entusiasmo: ``vocês não são apenas
a aristocracia de um partido, vocês são muito mais do que isso, vocês
são a expressão e a alma da nação italiana''. Já aos adversários o líder
lançou mais esta ameaça: ``é preciso que eles percebam que não vamos
retroceder em nossas conquistas, que nós estamos preparados para travar
as batalhas mais difíceis para defender nossa revolução''. E continuou
nestes termos:

\begin{quote}
Por outro lado, peço a vocês que reflitam, por favor, sobre o seguinte:
a revolução foi feita com bastões e cacetetes. O que vocês têm agora em
suas mãos? (Os fascistas, então, gritaram: ``rifles'', ``mosquetes''\ldots{},
e mostraram essas suas armas, levantando"-as acima de suas cabeças). Se
com bastões e cacetetes foi possível fazer a revolução, e isso graças ao
seu heroísmo e graças também à enorme covardia daqueles que enfrentamos,
agora a revolução é defendida e consolidada com armas, com as suas
armas.
\end{quote}

O \emph{Duce} continuou com um tom sempre ameaçador, dirigindo"-se aos
militantes com o estilo oratório inventado por Gabriele D'Annunzio em
Fiume três anos antes, para então recordar sinteticamente, com imagens
eficazes, as razões fundamentais da ideologia, ou melhor dizendo, da
mitologia do fascismo antidemocrático e imperialista:

\begin{quote}
Quero ter um diálogo com cada um de vocês e tenho certeza de que suas
respostas serão afiadas e formidáveis. Minhas perguntas e suas respostas
não serão apenas ouvidas por vocês, mas por todos os italianos e por
todas as pessoas, porque hoje, depois de séculos, mais uma vez é a
Itália que dá uma direção ao caminho da civilização do mundo. (Aplausos)

Camisas pretas, pergunto"-lhes: se os sacrifícios de amanhã forem mais
sérios do que os sacrifícios de ontem, vocês me apoiarão? (Gritos
imensos dos fascistas: ``Sim!'')

Se amanhã eu lhes pedisse o que poderia ser chamado de uma prova sublime
da disciplina, vocês me dariam essa prova? (``Sim!'', Os soldados repetem
com uma voz alta, com entusiasmo).

Se amanhã eu desse o sinal de alarme, o alarme dos grandes dias,
daqueles que decidissem o destino dos povos, vocês responderiam? (Nova
explosão entusiasmada de: ``Sim! Nós juramos!'')

Se amanhã eu lhes disser que precisamos retomar e continuar a marcha e
empurrá"-la para outras direções, vocês marchariam? (``Sim! Sim!''. E o
coro fascista sobe ao mais alto diapasão).

Vocês têm a alma pronta para todos os desafios que a disciplina exige,
mesmo para aqueles em que vocês devem demonstrar que são os mais
humildes, os mais ignorados, dia após dia? (A Milícia grita: ``Sim!'').

Vocês estão certamente fundidos em um espírito, um coração, uma
consciência. Vocês realmente representam o prodígio desta maravilhosa e
antiga raça Itálica, que conhece as tristes horas, mas nunca conheceu a
escuridão das trevas. Se às vezes nosso destino parece escurecido, de
repente ele reaparece sob uma luz maior.
\end{quote}

A conclusão do discurso continuou na forma dialógica para reafirmar com
evidente franqueza a vontade fascista de preservar firmemente o
monopólio do poder:

\begin{quote}
São apenas doze meses. Vocês acham que vai durar doze anos multiplicado
por cinco? (``Sim, sim!'', Os militantes e a multidão) formam uma única
voz.

Vai durar, camisas pretas. Durará porque nós, os negadores da doutrina
do materialismo, não expulsamos a vontade da história humana; vai durar
porque queremos que dure; durará porque vamos sistematicamente dispersar
nossos inimigos; durará porque não é apenas o triunfo de um partido e
uma crise ministerial: é algo muito, muito maior, infinitamente maior. É
primavera, é a ressurreição da raça. São as pessoas que se tornam uma
nação; é a nação que se torna um Estado; e é o Estado que busca as
linhas de sua expansão no mundo.\footnote{Mussolini, \emph{Opera Omnia},
  vol. \textsc{xx}, p.\,62.}
\end{quote}

E o regime fascista, pela vontade do Duce e dos fascistas, durou vinte
anos, desfeito apenas depois de ter sido esmagado por uma derrota
militar desastrosa na Segunda Guerra Mundial.

Por muitos meses após a ascensão do fascismo no poder, a grande maioria
dos políticos italianos, não"-fascistas e antifascistas, estava
convencida de que o governo de Mussolini não duraria muito e de que o
Partido fascista seria atingido por uma grave crise interna durante o
governo. Em 1923, ele parecia que seria desintegrado como uma
organização improvisada, um agregado de bandidos sem ideologia,
experiência e prática governamental, unidos apenas pela violência armada
contra oponentes. Apenas alguns observadores e políticos antifascistas,
raros, perceberam que a chegada ao poder de um partido da milícia,
dotado de um aparato militar que já havia imposto seu domínio violento
na maior parte do território italiano, significaria o fim da democracia
na Itália e o advento de uma ``ditadura do partido total''. Em 2 de
novembro de 1923, comentando as celebrações marciais da ``marcha sobre
Roma'', o antifascista liberal Giovanni Amendola, que no mês de abril
daquele ano cunhara o termo ``totalitário'' para definir o domínio
violento do partido fascista na política italiana, escreveu:

\begin{quote}
De fato, os que no futuro estudarão o movimento fascista não terão
dúvida em afirmar que sua característica mais marcante é o espírito
``totalitário''. Do mesmo modo como acreditamos que amanhã o sol
nascerá, os fascistas acreditam em seu Duce. Essa singular ``guerra
religiosa'' que se desenrola na Itália há mais de um ano não nos oferece
uma verdeira fé, nos nega o direito de ter uma consciência e nos impede
de ter esperanças, porque nosso futuro está hipotecado com chumbo.
\end{quote}

O ``espírito totalitário'', concluiu Amendola, agiu concretamente em uma
Itália ``com um governo cercado e defendido por uma milícia partidária,
com um partido livre para dedicar"-se à adulteração de toda a vida
nacional em benefício de seus adeptos, com um Parlamento que está
placidamente se movendo em direção à letargia preparada para isso pela
nova lei eleitoral''.\footnote{Amendola, G. \emph{La democrazia italiana
  contro il fascismo 1922--1924}, Milão/Nápoles: Ricciardi, 1960, p.\,193--197.} Dois anos depois, tendo superado a crise do governo causada
pelo assassinato do deputado socialista Giacomo Matteotti pelos
esquadristas, Mussolini tomou o caminho da destruição do regime liberal
para estabelecer o regime totalitário, como Amendola havia previsto de
forma realista. No dia 22 de junho de 1925, no quinto e último congresso
nacional do Partido fascista, Mussolini, depois de estrear dizendo, com
o costumeiro hábito, que havia superado ``o aborrecimento que tenho toda
vez que tenho que fazer um discurso'', proclamou as palavras de ordem do
novo curso: ``intransigência absoluta, ideal e prática'', ``todo o poder
para todo o fascismo'' e acrescentou, indiretamente aludindo a Amendola:

\begin{quote}
Trouxemos a luta para um terreno tão claro que agora é necessário marcar
nossa presença. E não apenas isso. Nosso objetivo deve estar bem
definido e deve ser perseguido com toda nossa ferrenha vontade
totalitária, deve ser perseguido com uma ferocidade cada vez maior. Esta
tem de ser verdadeiramente a maior preocupação. Em resumo, queremos
tornar toda a nação fascista, de tal modo que amanhã o italiano e o
fascista, assim como são mais ou menos a mesma coisa o italiano e o
católico, sejam a mesma coisa.\footnote{Mussolini, \emph{Opera Omnia},
  vol. \textsc{xxi}, p.\,362--363.}
\end{quote}

\section{O «Duce» e a massa}

Durante o regime fascista, a oratória de Mussolini continuou a
frequentar diversas de suas circunstâncias e de seus eventos,
preservando a trama fundamental entre os distintos temas de que costuma
tratar o discurso fascista, tal como ela havia sido elaborada por
Mussolini entre o ano seguinte à Primeira Grande Guerra e a celebração
do primeiro aniversário de sua ascensão ao poder. De nossa investigação
de Mussolini como orador, cujo percurso se estendeu desde a adolescência
do aluno do colegial até o início, duas décadas depois, da trajetória do
líder de um novo regime de partido único, podemos concluir que o aspecto
fundamental de sua oratória era o contato constante e direto entre
Mussolini e as massas.

Apesar da diversidade substancial das convicções ideológicas que o
acompanharam nas várias metamorfoses de sua vida política, passando do
socialista revolucionário ao líder do regime totalitário fascista,
Mussolini teve uma atitude constante e difusa em relação às massas. O
socialista revolucionário acreditava na possibilidade de emancipar as
massas proletárias, de lhes forjar uma consciência autônoma, por meio da
ação educativa do partido e da cultura marxistas. Assim, o futuro que se
projetava para as massas era o de uma comunidade de pessoas livres e
iguais na sociedade socialista. Já para o Mussolini fascista, a ideia de
emancipação individual e coletiva das massas mediante a formação de uma
consciência autônoma se tornara absolutamente estranha. O mandamento
``Acreditar, obedecer e lutar'' condensava a convicção de Mussolini a
propósito da natureza imutável da massa como uma coletividade
inevitavelmente gregária.

Com efeito, o \emph{Duce} não escondeu essa sua convicção, a despeito de
suas reiteradas exaltações retóricas do povo italiano e das multidões
que ele conheceu e às quais se dirigiu por vinte anos nas mais diversas
praças da Itália e, principalmente, na sacada do \emph{Palazzo Venezia},
o púlpito de seus mais solenes e dramáticos discursos para os italianos.
Em 1932, durante uma conversa entre Mussolini e o escritor austríaco,
Emil Ludwig, que acabara de assistir a um breve discurso do \emph{Duce}
proferido justamente da sacada do \emph{Palazzo Venezia}, Ludwig
mencionara a sinergia entre o orador e a multidão. Mussolini lhe
respondeu com uma franqueza brutal:

\begin{quote}
A massa para mim não passa de um rebanho de ovelhas, até que seja
organizada. Eu não estou sendo hostil, mas somente não acredito numa
capacidade de autogestão das multidões. Para liderar a massa, você deve
dominá"-la com duas rédeas: o entusiasmo e o interesse. Quem usar apenas
uma delas estará em perigo. O lado místico do entusiasmo e o lado
político do interesse se condicionam mutuamente. Este último sem o
primeiro se torna muito árido, porque o interesse sem entusiasmo se
perde diante de uma brisa. Não posso dedicar minha atenção à vida
desconfortável das massas. Isso cabe a alguns poucos. Quanto à
influência recíproca da qual você fala, ela consiste precisamente nisto:
hoje, eu disse apenas algumas palavras aos que estavam reunidos na
praça, amanhã, milhões de pessoas podem lê"-las, mas aqueles que estavam
lá embaixo acreditam mais profundamente no que ouviram com seus próprios
ouvidos e no que viram com seus próprios olhos. Cada discurso dirigido à
massa tem um duplo objetivo: esclarecer uma situação e sugerir alguma
coisa nova. É por isso que para se produzir uma guerra é indispensável
fazer um discurso ao povo.\footnote{Ludwig, E. \emph{Colloqui con
  Mussolini,} Milão: Mandadori, 1965, p.\,129.}
\end{quote}

O \emph{Duce} ficou lisonjeado com este comentário de Ludwig ``Hoje, você
é o maior especialista das massas'' e lhe respondeu com estas palavras:
``Eu conheço as massas há trinta anos. Em Milão, eles me chamavam de
\emph{Barbarossa}. Lá eu podia encher ou esvaziar as ruas com as
multidões''. Além disso, falando ainda com o escritor sobre seus
discursos ao povo, Mussolini lhe disse que não os improvisava, como
acabara de declarar publicamente aos seus ouvintes. Afirmou que levava
meses para prepará"-los. Ludwig ainda lhe perguntou: ``O que você faz para
que seu discurso possa mudar a visão das massas?''. O \emph{Duce} então
lhe respondeu, dizendo que a técnica de seus discursos para a multidão
consistia na ``arte de disciplinar as massas'', tal como também o fazia
Lênin, acrescentou Mussolini, elogiando a oratória do líder bolchevique.

\begin{quote}
A construção dos meus discursos é como a construção das casas
americanas. Primeiro, a estrutura é erguida, ou seja, montam a estrutura
em aço. Em seguida, o cimento ou tijolos são lançados. A depender da
oportunidade, acrescenta"-se o material mais nobre do acabamento. Para o
discurso que farei na festa de outubro, já tenho a estrutura pronta,
mas, depois, dependerá da atmosfera da praça, dos olhos e das vozes dos
milhares de homens que estarão por lá, para eu decidir se lhe
acrescentarei somente tijolos ou travertino, ou ainda se colocarei
cimento e mármore mais ou menos juntos.\footnote{Ludwig, 1965, p.\,130--131.}
\end{quote}

Na base dessa sua postura para com as massas e de sua técnica para falar
às multidões, estava sua total desconfiança na racionalidade humana como
uma qualidade individual e coletiva. Para reforçar essa desconfiança,
além de suas experiências políticas pessoais, contribuíram estudiosos da
irracionalidade do comportamento humano, como Vilfredo Pareto e Gustave
Le Bon. Deste último, Mussolini se declarou um admirador e leitor
frequente em várias ocasiões. Para o jornalista francês Pierre
Chanlaine, que o entrevistou em 1932, o \emph{Duce} declarou:

\begin{quote}
do ponto de vista filosófico, eu sou um dos mais fervorosos adeptos do
seu ilustre conterrâneo Gustave Le Bon, cuja morte lamento
profundamente. Li toda sua imensa e penetrante obra. Sua
\emph{Psicologia das massas} e sua \emph{Psicologia dos novos tempos},
em particular. São duas obras, juntamente com seu \emph{Tratado de
psicologia política}, às quais me reporto frequentemente. Aliás, me
inspirei em muitos dos seus princípios para edificar o atual regime de
governo da Itália.\footnote{Chanlaine, P. \emph{Mussolini parle},
  Paris: Jules Tallendier, 1932, p.\,61--63.}
\end{quote}

Com sua longa experiência em política de massa, primeiramente como
socialista e, em seguida, como fascista, é provável que Mussolini não
precisasse dos ensinamentos de Le Bon para conquistar as massas, mas,
certamente, as reflexões do psicólogo social francês o ajudaram a
refinar seu estilo oratório. Dessas reflexões deve derivar o predomínio
das imagens sobre os assuntos tratados e o da produção de repetições, de
emoções e de frases lapidares e sentenciosas sobre a elaboração de
raciocínios. Qualquer que tenha sido a influência direta de Le Bon sobre
Mussolini, o fato é que a oratória do líder fascista pode muito bem ser
considerada o exemplo prático mais bem acabado de suas teorias sobre a
psicologia das massas.

Ao longo dos anos do regime, a oratória de Mussolini foi sendo
desenvolvida como parte de um amplo aparato de propaganda, que
transformou cada um dos discursos que o líder dirigia às massas, da
sacada do \emph{Palazzo Venezia} ou das sacadas e dos palanques de
muitas cidades italianas, no momento culminante de um ritual
cuidadosamente preparado. Tratava"-se de um evento ritualizado, que havia
sido elaborado com a expertise de toda uma equipe, para que nele se
processasse uma espécie de fusão mística entre o líder e a multidão. Os
efeitos que os discursos desse líder produziam sobre as massas reunidas
para ouvi"-lo podem ser observados num relatório policial acerca de um
pronunciamento feito por Mussolini em Nápoles, em outubro de 1931. O
comentário do informante anônimo, afetado pelo entusiasmo coletivo,
torna"-se ainda mais significativo, porque não se destinava à publicação:

\begin{quote}
O \emph{Duce} falou\ldots{} mas ele era acima de tudo de tal modo tão
dramaticamente expressivo, que ele nem precisaria dizer nada. A multidão
delirante, em estado de graça, sentia, e o sentia por meio dos espasmos
e das contradições encarnadas naquele rosto e naquele corpo do líder
italiano, que algo grande e terrível estava sendo preparado para a
Itália e para o mundo. A multidão sentia que Benito Mussolini era o
artífice invencível dessa transformação.\footnote{Prezzolini, \emph{op.\,cit}., p.\,226.}
\end{quote}

Quanto aos efeitos que os encontros com as massas produziram no
\emph{Duce}, eles são confirmados por vários e confiáveis testemunhos.
Mussolini sofreu a sugestão do entusiasmo coletivo, ficando arrebatado
com a dedicação que a multidão lhe demonstrava, mesmo que ele soubesse
que boa parte da empolgação da massa e de seu efeito sobre ele próprio
eram devidos ao aparato propagandístico e que havia uma imposição do
Partido fascista à população para participar das suas ``reuniões
oceânicas''. Attilio Tamaro, um historiador fascista que conhecia bem
Mussolini, afirmou que o \emph{Duce} tinha ``uma visão encantada das
multidões'':

\begin{quote}
Se poderia dizer que um sólido vínculo se estabelecia, em particular,
entre o orador e as massas quando o \emph{Duce} lhes dirigia sua fala.
Era algo diferente de uma mera sugestão, porque, em certo sentido,
Mussolini não apenas possuía as multidões, mas era também possuído por
elas. Um prefeito nos contou que havia acompanhado o \emph{Duce} numa de
suas turnês na região de Marcas: enquanto o carro estava em movimento e
os dois conversavam pacificamente sobre problemas locais, o prefeito a
certa altura percebeu que Mussolini dava sinais de inquietação e de que
prestava menos ateção à conversa. Notara que pouco a pouco os traços de
seu rosto mudaram e endureceram, enquanto evidentemente sua atenção era
atraída para alguma coisa que o prefeito não via nem ouvia. Finalmente,
o prefeito tomou conhecimento da razão dessa mudança: Mussolini, que não
conhecia o lugar, não tinha certeza se eles estavam já próximos ou não
do centro da cidade para a qual estavam se dirigindo. Mas, ele pôde
sentir a proximidade da multidão que o esperava. Então, ele se tornou
outro homem. {[}\ldots{}{]} Ele era pessimista e cauteloso em relação aos
indivíduos, e às vezes se tornava cínico e até mesmo vulgar. Mas, as
massas eram para Mussolini um fenômeno mais complexo. Ele experimentava
intensamente o contato com as massas. Isso fazia com que sua oratória
não fosse somente algo simples e fascinante, mas se tornasse também uma
obra"-prima do governo político. Ele parecia confiar mais nas massas do
que em seus próprios colaboradores.\footnote{Tamaro, A. \emph{Venti anni
  di storia (1922--1943}). Roma: Tiber, 1953--1954. vol. \textsc{ii}, p.\,150--152.}
\end{quote}

Por quase duas décadas, o \emph{Duce} falou às massas para anunciar suas
decisões mais importantes, tais como as declarações de guerra à Etiópia
em 1935, à Grã"-Bretanha e à França em 1940, e aos Estados Unidos em
1941. Com discursos para as multidões, comemorou tanto a vitória na
Etiópia em 1936 quanto o ``reaparecimento do império nas colinas de
Roma''. Tudo isso culminaria nos momentos de sua apoteose, em que se deu
a mística comunhão com ``o povo imortal''. Mas, nos anos da Segunda Guerra
Mundial, quando o exército italiano sofreu derrota após derrota, o
\emph{Duce} ficou em silêncio.

Os italianos, então, modificaram seu nome: de Mussolini para
\emph{Muto}lini.\footnote{Nota do tradutor: em italiano
  \emph{muto} significa \emph{mudo}.} A última vez que o \emph{Duce}
apareceu na sacada do Palazzo Venezia para falar algumas palavras à
multidão reunida foi em 5 de maio de 1943. Também havia sido num 5 de
maio, mas do ano de 1936, que daquela mesma sacada o \emph{Duce}
anunciara a conquista da Etiópia. Igualmente, num 5 de maio, desta vez,
do ano de 1941, o imperador da Etiópia, deposto por Mussolini, retornava
triunfantemente ao trono, com a atuação de tropas britânicas. Já em 25
de julho de 1943, o \emph{Duce}, dominado por derrotas militares,
deserdado pela maioria do Grande Conselho, órgão supremo do fascismo,
foi preso por ordem do rei e seu regime fascista entrou em colapso.
Assim, as multidões, que antes ovacionavam Mussolini, agora, passaram a
aplaudir a liberdade recuperada e a sonhar com o fim da guerra.

Prisioneiro e isolado numa ilha do Mar Tirreno, na condição de ``ex"-chefe
do governo aprisionado para sua proteção contra a fúria
popular'',\footnote{Mussolini\emph{, Opera Omnia}, vol. \textsc{xxxiv}, p.\,295.}
como ele próprio definia, Mussolini meditou filosoficamente sobre a
volubilidade das massas: ``Das três almas, de que falou Platão, as
massas possuem somente as duas primeiras: a vegetativa e a sensível. Ela
não tem a alma que se eleva ao seu mais alto nível, a saber, a alma
intelectual. Por isso, para mim, não é difícil acreditar que milhões de
italianos, que me glorificavam até ontem, hoje, me odeiam e amaldiçoam o
dia em que nasci e a cidade onde vi a luz. Amaldiçoam todos os de minha
raça. Talvez, odeiem até meus antepassados, mas, certamente, odeiam a
mim e a todos de minha família que ainda estão vivos!''

\bigskip

\begin{flushright}
\textit{Emilio Gentile}
\end{flushright}

\chapter[Bolsonaro fala às massas, \emph{por Carlos Piovezani}]{Bolsonaro fala às massas \subtitulo{Do baixo clero político à presidência da República}}

\epigraph{A linguagem sempre revela o que uma pessoa tem dentro de si e
deseja encobrir, de si e dos outros, ou que conserva inconscientemente.
Uma pessoa pode fazer declarações mentirosas, mas o estilo deixará as
mentiras expostas}{\textsc{victor klemperer}}

\section{De capitão ganancioso a vereador populista}

\noindent{}No dia 03 de setembro de 1986, Jair Bolsonaro aparece pela primeira vez
no cenário público brasileiro. Publicara na seção ``Ponto de vista'' da
revista \emph{Veja} um artigo intitulado ``O salário está baixo''. Eis
algumas passagens desse artigo, cuja irrelevância o teria certamente
fadado ao esquecimento, não fosse a terrível ascensão política de seu
autor:

\begin{quote}
Há poucos dias a imprensa divulgou o desligamento de dezenas de cadetes
da Academia Militar das Agulhas Negras por homossexualismo, consumo de
drogas e uma suposta falta de vocação para a carreira. Em nome da
verdade, é preciso esclarecer que, embora tenham ocorrido casos
residuais de prática do homossexualismo,
consumo de drogas e mesmo
indisciplina, o motivo de fundo é outro. Mais de 90\% das evasões se
deram devido à crise financeira que assola a massa dos oficias e
sargentos do Exército brasileiro.

Agora, na Nova República, novamente sofremos uma grande perda salarial:
a maioria dos trabalhadores, através de lutas sindicais que nos são
expressamente proibidas, gozava de adiantamentos, trimestralidade, bônus
e outros ganhos que foram incorporados aos salários. Como não tínhamos
esse privilégio, perdemos novamente o equivalente a três meses de
inflação na época em que ela corroía consideravelmente o poder
aquisitivo da população.

Como capitão do Exército brasileiro, da ativa, sou obrigado pela minha
consciência a confessar que a tropa vive uma situação crítica no que se
refere a vencimentos.

Esse quadro é a causa sem retoques da evasão, até agora, de mais de
oitenta cadetes da \textsc{aman}. Eles solicitaram desligamento. Não foram
expulsos, como sugere o noticiário.

Não pleiteio aumento salarial. Reclamo --- como fariam, se pudessem, meus
colegas --- um vencimento digno da confiança que meus superiores
depositam em mim. Muitos reclamam da não tributação do imposto de renda
sobre vencimentos brutos dos oficiais e sargentos. Ora, se isso
ocorresse, depararíamos com a inconcebível circunstância de um aspirante
a oficial do Exército --- homem de elite e cheio de sonhos --- ter que
sobreviver com menos de 5.000 cruzados mensais.

Torno público este depoimento para que o povo brasileiro saiba a verdade
sobre o que está ocorrendo na massa de profissionais preparados para
defendê"-lo. Corro o risco de ver minha carreira de devoto militar
seriamente ameaçada, mas a imposição da crise e da falta de perspectiva
que enfrentamos é maior. Sou um cidadão brasileiro cumpridor de meus
deveres, patriota e portador de excelente folha de serviços. Apesar
disso, não consigo sonhar com as necessidades mínimas que uma pessoa do
meu nível cultural e social pode almejar. Amo o Brasil e não sofro de
nenhum desvio vocacional. Brasil acima de tudo.\footnote{Bolsonaro, J.
  ``O salário está baixo'', in \emph{Veja}, 03 de setembro de 1986, p.\,154.}
\end{quote}

Há nesse texto maquinações e construções de linguagem de Bolsonaro que
perdurarão ao longo do percurso que o conduziu do baixo clero político à
presidência da República. Uma delas é a produção da mentira contrafeita
com o que seria sua prerrogativa da verdade e com a denúncia de
falsidade de outrem, nesse caso, da imprensa. Outra é a tentativa de
constituição de um ethos investido de autenticidade, cujos benefícios
derivam dos valores positivos agregados a quem, supostamente impelido
por dever de consciência, tem coragem de confessar algo grave e de
enfrentar os riscos decorrentes de sua ação de dizer a verdade.
Bolsonaro mostra"-se disposto a um sacrifício, em nome do que seria uma
justa causa: ``um vencimento digno da confiança que meus superiores
depositam em mim''. Essa é a forma de escamotear o objetivo que ele
busca e nega manifestamente buscar.

O negacionismo é, portanto, mais um de seus procedimentos, atualizado
aqui nesta denegação: ``Não pleiteio aumento salarial''. A ele se somam
ainda seu interesse corporativo e a superestimação de si e dos seus.
``Homem de elite e cheio de sonhos'', ``portador de excelente folha de
serviços'' e ``uma pessoa do meu nível cultural e social'': a
insistência nessa superestimação tanto indica a insegurança de quem a
produz quanto revela a causa do menosprezo pelos de classe, grupo e
gosto distintos, tomados como inferiores. Enquanto o salário de um
capitão, uma das categorias para as quais Bolsonaro reivindicava aumento
salarial, era de 10.433 cruzados por mês, o salário mínimo era de 804
cruzados à mesma época. É por essa razão que as conquistas trabalhistas
serão chamadas de ``privilégios''. O mecanismo que vigora aí deriva de
visão excludente de direitos e repressora de desejos: se outros os têm,
luto para que sejam eliminados. Sob o pretexto de somente falar de
direitos e desejos alheios, o que se faz de fato é proceder à sua
condenação. As práticas de ``homossexualismo'', o consumo de drogas e os
direitos trabalhistas são o gozo de outrem, concebido como tormento.

``Brasil acima de tudo''. Como sabemos, esta última frase de seu texto
será mais do que nunca ouvida e reproduzida à exaustão trinta e dois
anos mais tarde. Dotada de relativa autonomia, a formulação poderia
figurar em diferentes passagens do texto e ainda ser dele extraída para
circular sob a forma de um \emph{slogan} em diversas outras
circunstâncias. Seu uso como ponto de chegada de uma peroração, onde bem
cabe o apelo patético, constrói o efeito de um grito patriótico que
arremata todo o nacionalismo exposto ao longo do texto. Ao ethos da
autenticidade se junta o do patriotismo, pois Bolsonaro integra uma
``massa de profissionais preparados para defender o povo brasileiro'' e
afirma em primeira pessoa do singular seu amor pelo Brasil. Além de lhe
agregar a virtude do patriotismo, a declaração de amor e o grito de
guerra dispensam"-no de concluir seu texto com um argumento. Em seu
lugar, vêm vagueza semântica, emoções e frases lapidares, que substituem
a elaboração de raciocínios.

Como metáfora e prenúncio do que viríamos décadas mais tarde, o grito
patriótico ufanista encarna um ato prototípico da linguagem fascista,
porque constrói uma identidade imaginária, reforça o amor pelos seus,
delineia uma alteridade com a qual se indispõe e fala não para se abrir
à réplica de um diálogo, mas para calar as vozes de uma pluralidade
democrática. A fanática declaração de amor é a escusa para o
silenciamento simbólico da diferença e para os gestos de violência
destinados aos que não compartilham desse amor e ainda mais aos que o
criticam. Diferentemente do debate que, ante as diferentes posições,
pode suprir o enfrentamento físico, um grito de guerra é a antecipação
do ataque e da morte do adversário. É linguagem de sinal fechado que
quer calar a linguagem de sinal aberto, pela qual a diferença pode
existir e se fazer ouvir.

A reclamação pelos baixos salários publicada na \emph{Veja} custou a
Bolsonaro um processo militar e uma leve punição que o levou a uma
rápida passagem pela prisão. Pouco mais de um ano mais tarde, as páginas
da revista seriam novamente um foco de luz sobre Bolsonaro, projetando
novamente seu nome para parte da sociedade brasileira. Uma vez que o
protesto e a reivindicação por melhores salários não foram ouvidos,
Bolsonaro e outros militares passariam a uma ação bem mais contundente:
``Falamos, falamos, e eles não resolveram nada''. Na edição de
\emph{Veja} publicada no dia 25 de outubro de 1987, a jornalista Cassia
Maria revelou os bastidores da operação ``Beco sem saída'', que
consistia num plano de explosão de bombas na Escola de Aperfeiçoamento
de Oficiais, na Academia Militar das Agulhas Negras e em outras
dependências militares, tal como lhe foram relatados por capitães
envolvidos na operação. No conforto do segredo, Bolsonaro destrata seus
superiores: ``São uns canalhas''. Sua coragem não resistiria, se posta à
prova. O relato de Cassia Maria se encerra com a insistência e a ameaça
do então capitão para que nada daquilo fosse publicado: ``Nervoso,
Bolsonaro advertiu"-me mais uma vez para não publicar nada sobre nossas
conversas: `Você sabe em que terreno está entrando, não sabe?',
perguntou''.\footnote{``Pôr bombas nos quartéis, um plano na Esao'', in
  \emph{Veja}, 25 de outubro de 1987, p.\,40--41.}

Bolsonaro fala para intimidar e tentar calar a exposição de uma verdade
a seu respeito. Nem essa sua atitude autoritária nem o grave plano de
explosão de bombas foram suficientemente denunciados e devidamente
punidos. Entre abril e junho de 1988, apesar das evidências factuais e
da condenação interna na instância designada Conselho de Justificação,
ele nega a existência da operação ``Beco sem saída'', protocola defesa
junto ao Superior Tribunal Militar e é absolvido. Todo o imbróglio
precipita o fim de sua carreira militar, culminando na reserva
remunerada como capitão do Exército. Ao invés de uma punição, seu
extremismo lhe rendeu esse e outros maiores benefícios. Em novembro
daquele mesmo ano, Bolsonaro seria eleito vereador da cidade do Rio de
Janeiro pelo Partido Democrata Cristão com 11.062 votos.

Sua primeira conquista eleitoral deveu"-se em larga medida à sua defesa
das causas militares. As reclamações de aumento salarial lhe expuseram a
embates e desgastes com os comandantes, mas também lhe proporcionaram
uma projeção e alguma admiração entre os militares de menor patente, que
passavam a vê"-lo como um porta"-voz que podia representá"-los. Mas
Bolsonaro não alcançaria essa condição sem a visibilidade midiática de
que começou a gozar com o artigo e com a reportagem da revista
\emph{Veja}. A exposição na mídia de um sujeito que se mostrava disposto
não somente a falar em nome dos seus, mas também a tomar medidas
extremas para a defesa dos interesses de graduados, oficiais subalternos
e oficiais intermediários, foi um fator decisivo no início de sua
ascensão política.

Missão dada é missão cumprida. Uma vez que Bolsonaro fora eleito com a
pauta da defesa dos militares, dele não se esperava que fizesse nem mais
nem menos do que isso como vereador. Foi"-lhe atribuída essa missão e ele
a cumpriu a seu modo: defendeu os militares com relativamente poucos,
mas veementes e inflamados pronunciamentos. Além de essa constante e
virulenta defesa continuar a lhe promover, tal como fizera com seu
projeto de lei em benefício do transporte público gratuito para os
militares, Bolsonaro já começara a se notabilizar pelo patrocínio da
moralidade e pelo destempero. Como a legislatura precedente havia sido
maculada por denúncias de clientelismo, empreguismo e regalias, a Câmara
havia se renovado: ``dos 42 eleitos, 29 estreavam na casa''. Nesse
cenário, o mote dos discursos de Bolsonaro ``era o do resgate da
moralidade''.\footnote{Saint"-Clair, Clóvis. \emph{O homem que peitou o
  exército e desafia a democracia}. Rio de Janeiro: Máquina de Livros,
  2018, cap.\,52--53.} Entre dezembro de 1989 e fevereiro de 1990, o
vereador do \textsc{pdc} frequentou a imprensa com estas pautas: voto contrário
ao aumento de \textsc{iptu}; denúncia da criação da Companhia Municipal de
Energia e Iluminação (RioLuz) e do que seria sua escusa função de
distribuir cargos a aliados políticos do executivo fluminense; e
acusação de que a Prefeitura do Rio estaria cobrando aluguéis irrisórios
de prédios públicos situados em áreas nobres da cidade e locados por
empresários aliados do prefeito.\footnote{Saint"-Clair, 2018, p.\,55.}

Na sessão da Câmara do dia 27 de março de 1990, a assembleia apreciava
uma emenda do vereador Alfredo Sirkis, do Partido Verde. O que se
propunha com ela era a extinção do artigo que punia o vereador que
depreciasse a imagem da Câmara com sua cassação.

\begin{quote}
Convocado para fazer a chamada dos parlamentares, Bolsonaro irritou"-se
quando o vereador Américo Camargo (\textsc{pl}) decretou a manutenção do artigo,
com o voto de minerva que determinou o placar de 18 a 17. Deu um soco na
mesa e vociferou:

--- Quero ver quem tem coragem de me cassar neste plenário!

Foi o bastante para que instaurasse uma grande confusão envolvendo o
Bloco Progressista, formado basicamente por integrantes dos partidos de
esquerda, e o centrão, liderado pelo vereador Maurício Azêdo (\textsc{pdt}). Esse
já havia xingado no início da sessão Chico Alencar (\textsc{pt}) e Francisco
Milani (\textsc{pcb}), presidente da Câmara, que abandonou os trabalhos em
protesto contra o que considerou `falta de respeito'. Involuntariamente,
o soco de Bolsonaro na mesa soou como gongo, e o plenário se transformou
num ringue. Azêdo partiu para cima de Alencar e lhe acertou um murro no
rosto. Ivo da Silva (\textsc{ptr}) pulou diversas cadeiras para atacar Guilherme
Haeser (\textsc{pt}).\footnote{Saint"-Clair, 2018, p.\,56.}
\end{quote}

Sua condição de protagonista e incentivador de ações violentas somada
aos expedientes populistas e à projeção midiática concorreram para o
crescimento de sua popularidade. Bolsonaro daria muita visibilidade à
sua recusa de pagamento por participar das sessões extraordinárias da
Câmara: ``Nós já recebemos uma remuneração razoável para exercer o
mandato, e não precisamos de expedientes desse tipo. Isso é
vergonhoso!''. E faria o mesmo com a devolução de cinco mil cartões de
Natal que recebera para enviar a seus eleitores: ``Desejar votos de boas
festas às custas do contribuinte é uma afronta à sociedade''. Posturas e
discursos populistas e agressivos tanto mais eficientes quanto mais
fossem repercutidos pela mídia:

\begin{quote}
De resto, o capitão reformado utilizou com frequência a seção de cartas
dos leitores dos jornais como tribuna para desferir ataques ao \textsc{pt};
denunciar os baixos salários da tropa e das pensões de ex"-combatentes; a
falta de isonomia entre os vencimentos da \textsc{pm} e dos Bombeiros em relação
ao soldo dos militares das Forças Armadas; as condições precárias dos
hospitais militares e do Fundo de Saúde do Exército.\footnote{Saint"-Clair,
  2018, p.\,57.}
\end{quote}

Defender com unhas e dentes os interesses dos seus, atacar os
concorrentes políticos, recusar prerrogativas e conseguir divulgação de
tudo isso na imprensa eram os meios pelos quais Bolsonaro se fazia cada
vez mais conhecido. A possibilidade de perder aquela via de projeção
midiática fez o vereador do \textsc{pdc} buscar outro meio para continuar
frequentando o noticiário. Os jornais haviam decidido suspender a
publicação de mensagens de políticos na seção de cartas dos leitores.
Diante disso, ``Bolsonaro descobriu um modo mais eficaz de garantir
espaço na mídia'', para continuar em sua ``defesa das pequenas e grandes
bandeiras que levanta em nome do que há de mais conservador ou
reacionário'': o vereador populista ``deixa sua impulsividade falar mais
alto e vocifera em lugar de argumentar''. Com frequência, fazia"-o
``distribuindo coices nos adversários políticos, ao melhor estilo do
Cavalão''.\footnote{Saint"-Clair, 2018, p.\,57.}

No final de 1973, o então jovem Bolsonaro havia passado no concurso de
admissão da Academia Militar as Agulhas Negras. Já na academia, o
aspirante defrontava"-se com suas limitações e com as dificuldades para
ser aprovado: ``O nível das disciplinas era bem maior do que aquele com
o qual estava acostumado. O cadete sofria nas aulas de Geometria
descritiva e pensou em desistir''. Incentivado pelo pai, ``Jair Messias
deu seu jeito e passou de ano. Destacava"-se mais, porém, nas atividades
físicas. Foi recordista na corrida de 4 Km fardado e começou a competir
no pentatlo. Seu vigor lhe valeu o apelido de Cavalão --- como os
militares se referem a quem exibe bom porte físico''.\footnote{Saint"-Clair,
  2018, p.\,23--24.} Os reforços positivos dessa sua disposição física e
as dificuldades enfrentadas no plano intelectual provavelmente
contribuíram para que se instalasse mais ou menos precocemente em
Bolsonaro o que vimos ser um dos traços do programa pedagógico de Hitler
e um dos traços do fascismo: a predileção pelo preparo físico e pelas
ações, em detrimento da reflexão, da formação intelectual e dos
conteúdos filosóficos e científicos, que ficam relegados a um último
plano e são concebidos com desconfiança e desprezo.

A conveniência dessas comparações entre os fascistas europeus do começo
do século \textsc{xx} e o representante de nosso neofascismo que venceu as
eleições presidenciais em 2018 não deve se impor sobre nosso objetivo de
retraçar uma narrativa histórica de Bolsonaro, que passa por três
momentos decisivos de sua carreira política: sua atuação como deputado
federal, seu desempenho eleitoral como candidato à presidência da
República e o início do cumprimento de seu mandato como presidente da
nação. Por essa razão, somente recorreremos a essas eventuais
comparações para que elas esclareçam propriedades mais ou menos comuns
da linguagem fascista. Iremos empreendê"-las ou ressaltá"-las, buscando
não descurar das especificidades históricas e dos fatores sociais
envolvidos nas condições de produção dos discursos de Bolsonaro em cada
um desses contextos.\footnote{O anti"-intelectualismo é um dos traços do
  fascismo e de sua linguagem. Manifesta"-se em Bolsonaro como se fosse
  um traço idiossincrático. Mas, na formação do anti"-intelectualismo
  deste último e dos bolsonaristas, há fatores decisivos da história
  brasileira, conforme veremos.} Tentaremos proceder do mesmo modo no
estabelecimento de relações entre as diferentes fases da carreira
política de Bolsonaro.

O soco na mesa, as vociferações do vereador populista e a repercussão
midiática de suas ações e pronunciamentos na Câmara municipal do Rio de
Janeiro talvez prenunciem, mas não são idênticos aos seus atos e
discursos cada vez mais investidos de poder, de efeitos nocivos e de
visibilidade, à medida que Bolsonaro ascende em sua trajetória política.
Com efeito, os poderes e os perigos de fascismos e neofascismos não se
limitam à sua linguagem. Não há dúvida de que as agressões e os
extermínios ultrapassam as ações linguísticas. A linguagem fascista não
pode, porém, ser subestimada. São as versões fascistas da história que
promovem a progressão da anuência de discursos de ódio e de atos
violentos e fatais. Sua eficácia reside não somente no que contam, mas
também em suas maneiras de contar. É por essa razão que os esforços que
empreendemos aqui para compreender a ascensão bolsonarista se soma
àqueles já realizados por cientistas políticos, filósofos e
historiadores, mas concentram"-se nas propriedades e transformações da
linguagem fascista, tal como ela se atualizou na boca de Bolsonaro em
diversas circunstâncias, concorrendo decisivamente para elevá"-lo da
política do baixo clero e do abjeto e atraente entretenimento midiático
à presidência da República.

\section{O deputado falastrão}

Abaixo da mediocridade. Não se poderia classificar de outro modo o
desempenho de Bolsonaro como deputado federal ao longo de quase três
décadas. Essa sua irrisória atuação é bastante conhecida\footnote{Conforme
  se pode verificar na síntese de sua biografia e de sua atuação como
  deputado divulgada pela Câmara Federal; e conforme ainda várias reportagens da imprensa. Entre estas
  últimas, ver, por exemplo,
  ``Após 25 anos de Congresso, Bolsonaro consegue aprovar 1ª emenda; `Sou discriminado'\,'', \textit{\textsc{bbc} News}, 17 de junho de 2015.}
e contrasta com toda a visibilidade que ele conseguiu adquirir sobretudo
em seus dois últimos mandatos tanto com suas defesas dos militares e da
ditadura de 1964, de punições cada vez mais severas na área de segurança
pública e de pautas pró moral e bons costumes quanto com seus ataques a
programas sociais e políticas afirmativas, a ideologias igualitárias e
aos direitos humanos. Se as coisas ditas nessas defesas e nesses ataques
foram necessárias para a projeção alcançada por Bolsonaro, elas não
seriam suficientes para alçá"-lo além de sua irrelevante atuação no
Congresso. Para tanto, foram fundamentais suas maneiras de dizer, a
ampla difusão midiática do que disse, a falta de devidas punições às
suas quebras de decoro, o reforço de um relativo consenso conservador e
a constituição de uma espiral de silêncio nos setores progressistas.

Bolsonaro deixa a condição de político insignificante e se torna
porta"-voz do pensamento reacionário no Brasil graças ao substrato
autoritário nas ações e no imaginário brasileiro, à nossa história de
atraso na redução de injustiças e desigualdades sociais e à sua
conformidade com a lógica do espetáculo que vigora em nossa mídia. As
declarações ofensivas e os pronunciamentos agressivos ganharam cada vez
mais repercussão e popularidade, porque se encaixam em consensos
compartilhados por boa parte da população, para cuja formação a grande
mídia diversionista contribui decisivamente, e porque causam
controvérsias e dão grande audiência aos veículos que lhes abrem espaço.
Depois de uma atuação relativamente discreta em sua estreia no cenário
político em Brasília, já em segundo mandato como deputado, Bolsonaro
``se sentiu mais à vontade para adotar a estratégia de verbalizar
declarações polêmicas para garantir mais espaço na mídia, revelando seu
desprezo pela democracia e pelos direitos humanos''. Quando do massacre
do Carandiru, em 02 de outubro de 1992, em que 111 detentos foram mortos
pela Polícia Militar, ``Bolsonaro vociferou: Morreram poucos. A \textsc{pm} tinha
que ter matado mil!''.\footnote{Saint"-Clair, 2018, p.\,69--70.} No ano
seguinte, ele ainda diria publicamente fora e dentro da própria Câmara
que era favorável ao fechamento do Congresso. A atitude abertamente
antidemocrática não lhe rendeu mais do que uma mera advertência.

São inúmeros os programas de entretenimento veiculados no rádio e na
televisão que o entrevistaram, são igualmente muito numerosas as
entrevistas concedidas por Bolsonaro a telejornais e as retransmissões
no rádio, na tevê e, mais recentemente, nas redes sociais de passagens
particularmente espetaculares de seus pronunciamentos nas sessões do
Congresso Nacional. Aproveitando"-se dessa lógica do espetáculo
midiático, Bolsonaro adotou deliberadamente o estilo impulsivo em suas
intervenções, o alto volume de voz, a tensão e a agressividade na
produção de suas falas, com os quais buscou atrair ainda mais o
interesse do público conservador e a atenção da imprensa irresponsável.
O próprio deputado o admitiu, embora tenha dourado bastante a pílula, ao
fazê"-lo, chamando de ``contundência'' a virulência de seus discursos:
``Sem contundência ninguém é ouvido. Temos excelentes deputados que
expressam suas ideias de forma polida e por isso não encontram eco na
mídia. Minhas declarações vendem jornais e revistas e dão audiência no
rádio e na \textsc{tv}''.\footnote{Bolsonaro, Jair. ``Sou preconceituoso, com
  muito orgulho''. Revista \emph{Época}, 02 de julho de 2011.}
Por essa razão, diferentemente da estratégia de outros políticos e de
outras pessoas públicas, que se caracteriza por mostrar ponderação e
bom"-senso na exposição de seus julgamentos ou por mais simplesmente
fugir de temas que dividem as opiniões, Bolsonaro investe na promoção e
na manutenção de controvérsias nas quais se posiciona aberta e
excessivamente e se apresenta como um dos principais porta"-vozes da
ideologia conservadora: ``Tenho paz na consciência e falo o que penso e
tenho apoio de considerável parcela da sociedade''.\footnote{Bolsonaro in
  revista \emph{Época} (2011).} Assim, ele produziu surpresa e
indignação entre alguns de nós, mas também foi considerado um político
excêntrico por outros tantos e ainda se estabeleceu solidamente como um
legítimo representante da extrema"-direita para uma terceira e não
negligenciável parcela da população.

A despeito de seus discursos em defesa da família, da moral e dos bons
costumes e do governo militar já lhe terem rendido alguma projeção,
Bolsonaro continuava numa espécie de limbo político para boa parte da
sociedade brasileira no curso de seus primeiros mandatos como deputado
federal. Foi principalmente na passagem do quinto para o sexto mandato,
e de modo particular a partir deste último, que ele adquiriu um
considerável protagonismo na oposição ao projeto que tornava crime a
homofobia e na derrota que ajudou a impor ao Ministério da Educação, que
pretendia distribuir material anti"-homofóbico nas escolas, em 2011. Para
a promoção de seu nome, como sempre, a cobertura e a repercussão
midiática foram fundamentais. Destacamos aqui somente algumas dessas
inúmeras circunstâncias nas quais grandes veículos da imprensa, do rádio
e da televisão consagraram atenção e espaço para a manifestação de sua
posição: ``O deputado Jair Bolsonaro (\textsc{pp"-rj}) protagonizou um novo
bate"-boca na Câmara nesta quarta"-feira, ao criticar homossexuais durante
uma audiência pública sobre segurança pública. Bolsonaro voltou a dizer
que nenhum pai pode `ter orgulho de ter um filho gay' e atacou o `kit
gay', material anti"-homofobia que o Ministério da Educação estuda
distribuir às escolas''.\footnote{Guimarães, Larissa. ``Bolsonaro volta a
  atacar `kit gay' do Ministério da Educação'', \emph{Folha de S.\,Paulo},
  Poder, 27 de abril de 2011.}
Aproximadamente três meses depois, como vimos, a revista \emph{Época}
promovia uma entrevista de Bolsonaro concedida a seus leitores. Nela, o
deputado disse o seguinte:

\begin{quote}
Minha luta vitoriosa no Congresso foi contra a distribuição do kit gay
nas escolas do 1º grau. Não podia me omitir diante do material que
estimulava nossos meninos e meninas a ser homossexuais. E deviam se
orgulhar dessa condição. No mais, tudo é demagogia, pois certamente não
acredito que nenhum pai possa se orgulhar de ter um filho gay. (\ldots{}) Se
lutar para impedir a distribuição do kit"-gay nas escolas de ensino
fundamental com a intenção de estimular o homossexualismo, em verdadeira
afronta à família, é ser preconceituoso, então sou preconceituoso, com
muito orgulho.
\end{quote}

Essa passagem da entrevista contém uma série de recursos e expedientes
de linguagem de que Bolsonaro já se valia desde quando era um vereador
quase anônimo e um deputado caricato e de que continuaria a se valer em
sua escalada para a condição de político bastante conhecido. O então
deputado pelo Partido Progressista forja a personalização belicosa de
sua atuação, que lhe rendera uma vitória e, ao precisar o adversário
combatido em sua batalha, emprega uma expressão que se cristalizaria,
graças à sua condição de fórmula fácil, simplista e muito pregnante.
Prova de sua grande pregnância foi o fato de que a expressão tenha
passado a circular e a ser reproduzida abundantemente na imprensa e
eventualmente mesmo entre alguns partidários mais desavisados de
ideologias igualitárias e inclusivas.

Além disso, Bolsonaro constrói um simulacro grosseiro, mas bastante
eficiente, da posição antagonista, mediante o qual o material
anti"-homofobia é transformado em ``kit gay'', cujo objetivo seria o de
estimular ``nossos meninos e meninas a ser homossexuais''. Há ainda o
uso de um argumento disjuntivo falacioso, segundo o qual não haveria
outra opção além destas duas opostas: ou se teria vergonha ou orgulho de
``ter um filho gay''. Assim, sua formulação exclui a possiblidade de que
a sexualidade não seja motivo nem de uma nem de outro.\footnote{Um dos
  traços do ``fascismo eterno'' seria o seguinte: ``Como tanto a guerra
  permanente quanto o heroísmo são jogos difíceis de jogar, o
  Ur"-Fascista transfere sua vontade de poder para questões sexuais. Esta
  é a origem de seu machismo (que implica desdém pelas mulheres e uma
  condenação intolerante de hábitos sexuais não conformistas, da
  castidade à homossexualidade. Como o sexo é também um jogo difícil de
  jogar, o herói Ur"-Fascista joga com as armas, que são seu Ersatz
  fálico: seus jogos de guerra se devem a uma \emph{invidia penis
  permanente}''. (Eco, 2018, p.\,54--55).} Finalmente, identificamos não
somente uma admissão, mas uma ostentação de seu preconceito, justificado
por sua equivalência semântica com o compromisso de proteção à família.

Poucos dias antes dessa entrevista à revista Época, Bolsonaro havia
participado do quadro ``O povo quer saber'' do programa de
entretenimento \textsc{cqc} exibido pela \emph{\textsc{tv} Bandeirantes}, tal como
fizera várias vezes nesse mesmo e em outros programas humorísticos e de
distração popularesca de rádio e televisão. Sua interlocução com a
cantora Preta Gil na segunda"-feira, dia 28 de março de 2011, lhe
proporcionou uma enorme visibilidade midiática. No momento em que a
cantora lhe perguntou qual seria sua reação, caso um de seus filhos
começasse a se relacionar com uma mulher negra, Bolsonaro lhe respondeu
assim: ``Preta, não vou discutir promiscuidade com quem quer que seja.
Eu não corro esse risco porque meus filhos foram muito bem educados e
não viveram em ambiente como lamentavelmente é o teu''.\footnote{``Preta
  Gil quer processar deputado por comentário racista'', \textit{Folha de S.\,Paulo}, Ilustrada, 29 de março de 2011.}
O episódio lhe renderia ainda muito mais projeção, tendo em vista a
grande repercussão na imprensa e nas redes sociais de seus
desdobramentos jurídicos. Isso porque, conforme ocorrera em tantos
outros casos, a postura agressiva nos embates, as declarações polêmicas,
os insultos e os discursos discriminatórios, tais como os racistas,
homofóbicos e anti"-humanitários, foram alvo de representações e
processos no Conselho de Ética da Câmara, em que se apresentaram
acusações de quebra do decoro parlamentar e pedidos de cassação do
mandato, e ainda em outras instâncias judiciais.

Nada disso conteve o deputado falastrão. Pelo contrário, Bolsonaro
investiu cada vez mais nessa sua persona. Havia um lastro histórico e um
bônus eleitoral que lhe faziam seguir nessa direção. Desde sua não
punição exemplar ao desrespeitar o alto comando do Exército e ao
empreender seu plano de explodir bombas em instalações militares para
forçar um aumento de salário até suas absolvições ou condenações sem
maiores consequências em vários outros processos, o capitão e depois o
vereador e o deputado se viram livres para continuar em seus ímpetos
extremistas. Mais do que a liberdade para fazê"-lo, Bolsonaro conseguira
com eles dividendos econômicos e políticos. Foi para a reserva com a
garantia do recebimento vitalício de seu soldo como capitão, elevou"-se à
condição de porta"-voz dos militares e foi eleito vereador na cidade do
Rio de Janeiro e, em seguida, deputado federal. A polêmica, a
agressividade e a grosseria tornaram"-se cálculo eleitoral e marketing
político.\footnote{Denunciados como tais desde o início da ascensão de
  Bolsonaro, quando já começara a deixar as sombras do baixo clero do
  Congresso Nacional: Nogueira, Marco Aurélio; Paulino, Mauro. ``Efeito
  Bolsonaro: discurso polêmico esconde cálculo eleitoral e marketing
  político''. \emph{Tv Folha}, \emph{Youtube}, 05 de abril de 2011.}

Antes de alcançar a celebridade política, Bolsonaro já empregava uma
linguagem eivada de traços fascistas. Contando com os efeitos positivos
que causaria ao menos junto a um considerável contingente que girava em
torno de 15\% da população brasileira,\footnote{Em pesquisas realizadas
  pelo \emph{DataFolha} nos anos de 2006 e 2010, os dados indicam que
  16\% dos entrevistados na primeira e 14\% na segunda identificam"-se
  como alguém que se posiciona na extrema"-direita do espectro político.
  ``Brasileiros se colocam mais à direita'', \textit{DataFolha}, 31 de maio de 2010.}
quando já estava em seu terceiro mandato, o então deputado federal pelo
Partido Progressista Brasileiro aproveitava"-se de qualquer oportunidade
em que pudesse alastrar seus radicalismos e construir a imagem de um
homem autêntico, que falaria o que pensa e que o faria sem papas na
língua, de maneira crua e vulgar. É o que ocorreu na entrevista que
Bolsonaro concedeu ao programa ``Câmara Aberta'' da \emph{\textsc{tv}
Bandeirantes}, no dia 23 de maio de 1999. Eis abaixo somente três
pequenos trechos dessa entrevista:

\begin{quote}\parindent=0em
\textbf{\textsc{entrevistador}: Você não acha que a \textsc{cpi} tem esse papel? e às
vezes até confunde o papel da \textsc{cpi}, quando vai alguém depor no Congresso
Nacional?}

\textsc{bolsonaro}: É o seguinte xará: tapa na mesa, querer ir pra
porrada, não é o caso. Dá porrada no Chico Lopes\ldots{} Eu até sou favorável
que na \textsc{cpi} no caso do Chico Lopes que tivesse pau de arara lá. Ele
merecia isso, pau de arara. Funciona. Eu sou favorável à tortura, tu
sabe disso. E o povo é favorável a isso também.

\medskip

\noindent\textbf{Se você fosse hoje o presidente da República,
você fecharia o Congresso Nacional?}

Não há a menor dúvida. Daria um golpe no mesmo dia,
no mesmo dia. Não funciona. E tenho certeza que pelo menos 90\% da
população ia fazer festa e bater palma. Porque não funciona. Pra quê? O
Congresso hoje em dia não serve pra nada, xará. Só vota o que o
presidente quer. Se ele é a pessoa que decide, que manda, que tripudia
em cima do Congresso, que dê logo o golpe, pô! Parte logo pra ditatura.
Agora, não vai falar em ditadura militar aqui. Só desapareceram 282, a
maioria marginais, assaltantes de bancos, sequestradores.

\medskip

\noindent\textbf{O senhor tem esperança, o senhor tem futuro? O
senhor imagina, o senhor vê o Brasil num lugar melhor? O senhor acredita
neste país? De que maneira o senhor enxerga o Brasil de todos nós?}

Só com crise, né? Não é com crise que cresce, essa
palhaçada que a gente vê na imprensa por aí, que é propaganda paga do
governo, o dinheiro de você contribuinte. Só com uma crise seríssima. Me
desculpa, né, mas através do voto, você não vai mudar nada nesse país.
Nada! Absolutamente nada! Só vai mudar, infelizmente, no dia em que nós
partirmos para uma guerra civil aqui dentro. E fazendo o trabalho que o
regime militar não fez, matando uns 30 mil. Começando com o \textsc{fhc}. Não
vamos deixar ele pra fora, não. Matando. Se vai morrer alguns inocentes,
tudo bem. Em tudo e quanto é guerra, morre inocente. Eu até fico feliz,
se eu morrer, mas desde que vá outros 30 mil, outros junto comigo,
né?\footnote{Bolsonaro, Jair. Entrevista concedida ao programa ``Câmara
  Aberta'' da \textsc{tv} Bandeirantes no dia 23 de maio de 1999. ``Jair Bolsonaro Defendendo Guerra Civil, Fim do Voto e Fechamento de Congresso'', \textit{Youtube}, 10 de abril de 2016.}
\end{quote}

No plano de seu conteúdo, tudo é bastante explícito. Bolsonaro é
partidário e mesmo entusiasta da violência física e da morte de
adversários, ao mesmo tempo em que nutre um enorme desprezo pela vida
humana. Desrespeita e afronta as instituições do regime democrático e
mostra"-se aficcionado pela ditadura militar. Além disso, sua fala produz
o efeito de incitação a uma guerra civil e apresenta a condenação à
morte de 30 mil pessoas como condição necessária para uma mudança no
país. Finalmente, destaca"-se em mais de uma passagem a construção de sua
qualidade de intérprete privilegiado e porta"-voz do povo: ``E o povo é
favorável a isso também''; ``pelo menos 90\% da população ia fazer festa
e bater palma''. Ocupando essa posição de representante do povo,
Bolsonaro busca instaurar um sentimento de identidade e de pertença a um
grupo, com base na reprodução de clichês que circulam no senso"-comum: o
povo é favorável à tortura, porque bandido deve ser punido, pra tomar
vergonha na cara; o povo é favorável ao fechamento do Congresso, porque
sabe que político é tudo corrupto.

Os efeitos de pertença ao grupo e de sua consolidação tornam"-se bem mais
presentes, à medida que o plano do conteúdo é estendido e reforçado pelo
da expressão. A direção do olhar de Bolsonaro que oscila fluentemente
entre o entrevistador e a objetiva da câmera, como se se dirigisse não
apenas ao interlocutor direto, mas também ao telespectador, o balanço
relativamente constante de sua cabeça e a aparente naturalidade dos
gestos de seus braços e de suas mãos dão a impressão de que o deputado
está numa conversa cotidiana, que lhe permite falar francamente de temas
dos quais fugiria a maioria de seus colegas congressistas. Já seus meios
de expressão verbal são fundamentais para a constituição do que se
apresenta como uma fala espontânea, sincera e autêntica. Para produzir
esses efeitos, Bolsonaro emprega uma linguagem simples, clara, direta e
figurada. Uma ocorrência típica dessa linguagem é a seguinte: ``E tenho
certeza que pelo menos 90\% da população ia fazer festa e bater palma''.
Isso porque, no lugar dessa formulação, o deputado poderia ter dito ``E
estou certo de que pelo menos 90\% da população apoiaria a medida com
entusiasmo''. Certos elementos prosódicos de sua fala contribuem para
tornar seu modo de dizer ainda mais autêntico e verdadeiro. A efetiva
pronúncia dessa mesma frase dita por Bolsonaro se aproxima desta
transcrição: ``I tenhu certeza qui pelo menus 90\% da população ia fazê
festa i batê palma''. Em sua fala, há tanto certa distensão e
espontaneidade, como se Bolsonaro experimentasse a tranquilidade de quem
sabe que diz a verdade, quanto a tensão e a virulência, como se sentisse
a revolta de quem sabe que a política é um antro de maldades. A despeito
de sua pertença ao campo há décadas, Bolsonaro não se identifica com
ele, o denuncia e prega a morte de boa parte de seus integrantes,
sobretudo, de seus oponentes políticos.

O enredo de desrespeitos, ataques e mentiras sem a devida
punição\footnote{Com base nas declarações de Bolsonaro na entrevista
  concedida ao ``Câmara Aberta'', tanto Antônio Carlos Magalhães, então
  presidente do Senado, quanto o próprio Fernando Henrique Cardoso,
  então presidente da República, nada mais fizeram do que manifestar seu
  repúdio às falas do deputado do \textsc{ppb}. Este texto do Senado deixava
  claro que o alto escalão da política brasileira não desconhecia o
  conteúdo das afirmações de Bolsonaro: ``Câmara estuda processo contra
  Bolsonaro. Deputado defendeu na \textsc{tv} fechamento do Congresso e disse que
  \textsc{fhc} deveria ter sido fuzilado pelos militares''.}
se repete e estimula Bolsonaro a conservar suas crenças e seus modos de
proceder e a aprofundar e aperfeiçoar suas estratégias políticas e
eleitorais. Em outras tantas ocasiões, o deputado boquirroto agiu do
mesmo modo e, assim, continuava a assistir ao incessante aumento de sua
popularidade. Aproximadamente um ano e meio depois de ter sido um dos
protagonistas da suspensão da distribuição de material anti"-homofobia no
início de 2011, Bolsonaro voltaria à cena política e midiática como um
dos principais atores em uma audiência na Comissão de Direitos Humanos
no Congresso, realizada no dia 28 de junho de 2012. A pretexto de
efetuar mais uma de suas defesas da família, manifestava"-se ali
novamente sua obsessão com a sexualidade alheia. Vejamos uma passagem
particularmente eloquente de sua grosseira perseguição à
homossexualidade:

\begin{quote}
Presidente, o comandante Jean Willys abandonou a tropa de homossexuais.
E a tropa de homossexuais está batendo agora em retirada. São
heterofóbicos. Quando veem um macho na frente, eles ficam doidos. O que
tá em jogo neste país aqui é a esculhambação da família. É isso que tá
em jogo. E são tão covardes, que atacam lá nas criancinhas, a partir de
3, 4 e 5 anos de idade. E o que eu falo aqui tá previsto em plano de
governo. Num é palavra minha, não. Fizeram aqui ó, no dia 15 de maio, o
\textsc{ix} Seminário \textsc{lgbt} infantil. Canalhas! Canalhas! Emboscando crianças nas
escolas. Canalhas, mil vezes! Homossexualismo? Direitos? Vai queimar tua
rosquinha onde tu bem entender, porra! Eu não tenho nada a ver com isso.
Não queiram estimular crianças, filhos de vocês aqui, humildes, que
ganham um salário mínimo, tão recebendo uma carga de material
homoafetivo na escola.
\end{quote}

Com vistas a não legitimar a fala de Bolsonaro com sua presença, o
deputado Jean Willys, do Partido Socialismo e Liberdade, conhecido
militante das causas \textsc{lgbtq}s, decide sair da sala em que ocorria a
audiência. Incomodado com a evasão de seu adversário político, Bolsonaro
o provoca, assim como importuna todos os demais adeptos das pautas
anti"-homofobia, debochando do próprio deputado e dos homossexuais, de
modo geral. Ele ironiza a saída de Willys por meio de uma alegoria pela
qual trata a homossexualidade e os homossexuais com termos do campo
semântico militar. Sua tentativa consiste em estabelecer o que lhe
parece ser o contraste entre a virilidade do universo das armas e a
debilidade do campo homossexual. Segundo Bolsonaro, aos homossexuais, em
geral, e ao deputado Jean Willys, em particular, faltaria coragem para
enfrentar uma batalha e, por isso, eles seriam menos homens que os
heterossexuais. Nesse extremismo da ótica conservadora, os homossexuais
são seres humanos reduzidos à sua sexualidade. Eles desprezariam os
heterossexuais, porque seriam ``heterofóbicos''. Porque completamente
controlados e descontrolados por seus instintos sexuais e por sua
covardia, ``quando veem um macho na frente, eles ficam doidos''.

Os usos da linguagem e a construção de uma narrativa novamente
desempenham função primordial no ataque de Bolsonaro. Nada mais claro do
que os termos empregados para designar e classificar os homossexuais:
``doidos'', ``covardes'' e ``canalhas''. Este último não só é repetido,
mas é também pronunciado aos gritos. Opção prosódica não muito distinta
de outras passagens de sua intervenção nas quais o deputado do \textsc{ppr}
vocifera, ao invés de falar. Também as ações atribuídas aos homossexuais
os constroem como se fossem membros de uma categoria execrável: eles
abandonam a tropa, ficam doidos ao verem um macho, esculhambam a
família, atacam criancinhas e as emboscam nas escolas. Assim, Bolsonaro
produz um contraste entre famílias indefesas e criancinhas, de um lado,
e os devassos e perversos homossexuais, de outro.

Seus estouros e suas denúncias apresentam"-se como uma corajosa defesa de
desamparados. Além disso, o deputado abusa indiscriminadamente de
notícias e informações falsas, envoltas por efeitos de verdade,
produzidos por um gesto dêitico, pela menção a uma data precisa e pela
referência mais ou menos verossimilhante a um seminário com edição e
título precisos. Por fim, ao encaminhar"-se para o arremate de sua fala,
Bolsonaro simula uma modificação de seu interlocutor, aparentando
dirigir"-se diretamente aos homossexuais. É assim que o deputado
contrafaz o que supõe ser um gesto franco e corajoso e é nesse exato
momento que ele leva sua linguagem chula ao mais baixo nível da
grosseria.\footnote{Para saber mais sobre o constante comportamento
  homofóbico de Bolsonaro, ver: ``Gays sempre na mira'' in Saint"-Clair,
  2018, p.\,81--102.} Ele o faz antes de produzir outra alteração de seu
interlocutor, quando passa a falar com os ``humildes'' pais das crianças
atacadas e emboscadas pelos gays.

Outros temas constantemente explorados por Bolsonaro e que lhe renderam
projeção midiática e identificação com setores conservadores são o
controle de natalidade das classes populares, a pena de morte, a
diminuição da maioridade penal e o combate aos direitos humanos, de modo
geral. No final de 1989, ele acabara de se eleger deputado federal.
Desde então um ou outro veículo da grande mídia brasileira já divulgava
suas propostas relativas a esses temas. Bolsonaro nem sequer havia
assumido sua vaga no Congresso e o jornal \emph{O Globo} já lhe abria
espaço para esta declaração de inspiração malthusiana e eco eugenista:
``Um filho indesejado, abandonado ou criado em condições precárias pode
se tornar um bandido no futuro. Por isso, acho que primeiro é preciso
controlar a natalidade e, somente depois, implementar a pena de morte
para alguns casos, como sequestros ou estupros seguidos de
morte''.\footnote{Saint"-Clair, 2018, p.\,60.} Praticamente vinte e cinco
anos mais tarde, em uma entrevista concedida no dia 11 de fevereiro de
2014 a vários canais de rádio e televisão, Bolsonaro dera a seguinte
declaração:

\begin{quote}
Olha, quando eu falo em pena de morte, é que uma minoria de marginais
aterrorizam a maioria de pessoas decentes. Quando se fala em menor
vagabundo, como esse que foi preso lá no poste no Rio de Janeiro, você
tem que ter uma política para aprisionar esses cara, buscar reduzir a
maioridade penal e não defender esses marginais, como se fossem
excluídos da sociedade. Não são excluídos, são vagabundos, que devem ter
um tratamento adequado. Então, buscar redução da maioridade penal, uma
política pra planejamento familiar; buscar uma maneira de dizer à
sociedade que eles foram enganados pelo estatuto do desarmamento. Só
desarmou cidadão de bem; os marginais continuam armados, tá \textsc{ok}? Lutar
por uma maneira de legítima defesa, não apenas pela legítima defesa da
vida própria e de outros, mas legítima defesa de seu patrimônio e de
outrem, pra dar uma resposta ao \textsc{mst}, que invade propriedade de quem
trabalha, que leva o terror ao campo. Tem que mudar\ldots{} A política de
direitos humanos é só pra humanos direitos. E não pra vagabundos,
marginais, que vivem às custas do governo.
\end{quote}

A defesa pessoal da pena de morte se justifica diante do que se
apresenta como um fato: o terror imposto por uma ``minoria de
marginais'' a uma ``maioria de pessoas de bem''. Constitui"-se aí uma
argumentação em que se contrasta o mal e o bem extremos e a pequena e a
grande parcela da população. Esse contraste é bastante chapado e se
formula em uma linguagem absolutamente simplista. Em vez da
simplificação ``Não são excluídos, são vagabundos'', o deputado ainda
afinado com a mesma posição ideológica poderia ter dito algo como:
``Embora por vezes esses jovens possam cair na criminalidade em
decorrência de exclusões que sofrem, há muitos outros que são
verdadeiros marginais, que cometem verdadeiras atrocidades''. A reflexão
ponderada, a nuance no pensamento e o estilo concessivo e modalizado de
linguagem não são marcas de Bolsonaro. Já a menção a um caso brutal de
abuso dos direitos humanos, no qual um adolescente negro fora espancado
e amarrado nu junto a um poste por ``justiceiros'', sob a alegação de
que seria o autor de pequenos furtos, ocorrido no dia 03 de fevereiro
daquele ano, é o ensejo para a justificação da redução da maioridade
penal e a refutação da posição antagonista, esta, sim, fundamentada em
estudos científicos sobre a violência urbana, que aponta causas
econômicas e sociais da exclusão adolescente e juvenil e seu posterior
aliciamento por grupos violentos.

Onde o fundamento científico afirma a existência da exclusão, Bolsonaro,
adeptos da extrema"-direita e neofascistas militantes dizem não haver
nada mais do que ``vagabundagem'', ``sem"-vergonhice'' e
``marginalidade''. Sem articulação sólida e manifesta, o deputado passa
dessa temática à do estatuto do desarmamento, valendo"-se uma vez mais da
estanque oposição entre ``cidadão de bem'' e ``marginais''. Novamente,
com uma débil coesão entre as coisas ditas, Bolsonaro passa a falar de
``legítima defesa da vida'', de ``legítima defesa do patrimônio'' e de
uma ``resposta ao \textsc{mst}''. Este último, assim como outros adversários
situados no campo da esquerda, é demonizado com duas acusações tão
cristalizadas como improcedentes: o movimento ``invade propriedade de
quem trabalha'' e ``leva o terror ao campo''. Em meio a tudo isso, ainda
há espaço para dourar a pílula, ao chamar de ``política pra planejamento
familiar'' sua desejada ação biopolítica de controle da geração da vida
das populações precarizadas.

Após ter arrolado um caótico conjunto de horrores, Bolsonaro se
encaminha para o final de sua declaração com mais uma simplificação
expressa, como todo o resto, cheia de fel, indignação e truculência. O
início de sua peroração constitui"-se de uma formulação tão vaga quanto
oportunista. Assemelhando"-se ao clichê ``Tem que mudar tudo isso que tá
aí, tá Ok?!'', inclusive explorado por humoristas, tamanhas eram sua
repetição e sua imprecisão, a expressão ``Tem que mudar\ldots{}''
beneficia"-se do ambiente de insatisfação da opinião pública para com o
campo político, de modo geral, e de sua intensificação em larga medida
produzida por veículos da grande mídia, setores conservadores da
sociedade civil e adversários políticos do governo do Partido dos
Trabalhadores. Em suma, o expediente reside em forjar um caos para que
se deva impor a ordem: quanto mais se fomentam os medos e os riscos que
assolam o presente e o futuro, tal como as ameaças reais e imaginárias
sentidas e projetadas pelas classes médias em relação aos empobrecidos e
marginalizados, mais facilmente se introjetam as necessidades de
disciplina, de segurança e de polícia.

É nesse ambiente que essa formulação produz seus efeitos. É também nele
que se fixam as fórmulas ``Direitos humanos só para humanos direitos'' e
``Vagabundos e marginais que vivem às custas do governo'' com as quais
Bolsonaro encerra sua intervenção. Seu taxativo modo de dizê"-las e a
condição quase proverbial do que é dito, formada pelo jogo e pela
inversão das palavras no primeiro caso e pelo consenso popularesco no
segundo, concorrem para recobrir o final de sua fala com uma espécie de
aura de sabedoria e franqueza. Para a eficácia dessas fórmulas, além do
consenso grosseiro, conta ainda e fundamentalmente o encantamento com a
linguagem, uma espécie de amor praticamente universal, ainda que
evidentemente constituído distintamente em contextos históricos, sociais
e culturais diversos, às eufonias e aos jogos de palavras. O bordão
``Bandido bom é bandido morto'' também tantas vezes dito e repetido por
Bolsonaro e bolsonaristas é prova disso.

Já o dissemos, mas é preciso reiterá"-lo, sua perseguição aos direitos
humanos e seus preconceitos contra a comunidade \textsc{lgbtq} sem sanção legal e
divulgados amplamente pela mídia fizeram com que um deputado quase
anônimo se tornasse uma celebridade do universo político, mas também do
mundo \emph{pop}. Além dos gays, lésbicas e trans, dos negros e
indígenas e dos miseráveis e excluídos, Bolsonaro também discrimina e
ofende as mulheres. Em que pesem suas inúmeras demonstrações de
misoginia, vamos aqui abordar somente o emblemático embate entre
Bolsonaro e a deputada federal pelo \textsc{pt} do Rio Grande do Sul, Maria do
Rosário. Dois episódios foram particularmente marcantes nesse embate.
Dez dias antes do primeiro confronto entre eles, ocorrera um bárbaro
crime, envolvendo um menor de idade. No dia 01 de novembro de 2003, o
adolescente Roberto Alves Cardoso, conhecido como Champinha, participara
com outros quatro maiores de idade do roubo, sequestro e morte de um
jovem casal de namorados. A repercussão do caso na imprensa foi muito
intensa. Bolsonaro não perderia a oportunidade para se apresentar como
defensor da pena de morte e da redução da maioridade penal.

Maria do Rosário, era então, além de deputada federal pelo \textsc{pt}/\textsc{rs}, a
relatora da \textsc{cpi} da Exploração Infantil, e Bolsonaro era deputado federal
pelo \textsc{ppb}/\textsc{rj}. No dia 11 de novembro, eles concediam entrevista a uma
equipe de reportagem da \emph{Rede \textsc{tv}} no Salão Verde da Câmara Federal,
sobre a redução ou não da maioridade penal, repercutindo ainda o trágico
assassinato do juvenil casal. Bolsonaro estava defendendo a redução da
maioridade penal ao microfone da emissora, no momento em que Maria do
Rosário o interrompeu, dizendo o seguinte: ``O senhor é que promove
essas violências\ldots{}''. Indignado por ter sido interrompido, Bolsonaro
rebateu com violência, mas também com esperteza, formulando um enunciado
que sugeria que Maria do Rosário o teria acusado de promover estupros:
``Eu que promovo estupros?''. Em vez de questionar a formulação de
Bolsonaro, refutando a implicação semântica produzida pelo deputado,
Maria do Rosário foi envolvida na artimanha argumentativa que ele
construíra: ``É, o senhor promove, sim\ldots{}'' A discussão se estenderia
até o ápice da misoginia de Bolsonaro em torno do qual a repercussão
social e midiática se fixou:

\begin{quote}
\forceindent{}--- Grava aí, grava aí que eu promovo estupro. Grava aí\ldots{}

--- É, o senhor, eu estou vendo isso, sim\ldots{}

--- Grava aí, grava aí, eu sou estuprador\ldots{}

--- Quem defende a violência é o senhor\ldots{}

--- Eu sou estuprador agora\ldots{}

--- É, sim\ldots{}

--- Olha, jamais eu ia estuprar você porque você não merece!
\end{quote}

Em sua quarta frase, a deputada parecia tentar se desvencilhar da
armadilha retórica em que Bolsonaro queria pegá"-la. Mas, ela não
resistiu à nova investida do deputado. Bolsonaro tomou a resposta
afirmativa à questão que lançara como ensejo para formular a oração
absolutamente chocante e seu pressuposto não menos abominável: Maria do
Rosário não seria estuprada por ele porque não mereceria sê"-lo. Haveria,
portanto, pessoas que mereceriam ser estupradas e vítimas de estupro que
o foram por tê"-lo merecido. O descontrole de ambos se instala, ainda que
desigualmente entre eles. Maria do Rosário reage ao ignóbil ultraje de
Bolsonaro, dizendo"-lhe o seguinte: ``Olha, eu espero que não\ldots{} Porque
senão, eu lhe dou uma bofetada''. Eles já haviam se aproximado o
bastante um do outro para que o deputado revidasse com gestos e palavras
violentos, ``Dá que eu te dou outra, dá que eu te dou outra\ldots{}'',
tocando e empurrando a deputada com a mão esquerda, enquanto levantava a
direita com dedo indicador em riste, sob a forma de uma ameaça e de uma
iminente agressão física. Bolsonaro ainda repetiria quatro vezes a frase
``Dá que eu te dou outra'', sempre com tons e gestos ameaçadores e
agressivos. Maria do Rosário reagiu, expressando seu assombro: ``O
senhor tá me empurrando?'' e ``Mas o que é isso?!'', ambos enunciados
repetidos duas vezes, antes de chamá"-lo de ``desequilibrado''. Foi o
suficiente para que o deputado passasse a insultá"-la: ``Você me chamou
de estuprador. Você é uma imoral, tá? Vagabunda!''\footnote{Saint"-Clair,
  2018, p.\,105--106.} Ele ainda repetiria aos gritos o xingamento
``Vagabunda!'' mais uma vez.

Conforme acontecera já tanto em sua carreira militar quanto na política,
os abusos e a agressividade de Bolsonaro foram registrados, denunciados
e perdoados. Já no dia 12 de novembro, o líder do \textsc{pt} no Congresso,
deputado Nelson Pellegrino, protocolou um pedido de abertura de processo
disciplinar contra Bolsonaro. Novamente, o deputado do \textsc{ppb} sairia ileso.
A impunidade certamente contribui para que algo semelhante ocorresse
onze anos mais tarde. Não se tratava mais de algo que aconteceria nos
bastidores da Câmara, mas em uma sessão deliberativa do Senado. No dia
09 de dezembro de 2014, o deputado Amauri Teixeira, do \textsc{pt} da Bahia,
presidia a mesa e Maria do Rosário lhe pediu a palavra. Em sua
intervenção, ela fez uma defesa da Comissão Nacional da Verdade.
Bolsonaro também havia solicitado espaço para uma sua intervenção, mas,
antes de cedê"-la, o presidente da sessão apoiou a posição de Maria do
Rosário:

\begin{quote}
Quero me somar à fala da senhora. Realmente, o Brasil e a América Latina
têm que ser passados a limpo. Crimes clandestinos cometidos pela
ditadura são revelados, atualmente, como a Operação Condor, documentos
que provam a articulação internacional para assassinar lideranças de
esquerda pelas ditaduras.

Deputado Jair Bolsonaro, o senhor tem três minutos
prorrogáveis\ldots{}\footnote{Saint"-Clair, 2018, p.\,108.}
\end{quote}

No início da réplica de Bolsonaro, quando estava advertindo o deputado
Amauri Teixeira, dizendo ``Um presidente não pode falar isso\ldots{}'', ele
percebeu que Maria do Rosário estava deixando o recinto. Com o lastro de
absolvições que já carregava consigo e agora com a condição de um dos
deputados federais mais bem votados do Brasil nas eleições de outubro
daquele mesmo ano,\footnote{Bolsonaro se elegera com 464.572 votos pelo
  \textsc{pp}/\textsc{rj}. Foi o deputado federal com mais votos no Rio de Janeiro, e o
  terceiro em todo o Brasil, ficando atrás somente de Celso Russomano e
  de Tiririca, eleitos pelo Estado de São Paulo.} Bolsonaro voltou à
carga com tudo:

\begin{quote}
Não saia não, Maria do Rosário! Não saia, não. Fica aí! Fica aí, Maria
do Rosário! Fica! Há poucos dias, tu me chamou de estuprador, no Salão
Verde. E eu falei que não estuprava você, porque você não merece. Fica
aqui pra ouvir. (\ldots{}) Maria do Rosário saiu daqui agora correndo. (\ldots{})

Maria do Rosário, por que não falou sobre sequestro, tortura e execução
do prefeito Celso Daniel, do \textsc{pt}? Nunca ninguém falou nada sobre isso
aqui\ldots{} tão preocupados com os direitos humanos. Vai catar coquinhos!
Mentirosa deslavada e covarde! Eu ouvi ela falando aqui as asneiras
dela. E fiquei aqui. Fala do teu governo. O governo mais corrupto da
história do Brasil! (\ldots{})

O Brasil tá quebrado! Vamos partir pra onde? Pra cubanização, como uma
forma de salvar o país? Volta de \textsc{cpmf}, nova alíquota de Imposto de
Renda, taxação das grandes fortunas, um governo canalha, corrupto e
imoral! Ditatorial! Queria também aqui mencionar as questões voltadas
para as eleições da Unasul. Descobriu que a urna eletrônica é a garantia
de se perpetuar no poder. Governo covarde! Comunista! Imoral!
Ladrão!\footnote{O pronunciamento editado de Bolsonaro está disponível no
  \emph{Youtube}, ``Dep.\,Jair Bolsonaro (\textsc{pp}) rebate a Dep.\,Maria do Rosário sobre discurso dos Direitos Humanos'', 9 de dezembro de 2014.}
\end{quote}

Antes de mais nada, é preciso ressaltar que Maria do Rosário não havia
se dirigido a Bolsonaro em sua intervenção. Mesmo assim, ele se endereça
direta e incisivamente a ela, empregando os pronomes de tratamento
``tu'' e ``você'', em vez de usar os protocolares ``Senhora'' ou ``Vossa
excelência'', a interpela e tenta lhe dar ordens. O deputado vale"-se
ainda de um expediente retórico para buscar escamotear a repetição da
gravíssima ofensa que já fizera a Maria do Rosário onze anos antes.
Bolsonaro retoma o episódio anterior como se o estivesse mencionando e
não reiterando a terrível injúria, de modo que essa suposta menção lhe
desse ensejo para repetir que não a estupraria porque ela não mereceria.
De fato, a ação de estuprar é formulada em um relativo distanciamento
produzido pelo imperfeito, mas o demérito imputado à deputada está
bastante próximo e continuaria atual, ao ser formulado no presente do
indicativo. Além disso, Bolsonaro a acusa de covardia e de dissimulação,
sem poupar nenhum dos presentes e menos ainda a própria deputada de suas
grosserias. O ``Cavalão'' ainda a insulta de modo agressivo e
desprezível tanto pelo que ela seria, ao afirmar que ela é ``Mentirosa
deslavada e covarde!'', quanto por suas ações, ao dizer que ela havia
falado ``as asneiras dela''. Enfim, o deputado do \textsc{pp} descompromete"-se
com a verdade factual, produz hipérboles e arroubos simplistas, ``O
governo mais corrupto da história do Brasil!'', ``O Brasil tá
quebrado!'', e finaliza seu pronunciamento vociferando as alucinações
neofascistas do alegado comunismo no Brasil, usado como aval para alçar
os adversários políticos a inimigos do país e as pessoas de bem para
combatê"-los e persegui"-los, inclusive, fisicamente.\footnote{Esse e
  outros procedimentos discursivos foram descritos por Fiorin, José
  Luiz. Operações enunciativas do discurso da extrema"-direita.
  \emph{Discurso \& Sociedad}, vol. 13(3), 2019, p.\,370--382.}

Nos dias seguintes, a intervenção agressiva de Bolsonaro seria mais uma
vez premiada com a vitrine midiática. Já no dia 10 de dezembro, ele foi
entrevistado longamente pelo jornal \emph{Zero Hora} e, dois dias
depois, publicou um texto de opinião em espaço privilegiado do jornal
\emph{Folha de S.\,Paulo}.\footnote{``Bolsonaro diz que não teme processos e faz nova ofensa: `Não merece ser estuprada porque é muito feia'\,'', \textsc{gzh}, 10 de dezembro de 2014; e ``Jair Bolsonaro: O grito dos canalhas'', \textit{Folha de S.\,Paulo}, 18 de dezembro de 2014.}
O Ministério Público Federal abriu uma ação judicial contra Bolsonaro e
a própria Maria do Rosário acionou o deputado no Tribunal de Justiça do
Distrito Federal e no Supremo Tribunal Federal. Bolsonaro não sairia
ileso dessas ações, mas recorreria em várias instâncias, arrastaria os
processos e, o mais importante, as leves sanções não o impediriam de
registrar oficialmente sua candidatura à presidência da República em
2018. Uma das principais vozes a declarar e repetir que ``Bandido bom é
bandido morto'' descobrira precoce e reiteradamente que os crimes de
calúnia e injúria compensam.

Menos de um ano mais tarde, o deputado federal Eduardo Cunha, então
presidente da Câmara, aceitava dar encaminhamento à denúncia de crime de
responsabilidade contra a presidenta Dilma Rousseff. O relatório da
comissão especial que avaliou a denúncia foi favorável ao seguimento do
trâmite do processo de impeachment. No dia 17 de abril de 2016, o
plenário do Congresso votava favorável ou contrariamente à posição do
relatório, ou seja, ao afastamento ou à manutenção de Dilma na
presidência da República. A sequência dos votos foi ao mesmo tempo
tediosa, repugnante e comicamente trágica e compreendeu votos em nome de
Deus e da família, em nome dos eleitores do Estado do parlamentar e do
futuro dos filhos. Havia certa expectativa de que Bolsonaro suspendesse
o tédio com seu voto. Infelizmente, ele não a decepcionou.

Conjuntamente com o que havia de anedótico e detestável naquela
sequência de votos, a abjeta intervenção de Bolsonaro tornou"-se uma
espécie de emblema do golpe jurídico e parlamentar sofrido por Dilma.
Mais do que quebrar aquele encadeamento modorrento, ele iria provocar
choque e indignação entre os democratas e identificação e entusiasmo
entre os partidários da extrema"-direita violenta. É verdade que outrora,
ainda praticamente anônimo no cenário nacional, quando o Brasil dava
sinais de começar a consolidar sua democracia, ele falava abertamente
que gostaria de fechar o Congresso e de participar de uma guerra civil
que mataria ``30 mil'', mesmo que morressem inocentes. Mas agora, em um
contexto de contestação de instituições democráticas e com uma inédita
relevância política, seu voto era esperado com certa ansiedade, estava
investido de importância e iria nos estarrecer:

\begin{quote}
Nesse dia de glória para o povo brasileiro, tem um nome que entrará para
a história nessa data, pela forma como conduziu os trabalhos nessa Casa.
Parabéns, presidente Eduardo Cunha! Perderam em 1964. Perderam agora em
2016. Pela família e pela inocência das crianças em sala de aula, que o
\textsc{pt} nunca teve. Contra o comunismo. Pela nossa liberdade. Contra o Foro
de São Paulo. Pela memória do coronel Carlos Alberto Brilhante Ustra, o
pavor de Dilma Rousseff. Pelo Exército de Caxias, pelas nossas Forças
Armadas. Por um Brasil acima de tudo e por Deus acima de todos, o meu
voto é sim!
\end{quote}

O voto de cada deputado poderia se limitar a dizer ``sim'' ou ``não''.
Bolsonaro usou 103 palavras em seu voto. Em um misto de cinismo e
bajulação, ele começa sua intervenção com um sorriso estampado no rosto
e com um gesto que aponta para a mesa da presidência da Câmara, onde
está Eduardo Cunha. O abalo na democracia é chamado de glória e as
elites conservadoras e seus adeptos populares, de ``o povo brasileiro''.
Bolsonaro reiteradamente confere o que vai dizer em uma anotação que
está em sua mão esquerda. Não é, pois, um voto espontâneo. Cada uma de
suas frases é pronunciada com uma cadência relativamente lenta, e elas
são sucedidas de pausas mais ou menos longas que lhes dão ainda mais
destaque. Ao cinismo, à bajulação e à encenação se somam o revanchismo e
as alucinações políticas. Tudo se passa como se a família e a inocência
das crianças tivessem sido ameaçadas pelo \textsc{pt}, como se houvesse algum
comunismo no Brasil, algum risco à liberdade e o famigerado Foro de São
Paulo.

A tudo isso se junta um puro exemplar do sadismo, sob a forma de uma
conexão histórica entre o passado, o presente e o futuro: a vitória do
horror em 1964, a que se começava a se consumar naquele 17 de abril de
2016 e a que poderia vir na esteira dessas duas. O sadismo de Bolsonaro
é ainda mais estarrecedor, porque ele concentra o terrível contraste
entre o sorriso com que inicia seu voto e a alegria revanchista com que
fala dessas vitórias, de um lado, e as dilacerantes dores físicas e os
irreversíveis traumas psíquicos sofridos por quem passou por sessões de
tortura. Ao dedicar seu voto à memória de um dos maiores torturados da
ditadura brasileira entre 1964 e 1985, Bolsonaro pronuncia seu nome
quase aos gritos e sílaba por sílaba, como se a altura excessiva e a
extensão duradoura de sua pronúncia revivessem, aumentassem e
distendessem o prazer de quem faz sofrer e a dor e a angústia de quem
sofre. Como se a dose de crueldade já não houvesse extrapolado limites
democráticos e humanitários, Bolsonaro ainda lhe expande e precisa com
um aposto que sucede o nome do torturador: ``o pavor de Dilma
Rousseff''.\footnote{``Brilhante Ustra não foi um patife qualquer. Teve
  papel de destaque no ``trabalho'' que, segundo o lamento de Bolsonaro,
  a ditadura não terminou. O \textsc{doi}-Codi que Ustra comandou entre 1970 e
  1974 foi chamado de ``casa dos horrores'' na sentença histórica,
  proferida em 2008, em que o juiz Gustavo Santini Teodoro condenou o
  coronel. Ustra foi o único oficial militar condenado civilmente pela
  Justiça brasileira pelo crime de tortura. Isso porque a Lei da
  Anistia, de 1979, serviu como escudo legal para impedir que os
  torturadores fossem levados penalmente ao banco dos réus.'' (Barros e
  Silva, F. Dentro do pesadelo. O governo Bolsonaro e a calamidade
  brasileira. In: revista Piauí, edição 164, maio de 2020, p.\,29)}

As penas leves, as absolvições e projeção midiática permitiram que o
deputado do \textsc{psc} pudesse cometer essa atrocidade verbal e institucional
em pleno Congresso Nacional. Como nos ensinou a interface entre a
psicanálise e a história, de modo análogo ao que ocorre com os
indivíduos, a experiência de traumas históricos e sociais, quando
recalcada e não elaborada, produz o retorno de suas causas e sintomas. A
opressão e os massacres perpetrados manifestamente contra negros e
indígenas durante quatro séculos não foram elaborados com reparação
histórica e se perpetuam até nossos dias por meios mais ou menos
velados, mas sem dúvida muito conhecidos. Em outra escala, a ditadura em
plena segunda metade do século \textsc{xx} matou, torturou e perseguiu
adversários reais e imaginários, sem mais tarde sofrer sanções por tudo
isso. A repetição histórica das impunidades e sua reiteração pessoal na
trajetória de Bolsonaro lhe deram respaldo para proferir seu voto
naqueles deploráveis e abjetos termos, fazendo ``reviver a própria
tortura, num exercício de sadismo de que pouca gente é capaz''.\footnote{Barros
  e Silva, 2020, p.\,29.}

Além da aberta crueldade, a defesa dos militares está presente como de
costume. Para defendê"-la, sugere"-se uma história gloriosa que seria a
sua, evocando um nome do passado ao qual o próprio Exército elege como
patrono e cobre de galardões, principalmente por seu maior sucesso
bélico que foi o massacre que comandou na guerra contra o Paraguai. A
homenagem às ``nossas Forças Armadas'' é coerentemente sucedida por um
grito de guerra que já se desprendia da frase com que Bolsonaro
finalizara o artigo publicado na \emph{Veja} em 1986, em que podia
ressoar o slogan nacionalista ``Alemanha acima de tudo'' retomado pelos
nazistas e ao qual, agora, se acrescenta a expressão ``Deus acima de
todos'', que comporá mais tarde a divisa nas eleições presidenciais.
``Grito'' aqui não é força de expressão, porque, de fato, o voto de
Bolsonaro é uma vociferação. Depois da adulação a Eduardo Cunha, o mais
célebre deputado do \textsc{psc} passa a elevar de tal modo o volume de sua voz
que o restante de seu pronunciamento se torna uma série de urros muito
exaltados. A ideologia violenta de extrema"-direita de que sua fala é
veículo privilegiado materializa"-se bastante bem na agressividade dos
gritos de Bolsonaro.

Seria difícil de imaginar que o deputado sádico e falastrão abriria mão
de toda essa violência verbal que passou a caracterizá"-lo e que lhe
tirou de um quase anonimato do baixo clero político para elevá"-lo à
categoria de um dos políticos mais conhecidos do Brasil e à de potencial
candidato à presidência da República. Porém, foi isso que de certo modo
ocorreu nas eleições de 2018.

\section{O candidato lacônico}

Jair Bolsonaro não falou muito durante sua campanha eleitoral à
presidência. Sua condição de candidato fez com que seu desempenho
oratório se tornasse objeto de discussão em várias ocasiões. Nessas
circunstâncias, não raras vezes, afirmou"-se que o candidato pelo Partido
Social Liberal (\textsc{psl}) cometia ``erros de comunicação'' e que sua dicção
era ``sofrível''. Nessas mesmas ou em ocasiões análogas, com certa
frequência, também se disse que ele não fala muito bem em público e que
não domina a norma culta do português.\footnote{Ver, entre outros, os
  seguintes textos: Polito, Reinaldo. ``Três erros de comunicação
  cometidos por Bolsonaro que você pode evitar'', \textit{Uol}, Economia, 23 de janeiro de 2019; Lago, Miguel. ``Bolsonaro fala outra língua'', revista \textit{Piauí}, 13 de agosto de 2018.} Em contrapartida, já antes das eleições, seus partidários
costumavam assegurar que ele fala as verdades que alguns não querem
ouvir e que outros não têm coragem para dizer.\footnote{``Bolsonaro
  falando a verdade, como sempre'', de Marco Antônio Felício da Silva
  (General de Brigada), publicado em 16 de janeiro de 2015 no
  \emph{Blog do Licio Maciel}.}
Também não são poucos os que diziam que Bolsonaro age, se comporta e
fala de modo simples e popular, como se fosse um homem do povo.

\begin{quote}
Bolsonaro parece falar a mesma língua da população não intelectualizada,
e sobre os mesmos assuntos. Como se ele incorporasse o ethos de padeiro,
carteiro, atendente de farmácia, dona de casa, falando exatamente do que
falam, com a linguagem que utilizam, enxergando a realidade que enxergam
e preparando suas respostas a partir disso. Bolsonaro não veria o Brasil
pelos óculos das abstrações intelectuais.

Não estaria preocupado com o enquadramento das câmeras; se deveria estar
em pé ou sentado na abertura do debate; se deveria, em nome do requinte
e dos bons modos, abotoar a manga da camisa ou fazer menção a grupos
humanitários internacionais; se deveria sorrir para a câmera e para os
demais candidatos ou ficar sério. Ele simplesmente ligava o celular e
começava a falar, assim como qualquer um de nós faz.\footnote{A
  chancela concedida nesses termos à autenticidade e à simplicidade
  construídas por Bolsonaro e sua equipe foi formulada por Pedro
  Henrique Alves, colunista do site ``Instituto Liberal'', cujo
  presidente do Conselho Deliberativo é Rodrigo Constantino. Ela foi
  mencionada por Moura, Maurício; Corbellini, Juliano. \emph{Eleição
  disruptiva}: por que Bolsonaro venceu. Rio de Janeiro/São Paulo:
  Record, 2019, p.\,122--123.}
\end{quote}

A despeito de Bolsonaro ter falado relativamente pouco entre o dia 22 de
julho, quando a convenção nacional do \textsc{psl} o aclamou candidato do
partido, e o dia 28 de outubro de 2018, quando ele foi eleito presidente
da República, ainda assim, o conjunto de seus pronunciamentos e
intervenções é bastante extenso para que fosse integralmente examinado
nos limites deste nosso capítulo. Por essa razão, serão analisados aqui
somente duas intervenções de Bolsonaro. A primeira delas foi realizada
no último programa do Horário Gratuito de Propaganda Eleitoral
(\textsc{hgpe})\footnote{Para uma análise de programas do \textsc{hgpe} de Bolsonaro, ver: Piovezani, Carlos. A retórica do mito: uma análise do desempenho
  oratório de Bolsonaro na propaganda eleitoral. Revista \emph{Discurso
  \& Sociedad}, vol. 13, n. 3, 2019, p.\,383--410.} e a segunda consiste
em uma fala dirigida a apoiadores que se encontravam na Avenida Paulista
a alguns dias da consumação das eleições.

Mediante o exame dessa amostra de suas falas públicas eleitorais,
veremos que a performance oratória de Bolsonaro foi marcada pela
produção dos efeitos de franqueza e identificação de grupo, de veemência
e antagonismo e ainda de ameaça e incitação à violência. Essa série de
efeitos fora produzida de modo mais ou menos constante por Bolsonaro
desde quando era vereador na cidade do Rio de Janeiro, passando pelos
vários mandatos como deputado federal, até seus últimos momentos como
candidato nas eleições presidenciais. Nessa última situação, porém, ao
menos, em suas falas públicas televisivas, ele aparenta ser menos
agressivo, se comparado ao que já se estabelecera como seu próprio
padrão oratório. Em que pesem suas semelhanças, os pronunciamentos da
propaganda eleitoral transmitida pela tevê, endereçados portanto a
partidários, mas também a eleitores ainda indecisos, são distintos do
discurso dirigido aos adeptos apaixonados que se encontravam no centro
de São Paulo a uma semana do pleito decisivo.

As participações de Bolsonaro em programas de tevê, sua presença
constante em textos da imprensa e em \emph{posts} e \emph{memes} das
redes sociais na internet e seus pronunciamentos na Câmara dos Deputados
e nos que circularam, principalmente, via \emph{WhatsApp},
\emph{Facebook} e afins foram fundamentais para a construção de sua
celebridade midiática. Foi desse modo que Bolsonaro passou a ``mitar''
cada vez mais no campo político e a se apresentar como o principal
expoente da extrema"-direita, de suas bandeiras e de seus modos de
expressão. Assim, ele encarnou a conjunção de uma ética da moralidade
pública e da salvaguarda dos comportamentos, de uma política da
intransigência policial e de uma estética populista da fala franca e
enérgica, simples e direta.

\section{O último programa no \textsc{hgpe}}

Exceto aos domingos, entre os dias 12 e 26 de outubro de 2018, ocorreu a
veiculação dos programas eleitorais no \textsc{hgpe} no segundo turno daquelas
eleições presidenciais. No dia 26 de outubro daquele ano, era, portanto,
transmitido o último programa da propaganda eleitoral gratuita de
Bolsonaro, assim como fora também o dia de exibição do último programa
de seu adversário, Fernando Haddad, do Partido dos Trabalhadores (\textsc{pt}).
Diferentemente do primeiro turno, desta vez, Bolsonaro não mais estava
em desvantagem na distribuição do tempo concedido às diversas
candidaturas e gozava dos benefícios de ter ficado na primeira colocação
na disputa ocorrida naquela primeira etapa e de liderar as pesquisas com
16\% à frente de Haddad. No segundo turno, a divisão do tempo foi
refeita. Os programas eleitorais eram exibidos no \textsc{hgpe} em dois blocos
diários: o primeiro a partir das 13h e o segundo, das 20h30. Cada um dos
candidatos contava com 5 minutos em cada um desses blocos diários para a
transmissão de seu programa.

A última propaganda eleitoral de Bolsonaro inicia"-se com uma sequência
relativamente longa, na qual se combinam a locução em voz \emph{off} de
um texto que, durante toda sua extensão, detrata exclusivamente o \textsc{pt} e
seus integrantes, em particular, o ex"-presidente Lula e Fernando Haddad.
Esse plano verbal e sonoro é acompanhado, no plano visual, de uma série
de imagens compostas de ícones de revistas, jornais e portais de
notícias da mídia tradicional brasileira, tais como \emph{Veja},
\emph{Isto é}, \emph{Uol notícias}, \emph{O Globo}, entre outros, e de
manchetes de suas reportagens que denunciam supostas irregularidades e
escândalos do \textsc{pt}. Há nessa passagem a combinação de uma seleção lexical
excessiva, que não se compromete com a verdade factual, e uma pronúncia
destacada justamente nas palavras que produzem de modo mais evidente
esse excesso: ``O \textsc{pt} ficou 13 anos no poder e \emph{que}brou o país.
Deixaram mi\emph{lhões} de desempregados, o maior índice de
criminalidade da his\emph{tó}ria e os maiores escândalos de corrupção do
\emph{mun}do''.

Há um flagrante contraste entre essa passagem e a sequência que se
inicia logo depois: ``Começa agora o programa do presidente livre e
independente. / Bolsonaro 17''. A ambiência cromática rubro"-negra
esfumaçada, que sugeria algo diabólico, é substituída pelas formas e
cores da bandeira brasileira, que ocupa toda a extensão da tela e, em
seguida, pela imagem de manifestantes pró Bolsonaro, que também portam,
muitos deles, bandeiras do Brasil em suas mãos. Entre tais
manifestantes, que se encontram em um espaço que parece ser a Avenida
Paulista, na cidade de São Paulo, se destaca, em primeiro plano, uma
grande bandeira brasileira que tremula. No mesmo sentido desse
contraste, segue a dimensão sonora desse trecho. O mesmo locutor emprega
agora outros recursos vocais: o andamento de sua voz se acelera, seu
volume aumenta ligeiramente e seu timbre recebe um aspecto entusiasmado,
em detrimento da gravidade que o caracterizava até então. Também faz
parte dessa paisagem sonora eufórica o canto entoado pelos partidários
do candidato do \textsc{psl}.

Em seguida, são exibidas, em um primeiro quadro, a inscrição central e
em letras verdes, sob o fundo amarelo, que preenche toda tela, ``O nosso
partido é o Brasil'', e, em um segundo, a imagem de Bolsonaro carregado
por militantes, em sua passeata na cidade de Juiz de Fora (\textsc{mg}), antes de
sofrer o conhecido atentado à faca. O plano sonoro é composto pelo mesmo
canto entoado na passagem anterior. Já esta última imagem é sucedida
pela legenda ``\textsc{presidente bolsonaro} 17'', em letras de grande dimensão,
em caixa alta, nas cores verde, azul e amarelo, abaixo da qual, em
letras menores, à direita, se vê, em azul, a seguinte inscrição: ``vice
General \textsc{mourão}''. Com o mesmo entusiasmo vocal, o locutor diz:
``Bolsonaro 17''. Desse trecho, decorre o início do pronunciamento do
candidato. Bolsonaro está sentado e não usa gravata. Ainda não consegue
eliminar totalmente a impressão de que lê o que está dizendo em um
\emph{teleprompter}, mas já o faz de modo menos perceptível do que nos
primeiros programas da propaganda eleitoral da \textsc{tv}. Vejamos o que foi
dito por ele neste seu último pronunciamento veiculado no \textsc{hgpe}:

\begin{quote}
Há quatro anos, eu decidi disputar a Presidência da República.

Num primeiro momento, eu confesso, era difícil, até para mim, aquela
situação. Como vencer um sistema? Como vencer uma máquina tão aferrada
no terreno, como é essa máquina que existe em Brasília? Políticos
poderosos. Sabia que não teria um grande ou médio partido ao meu lado,
não teria tempo de televisão, não teria fundo partidário, não teria
nada.

Mas, eu tinha algo dentro de mim: nós temos que fazer algo diferente.

Como cristão, eu adotei uma passagem bíblica, João (\textsc{viii}, 32): ``E
conhecereis a verdade, e a verdade vos libertará''. E mantive essa
bandeira em pé. Comecei a andar por todo o Brasil. Começamos a detectar
problemas. E como resolvê"-los, sem dinheiro? Porque sabemos das
dificuldades, depois das passagens desses últimos governos, que
mergulharam o Brasil na mais profunda crise ética, moral e econômica.

Mas, a fé, a vontade e a persistência se fez presente. Eu digo que o
milagre é eu estar vivo, depois daquele episódio em Juiz de Fora. Que eu
considero Juiz de Fora a minha segunda cidade natal. Lá, eu nasci de
novo. Salvaram a minha vida. Logicamente, a mão de Deus se fez presente.

Hoje, nós temos uma possibilidade concreta, real, de ganharmos as
eleições no próximo domingo. O que precisamos para tal? É nos mantermos
unidos. Combater as mentiras.

Meus irmãos, meus amigos, o momento é de união. Se essa for a vontade de
Deus, eu estarei pronto para cumprir essa missão.

Ninguém faz nada sozinho.

Com uma equipe boa ao meu lado, com pessoas maravilhosas, que são vocês,
nós temos como fazer um Brasil melhor para todos.

Estou aqui nessa missão, porque acredito em você, brasileiro. E você
está aí nos assistindo, porque acredita no Brasil.

Faremos um governo para todos.

Meu muito obrigado mais uma vez. Brasil acima de tudo, Deus acima de
todos.
\end{quote}

O início dessa intervenção é marcado pelo efeito de amenidade. Para
tanto, Bolsonaro vale"-se de propriedades prosódicas típicas de uma fala
relativamente distensa. Seu pronunciamento configura"-se como um relato
pessoal tranquilo, afável e espontâneo. A narrativa pessoal ganha
contornos de confissão já no segundo enunciado do candidato: ``Num
primeiro momento, eu confesso, era difícil, até para mim, aquela
situação''. A afirmação, digamos, objetiva, da dificuldade de um estado
de coisas, visto que o enunciado poderia ter sido ``Num primeiro
momento, era difícil aquela situação'', é atravessada pelos traços
subjetivos da confissão (``eu confesso'') e da presença manifesta do
próprio enunciador em seu enunciado (``até para mim''). Na declaração de
Bolsonaro, esses dois adendos íntimos são pronunciados de modo a serem
ressaltados.

Em princípio, a confissão poderia carregar a inconveniência de um erro
cometido, mas traz também e sobretudo a vantagem de seus reconhecimento
e admissão e o benefício do engajamento do interlocutor de conceder ao
confessor um julgamento positivo, em face de sua corajosa atitude de
contrição. Na declaração de Bolsonaro, não há nem mesmo aquela
inconveniência. Ao contrário, o que ocorre é uma soma de valores
agregados ao confessor, uma vez que ele se beneficia das virtudes da
sabedoria e da prudência, próprias de quem enxerga as dificuldades de
uma situação e reage judiciosamente à sua problemática condição. A tais
virtudes se agregam ainda a da coragem de quem não foge ao seu
enfrentamento e a da humildade de mostrar"-se comedido em sua resolução.

Já a expressão ``até para mim'' produz uma surpreendente inclusão do
candidato em um grupo do qual, em princípio, ele não faria parte.
Contrapondo"-se às virtudes da prudência e da humildade do ``eu
confesso'', aqui se constitui uma imagem superestimada de si: o nível de
dificuldade da situação é tamanho que, mesmo o sujeito em questão, uma
pessoa dotada, em tese, de capacidades distintas, teria de admitir que
encontraria percalços para resolver seus problemas. A superestimação não
seria a única coisa ali superdimensionada, porque a passagem seguinte do
pronunciamento indicaria as razões para que a situação fosse considerada
como algo de enormes dificuldades. Noutros termos, Bolsonaro reencarna o
``grande homem comum'' da propaganda fascista, de que fala
Adorno.\footnote{Adorno, 2018.}

Depois de um exórdio que busca captar a atenção e a benevolência do
auditório, mediante o anúncio de uma decisão e a confissão do embaraço
que decorria da dificuldade a ser enfrentada, Bolsonaro encadeia duas
interrogações e uma afirmação, que funciona como uma extensão das
primeiras: ``Como vencer um sistema? Como vencer uma máquina tão
aferrada no terreno, como é essa máquina que existe em Brasília?
Políticos poderosos''. Nesse início de suas proposições, ocorre um
reforço da carga dramática, porque o que se diz e a maneira de dizer se
retroalimentam. A dificuldade consistiria em enfrentar algo que concilia
organização sistêmica e funcionamento otimizado, enraizamento sólido e
concentração de poder, dispondo de bem pouca coisa ou ainda menos do que
isso: Bolsonaro ``não teria nada''. Não seria nem a primeira nem a
última vez que ele se descomprometeria de dizer a verdade:

\begin{quote}
O candidato à Presidência da República pelo \textsc{psl}, Jair
Bolsonaro, foi o primeiro a bater R\$ 1 milhão em doações de
apoiadores para a campanha eleitoral. A marca foi ultrapassada (exatos
R\$ 1.000.182) na noite de domingo. Somente 59 dias após o início da
arrecadação, iniciada no dia 5 de julho.

Empresas estão comprando pacotes de disparos em massa de mensagens
contra o \textsc{pt} no \emph{WhatsApp} e preparam uma grande operação na semana
anterior ao segundo turno. A prática é ilegal, pois se trata de doação
de campanha por empresas, vedada pela legislação eleitoral, e não
declarada. A \emph{Folha} apurou que cada contrato chega a R\$ 12
milhões e, entre as empresas compradoras, está a Havan. Os contratos são
para disparos de centenas de milhões de mensagens. As empresas apoiando
o candidato Jair Bolsonaro (\textsc{psl}) compraram um serviço chamado de
`disparo em massa', usando a base de usuários do próprio candidato ou
bases vendidas por agências de estratégia digital.\footnote{Respectivamente:
  Soares, Jussara. ``Bolsonaro é o primeiro a ultrapassar R\$ 1 milhão,
  em `vaquinha' para campanha eleitoral''. \emph{O Globo}, 03 de
  setembro de 2018; e Campos Mello, Patrícia. ``Empresários bancam campanha contra o \textsc{pt} pelo WhatsApp''. \emph{Folha de S.\,Paulo}, 18 de outubro de 2018.}
\end{quote}

A versão da história contada por Bolsonaro a seu próprio respeito oculta
dados factuais e o apresenta como uma espécie de mito ou, ao menos, como
um sujeito investido de muito brio. Após anunciado o contexto
profundamente adverso, ocorre uma sensível modificação introduzida pela
conjunção ``Mas''. Ela é pronunciada com elevação no volume de voz e é
seguida de uma pausa um pouco mais longa do que seria habitual.
Conjuntamente com essa alteração fonética e com a oposição semântica
promovida por essa conjunção, também se dá uma alteração no plano
visual: o candidato, que até então era focalizado frontalmente em plano
americano, passa a ser enfocado em um grande \emph{close"-up}, mediante o
qual a câmera se concentra somente em seu rosto, e a projetar um olhar
orientado para sua direita, sem encarar diretamente a objetiva.

O efeito é o de um depoimento sincero e espontâneo, ao qual o
telespectador passa a assistir como uma testemunha privilegiada. Por seu
intermédio, se revela ``algo dentro de mim''. Além dessa expressão, a
fisionomia e o discurso direto simulam o diálogo interior que Bolsonaro
teria tido consigo mesmo. Nada poderia ser mais franco e autêntico do
que um diálogo dessa natureza: ``nós temos que fazer algo diferente''. O
depoimento, particularmente, em seu diálogo interior, é, portanto, o
anúncio de um dever, mas é também uma resolução. Bolsonaro diz, então,
que é preciso ``fazer'', e não apenas dizer, ``algo diferente''. A
consciência de um dever conduz à tomada de uma decisão, que já é uma
declaração de ação iminente. Mas, o estado volitivo não é acompanhado de
um afeto agressivo, pois o que diz e as formas de dizer de Bolsonaro
produzem a imagem daquele que tem a serenidade de quem sabe o que deve
ser feito e que o fará, apesar das dificuldades, uma vez que se encontra
em boa companhia.

Em sua fala, Bolsonaro encadeia ``nós temos que fazer algo diferente'' e
``Ninguém faz nada sozinho''. Mas, além disso, reitera e reforça sua
inclusão na comunidade dos cristãos. Não bastasse o ostensivo apoio de
setores conservadores da igreja católica e de uma grande maioria das
igrejas evangélicas à sua candidatura, em meio aos recentes avanços da
teologia da prosperidade e do empreendedorismo popular evangélico e à
conservação mais ou menos modificada do integralismo brasileiro e seu
lema ``Deus, pátria e família'',\footnote{Sobre o integralismo brasileiro
  e suas relações com o fascismo italiano, ver: Bertonha, João Fábio.
  ``Entre Mussolini e Plínio Salgado: o Fascismo italiano, o Integralismo
  e o problema dos descendentes de italianos no Brasil'',
  \emph{Revista Brasileira de História}, vol.~21, n.~40, 2001.}
ele repete em seu pronunciamento as marcas dessa pertença: ``Como
cristão, eu adotei uma passagem bíblica, João (\textsc{viii}, 32): `E conhecereis
a verdade, e a verdade vos libertará'\,'', ``a fé'', ``o milagre'', ``a
mão de Deus se fez presente'', ``meus irmãos'', ``a vontade de Deus'',
``cumprir essa missão'' e ``Deus acima de todos''. Em meio a essas
tantas referências religiosas, se reafirma o esquema: descoberta de
problemas, ciência das dificuldades e resolução de enfrentar estas
últimas e de solucionar os primeiros.

É a alusão ao enfrentamento de dificuldades políticas, com ``fé'',
``vontade'' e ``persistência'', que dá a deixa para uma passagem
especialmente dramática do pronunciamento: ``Eu digo que o milagre é eu
estar vivo, depois daquele episódio em Juiz de Fora. Que eu considero
Juiz de Fora a minha segunda cidade natal. Lá, eu nasci de novo''.
Ocorrem uma elevação no volume vocal desde o início da sequência e uma
aceleração no andamento da fala, se comparados com os da passagem
anterior, que se estendem até ``vivo''. Nessa passagem, a fonação se
torna mais tensa. A partir daí, emerge uma nova conformação prosódica,
uma vez que o trecho ``depois daquele episódio em Juiz de Fora'' é
pronunciado de modo mais lento e com uma ligeira redução de volume da
voz. Esses contornos prosódicos, combinados com a escolha lexical de
``episódio'' e de seu determinante ``daquele'', concorrem para a
produção de um efeito de certo distanciamento, tomado por um sujeito
sereno, que, tendo Deus ao seu lado, não guarda rancor.

Ao se referir a ``Juiz de Fora'', Bolsonaro carrega na carga patética e
tenta constituir uma relação de empatia e apelar à compaixão de seus
interlocutores. Uma longa pausa, gestos de insistência, desta vez, com o
dedo indicador em riste, expressão fisionômica emocionada e movimento de
garganta de quem engole a saliva, na manifestação do que seria uma
tentativa de, estando bastante comovido, conter o choro, reforçam o já
considerável aditivo dramático dessa parte do pronunciamento na frase
``Lá:::: eu nasci de novo''. Já as duas frases seguintes são ditas em
meio a outra alteração de enfoque: Bolsonaro deixa de ser focalizado
frontalmente em plano americano e passa a ser enfocado em um grande
\emph{close"-up}, fechado sobre seu rosto. É nesse enquadramento que o
candidato diz: ``Salvaram minha vida. Logicamente, a mão de Deus se fez
presente''. A convicção na intervenção divina parece estar reservada à
troca da energia contida na pronúncia de ``Salvaram minha vida'' pela
tranquilidade da articulação de ``a mão de Deus se fez presente'', e
pela certeza construída pelo advérbio ``Logicamente''.

Na parte final do pronunciamento, é por meio de um intercâmbio entre a
primeira e segunda pessoa do discurso que ali se processa um dos efeitos
fundamentais da peroração: a comoção. De fato, o esquema argumentativo
do encerramento da intervenção de Bolsonaro agora segue mais de perto o
eixo \emph{docere}, \emph{delectare} e \emph{movere}, ou seja, ensinar,
emocionar e fazer agir. O primeiro componente desse esquema se concentra
nestas quatro frases, que contêm cinco ocorrências da primeira pessoa do
plural, a última delas, sob a forma de uma elipse: ``Hoje, nós temos uma
possibilidade concreta, real, de ganharmos as eleições no próximo
domingo. O que precisamos para tal? É nos mantermos unidos. Combater as
mentiras''. O efeito de inclusão do interlocutor em uma comunidade se dá
conjuntamente com a declaração de que essa comunidade de que faz parte
tem chance efetiva de uma iminente vitória eleitoral. Sua realização
exige, contudo, algo que o eleitor partidário, incluído na comunidade
quase já vitoriosa, ainda não saberia por completo.

A questão, também ela formulada em primeira pessoa do plural, reitera o
efeito comunitário, mas também reconfigura a identidade de seus membros,
uma vez que o candidato que a formula e que conhece sua resposta não se
confunde com os eleitores que a ouvem e que ainda não sabem da solução
que será proposta logo em seguida. Essa resposta estabelece uma
equivalência e um encadeamento discursivos entre manutenção da união e
combate a mentiras. Com base em sua evidência, o candidato faz equivaler
e encadeia a preservação da união e a luta contra as mentiras. Desse
modo, como cabe à peroração de uma fala em público, o orador busca a boa
disposição do ouvinte em seu favor e a disposição contrária a seu
adversário. Para fazê"-lo, se vale de uma fala enfática, dotada de
movimentos articulatórios vigorosos e de acentos de insistência. Além
disso, a última frase é pronunciada com a alteração no enfoque do
candidato. Uma vez mais se repete o recurso do grande \emph{close"-up},
em que a câmera focaliza somente seu rosto. A aproximação da face tenta
criar um efeito de proximidade entre os interlocutores e de franqueza de
quem fala. O candidato se posiciona, portanto, no lugar de um sujeito
que diz a verdade e que pode, por isso, mais bem impelir ao combate das
mentiras. Em seu discurso, se produzem essas equivalências: Deus,
verdade e amigos X demônio, mentira e inimigos.\footnote{``Os fascistas
  se apropriavam, como faz Bolsonaro, de metáforas e pensamentos
  religiosos. Essa crença numa forma de verdade sagrada tem claras
  conotações teológicas cristãs. Na Bíblia, a verdade do Senhor
  contrasta com as mentiras dos homens: `De modo algum! Seja Deus
  verdadeiro, e todo homem mentiroso'. Os que não acreditam na verdade
  de Deus são literalmente demonizados: `Quem é o mentiroso, senão quem
  nega que Jesus é o Cristo? Esse é anticristo, que nega o Pai e o
  Filho'. As mentiras dos infiéis emanavam do diabo. (\ldots{}) Se o líder
  encarnava uma verdade eterna, os fascistas concluíam que seus críticos
  mentiam, considerando"-os inimigos da verdade''. Finchelstein,
  Federico. O líder fascista como encarnação da verdade. \emph{Revista
  Serrote}. Edição especial, julho de 2020, p.\,42.}

Na passagem seguinte, Bolsonaro passa a falar na primeira pessoa do
singular. Assim, interpela os telespectadores, se refere a si mesmo como
o eleito de um anseio divino e manifesta a prontidão para o cumprimento
do seu desígnio. Além dos efeitos de interação, de laço identitário, de
proximidade e de afeição produzidos pelo ``Meus irmãos, meus amigos'',
esse vocativo se conjuga com outros termos religiosos: ``união'',
``vontade de Deus'' e ``cumprir essa missão''. A oração condicional ``Se
essa for a vontade de Deus'' realça a crença religiosa e a submissão do
sujeito que fala ao desejo divino. Em seguida ao afetuoso apelo ao
telespectador e à deferente menção a Deus, se insere um lugar"-comum da
humildade: ``Ninguém faz nada sozinho''. A humildade do candidato dá a
ocasião para o anúncio da ``equipe boa'' que o acompanhará e para a
produção de um processo de sedução, por meio da constituição de uma
imagem positiva dos interlocutores. Além da companhia de uma boa equipe,
Bolsonaro estará ainda na de ``pessoas maravilhosas''. Mediante a
escolha do adjetivo, não há somente uma tentativa de seduzir, mas também
a de bajular os eleitores, em um puro e emblemático \emph{delectare},
criando um regime de fala que, em princípio, não se coadunaria com o
debate público de assuntos políticos de uma sociedade. A primeira pessoa
do plural, com que se encerra a oração, compreende o locutor, a equipe e
os interlocutores. Investidos de suas virtudes, eles teriam competência
para tornar o país ``um Brasil melhor para todos''.

Depois da adulação particularmente presente nesta conjunção entre as
palavras e a forma como elas são pronunciadas, ``pessoas
maravi\emph{lho}sas:: que \emph{são} vo\emph{cês}'', Bolsonaro se coloca
em relevo e estabelece uma diferença de papéis e uma identidade de
afeto. Sua fala se torna mais fala enfática, mas não virulenta, como em
outras ocasiões, fora da propaganda eleitoral na \textsc{tv}, e nela se produz
uma crença quase recíproca: o candidato acredita no eleitor e este
último ``acredita no Brasil'', ou seja, confiaria no candidato que
dispõe de boa equipe, que o seduz ao lhe imputar qualidades
extraordinárias e que já anunciou seu potencial para a melhora do país.

Se o afeto seria mais ou menos compartilhado, os papéis desempenhados
são distintos: a Bolsonaro cabe o cumprimento de uma missão, enquanto ao
eleitor telespectador cabe a assistência. Exclui"-se sua participação
efetiva e se a limita ao mero testemunho. Assim, o que poderia ser a
inclusão dos interlocutores na promessa em primeira pessoa do plural,
``Faremos um governo para todos'', parece já, em boa medida, desdita
pelas oposições anteriores. Como se poderia esperar, a declaração desse
compromisso se materializa em uma fala enfática, que constrói o efeito
de certeza no futuro. O pacto, porém, está condicionado à esfera divina:
``Se for essa a vontade de Deus\ldots{}'' Já no agradecimento e na elocução
de seu \emph{slogan} de campanha, a busca por emocionar e por mover o
ânimo dos ouvintes se vale menos do brilho e do entusiasmo, prescritos
pela retórica, quando trata da peroração, do que de uma energia, entre
firme e anódina, no modo de fala.

\section{«Os marginais vermelhos serão banidos~de~nossa~pátria»}

Apesar de alguma veemência nesse último pronunciamento no \textsc{hgpe}, o
Bolsonaro que vemos e ouvimos ali destoa tanto pelo que diz quanto pelas
maneiras de dizer do sujeito agressivo e de tantas falas violentas.
Diferentemente da inclinação negociadora e conciliatória que marcou as
práticas e os discursos em toda a trajetória de Lula, de sindicalista
até seus dois mandatos como presidente da República,\footnote{Para
  análises de discursos de Lula, ver, entre outros: Cazarin, Ercília.
  \emph{Identificação e representação política: uma análise do discurso
  de Lula}. Ijuí: Editora da Unijuí, 2006; Ab'Sáber, Tales. A voz de
  Lula. Revista \emph{Serrote}, n. 10. 2012, p.\,63--71; Piovezani,
  Carlos. Falar em público na política contemporânea: a eloquência pop e
  popular brasileira na idade da mídia. In \emph{História da fala
  pública}. Petrópolis: Vozes, 2015, p.\,290--337; e Pereira, Maísa Ramos.
  Silenciamento e tomada de palavra: ambivalências discursivas em
  pronunciamentos de Evo Morales e Lula da Silva. (Tese de doutorado em
  Linguística), \textsc{ufsc}ar, 2019.} a real verve bolsonarista caracteriza"-se
pelo conflito e pela agressividade. Ainda assim, na propaganda
eleitoral, ele se apresenta praticamente em uma versão ``paz e amor''.
Essa persona mais parece um embuste, que não duraria muito tempo nem se
estenderia fora da cena eleitoral televisiva. Cinco dias antes desse
último programa do \textsc{hgpe} e a sete dias do segundo turno das eleições,
Bolsonaro fez um discurso dirigido a apoiadores reunidos na Avenida
Paulista em São Paulo em uma modalidade possivelmente inédita. Sua voz
chega até seus partidários via telefone e sua imagem é reproduzida em um
telão.

A despeito da distância, candidato e eleitores estão em alta sintonia,
interagem e estimulam"-se mutuamente. Além dessa difusão, o
pronunciamento estava já destinado a circular por outros meios, porque
foi gravado em vídeo. ``Para quem convalescia de uma facada, Bolsonaro
aparece em cena bastante disposto e corado, sorridente na maior parte do
tempo. Postado em pé no quintal dos fundos de casa, veste uma camiseta
verde e tem atrás de si, como cenário, algumas peças de roupa e lençóis
brancos pendurados no varal. Tudo é muito descontraído, casual,
calculadamente mambembe. Revendo o vídeo, tive a impressão de que
Bolsonaro lia o que falava de forma pausada, interagindo com a excitação
da massa. Foi um discurso atroz''.\footnote{Barros e Silva, 2020, p.\,29.}
Eis aqui algumas passagens dessa sua fala:

\begin{quote}
Nós somos a maioria. Nós somos o Brasil de verdade. Junto com esse povo
brasileiro construiremos uma nova nação. Não tem preço as imagens que
vejo agora da Paulista e de todo o meu querido Brasil. Perderam ontem,
perderam em 2016 e vão perder a semana que vem de novo. Só que a faxina
agora será muito mais ampla. Essa turma, se quiser ficar aqui, vai ter
que se colocar sob a lei de todos nós. Ou vão pra fora ou vão para a
cadeia. Esses marginais vermelhos serão banidos de nossa pátria.

Essa pátria é nossa. Não é dessa gangue que tem uma bandeira vermelha e
tem a cabeça lavada.

Aqui não terá mais lugar para a corrupção. E seu Lula da Silva, se você
estava esperando o Haddad ser presidente para soltar o decreto de
indulto, eu vou te dizer uma coisa: você vai apodrecer na cadeia. E
brevemente você terá Lindbergh Farias para jogar dominó no xadrez.
Aguarde, o Haddad vai chegar aí também. Mas não será para visitá"-lo,
não, será para ficar alguns anos ao teu lado.

Já que vocês se amam tanto, vocês vão apodrecer na cadeia. Porque lugar
de bandido que rouba o povo é atrás das grades.

Pretalhada, vai tudo vocês para a ponta da praia. Vocês não terão mais
vez em nossa pátria, porque eu vou cortar todas as mordomias de vocês.
Vocês não terão mais \textsc{ong}s para saciar a fome de mortadela de vocês. Será
uma limpeza nunca vista na história do Brasil. Vagabundos. Vai ter que
trabalhar. Vai deixar de fazer demagogia junto ao povo brasileiro.

Vocês verão as instituições sendo reconhecidas. Vocês verão umas Forças
Armadas altivas, que estará colaborando com o futuro do Brasil. Vocês,
pretalhada, verão uma Polícia Civil e Militar com retaguarda jurídica
para fazer valer a lei no lombo de vocês.

Bandidos do \textsc{mst}, bandidos do \textsc{mtst}, as ações de vocês serão tipificadas
como terrorismo. Vocês não levarão mais o terror ao campo ou às cidades.
Ou vocês se enquadram e se submetem às leis ou vão fazer companhia ao
cachaceiro lá em Curitiba.

Amigos de todo Brasil, esse momento não tem preço. Juntos, eu disse
juntos, nós faremos um Brasil diferente.

Meu muito obrigado a todos do Brasil que confiaram seu voto em mim por
ocasião do primeiro turno. Ainda não ganhamos as eleições. Mas esse
grito em nossa garganta será posto pra fora no próximo dia 28.

Nós ganharemos essa guerra. Vamos juntos trabalhar pra que no próximo
domingo aquele grito que está em nossa garganta, que simboliza tudo o
que nós somos, seja posto pra fora: Brasil acima de tudo, e Deus acima
de todos! Valeu! Um abraço meu Brasil!\footnote{A íntegra do discurso de
  Bolsonaro pode ser assistida em ``Jair Bolsonaro fala por telefone com os manifestantes da Av.\,Paulista, \textit{Youtube}, 21 de outubro de 2018.}
\end{quote}

Uma narrativa alicerça essa e praticamente todas as falas de Bolsonaro e
dos bolsonaristas: em uma origem idílica, em nosso reino, tudo eram
flores, até que com o correr dos tempos os inimigos ali se infiltraram e
produziram uma decadência ética, um declínio moral e uma degeneração
sexual. A pureza que conhecíamos fora maculada e precisa ser reintegrada
por meio de uma ``limpeza'' que nos livre dessa nódoa perigosa e
crescente. Há aí uma polarização simplista entre os amigos da pureza
(``nós'', ``nossa pátria'', ``Amigos'') e os inimigos que espalham a
sujeira (``marginais vermelhos'', ``gangue que tem a bandeira
vermelha'', ``Vagabundos'', ``bandidos'', ``petralhada''). Essa
polarização impõe uma ``guerra'', em que é preciso eliminar os
oponentes: ``esses marginais vermelhos serão banidos de nossa
pátria''.\footnote{João Cezar de Castro Rocha sustenta a hipótese de que
  a eliminação do inimigo presente nos discursos bolsonaristas vem da
  ``Doutrina de Segurança Nacional'' (\textsc{dsn}) formulada pela Escola
  Superior de Guerra e pelo regime de 1964: ``A guerra cultural
  bolsonarista realiza, de um lado, uma tradução inesperada, de
  consequências potencialmente funestas, da Doutrina de Segurança
  Nacional que foi desenvolvida durante a ditadura. Mas, mesmo antes,
  pela Escola Superior de Guerra. Na \textsc{dsn}, uma vez identificado o inimigo
  não há dúvida: é necessário eliminá"-lo''. Rocha, João Cezar de
  Castro. ``O verbo dominante nos vídeos dos intelectuais bolsonaristas
  é eliminar. E o substantivo é limpeza''. Entrevista concedida a
  Augusto Diniz e publicada no Jornal \emph{Opção}, no dia 08 de março
  de 2020.}

Para que não haja dúvida, Bolsonaro diz quem são esses inimigos, o que
os caracteriza e o que eles fazem. Seu pronunciamento pode ser resumido
a uma perseguição obsessiva dos adversários, porque ele se dedica muito
mais a apontá"-los, detratá"-los e a ameaçá"-los do que a agradecer seus
apoiadores e a lhes pedir que continuem em campanha até o final das
eleições. Não há uma única proposta de política pública. Mas abundam as
variações da violência e do banimento que recairão sobre os adversários:
``a faxina'', ``vão pra fora ou vão para a cadeia''; ``serão banidos'';
``apodrecer na cadeia''; ``vai tudo vocês para a ponta da praia''; ``uma
limpeza nunca vista na história do Brasil'', ``a lei no lombo de
vocês''.

Não há dúvidas de que ``tudo transpira ódio e recende a fascismo. A
`ponta da praia', talvez nem todos saibam, era o nome dado pela ditadura
a um local de desova de cadáveres no Rio de Janeiro. Bolsonaro fala como
torturador, não como candidato à Presidência''.\footnote{Barros e Silva,
  2020, p.\,29.} Isso não significa que a intervenção não compreenda
variações. Em seu início e em seu final, há espaço para o reforço da
identificação de grupo, para a consolidação de um efeito de pertença à
``maioria'', ao ``Brasil de verdade'', e ainda para certo entusiasmo,
ainda que também ele contaminado por não pouca animosidade. Mas a carga
patética raivosa mais ou menos bem distribuída por todo o corpo central
do discurso concentra"-se no ponto a partir do qual ocorre a simulação de
uma mudança de interlocutor.

De seu exórdio até certa passagem de seu pronunciamento, Bolsonaro
dirigia"-se diretamente aos seus partidários, quando então simula passar
a falar com o principal líder dos inimigos: Lula. A mudança acontece
neste trecho ``E seu Lula da Silva, se você estava esperando o Haddad
ser presidente para soltar o decreto de indulto, eu vou te dizer uma
coisa: você vai apodrecer na cadeia''. É justamente em meio a essa
modificação de interlocutor que o candidato do \textsc{psl} mais projeta um
discurso de ódio, tanto no que diz quanto nas maneiras de dizer, porque
é sempre vociferando que ele faz ameaças de violência física e até de
extermínio de adversários políticos. Mesmo que seja constituído dessa
manifesta truculência verbal, o discurso de Bolsonaro se beneficia do
fato de não raramente ser ``visto como algo folclórico, lúdico,
juvenil''. Seus modos de expressão seriam do ``tipo de manifestação que
domina o conjunto \emph{Facebook} / \emph{WhatsApp} / \emph{Youtube} /
\emph{Instagram} / \emph{Twitter}. É a forma do \emph{meme}. Bolsonaro
apostou na memeficação da política, dominou a arte de memeficar os temas
e, com isso, atraiu milhões de seguidores''.\footnote{Moura e Corbellini,
  2019, p.\,124.} Por essa e por outras razões bem mais escusas, a
virulência de Bolsonaro foi justificada e até bem recebida por veículos
da grande mídia.\footnote{O \emph{Estadão} publicou no dia 08 de outubro
  de 2018 um famigerado editorial intitulado ``Uma escolha muito
  difícil'', no qual equiparava Bolsonaro e Haddad: ``De um lado, o
  direitista Jair Bolsonaro (\textsc{psl}), o truculento apologista da ditadura
  militar; de outro, o esquerdista Fernando Haddad (\textsc{pt}), o preposto de
  um presidiário. Não será nada fácil para o eleitor decidir"-se entre um
  e outro.'' Já a \emph{Jovem Pan} em texto intitulado ``Bolsonaro fala em
  `marginais vermelhos', \textsc{pt} veste a carapuça'', de 23 de outubro de 2018, não só atenuou a violência contida nas declarações de Bolsonaro, mas também as
  defendeu: ``Claro que a frase seguinte, ``Esses marginais vermelhos
  serão banidos de nossa pátria'', era desnecessária e inadequada,
  porque o banimento da pátria soa autoritário, mas Bolsonaro, além de
  usar um termo genérico sem individualização, nem sequer disse que ele
  próprio banirá os marginais, mas que os marginais serão banidos.''}

Além disso, a impressão de fazer de Lula, Lindbergh Farias, Haddad,
Petralhada, Bandidos do \textsc{mst} e do \textsc{mtst} seus interlocutores diretos produz
um efeito de coragem. Bolsonaro investe"-se de bravura e simula falar de
modo franco e autêntico, aguerrido e sem rodeios aos seus piores e mais
poderosos adversários. O candidato do \textsc{psl} aparenta ser alguém que não
somente não tem medo de dizer o que pensa negativamente sobre alguém,
mas o faz diretamente para a pessoa concernida e com toda franqueza e
valentia de um soldado que encara e enfrenta o inimigo em uma guerra,
sem temer os riscos que corre. Em razão de um sólido amalgama entre
posição política e postura estética, a agressividade de Bolsonaro
promove a adesão de boa parte do eleitorado e catalisa os discursos de
ódio e de completo desrespeito pelos mais básicos direitos humanos.
Entre outras declarações de seus eleitores que vão nesta mesma direção,
eis estes dois emblemas do ódio condensado:

\begin{quote}
Jair Bolsonaro vai descer a borracha nesses vagabundos aí. (eleitor de
Minas Gerais, classe \textsc{b}, 28 anos)

Pra ele, se tem que matar, mata. Por isso, o povo está atrás dele.
(eleitor de Pernambuco, classe \textsc{d}, 39 anos).\footnote{Moura e Corbellini,
  2019, p.\,77.}
\end{quote}

A agressividade de Bolsonaro, ao indicar os inimigos políticos e ao
falar de aumento da corrupção e da violência, e a intransigência com que
trataria dos problemas da administração e da segurança pública geram
crença e adesão do eleitorado e fomenta a reprodução de discursos da
violência entre os próprios eleitores. Diante da crescente sensação de
que estaríamos no ápice de uma crise generalizada concomitante com o que
era apresentado como a existência de uma selvageria urbana, os
partidários de Bolsonaro não aderiam a um projeto bem definido para essa
área nem a uma experiência comprovada para conceber e propor soluções a
seus problemas. Em seus pronunciamentos, o candidato do \textsc{psl} não expunha
nem sequer um razoável plano de política pública aos eleitores.

Isso não parecia incomodar seus apoiadores. Entre eles, ocorria uma
adesão fervorosa provocada pela relevância que o candidato dava ao tema,
pela atitude de enfrentamento que ele corporificava, ``inclusive no
gestual da `arma', que se tornaria um dos símbolos de sua campanha'',
pela forma contundente que se posicionava a esse respeito e pelo
``combate sem tréguas ao que chamava de `bandidagem', até o limite da
exclusão do pacto social, de seu banimento do Estado de
direito''.\footnote{Moura e Corbellini, 2019, p.\,76.} Para a criação
dessa crença e desse vínculo, sua linguagem belicosa, com laivos
fascistas, em particular quando sua fala se endereça a públicos mais
homogeneamente fidedignos, foi sempre fundamental. Bolsonaro mostra"-se
loquaz e poderoso para lutar contra a corrupção e a violência,
valendo"-se em larga medida de sua fala para fazê"-lo. Nela, há
\emph{grosso modo} a conjunção entre essas duas facetas: ele é franco e
destemido para apontar o inimigo no que diz e forte e enérgico para
combatê"-lo em suas maneiras de dizer. Não foram exatamente toda essa
verbosidade e toda essa coragem que vimos ser demonstradas por Bolsonaro
nas situações em que ele poderia ter debatido e enfrentado seus
concorrentes nas eleições presidenciais.

De fato, Bolsonaro falou pouco durante sua campanha eleitoral. Dos 14
debates eleitorais televisivos previstos, o candidato do \textsc{psl} participou
de apenas 2.\footnote{``Eleições 2018: calendário, debates e programa dos candidatos à presidência do Brasil'', jornal \emph{El país}, 20 de setembro de 2018. Ocorreram, efetivamente, 7 debates
  eleitorais entre os candidatos à presidência da República nas eleições
  de 2018. No segundo turno, não houve nenhum debate. As emissoras de
  tevê desistiram de sua realização diante da recusa de Bolsonaro em
  participar dos debates.} A discrição, o laconismo e os silêncios não
lhe eram comuns antes das eleições presidenciais de que saiu vitorioso.
Em princípio, sua opção pelas antípodas da fala abundante e agressiva,
que frequentemente o caracterizou, poderia surpreender. Com mais forte
razão, deveria surpreender que um sucesso eleitoral decorresse em boa
medida de uma renúncia ao direito de ocupar as tribunas e de participar
dos debates. Diante dessa situação atípica e desse seu espantoso
resultado, poderiam ocorrer"-nos estas questões: por que Bolsonaro falou
bem menos do que poderia, em pleno contexto eleitoral? Como ele pôde
mesmo assim vencer as eleições?

Bolsonaro falou tão pouco porque boa parte das situações de fala às
quais estaria exposto eram fundamentalmente diálogos e debates de
ideias. Para muitos analistas, ele ``não teria um bom desempenho e
demonstraria despreparo em entrevistas e debates''. Tal como já ocorrera
mesmo antes do início da campanha eleitoral na \textsc{tv}, a performance
oratória de Bolsonaro era aventada como um seu calcanhar de
Aquiles.\footnote{``O desempenho de Bolsonaro em algumas entrevistas ---
  uma das mais marcantes talvez tenha sido aquela para a bancada da
  \emph{GloboNews}, em 3 de agosto de 2018 --- reforçou em muitos a
  certeza da previsão de sua candidatura se desidrataria na hora da
  verdade.'' (Moura e Corbellini, 2019, p.\,61).} Além disso, o
candidato do \textsc{psl} falou pouco, também porque falou ``outra língua'', a
língua das \textsc{tic}s, e substituiu os riscos de exposição dos
pronunciamentos e dos enfrentamentos verbais com adversários e
jornalistas pela conveniência das ``\emph{lives}''. Nelas e no idioma
das \textsc{tic}s, as falas circulam por vias e canais específicos, que se
caracterizam por romper ``com as referências normativas estabelecidas,
introduzindo algumas transformações centrais em nossa vida: ofuscam a
distinção entre realidade e virtualidade, invertem a lógica da escassez
da informação para a da abundância e, sobretudo, promovem a passagem de
um mundo em que a primazia é dada às entidades para outro, em que
predominam as interações''.\footnote{Lago, 2018.} Falou pouco, porque
ele já havia dito coisas e, pelas coisas ditas, já havia sido alçado a
uma espécie de celebridade política e midiática, por programas
televisivos e por redes sociais na internet. Bolsonaro já havia se
projetado pelas coisas que dizia, pelos seus modos de dizer e pelas
formas de circulação do que disse.

Na campanha de Bolsonaro, ocorreu e persistiu \emph{grosso modo} uma
distinção entre a \emph{persona} e o discurso mais palatáveis que
circulam pela \textsc{tv} e a \emph{persona} e o discurso agressivos que circulam
nas redes sociais. Além disso, há nestas últimas uma divisão entre seu
solo e seu subsolo. No primeiro, estão a posse absolutamente expandida
dos \emph{smartphones}, o uso amplamente difuso das redes sociais e os
efeitos que eles provocam: a forte crença nas mensagens recebidas gerada
por uma emissão não institucionalizada nos moldes tradicionais, a alta
adesão que essa crença constrói e o empoderamento participativo de
produzi"-las e reproduzi"-las. Já no segundo, encontram"-se uma consistente
inflexão da estratégia política e eleitoral da direita e da
extrema"-direita, a emergência e a consolidação de sites com conteúdos
conservadores e autoritários geridos e fomentados por militares da
reserva e ativistas de extrema"-direita e a compra milionária e irregular
de dados e disparos de mensagens, muitas delas \emph{fake news}, a
destinatários bem definidos e segmentados, valendo"-se de uma espécie de
\emph{politropia hi"-tech}.\footnote{Havia antes do período clássico da
  Antiguidade um conjunto de retores e sofistas que se filiavam a uma
  tradição pitagórica `irracional' para a qual os discursos que mais bem
  persuadem ``não são válidos para todos: bem ao contrário, há os
  válidos para os jovens, os para as mulheres, os para os arcontes, os
  para os efebos. (\ldots{}) Essa característica vem aí definida como
  \emph{polytropía} ou faculdade de encontrar os diversos modos de
  expressão convenientes a cada um''. (Plebe, Armando. \emph{Breve
  história da retórica antiga}. São Paulo: \textsc{epu}, 1978, p.\,3). Voltaremos
  a tratar desses aspectos logo adiante.}

Enquanto falava essa língua das tecnologias da informação e comunicação,
Bolsonaro via recrudescer o movimento de seus partidários que ecoavam o
que ele dizia. Três forças bastante conhecidas os moviam: i) a sensação
de aumento da violência urbana e de degradação ética e moral da
sociedade e os consequentes medos e ódios contra ``tudo o que tá aí'';
ii) o sentimento de empoderamento derivado da conquista de uma condição
de porta"-voz do grupo; e iii) as crenças em mensagens segmentadas, bem
dirigidas a públicos específicos e disparadas via rede sociais. No que
diz respeito aos afetos negativos em relação à política e aos partidos
de esquerda, em particular, é preciso ressaltar que em seu funcionamento
reside um ``impulso de eliminar o \emph{out"-group}''. No âmago desse
funcionamento, está o ``dispositivo `joio e trigo' empregado por todos
os demagogos fascistas''. Trata"-se de um mecanismo que ``age como uma
força negativamente integradora e essa integração negativa alimenta o
instinto de destrutibilidade''.\footnote{Adorno, 2018.} Mencionaremos
adiante alguns desses fatos, fenômenos e atores que fomentaram essa
adesão ao instinto de aniquilar os que são considerados exteriores ao
grupo na história pregressa e recente do Brasil.

Já a força do empoderamento meio \emph{fake} meio \emph{true}, mas,
integralmente narcisista, consiste no fato de que muitos sujeitos
periféricos e ignorados passam a se conceber como atores que se integram
a grupos identitários fortes\footnote{Todos os sujeitos não integrados a
  grupos e identidades minoritários, que conquistaram tardia e
  insuficientemente espaços de representação e atuação, podem na
  repetição só aparentemente anódina do \emph{slogan} ``Brasil acima de
  tudo. Deus acima de todos'' inscrever"-se na comunidade da nação, da
  religião e ainda da heterossexualidade.} e que desempenham papéis
importantes em um drama no qual até então só lhes estava reservada a
condição de plateia negligenciada. Um considerável conjunto de
brasileiros e brasileiras desconhecidos e menosprezados repentinamente
são elevados à condição de alguém que tem o que dizer, de quem agora tem
voz e vez para falar e ser ouvido. Emerge e funciona aí um narcisismo ao
mesmo tempo semelhante e distinto daquele indicado por Adorno na
formação de grupos fascistas: ``ao fazer do líder seu ideal, o sujeito
ama a si mesmo, por assim dizer, mas se livra das manchas de frustração
e descontentamento que estragam a imagem que tem de seu próprio eu
empírico''. Um dos meios privilegiados para o estabelecimento e o
reforço dessa identificação narcísica do líder é sua exposição como ``um
`grande homem comum' (\emph{great little man}), alguém que sugere tanto
onipotência quanto a ideia de que é apenas um de nós''. Essa condição
ambivalente ajuda a operar um ``milagre social'': ``A imagem do líder
satisfaz o duplo desejo do seguidor de se submeter à autoridade e de ser
ele próprio a autoridade''.\footnote{Adorno, 2018.}

Além da satisfação desse duplo desejo, a identificação do sujeito com o
líder e com o grupo fascistas lhe permite se desonerar do peso de seus
fracassos e frustrações nessa entrega à massa. Ele tanto se
desresponsabiliza pelo que lhe falta quanto atribui a culpa ao inimigo
apontado pelo consenso grupal. Por outro lado, essa autoridade pode ser
exercida por um sujeito medíocre cuja força provém da identificação com
o grupo e do exercício da condição de porta"-voz em seu interior. Assim,
abre"-se espaço para que o seguidor se torne também seguido, para a
superação momentânea e imaginária, mas efetivamente sentida, da situação
de quem não podia falar ou de quem até falava, mas não era ouvido. Em
uma sociedade profundamente injusta e desigual como a brasileira, dizer
e conquistar algum reconhecimento não é pouca coisa. Esses sujeitos
intensa e extensivamente desrespeitados passam a ser não só
representados por um líder poderoso com o qual se identificam, mas
também podem falar por si mesmos e se tornar eles próprios porta"-vozes
do grupo, praticamente sem os riscos de se defrontar com universos
externos desprovidos do fortalecimento que seu próprio grupo lhe deu e
com a frequente conveniência de falar aos seus no interior de um
consenso que se estabeleceu e que criou entre muitos discordantes uma
crescente onda de silêncio.

Em suma, o gozo narcísico fascista compreende a sugestão que a
propaganda lhe faz de que o ``seguidor, simplesmente por pertencer ao
\emph{in"-group}, é superior, melhor e mais puro que aqueles que estão
excluídos'',\footnote{Adorno, 2018.} mas provavelmente intensifica"-se à
medida que os resolutos partidários do fascismo e os ainda nem tão
convictos empoderados de última hora se tornam cada vez mais ativos no
processo de difusão e consolidação de suas crenças e afetos. Para tentar
suspender ou atenuar as exclusões que os vitimavam, não poucos sujeitos
de classes baixas e médias aderiram aos programas da direita e da
extrema"-direita. O arrefecimento do processo de integração nacional e de
seu clima de otimismo no Brasil inverteram a maré da alta popularidade
dos governos do \textsc{pt}, principalmente a partir da segunda gestão de Dilma
Rousseff. Essa ``inversão da maré, ajudada por técnicas recém"-inventadas
de propaganda enganosa, transformou aprovação em rejeição num passe de
mágica, aliás, assustador. Na falta de organização política para
aprofundar a democracia, ou melhor, a reflexão social coletiva, é
possível imaginar que os novos insatisfeitos, os favorecidos pelas
políticas esclarecidas anteriores, refaçam o seu cálculo e coloquem as
fichas na aposta anti"-ilustrada''.\footnote{Schwarz, Roberto. ``Neoatraso
  bolsonarista repete clima de 1964''. Entrevista concedida a Claudio
  Leal e publicada no caderno Ilustríssima da \emph{Folha de S.\,Paulo},
  15 de novembro de 2019.} Com medo, frustração e ódio no coração e um
celular com internet nas mãos, um gozo momentâneo e um grande estrago
estavam sendo gestados.

Finalmente, essas ``técnicas recém"-inventadas de propaganda enganosa''
produziram crenças fortes e reais em notícias frágeis e falsas. A
despeito da inerente fragilidade de sua contrafação, há uma narrativa,
uma maneira de contá"-las e um modo de pô"-las em circulação que
conjuntamente conseguem lhes impregnar uma condição de verdade e
infundir a crença em milhões de pessoas. A narrativa é esta bem
conhecida, segundo a qual ``as passagens desses últimos governos
mergulharam o Brasil na mais profunda crise ética, moral e econômica''.
Ela é contada em uma linguagem simples e mais ou menos alegórica, que
está baseada em uma concepção belicosa e neopentecostal do mundo, para a
qual há uma guerra formada de grandes combates políticos e de pequenas
batalhas cotidianas contra as múltiplas faces do mal. A compra ilícita
de dados e de disparos de mensagens bastante segmentadas, que chegavam
por uma via direta sem o peso e a desconfiança de instituições
tradicionais e com a legitimidade e a credibilidade de uma origem quase
sempre pessoal, suscitou a crença em patamares altíssimos nas \emph{fake
news}.\footnote{Uma pesquisa realizada pelo \emph{\textsc{ideia} Big Data},
  ``perguntou aos eleitores se tinham visto e acreditado em cinco das
  fake news mais populares nas redes sociais durante as últimas semanas
  das eleições. Impressionantes 98,2\% dos eleitores de Bolsonaro
  entrevistados tinham sido expostos a uma ou mais daquelas notícias
  falsas, e 89,8\% acreditaram que fossem embasadas em verdade.'' (Moura
  e Corbellini, 2019, p.\,129)}

Pouco antes do início oficial da campanha de Bolsonaro, ocorrera uma
elevação de patamar nessas técnicas de propaganda enganosa. Tudo
começara bem antes com a associação entre dois milionários
norte"-americanos, Robert Mercer e Steve Bannon, da qual resultou a
fundação da \emph{Cambridge Analytica}. Seu propósito era o de armar um
grande esquema com enormes bases de dados e, assim, conseguir manipular
resultados de diversos processos eleitorais: o Brexit e a eleição de
Donald Trump teriam sido os primeiros ensaios bem"-sucedidos. Tudo
acontecera do seguinte modo, desde seu início até sua chegada aos
assessores bolsonaristas:

\begin{quote}
O primeiro passo da operação foi coletar de forma ardilosa os dados de
cidadãos em celulares e redes sociais, sem que eles soubessem, por meio
de aplicativos, enquetes ou de testes do tipo \emph{quizz}. Assim, com
questionários ingênuos, milhões de pessoas deixaram dados sobre seus
traços de personalidade, desejos e medos.

Com esses dados em mãos, eles utilizaram técnicas de segmentação
comportamental, definindo o perfil psicológico dos eleitores. O
cruzamento com os dados que deixamos nas redes sociais, a cada curtida
ou comentário, formava um conjunto que permitiu catalogar as pessoas em
diferentes ``caixas psicológicas''.

A partir dessa operação, tornou"-se possível personalizar a mensagem
política. A segmentação permitia definir qual tipo de conteúdo cada
eleitor receberia, aquele que melhor se adequava à sua história digital.
Não é mais receber o mesmo santinho, ver a mesma placa, assistir à mesma
propaganda.

Para uma senhora diabética, a mensagem vai falar sobre os preços de
medicamentos. Para um homem que pesquisou cercas elétricas para sua
casa, mensagens sobre violência, medidas de segurança pública. Ou, na
versão brasileira, para uma senhora evangélica, mensagens sobre o kit
gay e mamadeira de piroca. Não foram todos que as receberam. Apenas
aqueles que teriam maior propensão de acreditar.

O modelo da \emph{Cambridge} contou ainda com uma estratégia de
intoxicação informativa. Para isso, Mercer adquiriu, antes das eleições
dos \textsc{eua}, o portal de notícias \emph{Breibarth News}, onde eram ancoradas
as narrativas em que seriam distribuídas as informações aos eleitores.
Os links, com ar de seriedade, eram depois disparados nas redes sociais.

Estamos falando de uma estratégia sofisticada de comunicação política.
Algo que alguém tão limitado como Carluxo jamais teria condições de
imaginar. Sim, a operação veio pronta, sob medida, para uso do
bolsonarismo. Carluxo é apenas um gestor na ponta.

A milícia digital bolsonarista constituiu"-se de forma profissional em
2018, na pré"-campanha à Presidência. Um dos momentos mais visíveis de
sua formação foi a greve dos caminhoneiros. Ali houve uma coordenação
para incidir naquele processo. Mensagens personalizadas eram criadas e a
vinculação de Bolsonaro com a greve começou a tomar as redes sociais.
Foi um ``laboratório'', um experimento do que viria pela frente e segue
rolando.\footnote{Boulos, Guilherme. ``A origem das milícias digitais''.
  In: Carta Capital, n. 1106, 20 de maio de 2020, p.\,25.}
\end{quote}

O solo brasileiro não era virgem, mas estava repleto de insumos. Assim,
nestas terras, as sementes plantadas por esse método escuso germinaram,
se transformaram em plantas saudáveis e geraram frutos apodrecidos. Não
é verdade, porém, que, por aqui, ``em se plantando, tudo dá''. Para
combater e impedir o nascimento de sementes transformadoras e
igualitárias, há entre nós combinações históricas, sociais e políticas,
que são tóxicas e muito eficientes. A \emph{politropia hi tech} tornou
possível o envio e a recepção de mensagens compósitas que satisfaziam
perfis cuja complexidade ultrapassava os diagnósticos mais esquemáticos
que dividiam a massa eleitoral em ``direita'' e ``esquerda''. A
``senhora evangélica'' que acredita no ``kit gay'' pode também ser uma
mãe pobre de adolescente viciado em crack e, por isso, estar
desesperadamente à espera de vaga em clínicas públicas de tratamento
antidrogas e ter muito medo e aversão à polícia, que já espancou e
humilhou seu filho. A complexidade que há em distintas porções de atraso
moderno ou de modernidade atrasada e seus diversos acordos é mais bem
contemplada e satisfeita com mensagens específicas do que com programas
mais coerentes e relativamente homogêneos, que apostassem mais ou menos
exclusivamente na modernidade das pautas progressistas ou no atraso das
conservadoras.

Não é possível e talvez nem seja necessário relembrar aqui as várias
experiências desastrosas de nossa história de longo e médio prazos de
persistentes retardos e de progressos ambivalentes e insuficientes e de
seus legados ainda tão presentes em nossos dias. Limitaremo"-nos a nos
referir somente a três episódios mais ou menos recentes que
conjuntamente contribuíram para o conveniente e relativo silêncio de
Bolsonaro nas eleições presidenciais e para sua não esperada, mas, nem,
por isso, surpreendente vitória. De envergaduras distintas e de efeitos
análogos, esses foram três episódios em que pudemos assistir à
fecundidade nefasta e à força deletéria da retórica reacionária no
Brasil.

O primeiro deles está muito bem descrito nestes termos:

\begin{quote}
Em 1985, depois de um trabalho de seis anos, foi publicado no Brasil um
livro que marcou época chamado ``Brasil: Nunca Mais''. Seria o livro
negro da ditadura militar. De maneira secreta, um grupo de pesquisadores
compilou aproximadamente 5 mil páginas de documentos do Superior
Tribunal Militar (\textsc{stm}) com processos de subversivos e guerrilheiros.

Portanto, todos os documentos que fazem parte do projeto ``Brasil: Nunca
Mais'' foram produzidos pela ditadura militar. Os pesquisadores
compilaram uma seleção dos documentos de modo a denunciar para a
sociedade brasileira a tortura, o assassinato e o desaparecimento
político. Eu tinha 20 anos quando o ``Brasil: Nunca Mais'' saiu. Foi uma
revolução na sociedade brasileira. Ficaram comprovadas de uma maneira
muito clara todas as arbitrariedades e a violência da ditadura militar.

No ano seguinte, sob a liderança do ministro do Exército do governo José
Sarney (\textsc{mdb}), que era o general da linha dura Leônidas Pires Gonçalves,
um grupo de militares resolveu revidar. Resolveu, a seu modo, escrever
outro livro. Já que o ``Brasil: Nunca Mais'' se tornou o livro negro da
ditadura militar, os militares comandados pelo Leônidas Pires Gonçalves
decidiram escrever o livro negro da luta armada, isto é, o livro negro
da esquerda.

Os militares compilaram material e documentos, sobretudo do serviço de
informação da Marinha, do Exército, da Aeronáutica e do Serviço Nacional
de Informação (\textsc{sni}), organizaram dois volumes de aproximadamente mil
páginas e queriam publicar o livro. Seria a resposta do Exército ao
``Brasil: Nunca Mais''.

José Sarney, em 1989, vetou a publicação temendo a radicalização e a
polarização que daí poderiam surgir. A partir deste momento, algumas
cópias produzidas manualmente circularam entre oficiais de alta patente
e poucos militantes de direita. Até que um jornalista, Lucas Figueiredo,
especialista na comunidade de informação brasileira, autor do mais
importante livro sobre o \textsc{sni}, ``Ministério do Silêncio'', descobriu e
teve acesso ao livro. O projeto dos militares se chamava
``Orvil''.\footnote{Rocha, 2020.}
\end{quote}

Desde então, envolto em certa aura de mistério, o ``Orvil'' --- nome
derivado da inversão da ordem das letras da palavra ``livro'' --- passou
a circular cada vez mais nas redes compostas por militares e por
militantes e adeptos da direita linha dura e da extrema direita. Foi do
``Orvil'' que o coronel Carlos Alberto Brilhante Ustra, frequentemente
louvado por Bolsonaro, extraiu material para publicar seu livro
intitulado ``A verdade sufocada''.\footnote{Sobre o uso de livros como
  objetos culturais legitimadores dos discursos de Bolsonaro e de sua
  conjunção com o anti"-intelectualismo, ver: Curcino, Luzmara.
  ``Conhecereis a Verdade e a Verdade vos libertará'': livros na eleição
  presidencial de Bolsonaro. In: Revista \emph{Discurso \& Sociedad},
  Barcelona, vol. 13, n. 3, 2019, p.\,468--494.} Em ambos, o que se busca
mostrar é que ``a esquerda da luta armada, na concepção do Exército, era
terrorista e provocou tantos assassinatos e tantas mortes quanto o
próprio Exército''. Além de ser uma compilação de documentos, o
``Orvil'' ``tem uma narrativa que procura interpretar a história
republicana brasileira a partir da década de 1920''. Teria havido ao
longo do século \textsc{xx} três tentativas do movimento comunista internacional
de implantar uma ditadura do proletariado no Brasil: ``É uma narrativa
delirante. É uma teoria conspiratória, simplesmente absurda''. Graças
aos militares, esses três ensaios teriam fracassado: ``O primeiro foi a
fundação do Partido Comunista do Brasil (\textsc{pc}do\textsc{b}), que assim se chamava em
1922, e a Intentona Comunista de 1935''. Já a segunda tentativa teria
começado depois do suicídio de Getúlio Vargas e se estenderia até a
``revolução'' de 1964. Enfim, o terceiro ensaio seria o da luta armada
já durante a ditadura, entre 1968 e 1974: ``Destaca"-se o ano de 1974,
porque nesse ano os últimos guerrilheiros do Araguaia são assassinados
pelo Exército. Não são feitos prisioneiros, são eliminados em fidelidade
à doutrina de segurança nacional''.\footnote{Rocha, 2020.}

As armas militares nos teriam livrado por três vezes da ditadura
comunista que se instalaria por aqui. Mas, sempre segundo essa narrativa
do ``Orvil'', o perigo ainda não havia passado por completo. Muito ao
contrário. Após terem vencido a última batalha, os militares da linha
dura, em vez de condecorados, viram"-se traídos por um dos seus: ``Ainda
em 1974, o presidente Ernesto Geisel começou a desmobilizar o aparato
repressivo, o que explica o ressentimento que se encontra na base do
`Orvil' e também esclarece a narrativa conspiratória do quarto e
derradeiro momento'': ``Na narrativa dos militares do Exército, em 1974,
a esquerda, derrotada militarmente mais uma vez, mudou de rumo e decidiu
adotar a técnica gramsciana; ela teria se infiltrado na cultura, acima
de tudo nas universidades e nas artes, para a médio prazo tomar
o poder. Entre outras palavras, a esquerda triunfou somente quando o
aparato repressivo foi desativado''.\footnote{Rocha, 2020. Ver também:
  Rocha, João César de Castro. ``A arquitetura da destruição'',
  \emph{Folha de S.\,Paulo}, Ilustríssima, 09 de agosto de 2020, p.\,\textsc{c6--c7}.} Em uma equação que nos parece absurda, há para Bolsonaro e
para bolsonaristas uma equivalência semântica entre o \emph{slogan}
``Brasil acima de tudo, Deus acima de todos'' e a palavra de ordem ``É
preciso eliminar o inimigo interno''. Isso quando não ocorre a ameaça
manifesta de morte: ``Pretalhada, vai tudo vocês para a ponta da
praia''.

Em novembro de 2011, foi criado o órgão temporário denominado Comissão
Nacional da Verdade. O governo de Dilma Rousseff viu"-se obrigado a
fazê"-lo, porque no final de 2010 o Brasil fora condenado pela Corte
Interamericana de Direitos Humanos da \textsc{oea}. Mais de vinte anos depois do
começo da difusão do ``Orvil'', já no início da terceira gestão do \textsc{pt} à
frente da presidência da República, era instituída uma comissão cuja
finalidade consistia em ``apurar graves violações de Direitos Humanos
ocorridas entre 18 de setembro de 1946 e 5 de outubro de
1988''.\footnote{``Relatório da \textsc{cnv}'', disponível no site da Comissão Nacional da Verdade, publicado em 10 de dezembro de 2014.}
Uma vez mais tentava"-se revelar à boa parte da sociedade brasileira as
atrocidades cometidas por agentes do Estado, elaborar coletivamente um
trauma de nossa história e proceder a uma reparação histórica para com
aqueles que foram perseguidos, exilados e torturados e para com os
familiares e amigos dos que foram assassinados e não poucas vezes
tiveram seus corpos ocultados, sem lhes dar o direito a um luto pleno.
Não se tratava, de fato, da possibilidade de punir os agentes envolvidos
nesses crimes, mas somente de conhecer mais precisamente seus atos e de
torná"-los públicos. Nem, por isso, se deixou de reavivar o negacionismo,
o revisionismo e o revanchismo de muitos setores militares e de
partidários da extrema direita.

No documentário ``Intervenção --- amor não quer dizer grande
coisa'',\footnote{Direção de Gustavo Aranda, Tales Ab'Sáber e Rubens
  Rewald. Confeitaria de Cinema / Cérebro Eletrônico, 2017.} assistimos
a uma longa e monológica sequência de manifestações alucinadas e
baseadas em teorias conspiratórias sobre o ``marxismo cultural'', sobre
a imaginada tentativa de implantação do comunismo no Brasil e sobre o
risco iminente de seu sucesso. A repetição da narrativa do ``Orvil'' não
poderia ser mais evidente. Em que pese a considerável diversidade dos
sujeitos que falam nas intervenções que o documentário compila, a
assistência produz ao mesmo tempo uma sensação de monotonia, por ser uma
diversificada repetição da mesma história, e inúmeros sobressaltos,
dadas as variações e as intensificações dos farsescos temas do comunismo
que nos ronda e da degradação da sociedade brasileira. Mediante a
observação desse extenso conjunto de discursos de extrema"-direita,
recolhidos na internet durante os anos de 2015 e 2016, intuímos sua
precedência e sua sucessão imediatas.

Desde o final de 2011 e o início de 2012, já havia começado a ocorrer
uma organização de declarações públicas contra a Comissão Nacional da
Verdade promovida principalmente por militares da reserva e de um
esquema de difusão dessas declarações em redes na internet entre
militantes de direita \emph{hard} e de extrema direita. Essa organização
já estava bem estabelecida, quando acontecem as rupturas entre os dois
polos ideológicos nas manifestações de 2013. Assim, no polo da extrema
direita, ``o novo fascismo já era bastante explícito nas ruas e na
internet em 2014, 2015, 2016 e 2017, como movimento de massas populares,
com várias faces e vários lugares sociais brasileiros, na construção do
impeachment de Dilma Rousseff, na criação política de seu \emph{pathos}
de mentira e de sua linguagem de agressão e calúnia, e do emudecimento
forçado do outro político, ao qual o jornalismo oficial se
acoplou''.\footnote{Ab'Sáber, Tales. ``O neofascismo no Brasil não é
  fruto de uma `parlamentada', mas é, possivelmente, o contrário''.
  Postagem no \emph{Facebook} feita no dia 11 de maio de 2020.}

A reação à instauração da Comissão Nacional da Verdade é um marco
importante do qual proveio o preparo e o fomento da convocação de uma
massa imbuída de postura intolerante e violenta na política nacional,
com base em um anticomunismo alucinado. A partir daí surgiram os
dispositivos de mentiras e produção de ódio à política, que já estavam
plenamente organizadas e muito atuantes nas manifestações com
considerável aderência popular contra o governo Dilma e na adesão de um
novo público do \textsc{psdb} à direita radical. Tudo isso se somou à
insatisfação dos bancos e do mercado financeiro à política de redução
dos juros adotada pelo governo a partir do segundo semestre de 2011 e à
cobertura e apoio da grande mídia à virada à direita de boa parte dos
protestos de junho de 2013.

Com as reações militares e direitistas contra a \textsc{cnv}, com o realinhamento
do grande capital contra o governo petista e com a intensificação da
detratação midiática do \textsc{pt}, reemerge uma personagem política lamentável,
mas poderosa: ``o homem conservador médio''. Essa figura ``antipetista
por tradição e anticomunista por natureza arcaica brasileira mais antiga
--- um homem de adesão ao poder por fantasia de proteção patriarcal e
agregada, fruto familiar do atraso brasileiro no processo da produção
social moderna --- pode entrar em cena como força política real, deixando
de expressar privadamente um mero ressentimento rixoso, carregado de
contradições, contra o relativo sucesso do governo lulo"-petista, que
jamais pôde ser verdadeiramente compreendido por ele''. Diante do
sucesso das gestões de Lula, das quais ele saiu com impressionantes
percentuais de popularidade e com as quais conseguiu eleger sua
sucessora, o ``governo só poderia ser vencido se lhe fosse projetado o
velho desejo autoritário brasileiro, o mais puro anticomunismo com
toques de moralismo neoudenista''.\footnote{Ab'Sáber, Tales. \emph{Dilma
  Rousseff e o ódio político}. São Paulo: Hedra, 2015, p.\,35 e 38.}

Fernando Henrique Cardoso não hesitou em propor passos decisivos nessa
direção. Movido por uma emulação constante e indisfarçável com Lula que
não lhe favorecia, ante a iminente terceira vitória eleitoral do \textsc{pt}, \textsc{fhc}
``propôs, de modo envergonhado, mas convicto, que o \textsc{psdb} guinasse à
direita e José Serra utilizou"-se abertamente da retórica anticomunista
em sua campanha contra Dilma Rousseff''.\footnote{Ab'Sáber, 2015, p.\,38.}
Consumada a vitória desta última, afetos negativos e conhecimentos
sociológicos combinaram"-se na formulação destes termos de Fernando
Henrique:

\begin{quote}
Enquanto o \textsc{psdb} e seus aliados persistirem em disputar com o \textsc{pt} a
influência sobre os ``movimentos sociais'' ou o ``povão'', isto é, sobre
as massas carentes e pouco informadas, falarão sozinhos. Isto porque o
governo ``aparelhou'', cooptou com benesses e recursos as principais
centrais sindicais e os movimentos organizados da sociedade civil e
dispõe de mecanismos de concessão de benesses às massas carentes mais
eficazes do que a palavra dos oposicionistas, além da influência que
exerce na mídia com as verbas publicitárias. Sendo assim, dirão os
céticos, as oposições estão perdidas, pois não atingem a maioria. Só que
a realidade não é bem essa. Existe toda uma gama de classes médias, de
novas classes possuidoras (empresários de novo tipo e mais jovens), de
profissionais das atividades contemporâneas ligadas à \textsc{ti} (tecnologia da
informação) e ao entretenimento, aos novos serviços espalhados pelo
Brasil afora, às quais se soma o que vem sendo chamado sem muita
precisão de ``classe C'' ou de nova classe média. Pois bem, a imensa
maioria destes grupos --- sem excluir as camadas de trabalhadores urbanos
já integrados ao mercado capitalista --- está ausente do jogo
político"-partidário, mas não desconectada das redes de internet,
\emph{Facebook}, \emph{YouTube}, \emph{Twitter}, etc. É a estes que as
oposições devem dirigir suas mensagens prioritariamente, sobretudo no
período entre as eleições, quando os partidos falam para si mesmos, no
Congresso e nos governos. Se houver ousadia, os partidos de oposição
podem organizar"-se pelos meios eletrônicos, dando vida não a diretórios
burocráticos, mas a debates verdadeiros sobre os temas de interesse
dessas camadas.

O discurso, noutros termos, não pode ser apenas o institucional, tem de
ser o do cotidiano, mas não desligado de valores. Obviamente em nosso
caso, o de uma democracia, não estou pensando em movimentos contra a
ordem política global, mas em aspirações que a própria sociedade gera e
que os partidos precisam estar preparados para que, se não os tiverem
suscitado por sua desconexão, possam senti"-los e encaminhá"-los na
direção política desejada. Seria erro fatal imaginar, por exemplo, que o
discurso ``moralista'' é coisa de elite à moda da antiga \textsc{udn}. A
corrupção continua a ter o repúdio não só das classes médias como de boa
parte da população. Na última campanha eleitoral, o momento de maior
crescimento da candidatura Serra e de aproximação aos resultados obtidos
pela candidata governista foi quando veio à tona o ``episódio Erenice''.
Mas é preciso ter coragem de dar o nome aos bois e vincular a ``falha
moral'' a seus resultados práticos, negativos para a população. Mais
ainda: é preciso persistir, repetir a crítica, ao estilo do ``beba Coca
Cola'' dos publicitários. Não se trata de dar"-nos por satisfeitos, à
moda de demonstrar um teorema e escrever ``cqd'', como queríamos
demonstrar. Seres humanos não atuam por motivos meramente racionais. Sem
a teatralização que leve à emoção, a crítica --- moralista ou outra
qualquer --- cai no vazio.\footnote{Cardoso, Fernando Henrique. O papel
  da oposição. In: Revista \emph{Interesse Nacional}, n. 13, 2011. Tales
  Ab'Sáber também reproduz parte desse artigo de \textsc{fhc} e afirma que ele
  ``sinalizou, em um discurso estranho e novo à política nacional, muito
  assemelhado aos cálculos sociais de marqueteiros americanos, a brecha
  possível para a emergente \emph{tea partização} do espaço público da
  política brasileira'' (2015, p.\,40).}
\end{quote}

O texto de Fernando Henrique não ficou sem repercussão e foi criticado
por Lula, por outros políticos e por cientistas sociais.\footnote{Lula
  disse o seguinte: ``Nós já tivemos políticos que preferiam cheiro de
  cavalo que o povo. Agora tem um presidente que diz que precisa não
  ficar atrás do povão, esquecer o povão. Eu sinceramente não sei como é
  que alguém estuda tanto e depois quer esquecer do povão''.
``Lula critica \textsc{fhc}: `Povão são todos os brasileiros'\,'', \emph{Rede Brasil Atual}, 14 de abril de 2011.
  Entre outras críticas de cientistas sociais, ver: Souza, Jessé. A
  parte de baixo da sociedade brasileira. In: Revista \emph{Interesse
  nacional}, n. 14, 2011.} A despeito dessas e ainda de outras
críticas, provenientes até mesmo de seus aliados ideológicos, sua
proposta contribuiu para a consolidação dos discursos de ódio à
política, ao \textsc{pt} e aos partidos de esquerda. No contexto de sua
publicação, formava"-se uma espécie de coro ressentido e raivoso, mas
tomado como legítimo e legitimador de vozes cada vez mais coléricas, que
se lhe foram agregando: ``no período de ascensão e queda petista, atacar
com a máxima retórica, isenta de responsabilidade, em jornais, blogs ou
revistas, o comunismo imaginado do governo, tornou"-se um dos modos mais
fáceis e oportunos de ganhar dinheiro no mercado dos textos e das ideias
no Brasil''. Não era preciso mais do que repisar esse pensamento
preguiçoso, anacrônico e perverso para engrossar um consenso forjado e
fazer sucesso:

\begin{quote}
Era suficiente reproduzir a rede de ideias comuns e fixadas, com sua
linguagem agressiva, indignada artificial, que sustentassem todo dia o
mesmo curto circuito do pensamento. Simplificação espetacular e ponto
certo no imaginário autoritário, jornalistas, articulistas, programas de
\textsc{tv} e de rádio e revistas inteiras passaram, durante anos, a ler as
atividades do governo do ponto de vista extremo, limitado, do
anticomunismo imaginário. O delírio interessado, farsesco, não conhecia
limite, uma vez que se desobrigava radicalmente de checar
realidades.\footnote{Ab'Sáber, 2015, p.\,41.}
\end{quote}

Entre o final da primeira e o início da segunda década deste século \textsc{xxi},
vimos a conjunção entre um aumento da circulação da narrativa à
``Orvil'', um ressurgimento do ``homem conservador médio'', uma série de
reações militares e ultradireitistas contra a \textsc{cnv} e uma adesão de
formadores da opinião pública à impostura em torno da ditadura comunista
que se estaria instalando no Brasil. A reunião dessas frentes estendeu e
consolidou um campo radical que congregava a sanha neoliberal de
endinheirados e dos que se tomam como tal, o gozo narcísico
protofascista de recém empoderados do \emph{smartphone}, o entusiasmo
real da defesa nem sempre sincera da ética pública e da moralidade de
comportamento e o ódio antipopular brasileiro. Formava"-se uma estranha,
mas compreensível, coalização entre aficionados pelas liberdades
individuais e entusiastas da doutrina de segurança nacional e com ela se
reforçava cada vez mais a tendência de se recusar a ouvir e de tentar
silenciar o pensamento divergente, entre uns, e a de se calar, entre
outros. Em nome do combate ao adversário a ser vencido a qualquer custo,
passava"-se a tolerar e mesmo a legitimar o que sempre deveria ter sido
intolerável:

\begin{quote}
Assim se produzia o campo extremo, algo delirante, em que a luta
democrática antipetista encontrava a velha tradição autoritária
brasileira. E, por isso, agora que o país, em seu neo"-transe, se levanta
contra os comunistas inexistentes, em uma ritualização do ódio e da
ideologia, elegantes \emph{socialites} peessedebistas e novos
empresários \emph{teapartistas} convivem bem, nas ruas, fechando os
olhos para o que interessa, com bárbaros defensores de ditadura, homens
que discursam armados em cima de trios elétricos, clamando por
intervenção militar urgente no Brasil e sonhando com o voto em Jair
Bolsonaro.\footnote{Ab'Sáber, 2015, p.\,43.}
\end{quote}

Ultraliberais e anticomunistas, ricos e pobres de direita davam"-se as
mãos e recebiam apoio tanto da grande mídia quanto de amplos setores do
aparato policial. Com a força dessa união, foram arregimentando um
considerável apoio popular. Assim, pavimentava"-se o caminho para o golpe
sofrido por Dilma Rousseff, que tiraria depois de 13 anos o \textsc{pt} da
presidência da República. A emergência e os primeiros passos desse
intenso movimento, que já se havia entranhado até mesmo em parte das
classes populares, não foram organizados nem estiveram diretamente
ligados a Bolsonaro. Mas lhe concediam espaço e legitimidade de fala. Ao
coro de que provinha o ódio ao \textsc{pt} e a criminalização da esquerda,
juntavam"-se os gritos anticorrupção. Mesmo que ela não os tenha prendido
em sua maioria, a Lava"-jato contribuiu para engrossar a desconfiança em
relação aos políticos e aos partidos do centro e da direita. A julgar
por uma conhecida fotografia em que aparecem reunidos Carlos Wizard,
Flávio Rocha, Abílio Diniz, Carlos José Marques, José Serra, Henrique
Meirelles, Geraldo Alckmin, Michel Temer, Sérgio Moro e Aécio Neves,
estes dois últimos em clima de intimidade e contentamento, o lavajatismo
acabou cumprindo à sua revelia um importante papel na detratação de
tucanos e afins. Já pouco antes delas, mas principalmente durante as
eleições presidenciais de 2018, Bolsonaro passou a encarnar a condição
de candidato do ``partido da Lava"-Jato'':

\begin{quote}
Esse foi o verdadeiro embate dessa eleição, a disputa entre o lulismo e
o `partido da Lava Jato', que encarnava o `ser contra tudo que está aí',
em substituição à antiga polarização entre \textsc{pt} e \textsc{psdb}. Todas as outras
tentativas de construção de discurso e posicionamento na campanha de
2018 se revelaram anódinas, sem impacto, sem relevância, sem capacidade
de significância ou de reter a atenção do eleitor.

O candidato do `partido da Lava Jato' muito dificilmente sairia de uma
das siglas partidárias tradicionais de oposição ao \textsc{pt}. Jair Bolsonaro
soube se posicionar para ser esse postulante. Apresentou"-se como um
inimigo visceral do \textsc{pt} e como um político `diferente de tudo o que está
aí', e sustentou um discurso politicamente incorreto e de enfrentamento
contra a `bandidagem', além de uma defesa conservadora dos valores da
família cristã.\footnote{Moura e Corbellini, 2019, p.\,56--57.}
\end{quote}

Usufruindo essa confortável posição, Bolsonaro pôde desfrutar da
conveniência de se abster de muitas circunstâncias em que seria obrigado
a falar, a debater e a se expor. O candidato do \textsc{psl} falou relativamente
pouco, porque uma série intensa e extensa de práticas e discursos
conservadores e reacionários continuava a ecoar fortemente na história
brasileira. Falou pouco, porque seu populismo se assentou nesta lógica:
para quem já está absolutamente convencido da existência de um inimigo
comum e da necessidade de eliminá"-lo, qualquer gesto bélico indicador já
é suficiente para o sentimento de integração comunitária às ``pessoas de
bem''. A esta e a outras, se soma a de que aos que já estão enredados em
um consenso e aos que só precisam entender ordens, aos que só cabe a
obediência, meia palavra basta\ldots{}

A explicação da vitória de Bolsonaro pelo surgimento do ``partido da
Lava"-Jato'' não é incorreta, mas também não é suficiente. Ela também não
poderia se restringir a mencionar o atentado à faca sofrido pelo
candidato do \textsc{psl}. Segundo a versão da história de alguns cientistas
sociais, houve uma série de motivos para que isso ocorresse. ``As razões
estruturais, as mais importantes, que levaram à sua eleição, já estavam
anteriormente estabelecidas: (1) a desmoralização das elites políticas e
do conjunto do sistema partidário tradicional provocada pela Lava"-Jato
(talvez essa seja a maior herança da operação sob a perspectiva do
eleitor)''; mas também ``(2) o aprofundamento da crise na segurança, que
adquire o status de maior problema nacional na percepção da opinião
pública''; e ainda ``(3) o crescimento da importância das redes sociais,
particularmente do WhatsApp como nova plataforma de comunicação, que
revoluciona a competição eleitoral e o modo de fazer campanha política
no Brasil''.\footnote{Moura e Corbellini, 2019, p.\,30.}

Não é preciso que subscrevamos integralmente a ideia de que essas teriam
sido as ``razões estruturais mais importantes''. Também não é necessário
que pretendamos dar uma resposta completa e definitiva a esta questão
que formulamos acima: como Bolsonaro pôde falar tão pouco e mesmo assim
vencer as eleições? A despeito disso, podemos arrolar um conjunto de
fatores que, articulados com os que já mencionamos, concorreram
fundamentalmente para a condução de Bolsonaro à presidência da
República. Esses fatores podem ser mais ou menos divididos em
estruturais, conjunturais e eventuais. Julgamos que a razão estrutural
mais decisiva resida no fato de a sociedade brasileira ser histórica e
profundamente injusta e desigual. É por isso que em seu interior grassam
os ataques às práticas e aos discursos que visam à sua transformação,
sem que lhes seja questionada a legitimidade, mesmo em suas
manifestações mais violentas.

Herdamos de um dos episódios mais trágicos de nossa história uma
terrível indiferença com o sofrimento ``da parte de baixo'' da população
do Brasil: ``A escravidão legou"-nos uma insensibilidade, um
descompromisso com a sorte da maioria que está na raiz da estratégia das
classes mais favorecidas, hoje, de se isolar, criar um mundo só para
elas, onde a segurança está privatizada, a escola está privatizada, a
saúde''.\footnote{Luiz Felipe de Alencastro citado por Schwarz, 2019.}
Já o legado de outro abominável acontecimento, a ditadura civil"-militar
de 64, cujos agentes não foram julgados e punidos, foi a aceitação do
inadmissível: a possibilidade de ampla circulação de uma narrativa que
promove um aumento progressivo na recepção de discursos de ódio e de
atos violentos. A essa estrutura de desigualdades e injustiças se soma
uma série de fatores conjunturais.

Entre esses fatores, encontra"-se a cessão do otimismo que frequentou o
Brasil durante os dois mandatos de Lula: ``A sequência de superações que
durante algum tempo deu a sensação de que o país decolava rumo ao
primeiro mundo pode ter chegado a seu limite, respeitadas as balizas da
ordem atual. Esgotada a conjuntura internacional favorável, em especial
a bonança das `commodities', o dinheiro necessário a novos avanços
desapareceu, interrompendo o processo de integração nacional e seu clima
de otimismo''.\footnote{Schwarz, 2019.} O entusiasmo cedeu à crença
pessimista na pior crise pela qual o país já teria passado. Nela se deu
a disseminação do consenso segundo o qual haveria um completo
esgotamento político das gestões do \textsc{pt}, que contribuía para o
recrudescimento da demonização do campo político, para a radicalização
da direita e para a consequente expansão da extrema"-direita. A postura e
a resposta discursiva de Bolsonaro estavam bem mais sintonizadas com
essa conjuntura: ``o discurso de Haddad mirava demais no passado,
enquanto o de Bolsonaro falava de uma ruptura com o presente: fazer algo
`diferente de tudo o que está aí'. O adversário de Haddad era o `golpe
que impediu Lula de ser candidato' e sua utopia, o governo que iria
`trazer o Brasil de Lula de volta'. Os adversários de Bolsonaro eram os
políticos corruptos e a `bandidagem', e a sua utopia, o governo de um
presidente honesto, que garantisse mais segurança nas ruas e que
protegesse a `inocência de nossas crianças'\,''.\footnote{Moura e
  Corbellini, 2019, p.\,107--108).}

Enfim, quanto aos fatores eventuais, há entre outros os seguintes: o
revanchismo de Aécio Neves, depois de sua derrota nas eleições
presidenciais de 2014. Ao retratar a birra inconsequente e cheia de
consequências de Aécio, Criolo escreveu estes versos em uma de suas
canções: ``Este abismo social requer atenção/ Foco, força e fé, já falou
meu irmão/ Meninos mimados não podem reger a nação/ Meninos mimados não
podem reger a nação''. À molecagem de Aécio se conjugou o aumento das
operações da Lava"-Jato já nos primeiros meses do segundo mandato de
Dilma Rousseff, cujo principal alvo era composto por políticos do \textsc{pt}, e
as prisões de políticos petistas ao longo de todo ano de 2015. Daí, em
conjunto com tudo mais que relatamos acima, viria o golpe jurídico e
parlamentar sofrido por Dilma Rousseff em 2016. Também ocorreram a
adesão elitista de primeira hora dos órfãos do \textsc{psdb} à candidatura de
Bolsonaro, com seu poder de disseminar consensos e formar a opinião
pública,\footnote{``Segundo diversas pesquisas públicas, os eleitores que
  `fidelizaram' mais cedo {[}à candidatura de Bolsonaro{]} foram os das
  classes A e B, de maior renda e escolaridade, sobretudo homens
  moradores de grandes centros urbanos. Todos os indícios apontavam para
  um grupo `órfão' do \textsc{psdb}''. (Moura e Corbellini, 2019, p.\,91)} e toda
consonância das posições econômicas do programa do candidato do \textsc{psl} com
o atual e hegemônico neoliberalismo meritocrático e financista.\footnote{``Há
  bastante em comum entre a vitória eleitoral de Bolsonaro, em 2018, e o
  golpe de 1964. Nos dois casos, um programa francamente pró"-capital
  mobilizou, para viabilizar"-se, o fundo regressivo da sociedade
  brasileira, descontente com os rumos liberais da civilização. (\ldots{})
  Cinquenta anos atrás, quem marchava com Deus, pela família e a
  propriedade, eram os preteridos pela modernização, representativos do
  Brasil antigo, que lutava para não desaparecer, mesmo sendo vencedor.
  É como se a vitória da direita, com seu baú de ideias obsoletas, não
  bastasse para desmentir a tendência favorável da história. Apesar da
  derrota do campo adiantado, continuava possível --- assim parecia ---
  apostar no trabalho do tempo e na existência do progresso e do futuro.
  Ao passo que o neoatraso do bolsonarismo, igualmente escandaloso, é de
  outro tipo e está longe de ser dessueto. A deslaicização da política,
  a teologia da prosperidade, as armas de fogo na vida civil, o ataque
  aos radares nas estradas, o ódio aos trabalhadores organizados etc.
  não são velharias nem são de outro tempo. São antissociais, mas
  nasceram no terreno da sociedade contemporânea, no vácuo deixado pela
  falência do Estado. É bem possível que estejam em nosso futuro, caso
  em que os ultrapassados seríamos nós, os esclarecidos. Sem esquecer
  que os faróis da modernidade mundial perderam muito de sua luz.''
  (Schwarz, 2019).}

Aos fatores eventuais ainda se somam estes outros: a série composta
pelas acusações, julgamentos e prisão de Lula, com uma enorme e parcial
cobertura midiática; o atentado sofrido por Bolsonaro em Juiz de Fora no
dia 06 de setembro de 2018 e toda sua imensa repercussão em todos os
veículos da mídia brasileira e em todas as redes sociais.\footnote{``O
  episódio trouxe benefícios eleitorais e políticos evidentes a
  Bolsonaro. As notícias sobre seu estado de saúde tiveram ampla
  cobertura de televisão'' (Moura e Corbellini, 2019, p.\,93)} Tudo sem
mencionar o fato decisivo de ele não ter sido punido severamente tanto
pelas quebras de decoro, pelos discursos discriminatórios, pelas ameaças
e pelas incitações à violência, quanto pelas acusações de nepotismo, de
uso de dinheiro público para pagamento de despesas privadas e de
utilização de verbas privadas para custear gastos da campanha
eleitoral.\footnote{Sobre acusações de crimes relativos à ética e à
  corrupção, ver o capítulo ``Questão de ética'' de Saint"-Clair (2018,
  p.\,131--158).} Essa ausência de punições foi fundamental para fazer de
Bolsonaro um ``mito'', uma entidade ilibada, que veio salvar o país da
degeneração total e que quase foi morto por forças do mal que ele veio
combater\ldots{} Se tudo isso já não bastasse, houve ainda a evasão de Ciro
Gomes depois do primeiro turno das eleições\ldots{}

\section{O presidente de palavra vã e a~necropolítica~na~presidência}

Vimos até aqui uma amostra das principais coisas ditas por Bolsonaro e
de seus mais recorrentes modos de dizer, desde quando ele ainda era um
ganancioso capitão do exército, passando pelo seu curto mandato como
vereador no Rio de Janeiro e por seus vários mandatos como deputado
federal, até sua condição de candidato à presidência da República nas
eleições de 2018. Subsidiado por essa conjunção de fatores estruturais,
conjunturais e eventuais que acabamos de mencionar, o Bolsonaro orador
pôde sair vitorioso desta sua última e mais ambiciosa empreitada. A
oratória agressiva, a retórica reacionária e os discursos violentos, que
chegaram a anunciar a eliminação de adversários, foram e continuam a ser
traços marcantes da linguagem do presidente. Isso não significa que não
tenham havido mudanças em sua trajetória. Os contrastes entre o deputado
falastrão, mas também sádico e indecoroso, e o candidato lacônico, mas
também colérico e incitador ao ódio, são provas dessas modificações.

Esta já longa história de um Bolsonaro que fala às massas e a outros
setores sociais em uma língua ora mais ora menos manifestamente
carregada com as marcas de um nosso fascismo tupiniquim cotidiano
poderia sem mais demora chegar ao seu fim. Antes disso, porém,
dedicaremos uma última e sucinta reflexão sobre as relações entre a
linguagem e a verdade em pronunciamentos mais recentes de Bolsonaro e
sobre os nefastos efeitos que eles têm promovido ou intensificado na
sociedade brasileira. Para tanto, apenas reproduziremos e comentaremos
brevemente alguns trechos de sua primeira ``live'' e de sua primeira
intervenção na tevê, realizadas logo após o anúncio de sua vitória nas
eleições presidenciais, e algumas de suas declarações no contexto da
pandemia provocada pelo coronavírus.

No domingo, dia 28 de outubro de 2018, depois de consumado seu último e
mais importante triunfo eleitoral, as primeiras palavras de Bolsonaro
não foram pronunciadas em um recinto público ou na sede de seu partido.
Sua primeira intervenção foi realizada em sua sala de estar, por meio de
uma ``\emph{live}'' via \emph{Facebook}. Nada surpreendente.\footnote{Bolsonaro
  já fora considerado ``um fenômeno na internet'' (Saint"-Clair, 2018) e
  o pleito presidencial vencido por ele, ``a eleição do WhatsApp''
  (Moura e Corbellini, 2019). Além disso, analistas já apontaram certa
  tendência de usos distintos de cada uma das redes sociais: ``Usa
  diferentes canais, com discursos diferentes dirigidos a diferentes
  públicos, explica Francisco Carvalho de Brito, diretor do
  \emph{Internet Lab}, consultoria de direito e tecnologia. `Bolsonaro
  usa o \textsc{fb} para divulgar sua agenda, para falar com suas bases,
  que não confiam na grande mídia. Quando quer moderar seu discurso,
  concorda em dar entrevistas para a televisão para enviar sinais aos
  mercados, às instituições \ldots{} Ele usa o \emph{Twitter} para responder
  rapidamente às questões (polêmicas). Usa os grupos de WhatsApp como
  fã"-clubes em que se pode fazer parte da sua rede''. Gortázar, Naiara
  Galarraga. ``Bolsonaro, um candidato que cresceu no Facebook e não quer
  sair de lá'', jornal \emph{El país}, 26 de outubro de 2018.}
Esta foi a via em que ele havia se refugiado praticamente durante toda
sua campanha. Sentado junto a uma mesa e ladeado por sua esposa,
Michelle Bolsonaro, à sua direita, e por uma intérprete de Libras, à sua
esquerda, Bolsonaro disse o seguinte:

\begin{quote}
Os médicos e demais profissionais de saúde da Santa Casa de Juiz de Fora
e do Hospital Albert Einstein, em São Paulo, operaram um verdadeiro
milagre, mantendo a minha vida.

Fizemos uma campanha não diferente dos outros, mas como deveria ser
feita. Afinal de contas, a nossa bandeira, o nosso slogan, eu fui buscar
no que muitos chamam de caixa de ferramentas para consertar o homem e a
mulher, que é a Bíblia sagrada. Fomos em João, 8, 32: ``E conhecereis a
Verdade e a Verdade vos libertará''. Nós temos que nos acostumar a
conviver com a verdade. A verdade tem de começar a valer dentro dos
lares até o ponto mais alto, que é a presidência da República.

Graças a Deus, essa verdade o povo entendeu perfeitamente.

Nada mais gratificante do que quando estive em Manacapuru, coração do
Amazonas, conversando com pessoas simples, mas que tinham sede de
conhecer a verdade e de conversar com alguém que realmente os tratava
com o devido respeito e consideração.
\end{quote}

Minutos mais tarde, ocorreria seu primeiro pronunciamento na \textsc{tv},
realizado também ele em sua casa na Barra da Tijuca no Rio de Janeiro.
Desta vez, Bolsonaro está de pé e, além de sua esposa e da intérprete de
Libras, estão ao seu lado ou logo às suas costas o senador e pastor
evangélico Magno Malta, um dos coordenadores da campanha, Gustavo
Bebbiano, e os deputados federais recém"-eleitos, Alexandre Frota e Hélio
Lopes, entre outros apoiadores. Desde que o microfone do repórter que
está ali presente consegue captar uma fala breve e introdutória de
Bolsonaro, que já está em curso, percebemos que ele está fazendo
referência a Magno Malta e agradecendo"-o pela ``aproximação com
evangélicos, com católicos e com as demais religiões''. Ele diz ainda
``Agradecer a Deus e pedir sabedoria pra que nós possamos continuar
nessa jornada rumo a presidência da República'', antes de passar a
palavra ao senador evangélico, que faz uma oração durante cerca de 2
minutos e meio. Encerrada a prece, depois desta fala do repórter,
``Estamos ao vivo aqui com o presidente eleito do Brasil, Jair Messias
Bolsonaro. Eu falo representando o pool de emissoras de televisão.
Parabéns pela votação\ldots{}'', Bolsonaro faz este pronunciamento:

Muito obrigado.

\begin{quote}
Primeiro, queria agradecer a Deus, que pelas mãos de homens e mulheres
da Santa Casa de Misericórdia de Juiz de Fora, bem como do Albert
Einstein, em São Paulo, me deixaram vivo. Com toda a certeza, essa é uma
missão de Deus.

``Conhecereis a Verdade e a Verdade vos libertará''

Faço de vocês minhas testemunhas de que esse governo será um defensor da
Constituição, da democracia e da liberdade. Isso é uma promessa; não de
um partido. Não é a palavra vã de homem. É um juramento a Deus. A
verdade vai libertar esse grande país, a liberdade vai nos transformar
em uma grande nação. A verdade foi um farol, que nos guiou até aqui e
que vai seguir iluminando o nosso caminho.
\end{quote}

No início das duas falas, há uma vez mais a repisada alusão ao atentado
que sofrera e à sua recuperação, o que lhe dá ensejo para mostrar"-se
grato aos profissionais de saúde, como intermediários, e a Deus, em
primeira e última instância. Com a caução divina, que operou ``um
verdadeiro milagre'' para que ele pudesse cumprir ``uma missão de
Deus'', Bolsonaro prossegue, expressando sua obsessão pela verdade.
Nesses dois curtos excertos, a palavra ``verdade'' é repetida 10 vezes.
O versículo bíblico transformado em mais um \emph{slogan} de campanha é
mencionado nos dois pronunciamentos. No primeiro deles, a verdade,
sobejamente conhecida pelo presidente recém"-eleito, torna"-se algo que se
deve impor: ``Nós temos que nos acostumar a conviver com a verdade. A
verdade tem de começar a valer dentro dos lares até o ponto mais alto,
que é a presidência da República''. Bolsonaro já a conhece e professa,
por isso, em que pese o uso da primeira pessoa do plural no primeiro
enunciado, ele próprio se exclui do grupo dos que ainda não estão
acostumados com ela. Agora, com sua presença, na qual o verbo faz"-se
carne, a verdade começará a valer na presidência.

Particularmente, no pronunciamento transmitido pela televisão, além das
reiteradas ocorrências do termo ``verdade'', há uma passagem em que a
obsessão bolsonarista com a verdade torna"-se ainda mais manifesta.
Trata"-se deste juramento: ``Faço de vocês minhas testemunhas de que esse
governo será um defensor da Constituição, da democracia e da liberdade.
Isso é uma promessa; não de um partido. Não é a palavra vã de homem. É
um juramento a Deus''. A instituição das ``testemunhas'', a indicação da
``promessa'', a denegação da ``palavra vã'' e a asserção do ``juramento
a Deus'' produzem uma saturação na garantia que se pretende outorgar ao
que é proferido. De modo geral, já fomos ensinados a ler o excesso como
sintoma da falta. Vimos aqui que, com Bolsonaro, algo semelhante já se
processara em sua insistência nas obsessivas demonstrações de coragem e
virilidade e também em não poucas simulações de humildade.

O juramento é o remédio contra o flagelo da violação da palavra dada.
Sua função consiste em tentar garantir a verdade do que é dito,
justamente porque sabemos que a linguagem humana nos permite dizer a
verdade, mas também mentir e guardar segredos, porque, diferentemente da
linguagem divina, que faz o que diz no próprio ato de dizer, na dos
humanos não há correspondência necessária entre palavras e coisas, falas
e ações. Ele destina"-se a impedir a desconfiança e a assegurar a verdade
de uma asserção ou de uma promessa. Mesmo já antes de um crescente e
relativo ``desencantamento do mundo'', o juramento se marcava menos pela
ira do Deus em nome do qual se jura do que pela busca por subtrair a
palavra de seu uso ordinário e pouco confiável, com vistas a torná"-la
digna de fé. Assim, ocorre uma desoneração dos riscos implicados na
checagem da palavra dada, uma vez que, se o juramento garante que o
sujeito diz a verdade, sua fala dispensa a verificação de
correspondência com o que foi dito e feito antes e a averiguação de
equivalência com o que será dito e feitos depois.\footnote{As ideias aqui
  expostas sobre o juramento provêm de: Agamben, Giorgio. \emph{O
  sacramento da linguagem: arqueologia do juramento}. Belo Horizonte:
  Editora \textsc{ufmg}, 2011.}

Em sua promessa, Bolsonaro evoca o testemunho de Deus e lhe dirige seu
juramento: ``É um juramento a Deus''. Independentemente de uma crença
cristã real ou simulada, considerando o histórico das absolvições e das
penas leves de que Bolsonaro se beneficiou tanto em sua carreira militar
quanto na política, se poderia dizer que ele tem razões para depositar
pouca fé na justiça ou para temer as devidas punições que ela lhe
poderia infligir. O testemunho e o endereçamento a Deus são
convenientes, se pensarmos que eles não podem de fato contestar o que
foi afirmado por quem jurou e que lhe dão garantia de que se diz a
verdade. Não há, portanto, motivo para recear tomar o santo nome em vão
nem para apresentar versões alternativas dos fatos. Já se disse que o
juramento ``é o compromisso mais grave que um homem pode contratar e a
falta mais grave que pode cometer, pois o perjuro depende não da justiça
dos homens, mas da sanção divina''.\footnote{Benveniste, Emile. A
  blasfêmia e a eufemia. In: \emph{Problemas de linguística geral}. Vol.
  \textsc{ii}. Campinas: Pontes, 1989, p.\,260--261.} Um lastro favorável de
impunidades pode tirar totalmente a gravidade desse compromisso e dessa
falta.

A promessa de Bolsonaro pode ser lida como seu avesso. Ora, ele toma o
santo nome de Deus em vão. Além disso, sua história está repleta de
casos em que ele não fez o que disse ou em que ele disse e desdisse o
dito sem maiores pudores. Finalmente, ele não parece temer o castigo por
fazê"-lo, porque foi muitas vezes perdoado. Sua jura, iniciada com um
sintomático verbo ``fazer'', no presente do indicativo da primeira
pessoa do singular, é um perfeito paradigma de um ato de fala
performativo. Bolsonaro fala e faz de seus interlocutores as testemunhas
do que diz, de modo que àqueles a quem ele se dirige não cabe resposta e
menos ainda contestação. Os atos de fala performativos caracterizam"-se
não somente porque produzem a ação que expressam no enunciado, mas
também porque são atos de linguagem autorreferenciais, ou seja, em vez
de proferirem uma constatação sobre um estado de coisas no mundo, tomam
a si mesmos como referentes. Eles remetem a uma realidade que eles
mesmos constituem. Assim, dispensam a verificação.

Nada poderia ser mais conveniente a Bolsonaro. O recém"-eleito presidente
da República não formula um enunciado promissório: ``Esse governo será
um defensor da Constituição, da democracia e da liberdade''. Essa sua
afirmação é precedida por um e sucedida por outro enunciado
performativo: pelo que institui as testemunhas e pelo que estabelece seu
próprio juramento. Sem eles, o que disse Bolsonaro poderia ser passível
de checagem. Com sua presença, pretende"-se suspender a possibilidade de
verificação, como se o \emph{factum} já estivesse no próprio
\emph{dictum}.\footnote{Trata"-se aí da distinção entre asserção e
  veridição: ``Enquanto a asserção tem um valor essencialmente
  denotativo, cuja verdade, no momento de sua formulação, é independente
  do sujeito e se mede com parâmetros lógicos e objetivos (condições de
  verdade, não contradição, adequação entre palavras e realidade), na
  veridição o sujeito se constitui e se põe em jogo como tal,
  vinculando"-se performativamente à verdade da própria afirmação''.
  (Agamben, 2011, p.\,68).} Ao formular o enunciado como o fez, é como
se Bolsonaro dissesse: ``Estou dizendo que digo a verdade. Não é preciso
verificar''. Ou como no chiste, neste caso, trágico: \emph{La garantía
soy yo}. Não somente busca dispensar a averiguação do que diz, mas
também tenta impedi"-la. No dia 05 de maio de 2020, já em seu segundo ano
de mandato, ao ser questionado por jornalistas de \emph{O Estado de São
Paulo} e da \emph{Folha de S.\,Paulo} sobre sua suposta interferência na
Polícia Federal, eis sua resposta: ``Cala a boca, não perguntei nada!
Cala a boca! Cala a boca!''\footnote{Della Coletta, Ricardo. ``Bolsonaro
  manda repórteres calarem a boca, ataca a \textit{Folha} e nega interferência na
  \textsc{pf}'', \emph{Folha de S.\,Paulo}, 05 de maio de 2020.
  Os veículos de imprensa em que trabalham os repórteres que tiveram de
  lidar com a estupidez de Bolsonaro são os que já disseram que a
  disputa entre Bolsonaro e Haddad impunha uma escolha difícil e que a
  ditadura brasileira não teria sido tão dura assim\ldots{}}

Nas falas de Bolsonaro, há uma vasta coleção de negacionismos: ``Não
pleiteio aumento salarial''; ``Eu não tenho nada a ver com isso'', ``Não
há mudança climática'', ``O nazismo não é de direita'', ``Num é palavra
minha, não'', ``Não é a palavra vã de homem'', ``Nunca quis interferir
na Polícia Federal'' etc. etc. A ela se soma o autoritarismo, bem
representado pelo ``Cala boca!'' e afins, e a ambos não raramente se
associa a grosseria. Esta última, para ficarmos apenas em dois casos, se
manifestou no insulto dirigido à jornalista Patrícia Campos Mello, da
\textit{Folha de S.\,Paulo}, quando no dia 18 de fevereiro disse ``Ela queria dar
o furo'',\footnote{Uribe, Gustavo. ``Bolsonaro insulta repórter da \textit{Folha}
  com insinuação sexual'', \textit{Folha de S.\,Paulo}, 18 de fevereiro de 2020.
  O insulto parecia desempenhar uma vez mais o papel de cortina de
  fumaça, além de ser um gesto de vingança: ``Em 18 de fevereiro, o
  antipresidente Jair Bolsonaro precisava tirar o foco da morte do
  miliciano Adriano da Nóbrega, pessoa"-chave para esclarecer o esquema
  de ``rachadinhas'' no gabinete de Flávio Bolsonaro, a relação da
  família Bolsonaro com as milícias que atuam no Rio de Janeiro e também
  quem mandou matar Marielle Franco --- e por quê. A eliminação de
  Nóbrega, com vários indícios de execução, voltava a colocar em
  destaque as relações dos Bolsonaros com as milícias. Era preciso
  desviar a atenção. Como de hábito, Bolsonaro usou o velho truque:
  criou um novo fato ao atacar a jornalista Patrícia Campos Mello, da
  \textit{Folha de S.\,Paulo}. A repórter, uma das mais competentes da sua
  geração, estava entre os jornalistas que denunciaram o uso fraudulento
  de nomes e \textsc{cpf}s para disparos de mensagens no WhatsApp em benefício de
  Bolsonaro.'' Brun, Eliane. ``Por que Bolsonaro tem problemas com furos'',
  jornal \emph{El país}, 11 de março de 2020.}
seguido por seu próprio riso e pelos risos de sua claque; e na defesa de
seus familiares e amigos em uma infame reunião ministerial, ocorrida em
22 de abril de 2020, na qual afirmou ``Eu não vou esperar foder a minha
família toda, de sacanagem, ou amigos meu''.\footnote{A reprodução na
  íntegra da reunião ministerial pode ser lida em ``Leia a íntegra da reunião ministerial de 22 de abril'', \textit{Uol}, Política, 22 de maio de 2020.}
Negacionismo e grosseria estavam ali mais do que nunca de mãos dadas.
Mas, a mentira também não esteve ausente de suas intervenções.

Ainda em seu primeiro mês de governo, no dia 22 de janeiro de 2019,
Bolsonaro fez o discurso de abertura do Fórum Econômico Mundial de
Davos. Inicialmente, o tempo de que ele poderia dispor para seu
pronunciamento era de 45 minutos. Sua equipe informou a organização do
Fórum que ele não precisaria de mais do que 30 minutos. Em uma fala
``relâmpago'', Bolsonaro não se estendeu nem sequer por 7 minutos.
Basicamente, sua intervenção consistiu em uma promessa de bons negócios
para capitalistas e financistas do Brasil e, principalmente, do mundo. A
postura e a movimentação do presidente faziam parecer que ele estava
engessado e a leitura do texto de sua fala no \emph{teleprompter} foi
sofrível. Enquanto seus apoiadores louvaram sua fala, muitos outros
consideraram que o breve discurso do presidente foi superficial e
decepcionou.\footnote{Ao registrarem seus comentários ao duvidoso texto
  de Reinaldo Polito, em que seu autor se esforça para encontrar méritos
  no desempenho oratório de Bolsonaro, intitulado ``Bolsonaro não
  comprometeu em Davos, mas precisa de treino para falar bem'',
  publicado no portal Uol no dia seguinte ao discurso de Bolsonaro, seus
  apoiadores disseram: ``Militar fala pouco mas sabe o que diz não fala
  aboborinhas e nem se estende para chamar atenção Parabéns Presidente
  continue com sua oratoria'', ``\textsc{blá blá blá papo furado. bolsonaro tem
    discurso simples, enxuto, verdadeiro e eficaz}. e de que adiantavam os
  discursos de lula recheados de piadinhas, marotagem e pura mentira pra
  depois inverter tudo e roubar? e os discursos malucos sem nexo e
  mentirosos de dilma? ou entao os reais mestres do discurso e
  dominantes da oratória como temer e renan que só mentem e enrrolam?'',
  ``Vcs deveriam parar de encher o saco do presidente, Não perceberam
  que o presidente é prático e direto, sem mínimi e papo furado. Ele é
  direto e objetivo, coisa que todos deveriam seguir''. Já as manchetes
  de não poucos veículos da impressa indicavam o inverso: ``O breve
  discurso de Bolsonaro decepciona em Davos''(Jornal \emph{El país}, 23
  de janeiro de 2019), ``Para especialistas, discurso de Bolsonaro em
  Davos foi superficial'' (Jornal \emph{Estado de Minas}, 23 de janeiro
  de 2019), ``Bolsonaro faz discurso genérico em Davos'' (portal
  \emph{G1}, 22 de janeiro de 2019)}

Por meio dessa sua medíocre performance oratória, podíamos ouvir os
versos de uma famosa canção de Raul Seixas: ``A solução pro nosso povo
eu vou dá / Negócio bom assim ninguém nunca viu / Tá tudo pronto aqui é
só vim pegar / A solução é alugar o Brasil / Nós não vamo paga nada /
Nós não vamo paga nada / É tudo \emph{free} / Tá na hora agora é
\emph{free} / Vamo embora / Dá lugar pros gringo entrar / Esse imóvel tá
pra alugar ah ah ah ah''.\footnote{``Aluga"-se''. Composição de Claudio
  Roberto Andrade De Azeredo e Raul Seixas. Álbum \emph{Abra"-te Sésamo}.
  Warner Chapell Music, 1980.} Enquanto a ironia de Raul Seixas ainda
dizia ``Os estrangeiros eu sei que eles vão gostar / Tem o Atlântico tem
vista pro mar / A Amazônia é o jardim do quintal / E o dólar dele paga o
nosso mingau'', a convicção de Bolsonaro afirmava ``Conheçam a nossa
Amazônia, nossas praias, nossas cidades e nosso Pantanal. O Brasil é um
paraíso, mas ainda é pouco conhecido!'' Logo depois desse seu
convite, ele não se acanhou em afirmar que ``Somos o país que mais
preserva o meio ambiente''. Aproximadamente, seis meses mais tarde,
quando o então diretor do Instituto Nacional de Pesquisas Espaciais
(\textsc{inpe}), Ricardo Galvão, divulgou números que revelavam um aumento
exponencial de desmatamento na Amazônia, Bolsonaro o acusou de estar
mentindo e o exonerou de seu cargo.\footnote{Brant, Danielle; Watanabe,
  Phillippe. ``Diretor do \textsc{inpe} será exonerado após críticas do governo a
  dados do desmate'', \textit{Folha de S.\,Paulo}, 02 de agosto de 2019.}

Ao final de seu primeiro mandato, depois de negacionismos e agressões
verbais, grosserias e mentiras, os partidários resolutos de Bolsonaro,
mas também grandes veículos de comunicação do Brasil ainda continuavam a
considerá"-lo um ``presidente sem filtro''.\footnote{Revista \emph{Veja}.
  Edição 2.666, ano 52, n. 52, 25 de dezembro de 2019.} Dois meses mais
tarde, a pandemia do coronavírus chegaria ao Brasil.\footnote{``O
  Ministério da Saúde confirmou, nesta quarta"-feira (26/2), o primeiro
  caso de novo coronavírus em São Paulo. O homem de 61 anos deu entrada
  no Hospital Israelita Albert Einstein, nesta terça"-feira (25/2), com
  histórico de viagem para Itália, região da Lombardia.'' Disponível no site do Ministério da Saúde.}
Além do tenebroso saldo de dezenas de milhares de mortes em apenas
poucos meses, assistimos a um tenebroso encontro entre negacionismo
renitente da doença, manifestações antidemocráticas e incitação da
violência contra a imprensa e o setor da saúde. No dia 17 de março, foi
confirmada a primeira morte no país. Pouco mais de quatro meses depois,
mais de 50 mil pessoas haviam morrido pela Covid"-19.

Enquanto essas milhares de vidas eram ceifadas, Bolsonaro produzia uma
série sinistra de ações e declarações que concorreram para esse aumento
exponencial de mortes. Ele estimulou a participação e tomou parte em
eventos que eram ao mesmo tempo atos de apoio a seu governo e protestos
contra instituições democráticas, nos quais os manifestantes pediam o
fechamento do Congresso e do \textsc{stf} e uma intervenção militar, em completo
desrespeito às recomendações da Organização Mundial da Saúde de
isolamento social, de não aglomeração e de uso de máscaras de proteção
em locais públicos. Bolsonaro ainda não revelou o resultado de seus
exames para o diagnóstico da doença, recomendou o uso de medicamento de
eficácia não comprovada e demitiu dois ministros da Saúde, que não
subscreveram integralmente seus desserviços à saúde pública. Tentou,
enfim, omitir dados de casos e de mortes por coronavírus, que deveriam
ser divulgados pelo Ministério da Saúde. Todas essas atitudes, que
mesclam irresponsabilidade, incompetência e desgoverno, sem dúvida,
contribuíram para o elevadíssimo número de mortes no Brasil. Além delas,
outras ações e afirmações de Bolsonaro durante esse período misturaram o
desatino e o tripúdio, o descompromisso e o delito.

Enquanto multiplicavam"-se os mortos, sempre sem máscara e sem se
distanciar de seu séquito, o presidente anunciou um churrasco e uma
``pelada'', passeou de \emph{jet ski} no Lago Paranoá, treinou tiros e
fez uma cavalgada. Na contramão do respeito pela vida e por tantas
mortes, Bolsonaro ainda produziu uma longa, desastrosa e já bem
conhecida sequência de declarações. Entre outras, relembremos as
seguintes:

\begin{quote}
Não podemos entrar em uma neurose como se fosse o fim do mundo. Outros
vírus mais perigosos aconteceram no passado e não tivemos essa crise
toda. Com toda certeza há um interesse econômico nisso tudo para que se
chegue a essa histeria.

Esse vírus trouxe uma certa histeria. Tem alguns governadores, no meu
entender, posso até estar errado, que estão tomando medidas que vão
prejudicar e muito a nossa economia

Depois da facada, não vai ser uma gripezinha que vai me derrubar, não,
tá ok?

Se for todo mundo com coronavírus, é sinal de que tem estado que está
fraudando a \emph{causa mortis} daquelas pessoas, querendo fazer um uso
político de números. (\ldots{}) Em São Paulo não estou acreditando nesses
números.

Ô, ô, ô, cara. Quem fala de\ldots{} eu não sou coveiro, tá?

Alguns vão morrer? Vão, ué, lamento. Essa é a vida.

E daí? Lamento. Quer que eu faça o quê? Eu sou Messias, mas não faço
milagre.
\end{quote}

Em uma rápida observação, percebemos que há aí acusação de histeria,
negação da doença e de seus riscos, insinuação de conspiração, visão
exclusivamente economicista, minimização de sua gravidade, sugestão de
fraude, desoneração de responsabilidade, lamento desmentido e piada
infame. Também a falta de empatia de Bolsonaro salta aos olhos, a ponto
de não poucos analistas acreditarem existir uma psicopatia em seu
comportamento.\footnote{Bächtold, Felipe; Arcanjo, Daniela. ``Psicanalistas
  veem Bolsonaro com atitude paranoica e onipotente diante da pandemia'',
  \textit{Folha de S.\,Paulo}, 04 de abril de 2020;
  e Gragnani, Juliana. ``Coronavírus: Falta de empatia de
  Bolsonaro com mortes por covid"-19 parece psicopatia'', \textit{\textsc{bbc} News}, 08 de junho de 2020.}
Como fica igualmente bastante claro nesta sua declaração, a principal
preocupação do presidente não é com as vidas perdidas, mas com tudo o
que possa desobrigá"-lo de ter de responder por suas ações: ``Querer
colocar a culpa de uma possível expansão do vírus na minha pessoa porque
vim saudar alguns na frente da Presidência da República, em um movimento
que eu não convoquei, é querer se ver livre da responsabilidade. Se eu
me contaminei, isso é responsabilidade minha''. Uma vez mais, Bolsonaro
mente e se desincumbe de sua carga: ele efetivamente havia participado
da convocação para as manifestações antidemocráticas e sua presença
naqueles atos demonstrava sua total irresponsabilidade pública. Isso
tudo sem ainda mencionar a ``live'' na qual estimulou seus partidários a
invadir hospitais, em busca de provas de que os números de enfermos, de
mortos e de gastos eram excessivos:

\begin{quote}
Tem hospital de campanha perto de você, hospital público, arranja uma
maneira de entrar e filmar. Muita gente está fazendo isso e mais gente
tem que fazer para mostrar se os leitos estão ocupados ou não. Se os
gastos são compatíveis ou não. Isso nos ajuda.
\end{quote}

O presidente agora fala para se desresponsabilizar, para fomentar
conluios, para dar comandos criminosos e para emplacar uma narrativa
única no meio da pandemia. Vimos que esta não é primeira vez que isso
ocorre. Ao longo de sua breve carreira militar e de sua longa trajetória
política, ele já falou para se descomprometer, para detratar e tentar
eliminar adversários tornados inimigos e criminosos, para incitar a
violência ao outro, para calá"-lo simbólica e fisicamente. Suas falas
fomentam direta e indiretamente a violência. Desde sua ascensão do baixo
clero político, as já muito agressivas, mas ainda, digamos,
institucionais, falas de Bolsonaro promoveram um grande aumento da
violência verbal entre apoiadores. Nesse sentido, houve uma chocante
ampliação do número de sites neonazistas durante o atual governo de
extrema direita.\footnote{Alessi, Gil; Hofmeister, Naira. ``Sites
  neonazistas crescem no Brasil espelhados no discurso de Bolsonaro,
  aponta \textsc{ong}'', jornal \emph{El país}, 09 de junho de 2020.}
Além dessa expansão da violência verbal, assistimos atônitos e
indignados ao crescimento de atos de violência propriamente ditos:
repórteres e profissionais da saúde foram agredidos, policiais foram
filmados agredindo e torturando jovens negros e pobres. Ante esse
recrudescimento dos abusos e da brutalidade, o Ministério da Família, da
Mulher e dos Direitos Humanos excluiu os casos de violência policial do
relatório anual sobre violações de direitos humanos.

Com Bolsonaro, a necropolítica chegou resolutamente à presidência da
República. Pelo histórico de sua obra, mas, sobretudo, por sua
desastrosa atuação durante a maior crise sanitária do século \textsc{xxi}, sua
gestão é acusada de ser um governo contra a vida.\footnote{Nunes,
  Rodrigo. ``O país de um futuro pior'', \emph{Folha de S.\,Paulo},
  Ilustríssima, 21 de junho de 2020, p.\,\textsc{b}11 e \textsc{b}12; e Solano, Esther.
  ``Mortos e mais mortos'', \emph{Carta Capital}, n.~1111. 24 de junho de
  2020, p.\,19.} De fato, o vereador populista já pregava a necessidade
de políticas que buscassem impedir famílias pobres de terem filhos, em
um pensamento no qual sem disfarce se articulam pobreza, negritude e
criminalidade. Já o deputado falastrão dizia que a ditadura matou pouco
e que deveriam ter morrido ao menos uns 30 mil. Mais recentemente, o
candidato lacônico, fora da televisão, em que expunha sua versão um
pouco menos bárbara, afirmava que era preciso fuzilar inimigos e
desová"-los na ``ponta da praia''. Frente a milhares de mortos, o
presidente disse: ``E daí?''. Bolsonaro intensifica e desempenha assim
um papel fundamental no funcionamento desse sistema em que o Estado e o
capital decidem quem pode viver e quem deve morrer.

No ``fascismo eterno, não há luta pela vida, mas antes `vida para a
luta'\,''.\footnote{Eco, 2018, p.\,52.} A natureza bélica fascista encontra
na história do Brasil os genocídios perpetrados contra negros e
indígenas, contra empobrecidos e marginalizados, a herança das
impunidades de quem manda e de quem executa as ações violentas e fatais
via aparato policial e a intensa e extensa circulação de narrativas que
não só toleram a intolerável selvageria, mas também estimulam o uso da
força bruta sobre quem pode menos. Assim, práticas e discursos já
antigos construíram a normalização do sofrimento e da morte dos sujeitos
da parte de baixo da sociedade brasileira. Outros mais ou menos recentes
consolidaram a naturalização de que devemos nos submeter a muitas dores
em nome da economia. Sua junção em um contexto de colapso sanitário é a
crônica de uma tragédia anunciada e consumada. Um enorme percentual
dessas mortes é de responsabilidade do governo Bolsonaro. Para tentar
escapar da culpa que lhe deveria pesar sobre os ombros e em sua
consciência, a presidência da República emprega um seu expediente
conhecido: uma soma de mentiras e segredos.

Amparado no suposto dilema entre a saúde ou a economia, Bolsonaro
consegue em boa medida emplacar a mentira de que a recessão econômica,
os altos índices de desemprego, as dificuldades das classes médias e as
penúrias dos pobres é culpa da pandemia e a de que a solução é
sacrificar o relativo bem"-estar físico em benefício da retomada do
crescimento econômico a ser revertido em melhora das condições de vida
para todos. Noutros termos, é a promessa sempre feita e nunca cumprida
por aqui de fazer crescer o bolo para depois dividi"-lo. Ela está
destinada a se cumprir ainda menos no descomunal neoliberalismo
necropolítico bolsonarista, no qual a economia das pessoas de bens joga
sempre contra a vida dos descartáveis. Essa mentira do governo e a
cínica sinceridade dos endinheirados\footnote{O Brasil somava mais de
  100 mil casos de covid"-19 e mais de 7 mil mortes e uma curva
  ascendente no dia anterior à seguinte declaração de Guilherme
  Benchimol, milionário e presidente da \textsc{xp} investimentos: ``Acompanhando
  um pouco os nossos números, eu diria que o Brasil está bem. O pico da
  doença já passou quando a gente analisa a classe média, classe média
  alta. O desafio é que o Brasil é um país com muita comunidade, muita
  favela, o que acaba dificultando o processo todo. (\ldots{}) É um desafio
  você pedir que a população inteira fique presa em casa. Um terço da
  população vive de diária e se não trabalhar hoje não vai comer, no
  máximo, na semana que vem. (\ldots{}) Não me lembro do Brasil viver sem
  instabilidade política. Se não afetar a economia e as reformas
  continuarem avançando, a crise política não atrapalha, é muito mais um
  barulho de curto prazo. (\ldots{}) Vamos continuar crescendo independente
  do cenário''. Moura, Júlia. ``Pico de Covid"-19 nas classes altas já
  passou; o desafio é que o Brasil tem muita favela, diz presidente da
  \textsc{xp}'', \emph{Folha de S.\,Paulo}, 05 de maio de 2020.}
assentam"-se em uma triste realidade para os mais pobres. Para estes,
ficar em casa é um privilégio e não um direito. Com essa experiência de
vida, dada a força da ideologia reacionária, ao invés de engrossar o
coro da luta por conquista de direitos, entre muitos desses desvalidados
estabelece"-se a ideia de acabar com o que concebem como luxo
inalcançável.

Falso dilema e mentira, sinceridade cínica e experiência real
encaminham"-nos à concretização do propósito político da propaganda
fascista: o de transformar pessoas em massas e conduzi"-las a apoiar
medidas que as oprimem, exploram e censuram. Para isso, a opção pela
economia, em detrimento da vida, a segurança dos ricos e remediados e as
limitações impostas às vidas dos pobres arrastam consigo outras crenças
ou segredos: oculta"-se a gravidade da crise sanitária. Omitem"-se as
alternativas para o que poderia ser efetivamente a salvação de vidas e
da economia, como uma sólida e ampla ação do Estado brasileiro que
garantisse renda e emprego até que a pandemia estivesse sob controle. O
governo Bolsonaro, que pretendia conceder um auxílio emergencial no
valor de R\$ 200,00, repisa a ideia de que é impossível arcar com mais
duas parcelas de R\$ 600,00 a milhões de brasileiros que precisam desse
dinheiro para sobreviver. Não cogita em hipótese alguma de outorgar o
direito a uma renda básica universal. Dadas as injustiças da história
brasileira, as crenças e a trajetória de Bolsonaro, suas ações, omissões
e declarações, não há surpresa em sua aposta: lavar as mãos em uma bacia
muito mais cheia com o sangue das vidas descartáveis do que com o das
vidas protegidas.

Nas palavras de um artista brasileiro, eis o que nos ocorre nestes
tempos:

\begin{quote}
A indiferença pela vida, o Brasil sempre teve. Não é criação
bolsonarista. A gente sempre topou deixar muita gente morrer, tolerando
o intolerável. Temos mais de 100 mil mortes ao ano que dava para dar um
jeito, né? Agora, o grau de elaboração que isso tomou nos levou ao plano
do inominável.\footnote{Ramos, Nuno. ``Elegemos o pior brasileiro entre
  210 milhões''. Entrevista concedida a Guilherme Amado para a revista
  \emph{Época}, 17 de junho de 2020.}
\end{quote}

Ao compreensível e necessário pessimismo no diagnóstico, acrescentemos
um relativo otimismo na ação. Uma ação não suficiente, mas necessária,
nos parece ser esta de examinar as propriedades da linguagem de um
neofascismo brasileiro, para mais bem compreendê"-la, criticá"-la e
denunciá"-la. Em suas histórias, tanto o que ela diz quanto suas maneiras
de dizer buscam transformar o inominável em alternativa única e, assim,
nos imobilizar nessa apoteose da necropolítica bolsonarista, nesse ápice
do retrocesso em nosso cambaleante e contraditório processo
civilizatório. A elaboração de nossa versão crítica dessas histórias que
tornaram aceitável toda sorte de descaso com a vida humana é mais um
passo na luta sem fim, mas cheia de propósitos, pela emancipação dos que
sem tréguas continuam a ser acossados, oprimidos e excluídos. Seu suor,
seu sangue e suas lágrimas continuam a jorrar aos borbotões entre nós,
enquanto os dos remediados e abastados não pingam mais do que algumas
gotas.

\bigskip

\begin{flushright}
\textit{Carlos Piovezani}
\end{flushright}


